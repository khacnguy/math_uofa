\documentclass[11pt]{article}
    \title{\textbf{Math 217 Homework I}}
    \author{Khac Nguyen Nguyen}
    \date{}
    
    \addtolength{\topmargin}{-3cm}
    \addtolength{\textheight}{3cm}
    
\usepackage{amsmath}
\usepackage{mathtools}
\usepackage{amsthm}
\usepackage{amssymb}
\usepackage{pgfplots}
\usepackage{xfrac}
\usepackage{hyperref}
\usepgfplotslibrary{polar}
\usepgflibrary{shapes.geometric}
\usetikzlibrary{calc}
\pgfplotsset{compat = newest}
\pgfplotsset{my style/.append style = {axis x line = middle, axis y line = middle, xlabel={$x$}, ylabel={$y$}, axis equal}}
\begin{document}
\section*{A1.1}
We have
\[
    -1 = \phi(-1) = \phi(i \cdot i) = \phi(i) \cdot \phi(i) \implies \phi(i) = i \text{ or } \phi(i) = -i   
\]
First, consider the case $\phi(i) = i$, for any $z = x + iy$ where $x, y \in \mathbb{R}$
we have that 
\[
    \phi(z) = \phi(x+ iy) = \phi(x) + \phi(iy) = x + \phi(i) \cdot \phi(y) = x + iy = z
\]
In the other case $\phi(i) = -i$, for any $z = x + iy$ where $x, y \in \mathbb{R}$
\[
    \phi(z) = \phi(x + iy) = \phi(x) + \phi(i) \cdot \phi(y) = x - iy = \overline{z}
\]
Hence, $\phi$ is either the identity or the conjugate map.
\newpage
\section*{A1.2}
\subsection*{a}
According to the Fundamental Theorem of Algebra, we know that every polynomials equation of degree $n$ 
with complex number coefficents has $n$ roots in $\mathbb{C}$, hence for every polynomials $f(z) \in \mathbb{R}[x]$, we can rewrite it in $\mathbb{C}$ as 
\[
    f(z) = \sum_{i=1}^n a_iz^i = a_n\prod_{i=1}^n (z-z_i)
\]
for some $z_i \in \mathbb{C}$. Suppose $z$ is a root of the polynomial, then 
\begin{equation*}
    \begin{aligned}
        f(\overline{z}) &= \sum_{i=1}^n a_i \overline{x}^i \\
        &= \sum_{i=1}^n \overline{a_i} \overline{x^i}\\
        &= \sum_{i=1}^n \overline{a_ix^i}\\ 
        &= \overline{\sum_{i=1}^n a_i x^i}  \\
        &= 0
    \end{aligned}
\end{equation*}
Hence, $\overline{z}$ is also a root. \\
Now suppose there is an irreducible polynomials in $\mathbb{R}[x]$ of degree greater than 2. 
If the degree of that polynomial f(x) is odd, then applying the intermediate value theorem
when we consider $\lim_{x \to \infty} g(x)$ and $\lim_{x \to - \infty}g(x)$, yields that there is a real root. 
Therefore, the irreducible polynomial must have an even degree. Since it has an even degree, we can rewrite $f(x)$ as 
\[
    f(x) = a_n\prod_{i=1}^{n/2}(x-z_i)(x-\overline{z_i}) = a_n \prod_{i=1}^{n/2} (x^2 - 2\text{Re}(z_i)x + |z_i|^2)
\]
\subsection*{b}
\begin{equation*}
    \begin{aligned}
         &x^8+8 = 0 \\
        \implies &x^8 = -8 \\
        \implies &x^8 = 8(e^{i\pi}) \\
        \implies &x = x_k = \sqrt[8]{8}\left(\cos\left(\frac{\pi}{8} + \frac{2\pi k}{8}\right) + i \sin \left(\frac{\pi}{8} + \frac{2\pi k}{8}\right)\right)
    \end{aligned}
\end{equation*}
for natural $0\le k \le 7$. \\
We have for any $0\le k \le 3$,
\begin{equation*}
    \begin{aligned}
        \overline{x_k}&=  \sqrt[8]{8}\left(\cos\left(\frac{\pi}{8} + \frac{2\pi k}{8}\right) - i \sin \left(\frac{\pi}{8} + \frac{2\pi k}{8}\right)\right) \\
        &= \sqrt[8]{8} \left(\cos\left(\frac{15\pi}{8} - \frac{2\pi k}{8}\right) + i \sin \left(\frac{15\pi}{8} - \frac{2\pi k}{8}\right)\right) \\
        &= \sqrt[8]{8} \left(\cos\left(\frac{\pi}{8} + \frac{2\pi (7-k)}{8}\right) + i \sin \left(\frac{\pi}{8} + \frac{2\pi (7-k)}{8}\right)\right) \\
        &= x_{7-k}
    \end{aligned}
\end{equation*}
Therefore, 
\[
    (x-x_k)(x-x_{7-k}) = x^2 -2x \left(\sqrt{2}\cos\left(\frac{\pi}{8} + \frac{2\pi k}{8} \right) \right) + \sqrt[4]{8}
\]
Hence, 
\begin{equation*}
    \begin{aligned}
        x^8 +8 &= \prod_{k=0}^3 \left( x^2 -2x \left(\sqrt{2}\cos\left(\frac{\pi}{8} + \frac{2\pi k}{8} \right) \right) + \sqrt[4]{8}\right) \\
    \end{aligned}
\end{equation*}

\newpage
\section*{A1.3}
First consider the isomorphic map:
\[
    \phi: (x,y) \to x+iy, \indent \mathbb{R}^2 \to \mathbb{C}    
\]
We have that for all $n \in \mathbb{N}^*$
\[
    \phi(P_n) - \phi(P_{n-1}) = n \cdot e^{\frac{2\pi(n-1)}{3}i} 
\]
$e^{2\pi i/3} = \frac{-1+i\sqrt{3}}{2}$
Hence, we have that 
\begin{equation*}
    \begin{aligned}
        \phi(P_n) &= \sum_{j=1}^n (\phi(P_j) - \phi(P_{j-1})) + \phi(P_0) \\
        &= \sum_{j=1}^n j e^{\frac{2\pi(j-1)}{3}i} \\
        &= e^{-2i\pi/3}\sum_{j=1}^n j\left(e^{2\pi i/3}\right)^j \\
        &= \frac{(ne^{2\pi i/3}-n-1)e^{2\pi i(n+1)/3}+e^{2\pi i/3}}{(1-e^{2\pi i/3})^2 e^{2\pi i/3}} \\
        &= \frac{(ne^{2\pi i/3}-n-1)e^{2\pi in/3} + 1}{(1-e^{2\pi i/3})^2} \\
        &= \frac{(n\left(\frac{-1+i\sqrt{3}}{2}\right)-n-1)e^{2\pi in/3}+1 }{(1-\left(\frac{-1+i\sqrt{3}}{2}\right))^2 } \\
        &= \frac{e^{2\pi in/3} \left(\frac{-3n + ni\sqrt{3}}{2} -1\right) + 1}{\left(\frac{3 - i\sqrt{3}}{2} \right)^2} \\
        &= \frac{2}{3} \cdot \frac{e^{2\pi in/3} \left(\frac{-3n + ni\sqrt{3}}{2} -1\right) + 1}{1 - i\sqrt{3}} \\ 
        &= \frac{2}{3} \cdot \frac{\left(e^{2\pi in/3} \left(\frac{-3n + ni\sqrt{3}}{2} -1\right) + 1 \right) (1+i\sqrt{3})}{(1 - i\sqrt{3})(1+i\sqrt{3})} \\
        &= \frac{1}{6} \cdot  \left(e^{2\pi in/3} \left(\frac{-3n + ni\sqrt{3}}{2} -1\right) + 1 \right) (1+i\sqrt{3}) \\
        &= \frac{1}{6} \left(  e^{2\pi in/3} \left(\frac{-3n + ni\sqrt{3}}{2} -1\right) (1+i\sqrt{3}) + 1 + i\sqrt{3} \right) \\
        &= \frac{1}{6} \cdot (e^{2\pi in/3} (-1 - i\sqrt{3} - 3 n - i\sqrt{3} n) + 1 + i\sqrt{3})
    \end{aligned}
\end{equation*}
When $3|n$, $e^{2\pi in/3} = 1$ and hence, 
\[
    \phi(P_n) = \frac{1}{6} \cdot (-1 - i\sqrt{3} - 3 n - i\sqrt{3} n + 1 + i\sqrt{3}) = -\frac{n}{2} - \frac{n\sqrt{3}}{6}i
\]
which means 
\[
    P_n = \left(-\frac{n}{2},- \frac{n\sqrt{3}}{6}\right)
\]
If remainder of n divides 3 is 1, then $e^{2\pi in/3} = -1/2 + \sqrt{3}i/2$ and hence
\begin{equation*}
    \begin{aligned}
        \phi(P_n) &= \frac{1}{6} \cdot \left(\left(-\frac{1}{2} + \frac{\sqrt{3}i}{2} \right)(-1 - i\sqrt{3} - 3 n - i\sqrt{3} n )+ 1 + i\sqrt{3}\right) \\
        &= -i \frac{\sqrt{3} (-1 + n)}{6} +  \frac{1 + n}{2}
    \end{aligned}
\end{equation*}
which means 
\[
    P_n = \left(   \frac{1 + n}{2} , - \frac{\sqrt{3} (-1 + n)}{6}\right)
\]
If remainder of n divides 3 is 2, then $e^{2\pi in/3} = -1/2 - \sqrt{3}i/2$ and hence
\begin{equation*}
    \begin{aligned}
        \phi(P_n) &= \frac{1}{6} \cdot \left(\left(-\frac{1}{2} - \frac{\sqrt{3}i}{2} \right)(-1 - i\sqrt{3} - 3 n - i\sqrt{3} n )+ 1 + i\sqrt{3}\right) \\
        &= \frac{\sqrt{3}}{3}(n+1)i 
    \end{aligned}
\end{equation*}
which means 
\[
    P_n = \left(0, \frac{\sqrt{3}}{3}(n+1) \right)
\]
\newpage
\section*{A1.4}
Let $z=x+iy$, then define \\
$a = \cfrac{if(1) + f(i)}{2i}$ \\
$b = \cfrac{if(1) - f(i)}{2i}$ \\
We have that 
\begin{equation*}
    \begin{aligned}
        az + b\overline{z} &= a(x+iy) + b(x-iy) \\
        &= x(a+b) + iy(a-b) \\
        &= x \cdot f(1) + iy \cdot \frac{f(i)}{i} \\
        &= f(x) + f(iy) \\
        &= f(x+iy) \\
        &= f(z)
    \end{aligned}
\end{equation*}
\newpage
\section*{A1.5}
Define a linear transformation 
\[
    g: \mathbb{C} \to \mathbb{C}, \indent z \to \frac{i\overline{f(1)} - \overline{f(i)}}{\overline{f(1)}f(i) - f(1)\overline{f(i)}} z + \frac{if(1) - f(i)}{\overline{f(i)}f(1) - f(i)\overline{f(1)}} \overline{z}
\]
so that 
\[
    g(f(1)) = 1
\]
\[
    g(f(i)) = i    
\]
Notice that the denominator $\overline{f(1)}f(i) - f(1)\overline{f(i)} \ne 0$ as it means that Im$(f(1)\overline{f(i)}) = 0$, which means both Im($f(1)$) and Im($f(i)) = 0$. Hence, $|f(1)-f(i)| = 2 \ne \sqrt{2} = |1-i|$.\\
\begin{equation*}
    \begin{aligned}
        g \circ f(z) &= \cfrac{i\overline{f(1)} - \overline{f(i)}}{\overline{f(1)}f(i) - f(1)\overline{f(i)}}f(z) + \cfrac{if(1) - f(i)}{\overline{f(i)}f(1) - f(i)\overline{f(1)}} \overline{f(z)} \\
    \end{aligned}
\end{equation*}
\[
    |g(f(z_1 -z_2))|=  |g(f(z_1) -f(z_2))| = |g(f(z_1)) -g(f(z_2))|
\]
We know that $|f(z_1)-f(z_2)| = |z_1-z_2| = |f(z_1-z_2) - f(0)| = |f(z_1-z_2)|$ and $g$ is a linear transformation, therefore
\[
    |g(f(z_1-z_2))| = |g(f(z_1) - f(z_2))| = |g(f(z_1)) - g(f(z_2))|
\]
Which means that $h:=g\circ f$ is also an isometry. Since $g \circ f$ fixes $0,1,i$, we have the following
\[
    |h(z)| = |z|, \indent |h(z) - 1| = |z-1|, \indent |h(z)-i| = |z-i|
\]
Square the three equations, we have
\[
    h(z) \overline{h(z)} = z \overline{z}, \indent (h(z)-1)(\overline{h(z)}-1) =(z-1)(\overline{z}-1), \indent (h(z)-i)(\overline{h(z)}+i) =(z-i)(\overline{z}+i)
\]
Expanding the second and third equations yield, 
\[
    h(z)\overline{h(z)} - h(z) - \overline{h(z)} = z\overline{z} -z - \overline{z}, \indent h(z)\overline{h(z)} + ih(z) - i\overline{h(z)} = z\overline{z} +iz - i\overline{z} 
\]
Substitute in the 2 equations in the first equation $h(z)\overline{h(z)} = z \overline{z}$, we have
\[
    h(z)+\overline{h(z)} = z + \overline{z}, \indent h(z) - \overline{h(z)} = z - \overline{z}
\]
Hence, $h(z) = z$, which means $g\circ f$ is the identity map. \\
$f$ is injective as $|f(z_1) - f(z_2)| = 0 \implies |z_1 - z_2|=0$. \\
Consider $B_r$ where $r$ is the radius. Suppose $f$ is not bijective. Then we can find some $z \in B_r \backslash f(B_r)$. Then the sequence $z_0 = z,z_1 = f(z_0),z_2 = f(z_1), \hdots$ has a convergent subsequence. However, we can choose $0< \epsilon < \inf \{|z-z^*| : z^* \in \text{Img}(f) \}$. 
so that $|f(z_n)-f(z_m)| = |f(z_0) - f(z_{m-n})| \ge \epsilon$ for $m>n$, which means that there is no convergent subsequence and therefore a contradiction. 
Hence, $f$ is bijective, and thus $f$, the inverse of $g$ is also linear.
\end{document}
