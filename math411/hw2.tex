\documentclass[11pt]{report}
    \title{\textbf{Math 217 Homework I}}
    \author{Khac Nguyen Nguyen}
    \date{}

    \addtolength{\topmargin}{-3cm}
    \addtolength{\textheight}{3cm}

\usepackage{amsmath}
\usepackage{mathtools}
\usepackage{amsthm}
\usepackage{amssymb}
\usepackage{pgfplots}
\usepackage{xfrac}
\usepackage{hyperref}

\usepgfplotslibrary{polar}
\usepgflibrary{shapes.geometric}
\usetikzlibrary{calc}
\pgfplotsset{compat = newest}
\pgfplotsset{my style/.append style = {axis x line = middle, axis y line = middle, xlabel={$x$}, ylabel={$y$}, axis equal}}
\begin{document}
\section*{A2.1}
\subsection*{a. $\Rightarrow$}
If $f$ is anti-holomorphic at $z_0 = x_0 + iy_0$ then consider the function $g = \overline{f}$, we know that
\[
    \lim_{z \to z_0} \frac{f(z)-f(z_0)}{\overline{z-z_0}}
\]
exists. Hence,
\[
    \lim_{z \to z_0} \frac{g(z) - g(z_0)}{z-z_0}
\]
also exists. Therefore, $g$ is complex differentiable at $z_0$. And hence $f$ is complex differentiable at $z_0$. Therefore,
\[
    \frac{\partial u}{\partial x} = -\frac{\partial v}{\partial y}  \indent \frac{\partial u}{\partial y} = \frac{\partial v}{\partial x}
\]
Hence at $z_0$,
\[
    \frac{\partial f}{\partial x} = \frac{\partial u}{\partial x} + i\frac{\partial v}{\partial x}
    = - \frac{\partial v}{\partial y} + i \frac{\partial u}{\partial y}
    = i \left( \frac{\partial u}{\partial y} +i \frac{\partial v}{\partial y} \right)
    = \frac{\partial f}{\partial y}
\]
Define
\[
    T: \mathbb{C} \to \mathbb{C}, \indent z \to \overline{ z g'(z_0)}
\]
\[
    \frac{|f(z) - f(z_0) - T(z-z_0)|}{|z-z_0|} = \left| \frac{f(z) - f(z_0)}{\overline{z-z_0}} - \overline{g'(z_0)}\right| \to 0
\]
as $z\to z_0$. Therefore, $f$ is totally differentiable.
\subsection*{b. $\Leftarrow$}
We know that $g = \overline{f}$ is totally differentiable at $z_0$ as $f$ is totally differentiable at $z_0$ and since
\[
    \frac{\partial f}{\partial x} = i \frac{\partial f}{\partial y} \implies \frac{\partial g}{\partial x} = -i \frac{\partial g}{\partial y}
\]
We know that $g$ is complex differentiable at $z_0$ and hence
\[
    \lim_{z \to z_0} \frac{g(z) - g(z_0)}{z-z_0}
\]
exists which means that
\[
    \lim_{z \to z_0} \frac{f(z)-f(z_0)}{\overline{z-z_0}}
\]
exists. Thus, $f$ is anti-holomorphic.


\newpage
\section*{A2.2}
\subsection*{a.}
\begin{equation*}
    \begin{aligned}
        \lim_{z \to z_0} \frac{f(z) - f(z_0)}{z-z_0}
        &= \lim_{z \to z_0} \frac{a(z^2 - z_0^2) + b(z\overline{z} - z_0 \overline{z_0}) + c(\overline{z}^2-\overline{z_0}^2)}{z-z_0}  \\
        &= 2az_0 + \lim_{z \to z_0} \frac{b(z\overline{z} - z_0 \overline{z_0}) + c(\overline{z}^2-\overline{z_0}^2)}{z-z_0} \\
        &= 2az_0 + \lim_{z \to z_0} \frac{ bz\overline{z} -b\overline{z}z_0 + b\overline{z}z_0 -bz_0\overline{z_0} + c (\overline{z}^2 - \overline{z_0}^2)}{z-z_0} \\
        &= 2az_0 +b\overline{z} + \lim_{z \to z_0} \frac{bz_0(\overline{z} - \overline{z_0}) + c(\overline{z}-\overline{z_0})(\overline{z} + \overline{z_0})}{z-z_0} \\
        &= 2az_0 +b\overline{z} + \lim_{z \to z_0} \frac{(\overline{z} - \overline{z_0})(bz_0 + c(\overline{z} + \overline{z_0}))}{z-z_0} \\
        &= 2az_0 +b\overline{z} + \lim_{z \to z_0} \frac{(\overline{z} - \overline{z_0})(bz_0 + c(\overline{z} + \overline{z_0}))}{z-z_0} \\
    \end{aligned}
\end{equation*}
Since the map $z \to \overline{z}$ is not complex differentiable, $\lim_{z \to z_0} \frac{f(z) - f(z_0)}{z-z_0}$ exists iff
$\lim_{z \to z_0} \frac{(\overline{z} - \overline{z_0})(bz_0 + c(\overline{z} + \overline{z_0}))}{z-z_0}$ iff $bz_0 + 2c(\overline{z_0}) = 0$ \\
\subsection*{b.}
Let $z_0 = x_0 + iy_0$, then $f$ is differentiable at $z_0$ if
\[
    bx_0 + 2cx_0 = 0 \indent by_0 - 2cy_0 = 0
\]
If $b = -2c, bx_0 + 2cx_0 = 0$ is true for all $x_0$. However, $by_0 - 2cy_0 = 2by_0 = 0 \implies y_0=0$. \\
If $b = 2c, by_0 - 2cy_0 = 0$ is true for all $y_0$. However, $bx_0 + 2cx_0 = 2bx_0 =0 \implies x_0 = 0$. \\
In other cases, $x_0 = y_0 = 0$ must be satisfied so that $f$ is differentiable at $z_0$.
\newpage
\section*{A2.3}
$g(z) = zf(z)$ is analytic in $D$, hence for all $z \in D$
\[
    \frac{\partial g}{\partial x} = -i \frac{\partial g}{\partial y}
\]
We also have
\[
    \frac{\partial g}{\partial x} = \frac{\partial z}{\partial x}f(z) + \frac{\partial f}{\partial x} z = f(z) + \frac{\partial f}{\partial x} z
\]
\[
    -i\frac{\partial g}{\partial y} = -i \left( \frac{\partial z}{\partial y}f(z) + \frac{\partial f}{\partial y}z \right) = -i \left(if(z) + \frac{\partial f}{\partial y}z\right) = f(z) - i\frac{\partial f}{\partial y}z
\]
Therefore,
\[
    \frac{\partial f}{\partial x} z = -i\frac{\partial f}{\partial y}z
\]
However, $\overline{f(z)}$ is analytic in $D$, hence for all $z \in D$
\[
    \frac{\partial f}{\partial x} = \frac{\partial f}{\partial y} = 0
\]
Thus $f$ is constant on $D$.
\newpage
\section*{A2.4}
Since $f$ and $g$ are anti-holomorphic at $z_0$ and $g(z_0)$, they are totally differentiable at $z_0$ and $g(z_0)$,
hence $f \circ g$ is also totally differentiable. We have
\begin{equation*}
    \begin{aligned}
        J_{f\circ g}(x_0) &= J_f(g(z_0))J_g(x_0) \\
        &=
        \begin{bmatrix}
            \frac{\partial u_f}{\partial x}(g(z_0)) & \frac{\partial u_f}{\partial y}(g(z_0)) \\
            \frac{\partial v_f}{\partial x}(g(z_0)) & \frac{\partial v_f}{\partial y}(g(z_0))
        \end{bmatrix}
        \begin{bmatrix}
            \frac{\partial u_g}{\partial x}(z_0) & \frac{\partial u_g}{\partial y}(z_0) \\
            \frac{\partial v_g}{\partial x}(z_0) & \frac{\partial v_g}{\partial y}(z_0)
        \end{bmatrix} \\
    \end{aligned}
\end{equation*}
\[
    \frac{\partial u_{f \circ g}}{\partial x} = \frac{\partial u_f}{\partial x}(g(z_0)) \frac{\partial u_g}{\partial x}(z_0) + \frac{\partial u_f}{\partial y}(g(z_0)) \frac{\partial v_g}{\partial x}(z_0) 
\]
\[
    \frac{\partial u_{f \circ g}}{\partial y} = \frac{\partial u_f}{\partial x}(g(z_0)) \frac{\partial u_g}{\partial y}(z_0) + \frac{\partial u_f}{\partial y}(g(z_0)) \frac{\partial v_g}{\partial y}(z_0) 
\]
\[
    \frac{\partial v_{f \circ g}}{\partial x} = \frac{\partial v_f}{\partial x}(g(z_0)) \frac{\partial u_g}{\partial x}(z_0) + \frac{\partial v_f}{\partial y}(g(z_0)) \frac{\partial v_g}{\partial x}(z_0) 
\]
\[
    \frac{\partial v_{f \circ g}}{\partial y} = \frac{\partial v_f}{\partial x}(g(z_0)) \frac{\partial u_g}{\partial y}(z_0) + \frac{\partial v_f}{\partial y}(g(z_0)) \frac{\partial v_g}{\partial y}(z_0) 
\]
Since $f$ and $g$ are anti-holomorphic, we have that 
\[
    \frac{\partial u}{\partial x} = -\frac{\partial v}{\partial y}, \indent \frac{\partial u}{\partial y} = \frac{\partial v}{\partial x}
\]
for both function at $z_0$ and $g(z_0)$ respectively. Hence, 
\[
    \frac{\partial u_{f \circ g}}{\partial x} = \frac{\partial v_{f \circ g}}{\partial y}, \indent \frac{\partial u_{f \circ g}}{\partial y} = -\frac{\partial v_{f \circ g}}{\partial x}    
\]
Therefore, $f \circ g$ is complex differentiable.
\newpage
\section*{A2.5}
We have that 
\[
    \frac{\partial g}{\partial x}(z) = \frac{\partial }{\partial x} \int_0^1 f(t,z) dt = \int_0^1 \frac{df}{dx}(t,z) dt
\]
And since $\frac{\partial f}{\partial x}$ exists and are continuous on $[0,1]\times D$ for $z = x+iy$, 
$\frac{\partial g}{\partial x}$ and similarly $\frac{\partial g}{\partial y}$ exists and are continuous on $[0,1] \times D$. \\
Therefore, $g$ is totally differentiable.
Notice that 
\[
    \frac{\partial g}{\partial x}(z) = \int_0^1 \frac{df}{dx}(t,z) dt = \int_0^1  -i\frac{df}{dy}(t,z) dt  = -i \int_0^1 \frac{df}{dy}(t,z) dt = -i\frac{\partial g}{\partial y}
\]
which confirms that $g$ is indeed complex differentiable.
\end{document}