\documentclass[11pt]{article}
    \title{\textbf{Math 217 Homework I}}
    \author{Khac Nguyen Nguyen}
    \date{}

    \addtolength{\topmargin}{-3cm}
    \addtolength{\textheight}{3cm}

\usepackage{amsmath}
\usepackage{mathtools}
\usepackage{amsthm}
\usepackage{amssymb}
\usepackage{pgfplots}
\usepackage{xfrac}
\usepackage{hyperref}
\usepackage{graphicx}
\long\def\comment#1{}

\usepgfplotslibrary{polar}
\usepgflibrary{shapes.geometric}
\usetikzlibrary{calc}
\pgfplotsset{compat = newest}
\pgfplotsset{my style/.append style = {axis x line = middle, axis y line = middle, xlabel={$x$}, ylabel={$y$}, axis equal}}
\begin{document}
\section*{1.}
\subsection*{a.}
We have that 
\[
    \sqrt[n]{5^n} = 5
\]
and 
\[
    \lim \frac{(n+1)^5}{n^5} = 1
\]
Hence, 
\[
    \sum_{n=0}^\infty 5^n z^n    
\]
converges with $R = \frac{1}{5}$ and 
\[
    \sum_{n=0}^\infty n^5 z^n    
\]
converges with $R = 1$. 
Hence, the radius of convergence of the original power series is $\frac{1}{5}$.
\subsection*{b.}
We have that 
\[
    \lim \left(\left( 1- \frac{1}{n} \right)^{n^2} \right)^{1/n} = \lim \left(1- \frac{1}{n} \right)^n = \lim_{x \to 0}(1+x)^{-1/x} = \frac{1}{e}
\]
Hence, the radius of convergence is $e$.
\subsection*{c.}
Let $a_n = 0$ if n is odd and $a_n = 2^{n/2}$ if n is even. Then 
\[
    \sum_{n=0}^\infty 2^n z^{2n} = \sum_{n=0}^\infty a_n z^n     
\]
Then 
\[
    R = \frac{1}{\limsup |a_n|^{1/n}} = \left(\frac{1}{\limsup |2^n|^{1/n}} \right)^{1/2} = \frac{1}{\sqrt{2}}    
\]
\subsection*{d.}
Let $a_n = 0$ if $n$ is not a power of 2, $a_n = 2 \log_2(n)$ otherwise. Then 
\[
    \sum_{n=0}^\infty 2n z^{2^n} = \sum_{n=0}^\infty a_n z^n
\]
Then 
\[
    R = \frac{1}{\limsup |a_n|^{1/n}} = \frac{1}{\limsup|2\log_2(n)|^{1/n}} =  \frac{1}{\limsup|2^{\frac{1}{n}}2^{\frac{1}{n} \log_2(\log_2(n))}| } = 1   
\]
\section*{2.}
Let $f(z) = \sum_{n=0}^\infty a_n z^n$, then 
\[
    f'(z) = \sum_{n=2}^\infty na_nz^{n-1} + a_1
\]
As $a_1 \ne 0, f'(0) = a_1 \ne 0$, hence we can find $r_1> 0$ such that $f'(z) \ne 0$ in $\{ |z|< r_1\}$.
Thus from the inverse function theorem, $f$ is injective in $\{ |z|< r_1\}$. Hence, choose $r = \min(R, r_1)$, we have that $f$ is 
bijective on $\{ |z|< r\}$.
\comment{We know that $f(z)$ is surjective in $B_r(0)$ if $r < R$. \\
If  $f(z_1) = f(z_2)$, then 
\[
    \sum_{n=0}^\infty a_n (z_1^n - z_2^n) = (z_1 - z_2) \left(a_1 + \sum_{n=2}^\infty a_n (z_1^{n-1} + \hdots + z_2^{n-1})\right) = 0     
\]
Hence, either $z_1 = z_2$ or $\sum_{n=1}^\infty a_n (z_1^{n-1} + \hdots + z_2^{n-1}) = 0$
Therefore, we want to prove that there is $r < R$ such that $|z_1| < r, |z_2|< r$ satisfies 
\[
    a_1 + \sum_{n=2}^\infty a_n (z_1^{n-1} + \hdots + z_2^{n-1}) \ne 0
\]
\begin{equation*}
    \begin{aligned}
        \left| \sum_{n=2}^\infty a_n(z_1^{n-1} + \hdots + z_2^{n-1}) \right| 
        &\le \sum_{n=2}^\infty |a_n||z_1^{n-1} + \hdots + z_2^{n-1}| \\
        &< \sum_{n=2}^\infty n|a_n|r^{n-1} \\
    \end{aligned}
\end{equation*}}
\newpage
\section*{3.}
\begin{equation*}
    \begin{aligned}
        \cos(\overline{z})&= \sum_{n = 0}^\infty (-1)^n \frac{\overline{z}^{2n}}{2n!} \\
        &= \overline{\sum_{n = 0}^\infty (-1)^n \frac{z^{2n}}{2n!}} \\
        &= \overline{\cos(z)}
    \end{aligned}
\end{equation*}
Hence, 
\begin{equation*}
    \begin{aligned}
        &|\cos(z)|^2 \\
        =& (\cos(x+iy))(\cos(x-iy)) \\
        =& (\cos(x) \cos(iy) - \sin(x)\sin(iy)) (\cos(x) \cos(iy) + \sin(x)\sin(iy)) \\
        =& (\cos(x))^2(\cosh(y))^2 + (\sin(x))^2 (\sinh(y))^2 \\
        =& (\cos(x))^2(1 + \sinh(y))^2 + (\sin(x))^2 (\sinh(y))^2 \\
        =& (\cos(x))^2 + (\sinh(y))^2
    \end{aligned}
\end{equation*}
because of $\cos(iy) = \cosh(y)$ and $\sin(iy) = i \sinh(y)$. Similarly, 
\begin{equation*}
    \begin{aligned}
        \sin(\overline{z})&= \sum_{n = 0}^\infty (-1)^n \frac{\overline{z}^{2n+1}}{(2n+1)!} \\
        &= \overline{\sum_{n = 0}^\infty (-1)^n \frac{z^{2n+1}}{(2n+1)!}} \\
        &= \overline{\sin(z)}
    \end{aligned}
\end{equation*}
Hence, 
\begin{equation*}
    \begin{aligned}
        &|\sin(z)|^2 \\
        =& (\sin(x+iy))(\sin(x-iy)) \\
        =& (\sin(x) \cos(iy) + \cos(x)\sin(iy)) (\sin(x) \cos(iy) - \cos(x)\sin(iy)) \\
        =& (\sin(x))^2(\cosh(y))^2 + (\cos(x))^2 (\sinh(y))^2 \\
        =& (\sin(x))^2(1 + \sinh(y))^2 + (\cos(x))^2 (\sinh(y))^2 \\
        =& (\sin(x))^2 + (\sinh(y))^2
    \end{aligned}
\end{equation*}


\newpage
\section*{4.}
\subsection*{a.}
We have that $|a_n| \le p_n, |b_n| \le q_n$. Therefore, 
\[
    |a_n \pm b_n| \le |a_n| + |b_n| \le p_n + q_n      
\]
and hence 
\[
    \sum_{n=0}^\infty a_n z^n \pm \sum_{n=0}^\infty b_n z^n \prec \sum_{n=0}^\infty p_n z^n + \sum_{n=0}^\infty q_n z^n     
\]
We have that 
\[
    \left|\sum_{l=0}^k a_l b_{k-l} \right| \le \sum_{l=0}^k |a_l| |b_{k-l}| \le \sum_{l=0}^k p_l q_{k-l} 
\]
Hence, 
\[
    \left(\sum_{n=0}^\infty a_n z^n \right) \left( \sum_{n=0}^\infty b_n z^n \right) \prec \left(\sum_{n=0}^\infty p_n z^n\right) + \left(\sum_{n=0}^\infty q_n z^n\right)     
\]
\subsection*{b.}
\[
    \frac{M}{r - z} = \frac{M/r}{1 - z/r} = \sum_{n=0}^\infty \frac{M}{r} \left(\frac{z}{r}\right)^n = \sum_{n=0}^\infty \frac{M}{r^{n+1}} z^n
\]
We know that for every $r \in [0,R)$, 
\[
    \limsup \sqrt[n]{|a_n|} = \frac{1}{R} < \frac{1}{r} 
\]
Therefore, there exists $n_0 \in \mathbb{N}$ such that $|a_n| < \left( \frac{1}{r} \right)^n$,
and hence we can choose $M \ge r$ such that $|a_n| < \frac{M}{r^{n+1}}$. Thus, 
\[
    \sum_{n=0}^\infty a_n z^n \prec \frac{M}{r-z}    
\]
\newpage
\section*{5.}
From A3.4, we know that there exists $M>0$ such that 
\[
    \sum_{m=1}^\infty a_m z^m \prec \frac{M}{r - z}    
\]
Hence, 
\[
    \left|\frac{1}{f(z)} \right| = \left| \sum_{n=0}^\infty \frac{(-1)^n}{a_0^{n+1}} \left( \sum_{m=1}^\infty a_m z^m \right)^n \right| < \sum_{n=0}^\infty \frac{1}{\left| a_0^{n+1} \right|} \left| \frac{M}{r-z} \right|^n
\]
which means that
\[
    \left| \frac{1}{f((r+M)/z)} \right| < \sum_{n=0}^\infty \frac{1}{\left| a_0^{n+1} \right|} |z|^n
\]
However, 
\[
    \left| \frac{\cfrac{1}{a_0^{n+1}}}{\cfrac{1}{a_0^{n+2}}} \right| = |a_0|    
\]
which means $1/f((r+M)/z)$ has a positive radius of convergence, thus also $1/f(z)$
\end{document}