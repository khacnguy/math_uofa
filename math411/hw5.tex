\documentclass[11pt]{article}
    \title{\textbf{Math 217 Homework I}}
    \author{Khac Nguyen Nguyen}
    \date{}

    \addtolength{\topmargin}{-3cm}
    \addtolength{\textheight}{3cm}

\usepackage{amsmath}
\usepackage{mathtools}
\usepackage{amsthm}
\usepackage{amssymb}
\usepackage{pgfplots}
\usepackage{xfrac}
\usepackage{hyperref}
\usepackage{graphicx}
\long\def\comment#1{}

\usepgfplotslibrary{polar}
\usepgflibrary{shapes.geometric}
\usetikzlibrary{calc}
\pgfplotsset{compat = newest}
\pgfplotsset{my style/.append style = {axis x line = middle, axis y line = middle, xlabel={$x$}, ylabel={$y$}, axis equal}}
\begin{document}
\section*{1.}
\subsection*{a.}
Consider the function $f(z) = \sin(z)$
\begin{equation*}
    \begin{aligned}
        \int_{|z|=2} \frac{\sin(z)}{z^2+1} dz 
        &= \int_{|z|=2} \frac{1}{2i}\left(\frac{\sin(z)}{z-i} - \frac{\sin(z)}{z+i}\right) dz \\
        &= \frac{1}{2i} \left(\int_{|z|=2}\frac{\sin(z)}{z-i}dz - \int_{|z|=2} \frac{\sin(z)}{z+i}dz\right) \\
        &= \frac{1}{2i} \left(\sin(i) - \sin(-i)\right) \\
        &= \frac{1}{i} \sin(i)
    \end{aligned}
\end{equation*} 
\subsection*{b.}
Consider the constant function $f(z) = 1$, then
\begin{equation*}
    \begin{aligned}
        &\int_\gamma \frac{z}{z^3-1} dz \\
        =& \int_\gamma \left( \frac{\frac{1}{3}}{z-1} + \frac{-\frac{1}{6} + \frac{i}{2\sqrt{3}}}{z - \left(-\frac{1}{2}-\frac{\sqrt{3}}{2}i\right)} + \frac{-\frac{1}{6} - \frac{i}{2\sqrt{3}}}{z - \left(-\frac{1}{2} +\frac{\sqrt{3}}{2}i\right)} \right) dz \\
        =& 2\pi i \left(\frac{1}{3} - \frac{1}{6} + \frac{i}{2\sqrt{3}} - \frac{1}{6} - \frac{i}{2\sqrt{3}} \right) \\
        =& 0
    \end{aligned}
\end{equation*}
\newpage
\section*{2.}
\[
    \cot(z) = \cos(z) / \sin(z)    
\]
and since $\sin(z) \ne 0 \iff z = n\pi$ for some $n \in \mathbb{Z}$. 
$\cot(z)$ has a complex antiderivative on $D$ as $D$ is starshaped with center $\pi/2$.
We know that 
\[
    f(\pi / 2) = \ln|\sin(\pi/2) | + C = 0 \implies C = 0    
\]
Hence, 
\[
    f(i) = \ln|sin(i)|    
\]
\newpage
\section*{3.}
We have that 
\begin{equation*}
    \begin{aligned}
        &\frac{1}{2\pi i} \int_{|z|=R} \frac{z^mf'(z)}{f(z)} dz \\
        =& \frac{1}{2\pi i} \int_{|z| = R} \sum_{i=1}^n \frac{z^m}{z-r_1} dz \\
        =&  \sum_{i=1}^n \frac{1}{2\pi i} \int_{|z| = R} \frac{z^m}{z-r_1} dz \\
        =& \sum_{i=1}^n z_i^m = b_m
    \end{aligned}
\end{equation*}
and 
\begin{equation*}
    \begin{aligned}
        &b_{m+n} + a_1 b_{m+n-1} + \hdots + a_n b_m \\
        =& \sum_{i=1}^n r_i^{m+n} + a_1 \sum_{i=1}^n r_i^{m+n-1} + \hdots + a_n \sum_{i=1}^n r_i^m \\
        =& \sum_{i=1}^n r_i^{m+n} + a_1 r_i^{m+n-1} + \hdots + a_n r_i^m \\
        =& \sum_{i=1}^n r_i^m (r_i^n + a_1 r_i^{n-1} + \hdots + a_n)\\ 
        =& 0
    \end{aligned}
\end{equation*}
\newpage
\section*{4.}
Let 
\[
    h(z) = f(\exp(z)) = g(\exp(iz))    
\]
It is obvious that 
\[
    f(\exp(z+i2\pi)) = f(\exp(z)) \text{ and } g(\exp(iz)) =    g(\exp(i(z+2\pi)))
\]
which means that 
\[
    h(z) = h(z+2\pi) = h(z+2\pi i)    
\]
Consider $D = \{ x+iy: 0\le x\le 2\pi, 0 \le y \le 2\pi\}$. Then there exists $M$ such that 
\[
    |h(z)| \le M \text{ for every } z \in D  
\]
Then for every $z \in \mathbb{C}$, there is a $z_0 \in D$ such that $h(z) = h(z_0)$ and hence $h$ is bounded. 
Therefore, $h$ is constant.
\newpage
\section*{5.}
Let $f(z) = u(z) + iv(z)$, then $f^2 = u^2-v^2+2uvi$. We have that 
\[
    \frac{\partial }{\partial x}(u^2-v^2) = \frac{\partial }{\partial y}(2uv)    
\]
and 
\[
    \frac{\partial }{\partial y}(u^2-v^2) = -\frac{\partial }{\partial x}(2uv)
\]
Therefore, we can get the system of equations
\[
\begin{cases}
    u \left(\frac{\partial u}{\partial x} -  \frac{\partial v}{\partial y}\right) - v \left(\frac{\partial u}{\partial y} +  \frac{\partial v}{\partial x}\right) = 0 \\
    u \left(\frac{\partial u}{\partial y} +  \frac{\partial v}{\partial x}\right) + v \left(\frac{\partial u}{\partial x} -  \frac{\partial v}{\partial y}\right) = 0 \\
\end{cases}
\]
Hence if $u \ne 0$ or $v\ne 0$, $\frac{\partial u}{\partial x} = \frac{\partial v}{\partial y}$ and $\frac{\partial u}{\partial y} = - \frac{\partial v}{\partial x}$, which means $f$ is analytic.
If $u = v =0$, then if $f^{(n)}(z) = 0$ for all $n$, $f \equiv 0$ in any ball around it.
In the other case, $f'(z)$ does not vanish. Hence, we can find $r$ such that $f(z) \ne 0$ for $0<|z-z_0| < r$, which means that the Morera condition holds (Lemma 5.1).
\end{document}