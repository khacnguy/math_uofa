\documentclass[11pt]{article}
    \title{\textbf{Math 217 Homework I}}
    \author{Khac Nguyen Nguyen}
    \date{}
    
    \addtolength{\topmargin}{-3cm}
    \addtolength{\textheight}{3cm}
    
\usepackage{amsmath}
\usepackage{mathtools}
\usepackage{amsthm}
\usepackage{amssymb}
\usepackage{pgfplots}
\usepackage{xfrac}
\usepgfplotslibrary{polar}
\usepgflibrary{shapes.geometric}
\usetikzlibrary{calc}
\pgfplotsset{compat = newest}
\pgfplotsset{my style/.append style = {axis x line = middle, axis y line = middle, xlabel={$x$}, ylabel={$y$}, axis equal}}
\begin{document}
\section*{1.}
\begin{equation*}
    \begin{aligned}
        \int_{-\infty}^\infty (x^2-x)1_{[0,a]}(x)dx 
        &= \int_0^a (x^2-x)dx \\
        &= \left.\left(\frac{x^3}{3} - \frac{x^2}{2}\right)\right|_0^a \\
        &= \frac{a^3}{3} - \frac{a^2}{2} = 0 \\
        \implies &a= \frac{3}{2}
    \end{aligned}
\end{equation*}
\pagebreak
\section*{2.}
\begin{equation*}
    \begin{aligned}
        \iint_R xy dx dy &= \int_0^1 \int_0^{x^2} xy dy dx \\
        &= \int_0^1 x \cdot \left.\frac{y^2}{2} \right|^{x^2}_{y=0} dx \\
        &= \int_0^1 \frac{x^5}{2} \\
        &= \left.\frac{x^6}{12} \right|_0^1 \\
        &= \frac{1}{12}
    \end{aligned}
\end{equation*}
From the Fubini's theorem, we know that
\[
    \iint_R xy dx dy &= \int_0^1 \int_0^{x^2} xy dy dx = \int_0^1 \int_0^{x^2} xy dx dy   
\]
To be clear, 
\begin{equation*}
    \begin{aligned}
        \iint_R xy dx dy &= \int_0^1 \int_{\sqrt{y}}^1 xy dx dy  \\
        &= \int_0^1 y \cdot \left.\frac{x^2}{2} \right|^{1}_{x=\sqrt{y}} dy \\
        &= \int_0^1 \left(\frac{y}{2}- \frac{y^2}{2}\right) dy \\
        &= \left.\left(\frac{y^2}{4} - \frac{y^3}{6} \right)\right|_0^1 \\
        &= \frac{1}{4}- \frac{1}{6} = \frac{1}{12} 
    \end{aligned}
\end{equation*}
\pagebreak
\section*{3.}
Since $0\le 0 < \lambda \le \lambda$, we have that
\[
    P([0,\lambda]) = e^\lambda - e^0 = e^\lambda - 1 = 1
\]
and hence
\[
    e^\lambda = 2 \implies \lambda = \ln2    
\]
\pagebreak
\section*{4.}
\begin{equation*}
    \begin{aligned}
        P((-2,-1) \cup (1,2]) &= F(-1) - F(-2) + F(2) - F(1) \\
        &= 0.202
    \end{aligned}
\end{equation*}
\end{document}