\documentclass[11pt]{article}
    \title{\textbf{Math 217 Homework I}}
    \author{Khac Nguyen Nguyen}
    \date{}
    
    \addtolength{\topmargin}{-3cm}
    \addtolength{\textheight}{3cm}
    
\usepackage{amsmath}
\usepackage{mathtools}
\usepackage{amsthm}
\usepackage{amssymb}
\usepackage{pgfplots}
\usepackage{xfrac}
\usepgfplotslibrary{polar}
\usepgflibrary{shapes.geometric}
\usetikzlibrary{calc}
\pgfplotsset{compat = newest}
\pgfplotsset{my style/.append style = {axis x line = middle, axis y line = middle, xlabel={$x$}, ylabel={$y$}, axis equal}}
\begin{document}
\section*{1.}
Let $T$ be the time expected for the next first call. Then, since we have that 
\[
    P(T>t) = P(\text{no call occurs before t seconds}) = e^{-10t/60} = e^{-t/6}   
\]
which means that $T$ is an exponential distribution with $\lambda = 1/6$. Hence, we
can calculate
\[
    E[T|T>10] = \frac{\int_{10}^\infty \frac{t}{6} e^{-t/6} dt}{\int_{10}^\infty \frac{1}{6} e^{-t/6} dt} = \frac{16e^{-5/3}}{e^{-5/3}} = 16    
\]
Hence, the expected time for the 90-th call to be processed is $16 \cdot 90 / 60 = 24$ minutes.
\pagebreak
\section*{2.}
Let $T_1$ be the time she has to wait to pass the entrance and $T_2$ be the time she passes the bar line. \\
For the first process: \\
We have that $T_{1_C}^SS$ is an exponential distribution with $\lambda = \lambda_S - \lambda_A = 1$. \\
For the second process: \\
We have that $r = \frac{\lambda_A}{\lambda_S} = 2.5 \in (2,3) = (k-1,k)$, hence
\[
    \pi(3) 
    = \pi(0) \cdot \frac{r^3}{3!} 
    = \left(\sum_{j=0}^1 \frac{2.5^j}{j!} + \frac{3 \cdot 2.5^{3-1}}{(3-1)!(3-2.5)} \right)^{-1} \cdot \frac{2.5^3}{3!} = \frac{125}{1068}
\]
\[
    B_K = -\pi(k) \left( \frac{k\lambda_S^2}{((k\lambda_S - \lambda_A - \lambda_S)(k \lambda_S - \lambda_A)}\right) = - \frac{125}{1068} \cdot (-12) = \frac{125}{89}
\]
\[
    f_{T_C^SS}(t) = -\frac{36}{89} \cdot 2e^{-2t} + \frac{125}{89} e^{-t}    
\]
Then $T = T_1 + T_2$. Applying convolution twice, we have that 
\begin{equation*}
    \begin{aligned}
        f_T(t) &= \frac{-36}{89} \int_0^t 2e^{-2\tau} e^{-(t-\tau)} d\tau + \frac{125}{89} \int_0^t e^{-\tau}e^{-(t-\tau)} d\tau \\
        &= \frac{-72}{89} (e^{-t}-e^{-2t}) + \frac{125}{89} te^{-t} \\
    \end{aligned}
\end{equation*}
\pagebreak
\section*{3.}
\begin{equation*}
    \begin{aligned}
        E[Q(\infty)^2] &= \sum_{i=0}^\infty i^2 \cdot \pi(i) \\
        &= \sum_{i=0}^\infty i^2 \cdot \frac{p_S - p_A}{p_S(1-p_S)} \left(\frac{p_A(1-p_S)}{p_S(1-p_A)}\right) \\
        &= \frac{0.1}{0.2 \cdot 0.8} \sum_{i=0}^\infty i^2 \left(\frac{0.1 \cdot 0.8}{0.2 \cdot 0.9} \right)^i \\
        &= \frac{5}{8} \sum_{i=0}^\infty i^2 \cdot \left(\frac{4}{9}\right)^i \\
        &= \frac{5}{8} \frac{\left(\frac{4}{9}\right)^2 + \frac{4}{9}}{\left(1-\frac{4}{9}\right)^3} \\
        &= \frac{117}{50} 
    \end{aligned}
\end{equation*}
\pagebreak
\section*{4.}
We have that 
\[
    \pi(i) = 
    \begin{pmatrix}
        n \\
        i    
    \end{pmatrix}
    p^i (1-p)^{n-i}
    = \frac{n!}{i! (n-i)!} p^i (1-p)^{n-i}
\]
We also have 
\[
    \pi(i) = \frac{\prod_{j=0}^{i-1} a_j}{\prod_{j=1}^{i} s_j} \pi(0) 
\]
Hence, 
\begin{equation*}
    \begin{aligned}
        \prod_{j=0}^{i-1} a_j &= \frac{n!}{\pi(0) i! (n-i)!} p^i (1-p)^{n-i} \cdot \prod_{j=1}^{i} j (1-p) \\
        &= \frac{n!}{\pi(0) i! (n-i)!} p^i (1-p)^{n-i} \cdot i! (1-p)^i \\
        &= \frac{n! p^i (1-p)^n}{\pi(0) (n-i)!}
    \end{aligned}
\end{equation*}
and thus,
\begin{equation*}
    \begin{aligned}
        a_i 
        &= \frac{\prod_{j=0}^{i} a_j}{\prod_{j=0}^{i-1} a_j} \\
        &= \cfrac{\cfrac{n! p^{i+1} (1-p)^n}{\pi(0) (n-(i+1))!}}{\cfrac{n! p^i (1-p)^n}{\pi(0) (n-i)!}} \\
        &= \frac{p (n-i)!}{(n-i-1)!} \\
        &= p(n-i)
    \end{aligned}
\end{equation*}
\end{document}