\documentclass[11pt]{article}
    \title{\textbf{Math 217 Homework I}}
    \author{Khac Nguyen Nguyen}
    \date{}
    
    \addtolength{\topmargin}{-3cm}
    \addtolength{\textheight}{3cm}
    
\usepackage{amsmath}
\usepackage{mathtools}
\usepackage{amsthm}
\usepackage{amssymb}
\usepackage{pgfplots}
\usepackage{xfrac}
\usepgfplotslibrary{polar}
\usepgflibrary{shapes.geometric}
\usetikzlibrary{calc}
\pgfplotsset{compat = newest}
\pgfplotsset{my style/.append style = {axis x line = middle, axis y line = middle, xlabel={$x$}, ylabel={$y$}, axis equal}}
\begin{document}
\section*{1.}
\subsection*{a.}
Since the number of customers arriving every minutes follows a poisson distribution with mean 2, 
the number of customers arriving every 5 minutes follows a poisson distribution with mean 10.
\[
    \frac{10^6 e^{-10}}{6!} \cdot \frac{10^8 e^{-10}}{8!} = 0.0071  
\]
\subsection*{b.}
Since the number of customers arriving every minutes follows a poisson distribution with mean 2, 
the number of customers arriving every 20 seconds follows a poisson distribution with mean 2/3.
\[
    C^{15}_6 \cdot C^{15}_8 \cdot \left(\frac{2}{3}\right)^{14} \left( 1 - \frac{2}{3} \right)^{16} = 0.00256
\]
\subsection*{c.}
\[
    C^{60}_6 \cdot C^{60}_8 \cdot \left(\frac{1}{6}\right)^{14} \left( 1 - \frac{1}{6} \right)^{106} = 0.0066
\]
\subsection*{d.}
We can see that the answers are close to each other, and since $5 < 20$,  the answer in part c is closer to 
the true answers, which is the answers in part a.
\pagebreak
\section*{2.} 
\[
    \lambda = \frac{p}{\Delta} = 3    
\]
\begin{equation*}
    \begin{aligned}
        P(X(1)X(2)X(3) = 4) &= P(X(1)=1, X(2)=1, X(3) = 4) + P(X(1) = 1, X(2) = 2, X(3) = 3) \\
        &= e^{-3\cdot 3} \frac{(3 \cdot 3)^4}{4!} 
        \begin{pmatrix}
              & 4 \\
            1 & 0 & 3
        \end{pmatrix}
        \left( \frac{1}{3} \right)^1 \cdot 
        \left( \frac{1}{3} \right)^0 \cdot
        \left( \frac{1}{3} \right)^3 \\
        &+ e^{-3\cdot 3} \frac{(3 \cdot 3)^2}{2!} 
        \begin{pmatrix}
              & 2 \\
            1 & 1 & 0
        \end{pmatrix}
        \left( \frac{1}{3} \right)^1 \cdot 
        \left( \frac{1}{3} \right)^1 \cdot
        \left( \frac{1}{3} \right)^0 \\
        &= 0.00277672
    \end{aligned}
\end{equation*}
\pagebreak
\section*{3.}
\begin{equation*}
    \begin{aligned}
        P(X=k) &=  \sum_{i=k}^\infty P(Y=i) \cdot \left(\frac{3}{4} \right)^k \cdot \left(\frac{1}{4} \right)^{i-k} \cdot 
        \begin{pmatrix}
            i \\
            k
        \end{pmatrix} \\
        &= \sum_{i=k}^\infty \frac{8^i e^{-8}}{i!} \cdot 3^k \cdot \left(\frac{1}{4}\right)^i \cdot \frac{i!}{k!(i-k)! } \\
        &= \frac{3^k}{k!}\sum_{i=k}^\infty \frac{2^i e^{-8}}{(i-k)!} \\
        &= \frac{3^k}{k!}\sum_{i=0}^\infty \frac{2^{i+k} e^{-8}}{i!} \\
        &= \frac{6^k e^{-6}}{k!}\sum_{i=0}^\infty \frac{2^i e^{-2}}{i!} \\
        &= \frac{6^k e^{-6}}{k!}
    \end{aligned}
\end{equation*}
\pagebreak
\section*{4.}
\subsection*{a.}
\[
    P(N(T_1 + 1) \ge 4) =  P(N(T_1 + 1) = 4 |N(T_1)=1) = 1 - \sum_{i=0}^2 \frac{5^i e^{-5}}{i!} = 0.875348
\]
\subsection*{b.}
\[
    \hat{\lambda} = \frac{1}{1/2} = 2    
\]
\[
    P(\hat{N}(T_1 + 1) \ge 4) =  P(\hat{N}(T_1 + 1) = 4 |\hat{N}(T_1)=1) = 1 - \sum_{i=0}^2 \frac{2^i e^{-2}}{i!} = 0.3233  
\]



\end{document}