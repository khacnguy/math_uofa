\documentclass[11pt]{article}
    \title{\textbf{Math 217 Homework I}}
    \author{Khac Nguyen Nguyen}
    \date{}
    
    \addtolength{\topmargin}{-3cm}
    \addtolength{\textheight}{3cm}
    
\usepackage{amsmath}
\usepackage{mathtools}
\usepackage{amsthm}
\usepackage{amssymb}
\usepackage{pgfplots}
\usepackage{xfrac}
\usepgfplotslibrary{polar}
\usepgflibrary{shapes.geometric}
\usetikzlibrary{calc}
\pgfplotsset{compat = newest}
\pgfplotsset{my style/.append style = {axis x line = middle, axis y line = middle, xlabel={$x$}, ylabel={$y$}, axis equal}}
\begin{document}
\section*{1.}
\begin{equation*}
    \begin{aligned}
        \int_0^\infty yf(y) dy 
        &= \int_0^\infty y^2e^{-y} \\
        &= \left.y^2e^{-y} \right|^\infty_0 + \int_0^\infty 2ye^y dy \\
        &= 0 - \left.ye^{-y}\right|^\infty_0 + 2\int_0^\infty e^{-y}dy \\
        &= 2\left(-e^{-y}|_0^\infty\right) = 2
    \end{aligned}
\end{equation*}
\pagebreak
\section*{2.}
Since $g$ can be any function. Let the function $g$ be the indicator function $1_{\left(\frac{1}{4}, \frac{9}{16}\right)}$.
Then we have that 
\begin{equation*}
    \begin{aligned}
        P\left(\frac{1}{2} < X < \frac{3}{4}\right) &= E[g(X)] \\
        &= \int_{0}^{1}\left(1_{\frac{1}{4}<x^{2}<\frac{9}{16}}\right)dx \\ 
        &=\int_{\frac{1}{2}}^{\frac{3}{4}}1 dx \\
        &= x|_\frac{1}{2}^\frac{3}{4} \\
        &= \frac{1}{4}
    \end{aligned}
\end{equation*}
\pagebreak
\section*{3.}
Since the payment is none if it is below the amount $D$. We have that 
\begin{equation*}
    \begin{aligned}
        0.8 \cdot \frac{100000-0}{2} 
        &= \int_D^{100000} (x-D) \cdot \frac{1}{100000} dx \\
        &= \left(50000 - D + \frac{D^2}{200000}\right)
    \end{aligned}
\end{equation*}
Hence,  we have that $D \simeq 10557.281$
\pagebreak
\section*{4.}
Since $\lim_{y\to 0.5^-} F_Y(y) = 0.5$ and $\lim_{y\to 0.5^+} F_Y(y) = 1$. \\ 
We have that $P(Y=0.5) = 1-0.5 = 0.5$
Therefore, 
\[
    E[Y] = \int_0^{0.5} (y)' \cdot y dy + \frac{1}{2} \cdot \frac{1}{2} 
    = \left.\frac{y^2}{2} \right|_0^{0.5} + \frac{1}{4} 
    = \frac{3}{8}
\]
It is not continuous at 0.5 and hence is not continuous.
It is not discrete because for any open ball around (0.25,f(0.25)) = (0.25,0.25), there is a point in the form 
$(x,f(x))$ such that it is contained in that ball because $f$ is continuous at 0.25.
\end{document}