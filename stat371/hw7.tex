\documentclass[11pt]{article}
    \title{\textbf{Math 217 Homework I}}
    \author{Khac Nguyen Nguyen}
    \date{}
    
    \addtolength{\topmargin}{-3cm}
    \addtolength{\textheight}{3cm}
    
\usepackage{amsmath}
\usepackage{mathtools}
\usepackage{amsthm}
\usepackage{amssymb}
\usepackage{pgfplots}
\usepackage{xfrac}
\usepgfplotslibrary{polar}
\usepgflibrary{shapes.geometric}
\usetikzlibrary{calc}
\pgfplotsset{compat = newest}
\pgfplotsset{my style/.append style = {axis x line = middle, axis y line = middle, xlabel={$x$}, ylabel={$y$}, axis equal}}
\begin{document}
\section*{1.}
\[
    e^{-\lambda} = 1-0.2 = 0.8    
\]
\[
    1 - \sum_{i=0}^2 e^{-\lambda} \frac{\lambda^i}{i!} = 0.00157
\]
\pagebreak
\section*{2.}
\begin{equation*}
    \begin{aligned}
        E[X] 
        &= E[X|Y=0] \cdot P(Y=0) + E[X|Y>0] \cdot P(Y>0)  \\
        &= E[Y\cdot 8000|Y=0] \cdot P(Y=0) + E[(Y-1)\cdot 8000|Y>0] \cdot (1-P(Y=0)) \\
        &= 8000 E[Y|Y=0] \cdot P(Y=0) + (8000 E[Y|Y>0] - 8000) \cdot (1-P(Y=0)) \\
        &= 8000 \cdot E[Y] - 8000 \cdot (1-P(Y=0)) \\
        &= 8000 \cdot 4 -8000\cdot (1-e^{-4}) \\
        &= 24146.525
    \end{aligned}
\end{equation*}
\pagebreak
\section*{3.}
\subsection*{a.}
We have that the probability that 
the fire appears in any specific quadrant follows a poisson distribution with mean 
$10^{-3} \cdot (10\cdot 10 /4) = 0.025$. Hence, the probability that fire appears in any quadrant is 
\[
    1 - \frac{0.025^0 e^{-0.025}}{0!} = 1 - e^{-0.025}  
\]
Hence, the probability that fire appears in three or more quadrant is 
\[
    C_3^4 (1-e^{-0.025})^3 (e^{-0.025}) + C_4^4 (1-e^{-0.025})^4 = 0.0000590895
\]
\subsection*{b.}
The number of fires appears in the city follows a poisson distribution with mean 
$10^{-3} \cdot 10 \cdot 10 = 0.1$.
Hence, the expected value of fire happens in the city is 0.1
\pagebreak
\section*{4.}
Since the number of jobs arrive in 1 minute follows a poisson distribution with mean 1,
N, the number of jobs arrive in 15 minutes follows a poisson distribution with mean 15.
\begin{equation*}
    \begin{aligned}
        P(\text{there will be lost job}) &= 1 - P(\text{there will not be lost job}) \\
        &= 1 - \sum_{i=1}^\infty P(N = i, B > N) \\
        &= 1 - \sum_{i=1}^\infty P(N=i)P(B>i) \\
        &= 1 - \sum_{i=1}^\infty \frac{15^i e^{-15}}{i!} 2^{-i} \\
        &= 1 - e^{-15/2} \sum_{i=1}^\infty \frac{\left(\frac{15}{2}\right)^i e^{-15/2}}{i!} \\
        &= 1 - e^{-15/2} \\
        &= 0.999447
    \end{aligned}
\end{equation*}
\end{document}