\documentclass[11pt]{article}
    \title{\textbf{Math 217 Homework I}}
    \author{Khac Nguyen Nguyen}
    \date{}
    
    \addtolength{\topmargin}{-3cm}
    \addtolength{\textheight}{3cm}
    
\usepackage{amsmath}
\usepackage{mathtools}
\usepackage{amsthm}
\usepackage{amssymb}
\usepackage{pgfplots}
\usepackage{xfrac}
\usepgfplotslibrary{polar}
\usepgflibrary{shapes.geometric}
\usetikzlibrary{calc}
\pgfplotsset{compat = newest}
\pgfplotsset{my style/.append style = {axis x line = middle, axis y line = middle, xlabel={$x$}, ylabel={$y$}, axis equal}}
\begin{document}
\section*{1.}
Let $A$ be random value picked between 0 and 2 and $B$ be the random value picked between 1 and 3. 
\begin{equation*}
    \begin{aligned}
        F_X(x) &= P(\max(A,B) \le x) = P(A\le x, B\le x) \\
        &= P(A\le x) \cdot P(B\le x) \\
    \end{aligned}
\end{equation*}
If $0<x<1$ then $P(B \le x) = 0$ and hence $F_X(x) = 0$. \\
If $1<x<2$ then 
\[
    F_X(x) = \int_0^x \frac{1}{2} da + \int_1^x \frac{1}{2} db = \frac{x}{2}\cdot \left(\frac{x-1}{2}\right)
\]
\[
    \implies f_X(x) = \frac{x}{2}-\frac{1}{4}    
\]
If $2<x<3$ then $P(A\le x) = 1$
\[
    F_X(x) = \int_1^x \frac{1}{2} db = \frac{x-1}{2}
\]
\[
    \implies f_X(x) = \frac{1}{2}
\]
\[
    E[X] = \int_1^2 \frac{x^2}{2}-\frac{x}{4} dx + \int_2^3 \frac{x}{2} dx = \frac{49}{24}
\]
\pagebreak
\section*{2.}
Let $A$ be the minute that the man arrives, $B$ be the minute that the woman arrives. Then since $A$ and $B$ is independent, we can find the joint probability
\[
    f_{A,B}(a,b) = \frac{1}{60} \cdot \frac{1}{30} = \frac{1}{1800}    
\] the probability that 
the first to arrive waits no longer than 5 minutes is 
\[
    P(|A-B|\le 5) = P(B-5\le A \le B+5) = \int_{15}^{45} \int_{a-5}^{a+5} \frac{1}{1800} dbda = \frac{1}{6}
\]
\[
    P(A<B) = \int_{15}^{45} \int_0^a \frac{1}{1800} dbda = \frac{1}{2}
\]
\pagebreak
\section*{3.}
Gamma(3,2)
\[
    1-\int_0^8  \frac{x^2e^{-x/2}}{2^3 \cdot 2!} dx = 0.238 
\]
\pagebreak
\section*{4.}
\[
    P(Y \in (60,75] \cup [70,90) \cup [120,240) \cup {59}) = P(Y \in (60,90) \cup (120,240))
\]
\[
    = \int_{60}^{90} \frac{y^2e^{-y/60}}{60^3 2!} dy + \int_{120}^{240} \frac{y^2e^{-y/60}}{60^3 2!} dy = 0.54942
\]
\end{document}