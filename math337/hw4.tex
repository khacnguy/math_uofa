
\documentclass[11pt]{article}
    \title{\textbf{Math 217 Homework I}}
    \author{Khac Nguyen Nguyen}
    \date{}
    
    \addtolength{\topmargin}{-3cm}
    \addtolength{\textheight}{3cm}
    
\usepackage{amsmath}
\usepackage{mathtools}
\usepackage{amsthm}
\usepackage{amssymb}
\usepackage{pgfplots}
\usepackage{xfrac}  
\usepackage{hyperref}
\usepackage{xcolor}
\definecolor{Mybackground}{RGB}{40,49,51}
\pagecolor{Mybackground}
\color{white}


\newtheorem{definition}{Definition}[section]
\newtheoremstyle{mystyle}%                % Name
  {}%                                     % Space above
  {}%                                     % Space below
  {\itshape}%                                     % Body font
  {}%                                     % Indent amount
  {\bfseries}%                            % Theorem head font
  {}%                                    % Punctuation after theorem head
  { }%                                    % Space after theorem head, ' ', or \newline
  {\thmname{#1}\thmnumber{ #2}\thmnote{ (#3)}}%                                     % Theorem head spec (can be left empty, meaning `normal')

\theoremstyle{mystyle}
\newtheorem{theorem}{Theorem}[section]
\theoremstyle{definition}
\newtheorem*{exmp}{Example}
\begin{document}
\section*{1.}
Since $u(t, x, y) = T(t) f(x, y)$, then 
\[
  T'(t) f(x,y) = k T(t) (f_{xx}(x,y) + f_{yy}(x,y))
\]
thus 
\[
  \displaystyle\frac{T'(t)}{T(t)} = k\displaystyle\frac{ f_{xx}(x,y) + f_{yy}(x,y)}{f(x,y)} = -\lambda
\]
We have that 
\[
  T'(t) + \lambda T(t) = 0 \implies T_n(t) = Ce^{-\lambda t}
\]
and then apply $f(x,y) = X(x) Y(y)$, we get the 2 dimensional eigenvalue problem: 
\[
  \begin{cases}  
    f_{xx}(x,y) + f_{yy}(x,y) = -\displaystyle\frac{\lambda}{k} f(x,y) \\
    u(x,0,t) = X(x) Y(0) T(t) = 0 \implies Y(0) = 0 \\  
  u(x,H,t) = X(x) Y(H) T(t) = 0 \implies Y(H) = 0 \\ 
  u_x(0,y,t) = X'(0) Y) T(t) = 0 \implies X'(0) = 0 \\ 
  u_x(L,y,t) = X'(L) Y(y) T(t) = 0 \implies X'(L) = 0
  \end{cases}
\]
We have 
\[
  X''(x)Y(y) + X(x)Y''(y) = -\displaystyle\frac{\lambda}{k} X(x)Y(y)
\]
thus 
\[
  \displaystyle\frac{Y''(y)}{Y(y)}= -\displaystyle\frac{\displaystyle\frac{\lambda_1}{k} X(x)+ X''(x)}{X(x)} = - \mu 
\]
Therefore,  
\[
  \mu = \left(\displaystyle\frac{n\pi}{H} \right)^2
\]
for $n = 1,2, \hdots$ and the eigenfunctions are
\[
  Y_n(y) = \sin \displaystyle\frac{n\pi y}{H}
\]
Now, we solve for 
\[
  -\displaystyle\frac{\lambda}{k} X(x) + X''(x) = - \left(\displaystyle\frac{n\pi}{H}\right)^2 X(x)
\]
which can be rewritten as 
\[
  X''(x) = -X(x) \left(\mu -\displaystyle\frac{\lambda}{k} \right) 
\] 
Thus the eigenvalue for the equations with is for $m = 0,1,2, \hdots$
\[
  \mu - \displaystyle\frac{\lambda}{k} = \left( \displaystyle\frac{m \pi}{L}\right)^2  \implies \lambda = k \left( \left(\displaystyle\frac{n\pi}{H}\right)^2 - \left(\displaystyle\frac{m\pi}{L}\right)^2 \right)
\]
and 
\[
  X_m(x) = \cos\left(\displaystyle\frac{m\pi x}{L}\right)
\]
Therefore, we have that 
\[
  f_{m,n}(x,y) = \sin \displaystyle\frac{n\pi y}{H} \cos \displaystyle\frac{m\pi x}{L}
\]
and 
\begin{align*}
  u(x,y,t) = \sum_{m=0}^\infty \sum_{n=1}^\infty A_{mn} e^{-\lambda_{mn}t} \cos \left( \displaystyle\frac{m\pi x}{L}\right) \sin\left(\displaystyle\frac{n\pi y}{H}\right)
\end{align*}
where 
\[
  \lambda_{mn} = -k \left( \left(\displaystyle\frac{n\pi}{H}\right)^2 - \left(\displaystyle\frac{m\pi}{L}\right)^2\right) 
\]
We have that 
\[
  u(x,y,0) = \sum_{m=0}^\infty \left( \sum_{n=1}^\infty A_{mn} \sin\left(\displaystyle\frac{n\pi y}{H}\right) \right) \cos \left( \displaystyle\frac{m\pi x}{L}\right) = \alpha(x,y)
\]
Thus for $m > 0$,  
\[
  \sum_{n=1}^\infty A_{mn} \sin \displaystyle\frac{n\pi y}{H} = \displaystyle\frac{2}{L} \int_0^L \alpha(x,y) \cos \displaystyle\frac{n\pi x}{L} dx 
\]
and 
\[
  \sum_{n=1}^\infty A_{0n} \sin \displaystyle\frac{n\pi y}{H} = \displaystyle\frac{1}{L} \int_0^L \alpha(x,y) dx 
\]
Therefore, for $m>0$, 
\begin{align*}
  A_{mn} 
  &= \displaystyle\frac{2}{H} \int_0^H \sin \displaystyle\frac{n\pi y}{H} \cdot \left(  \displaystyle\frac{2}{L} \int_0^L \alpha(x,y) \cos \displaystyle\frac{n\pi x}{L} dx \right) dy \\
  &= \displaystyle\frac{4}{HL} \int_0^H \sin \displaystyle\frac{n\pi y}{H} \cdot \left( \int_0^L \alpha(x,y) \cos \displaystyle\frac{n\pi x}{L} dx \right) dy 
\end{align*}
and 
similarly, 
\[
  A_{0n} = \displaystyle\frac{2}{HL} \int_0^H \sin \displaystyle\frac{n\pi y}{H} \cdot \left( \int_0^L \alpha(x,y) dx \right) dy \\
\]
\newpage
\section*{2.}
Separation of variables $u(x,y,z) = f(x,y) Z(z)$, we have 
\[
  f_{xx}(x,y) Z(z) + f_{yy}(x,y) Z(z) + f(x,y)Z''(z) = 0
\]
thus 
\[
  \displaystyle\frac{Z''(z)}{Z(z)} = - \displaystyle\frac{f_{xx}(x,y) + f_{yy}(x,y)}{f(x,y)} = -\lambda
\]
and thus we have the system of equations 
\[
    \begin{cases}
      Z''(z) = -\lambda Z(z) \\ 
      u_z(x,y,0) = X(x) Y(y) Z'(0) = 0 \implies Z'(0) = 0 
    \end{cases}
\]
Applying $f(x,y) = X(x) Y(y)$, the other equation is
\[
  X''(x) Y(y) + Y''(y) X(x) = \lambda X(x) Y(y) \implies \displaystyle\frac{X''(x)}{X(x)} = \lambda - \displaystyle\frac{Y''(y)}{Y(y)} = -\mu 
\]
Thus we have 
\[
  \begin{cases}
    X''(x) = -\mu X(x) \\ 
    u(x,y,z) = 0 \text{ for } (x,y) \in \partial \Gamma \implies X(0) = X(L) = 0
  \end{cases}
\]
and 
\[
  \begin{cases}
    Y''(x) = -(-\lambda - \mu) X(x) \\ 
    u(x,y,z) = 0 \text{ for } (x,y) \in \partial \Gamma \implies X(0) = X(L) = 0
  \end{cases}
\]
hence 
\[
  \mu = \left( \displaystyle\frac{n\pi}{L}\right)^2
\]
for $n=1,2,\hdots$ and 
\[
  -\lambda - \mu = \left( \displaystyle\frac{m\pi }{H}\right)^2
\]
for $m= 1,2,\hdots$, thus 
\[
  \lambda_{mn} = - \left( \displaystyle\frac{n\pi}{L}\right)^2 - \left( \displaystyle\frac{m\pi}{H}\right)^2  < 0 
\]
Therefore, 
\[
  Z(z) = c_1 \exp (\sqrt{-\lambda}z) + c_2 \exp(-\sqrt{-\lambda}z)
\]
and 
\[
  Z'(z) = \sqrt{-\lambda}(c_1 \exp (\sqrt{-\lambda}z) - c_2 \exp(-\sqrt{-\lambda}z))
\]
and plugging in $Z'(0) = 0$, we have $c_1 = c_2$ thus 
\[
  Z(z) = c_2 \cosh(\sqrt{-\lambda}z)
\]
Therefore, 
\begin{align*}
  u(x,y,z) = \sum_{n=1}^\infty \sum_{m=1}^\infty A_{mn} \sin \left( \displaystyle\frac{n\pi x}{L}\right) \sin \left( \displaystyle\frac{m\pi y}{H}\right) \cosh\left(z \sqrt{-\lambda_{mn}} \right)
\end{align*}
where 
\[
  \lambda_{mn} = -\left(\displaystyle\frac{n\pi}{L}\right)^2 - \left(\displaystyle\frac{m\pi}{H}\right)^2
\]
Applying $u(x,y,H) = \alpha(x,y)$, then using orthogonality we have 
\begin{align*}
  &\int_0^L \int_0^H \alpha(x,y) \sin \displaystyle\frac{n\pi x}{L} \sin \displaystyle\frac{n\pi y}{H} dx dy \\ 
  =& A_{mn} \cosh (H\sqrt{-\lambda_{mn}}) \int_0^L \int_0^H \sin^2 \displaystyle\frac{n\pi x}{L} \sin^2 \displaystyle\frac{m\pi y}{H} dx dy \\
  =& A_{mn} \cosh(H \sqrt{-\lambda_{mn}}) \displaystyle\frac{LH}{4}
\end{align*}
Therefore, 
\[
  A_{mn} = \displaystyle\frac{4}{LH \cosh(H \sqrt{-\lambda_{mn}})} \int_0^L \int_0^H \alpha(x,y) \sin \displaystyle\frac{n\pi x}{L} \sin \displaystyle\frac{n\pi y}{H} dx dy \\ 
\]
\newpage
\section*{3.}
Apply separation of variables $u(r, \theta, t) = f(r, \theta) T(t)$, we have 
\[
  T''(t) f(r, \theta) = c^2 \left(\displaystyle\frac{1}{r} \displaystyle\frac{\partial}{\partial r} \left( r f_r(r, \theta) T(t)\right) + \displaystyle\frac{1}{r^2} f_{\theta \theta}(r, \theta)T(t)\right)
\]
Thus 
\[
  \displaystyle\frac{T''(t)}{c^2 T(t)} = \displaystyle\frac{\displaystyle\frac{1}{r} \displaystyle\frac{\partial}{\partial r} (r f_r(r, \theta)) + \displaystyle\frac{1}{r^2}f_{\theta \theta}(r, \theta)}{f(r, \theta)} = -\lambda
\]
Now, apply separation of variables $f(r, \theta) = R(r) \phi(\theta)$, we have 
\[
  \displaystyle\frac{1}{r} \displaystyle\frac{\partial}{\partial r} (rR'(r) \phi(\theta)) + \displaystyle\frac{1}{r^2} R(r) \phi''(\theta) = - \lambda R(r) \phi(\theta) 
\]
and thus 
\[
  \displaystyle\frac{1}{r^2} R(r) \phi''(\theta) = - \phi(\theta) \left(\lambda R(r) + \displaystyle\frac{1}{r} (rR''(r) + R'(r)) \right)
\]
and 
\[
  \displaystyle\frac{\phi''(\theta)}{\phi(\theta)} = -\displaystyle\frac{ r^2\lambda R(r) + r^2R''(r) + rR'(r))}{R(r)} =  -\mu 
\]
Now since 
\[
  u(r, -\pi, t) = u(r, \pi, t) \implies \phi(-\pi) = \phi(\pi) 
\]
and 
\[
  u_\theta(r, -\pi, t) = u_\theta(r, \pi, t) \implies \phi'(-\pi) = \phi'(\pi) \\
\]
We have for $n = 0,1,2, \hdots$, 
\[
  \mu_n = n^2 
\]
and for $n=1,2,\hdots$
\[
  \phi_n(\theta) = \sin(n\theta), \cos(n\theta)
\]
with 
\[
  \phi_0(\theta) = 1
\]
Now to solve for $R$, we look at the equation 
\[
  r^2R''(r) + rR'(r) + \left(\lambda r^2 - \mu \right) R(r) = 0
\]
Let $z = \sqrt{\lambda }r$, we have 
\[
  z^2\displaystyle\frac{\partial^2 R}{\partial z^2} + z \displaystyle\frac{\partial R}{\partial z} + \left(z^2 - n^2 \right) R(z) = 0
\]
which is a bessel DE of order $n$. Thus 
\[
  R(z) = c_1 J_n(z) + c_2 Y_n(z)
\]
But since $R(0)$ is bounded, 
\[
  R(r) = c_1 J_n(\sqrt{\lambda }r)
\]
But $R(a) = 0$ so $\sqrt{\lambda} a$ is zeros of the $J_n$ function. Thus let $z_{mn}$ be the m-th zeros, we get 
\[
  \lambda_{mn} = \displaystyle\frac{z^2_{mn}}{a^2} > 0
\]
Since $\lambda_{mn} > 0$, we can find 
\[
  T(t) = cos( c\sqrt{\lambda}t), \sin(c\sqrt{\lambda}t)
\]
Thus 
\begin{align*}
  u(r, \theta, t) 
  =& \sum_{n=0}^\infty \sum_{m=1}^\infty A_{mn} J_n(\sqrt{\lambda_{mn}}r) \cos(n \theta) \cos(c \sqrt{\lambda_{mn}}t) \\
  +&\sum_{n=0}^\infty \sum_{m=1}^\infty A_{mn} J_n(\sqrt{\lambda_{mn}}r) \cos(n \theta) \sin(c \sqrt{\lambda_{mn}}t) \\
  +&\sum_{n=1}^\infty \sum_{m=1}^\infty A_{mn} J_n(\sqrt{\lambda_{mn}}r) \sin(n \theta) \cos(c \sqrt{\lambda_{mn}}t) \\
  +&\sum_{n=1}^\infty \sum_{m=1}^\infty A_{mn} J_n(\sqrt{\lambda_{mn}}r) \sin(n \theta) \sin(c \sqrt{\lambda_{mn}}t)
\end{align*}
\newpage
\section*{4.}
Apply separation of variables $u(r, \theta, t) = f(r, \theta) T(t)$, we have 
\[
  T''(t) f(r, \theta) = c^2 \left(\displaystyle\frac{1}{r} \displaystyle\frac{\partial}{\partial r} \left( r f_r(r, \theta) T(t)\right) + \displaystyle\frac{1}{r^2} f_{\theta \theta}(r, \theta)T(t)\right)
\]
Thus 
\[
  \displaystyle\frac{T''(t)}{c^2 T(t)} = \displaystyle\frac{\displaystyle\frac{1}{r} \displaystyle\frac{\partial}{\partial r} (r f_r(r, \theta)) + \displaystyle\frac{1}{r^2}f_{\theta \theta}(r, \theta)}{f(r, \theta)} = -\lambda
\]
Now, apply separation of variables $f(r, \theta) = R(r) \phi(\theta)$, we have 
\[
  \displaystyle\frac{1}{r} \displaystyle\frac{\partial}{\partial r} (rR'(r) \phi(\theta)) + \displaystyle\frac{1}{r^2} R(r) \phi''(\theta) = - \lambda R(r) \phi(\theta) 
\]
and thus 
\[
  \displaystyle\frac{1}{r^2} R(r) \phi''(\theta) = - \phi(\theta) \left(\lambda R(r) + \displaystyle\frac{1}{r} (rR''(r) + R'(r)) \right)
\]
and 
\[
  \displaystyle\frac{\phi''(\theta)}{\phi(\theta)} = -\displaystyle\frac{ r^2\lambda R(r) + r^2R''(r) + rR'(r))}{R(r)} =  -\mu 
\]
Now since 
\[
  u(r, -\pi, t) = u(r, \pi, t) \implies \phi(-\pi) = \phi(\pi) 
\]
and 
\[
  u_\theta(r, -\pi, t) = u_\theta(r, \pi, t) \implies \phi'(-\pi) = \phi'(\pi) \\
\]
We have for $n = 0,1,2, \hdots$, 
\[
  \mu_n = n^2 
\]
and for $n=1,2,\hdots$
\[
  \phi_n(\theta) = \sin(n\theta), \cos(n\theta)
\]
with 
\[
  \phi_0(\theta) = 1
\]
Now to solve for $R$, we look at the equation 
\[
  r^2R''(r) + rR'(r) + \left(\lambda r^2 - \mu \right) R(r) = 0
\]
Let $z = \sqrt{\lambda }r$, we have 
\[
  z^2\displaystyle\frac{\partial^2 R}{\partial z^2} + z \displaystyle\frac{\partial R}{\partial z} + \left(z^2 - n^2 \right) R(z) = 0
\]
which is a bessel DE of order $n$. Thus 
\[
  R(z) = c_1 J_n(z) + c_2 Y_n(z)
\]
But since $R(0)$ is bounded, 
\[
  R(r) = c_1 J_n(\sqrt{\lambda }r) 
\]
But $R'(a) = 0$ so $\sqrt{\lambda} a$ is the extrema of the $J_n$ function. Thus let $z_{mn}$ be the m-th extrema, we get 
\[
  \lambda_{mn} = \displaystyle\frac{z^2_{mn}}{a^2} > 0
\]
Since $\lambda_{mn} > 0$, we can find 
\[
  T(t) = cos( c\sqrt{\lambda}t), \sin(c\sqrt{\lambda}t)
\]
Thus 
\begin{align*}
  u(r, \theta, t) 
  =& \sum_{n=0}^\infty \sum_{m=1}^\infty A_{mn} J_n(\sqrt{\lambda_{mn}}r) \cos(n \theta) \cos(c \sqrt{\lambda_{mn}}t) \\
  +&\sum_{n=0}^\infty \sum_{m=1}^\infty A_{mn} J_n(\sqrt{\lambda_{mn}}r) \cos(n \theta) \sin(c \sqrt{\lambda_{mn}}t) \\
  +&\sum_{n=0}^\infty \sum_{m=1}^\infty A_{mn} J_n(\sqrt{\lambda_{mn}}r) \sin(n \theta) \cos(c \sqrt{\lambda_{mn}}t) \\
  +&\sum_{n=0}^\infty \sum_{m=1}^\infty A_{mn} J_n(\sqrt{\lambda_{mn}}r) \sin(n \theta) \sin(c \sqrt{\lambda_{mn}}t)
\end{align*}
\newpage
\section*{5.}
If equilibrium exists then 
\[
  ku_{xx} = - Q(x)
\]
Thus if $Q = 0$ 
\[
  u_{E} = c_1 x + c_2
\]
and 
\[
  u_{E}' = c_1
\]
\subsection*{a.}
Substituting the boundary conditions, we have  
\[
  u_{E} = Bx + A
\]
Now let 
\[
  u(x,t) = v(x,t) + u_E(x)
\]
Thus as $u_E$ is independent of $t$ and $u_E'' = 0$
\[
  v_t(x,t) = k v_{xx}(x,t)
\]
with boundary and initial conditions  
\[
  \begin{cases}
    v(0,t) = 0 \\
    v_x(L,t) = 0 \\
    v(x,0) = u(x,0) - u_{eq}(x) = f(x) - Bx - A
  \end{cases}
\]
$v(x,t) = X(x) T(t)$, we have 
\[
  X(x)T'(t) = k X''(x)T(t) \implies \displaystyle\frac{T'(t)}{kT(t)} = \displaystyle\frac{X''(x)}{X(x)} = -\lambda
\]
with boundary conditions
\[
  X(0) = 0 \text{ and } X'(L) = 0
\]
which has the eigenfunctions with $n=1,3,5, \hdots$ 
\[
  X_n(x) = \sin(\sqrt{\lambda_n }x)
\]
where 
\[
  \lambda_n = \left(\displaystyle\frac{n\pi}{2L}\right)^2
\]
and 
\[
  T_n(t) = \exp(-k\lambda_n t)
\]
Thus 
\[
  v(x,t) = \sum_{n = 1,3,\hdots}^\infty A_n \sin(\sqrt{\lambda_n}x)\exp(-k\lambda_n t)
\]
Applying the initial conditions, we have 
\[
  v(x,0) = \sum_{n=1,3,\hdots}^\infty A_n \sin(\sqrt{\lambda_n}x) = f(x) - Bx - A
\]
Therefore, 
\[
  A_n = \displaystyle\frac{2}{L} \int_0^L (f(x) - Bx - A) \sin(\sqrt{\lambda_n}x) dx
\]
and the final solution is 
\[
  u(x,t) = v(x,t) + Bx + A 
\]
where $v(x,t)$ is given above. 
\subsection*{b.}
The equilibrium does not exists as $u_x(0,t) = 0 < B = u_x(L,t)$, $u_{xx}(x_0,t) \ne 0$ for some $x_0$ thus $u_t(x_0, t) \ne 0$. Since equilibrium does not exists and the known boundary conditions for $u$, we can find the function 
\[
  r(x) = \displaystyle\frac{B}{2L}x^2 
\]
so that 
\[
  r'(0) = 0 \text{ and } r'(L) = B
\]
Then let
\[
  u(x,t) = v(x,t) + r(x)
\]
we find that 
\[
  \begin{cases}
    v_t(x,t) = u_t(x,t) = ku_{xx}(x,t) = k(v_{xx}(x,t) + \displaystyle\frac{B}{L}) \\
    v_t(0,t) = 0 \\
    v_t(L,t) = 0 \\
    u(x,0) = v(x,0) + r(x) \implies v(x,0) = f(x) - \displaystyle\frac{B}{2L}x^2
  \end{cases}
\]
\subsection*{c.}
If equilibrium exists then 
\[
  u_E''(x) = \displaystyle\frac{-Q(x)}{k}
\]
Thus 
\[
  u_E(x) = \displaystyle\frac{L^2}{k 4\pi^2}\sin \displaystyle\frac{2\pi x}{L} + c_1 x 
\]
Thus, 
\[
  u_E'(x) = \displaystyle\frac{L}{2k\pi}\cos \displaystyle\frac{2\pi x}{L} + c_1 
\]
and substitute the boundary conditions in 
\[
  u_E'(0) = u_E'(L) = c_1 + \displaystyle\frac{L}{2k\pi}= 0 \implies c_1 = - \displaystyle\frac{L}{2k\pi}  
\]
Thus, 
\[
  v_t(x,t) = u_t(x,t) = ku_{xx}(x,t) + Q(x,t) = kv_{xx}(x,t) + ku_E''(x) + Q(x,t) 
\]
Therefore, 
\[
  v_t(x,t) = kv_{xx}(x,t)
\]
with boundary and initial conditions 
\[
  \begin{cases}
    v_x(0,t) = 0 \\
    v_x(L,t) = 0 \\
    v(x,0) = f(x) - u_E(x)
  \end{cases}
\]
$v(x,t) = X(x) T(t)$, then 
\[
  X'(0) = X'(L) = 0
\]
Thus for $n=0,1,2, \hdots$ 
\[
  X_n(x) = \cos(\sqrt{\lambda_n }x)
\]
where $\lambda_n = \left( \displaystyle\frac{n\pi}{L}\right)^2$
and thus similar to part a
\[
  v(x,t) = \sum_{n=0}^\infty A_n \cos(\sqrt{\lambda_n}x) \exp(-k \lambda_n t)
\]
and by using the initial conditions, we get 
\[
  A_n = \displaystyle\frac{2}{L} \int_0^L (f(x) - u_E(x)) \cos(\sqrt{\lambda_n}x) dx
\]
and the final solution is 
\[
  u(x,t) = v(x,t) + u_E(x)
\]
where $v(x,t)$ and $u_E(x)$ are given above. 
\newpage
\section*{6.}
Let 
\[
  u(x,t) = v(x,t) + r(x,t)
\]
\subsection*{a.}
Let 
\[
  r(x,t) = c_1(t) x^2 + c_2(t) x
\]
then 
\[
  r_x(x,t) = 2c_1(t)x + c_2(t) 
\]
and thus using the boundary conditions, 
\[
  r(x,t) = \displaystyle\frac{B(t) - A(t)}{2L} x^2 + A(t)x
\]
We also have 
\[
  v_t(x,t) - r_t(x,t) = k v_{xx}(x,t) - kr_{xx}(x,t) + Q(x, t)
\]
Thus the equation is 
\[
  v_t(x,t) = kv_{xx}(x,t) + Q(x,t) + k\displaystyle\frac{B(t) -A(t)}{L} + r_t(x,t)
\]
where 
\[
  r_t(x,t) = \displaystyle\frac{B'(t) - A'(t)}{2L}x^2 + A'(t)x
\]
with the boundary conditions 
\[
  v_t(0,t) = v_t(L, t) = 0
\]
and initial conditions 
\[
  v(x,0) = u(x,0) - r(x,0) = f(x) + \displaystyle\frac{B(0)-A(0)}{2L}x^2 - A(0)x
\]
\subsection*{b.}
Let 
\[
  r(x,t) = c_1(t)x + c_2(t)
\]
then 
\[
  r_x(x,t) = c_1(t)
\]
and thus using the boundary conditions
\[
  r(x,t) = A(t) x + B(t) - A(t) L
\]
Therefore, 
\[
  v_t(x,t) = kv_{xx}(x,t) + Q(x,t) + r_t(x,t)
\]
with the boundary conditions 
\[
  v_t(0,t) = v_t(L,t) = 0 
\]
and initial conditions 
\[
  v(x,0) = u(x,0) - r(x,0) = f(x) - A(0) x + B(0) - A(0)L
\]
\subsection*{c.}
Let 
\[
  r(x,t) = c_1(t)x  + c_2(t)
\]
then 
\[
  r_x(0,t) = c_1(t) = 0 
\]
and 
\[
  h(c_3(t) - B(t)) = 0 \implies c_2(t) = B(t) 
\]
Thus 
\[
  v_t(x,t) = kv_{xx}(x,t) + Q(x,t) + r_t(x,t)  = kv_{xx}(x,t) + Q(x,t) + B'(t)
\]
with the boundary conditions 
\[
  v_t(0,t) = v_t(L,t) = 0 
\]
and initial conditions 
\[
  v(x,0) = u(x,0) - r(x,0) = f(x) - B(0)
\]
\newpage
\section*{7.}
\subsection*{a.}
Suppose there is an equilibrium 
\[
  u_E''(x) = - \displaystyle\frac{Q(x)}{k} = -\displaystyle\frac{1}{k}
\]
Then let 
\[
  u_E(x) = - \displaystyle\frac{1}{2k}x^2 + c_1 x + c_2
\]
Plugging in the boundary conditions $u_E(0) = A, u_E(L) = B$, we have 
\[
  u_E(x) = -\displaystyle\frac{1}{2k}x^2 + \left(\displaystyle\frac{L}{2k} + \displaystyle\frac{B-A}{L}\right)x+ A 
\]
Then let 
\[
  v(x,t) = u(x,t) - u_E(x)
\]
We have that 
\[
  \begin{cases}
    v_t = k v_{xx} \\
    v(0,t) = u(0,t) - u_E(0) = 0 \\
    v(L,t) = u(L,t) - u_E(L) = 0 \\
    v(x,0) = u(x,0) - u_E(x) = f(x) - u_E(x) \\
    v_t(x,0) = u_t(x,0) = g(x)
  \end{cases}
\]
Let $v(x,t) = X(x) T(t)$, we know the boundary conditions for $X$ are 
\[
  X(0) = 0 \text{ and } X(L) = 0
\]
Thus the eigenfunctions are for $n=1,2,3, \hdots$, 
\[
  X_n(x) = \sin(\sqrt{\lambda_n}x)
\]
where $\lambda_n = \left( \displaystyle\frac{n\pi}{L}\right)^2$and 
\[
  T_n(t) = \exp(-k \lambda_n t)
\]
Thus 
\[
  v(x,t) = \sum_{n=1}^\infty A_n \sin(\sqrt{\lambda_n} x) \exp(-k\lambda_n t)
\]
where we can find $A_n$ using the initial conditions $v(x,0) = f(x) - u_E(x)$
\[
  A_n = \displaystyle\frac{2}{L} \int_0^L (f(x) - u_E(x)) \sin \displaystyle\frac{n\pi x}{L} dx
\]
However, we have another initial conditions to satisfy
\[
  v_t(x,0) = - \sum_{n=1}^\infty A_n k \lambda_n \sin(\sqrt{\lambda_n}x) = g(x)
\]
Thus 
\[
  -A_n k\lambda_n =  \displaystyle\frac{2}{L} \int_0^L g(x) \sin \displaystyle\frac{n\pi x}{L} dx
\]
which means
\[
  A_n = -\displaystyle\frac{2}{Lk \lambda_n} \int_0^L g(x) \sin\displaystyle\frac{n\pi x}{L} dx
\]
Therefore, there is a solution only when the two version of $A_n$ are equal to each other, that is 
\[
  \int_0^L (f(x) - u_E(x)) \sin \displaystyle\frac{n\pi x}{L} dx = -\displaystyle\frac{1}{k \lambda_n} \int_0^L g(x) \sin \displaystyle\frac{n\pi x}{L} dx
\]
and the solution is 
\[
  u(x,t) = v(x,t) + u_E(x)
\]
where $v(x,t)$ and $u_E(x)$ are mentioned above. 
\subsection*{b.}
Suppose that there is an equilibrium solution, that is 
\[
  u_E''(x) = - \displaystyle\frac{\sin \displaystyle\frac{\pi x}{L}}{k}
\]
Then let 
\[
  u_E(x) = \displaystyle\frac{L^2}{k\pi^2} \sin \displaystyle\frac{\pi x}{L} + c_1x
\]
Plugging in the boundary conditions, 
\[
  \begin{cases}
    u_E(0) = \displaystyle\frac{L^2}{k\pi^2} \sin(0) = 0 \\
    u_E(L) = \displaystyle\frac{L^2}{k\pi^2}\sin(\pi) + c_1 L = 0
  \end{cases}
\]
Thus $c_1 = 0$ and we have 
\[
  \begin{cases}  
    v_t(x,t) = kv_{xx}(x,t) \\
    v(0,t) = u(0,t) + u_E(0) = u(0,t) = 0 \\
    v(L,t) = u(L,t) + u_E(L) = u(L,t) = 0 \\
    v(x,0) = u(x,0) - u_E(x) = f(x) - u_E(x) \\
    u_t(x,0) = v_t(x,0) = g(x)
  \end{cases}
\]
Let $v(x,t) = X(x) T(t)$, we know the boundary conditions for $X$ are 
\[
  X(0) = 0 \text{ and } X(L) = 0
\]
Thus the eigenfunctions are for $n=1,2,3, \hdots$, 
\[
  X_n(x) = \sin(\sqrt{\lambda_n}x)
\]
where $\lambda_n = \left( \displaystyle\frac{n\pi}{L}\right)^2$and 
\[
  T_n(t) = \exp(-k \lambda_n t)
\]
Thus 
\[
  v(x,t) = \sum_{n=1}^\infty A_n \sin(\sqrt{\lambda_n} x) \exp(-k\lambda_n t)
\]
where we can find $A_n$ using the initial conditions $v(x,0) = f(x) - u_E(x)$
\[
  A_n = \displaystyle\frac{2}{L} \int_0^L (f(x) - u_E(x)) \sin \displaystyle\frac{n\pi x}{L} dx
\]
However, we have another initial conditions to satisfy
\[
  v_t(x,0) = - \sum_{n=1}^\infty A_n k \lambda_n \sin(\sqrt{\lambda_n}x) = g(x)
\]
Thus 
\[
  -A_n k\lambda_n =  \displaystyle\frac{2}{L} \int_0^L g(x) \sin \displaystyle\frac{n\pi x}{L} dx
\]
which means
\[
  A_n = -\displaystyle\frac{2}{Lk \lambda_n} \int_0^L g(x) \sin\displaystyle\frac{n\pi x}{L} dx
\]
Therefore, there is a solution only when the two version of $A_n$ are equal to each other, that is 
\[
  \int_0^L (f(x) - u_E(x)) \sin \displaystyle\frac{n\pi x}{L} dx = -\displaystyle\frac{1}{k \lambda_n} \int_0^L g(x) \sin \displaystyle\frac{n\pi x}{L} dx
\]
and the solution is 
\[
  u(x,t) = v(x,t) + u_E(x)
\]
where $v(x,t)$ and $u_E(x)$ are mentioned above. 
\newpage
\section*{8.}
\subsection*{a.}
Let $r(x)$ satisfy the $r(0) = 0, r'(L) = 0$, thus let 
\[
  r(x) = c_1 x^2 + c_2x + c_3
\]
then applying the boudary conditions, we have 
\[
  r(x) = c_1 x^2 - 2c_1 Lx
\]
Then let 
\[
  v(x,t) = u(x,t) - r(x)
\]
and 
\[
  r''(x) = - \displaystyle\frac{Q}{k} \implies c_1 = -\displaystyle\frac{Q}{k}
\]
and therefore, 
\[
  \begin{cases}
    v_t(x,t) = kv_{xx}(x,t) \\
    v(0,t) = u(0,t) - r(0) = 0 \\
    v_x(L,t) = u_x(L,t) - r'(L) = 0 \\
    v(x,0) = u(x,0) - r(x) = f(x) - r(x)
  \end{cases}
\]
Thus let $v(x,t) = X(x) T(t)$, we have the boundary conditions for $X$
\[
  \begin{cases}
    v(0,t) = X(0) T(t) =0 \implies X(0) = 0 \\
    v_x(L,t) = X'(L) T(t) = 0 \implies X'(L) = 0 
  \end{cases}
\]
Then we know that for $n=1,3,5, \hdots$
\[
  X_n(x) = \sin(\sqrt{\lambda_n}x)
\]
where $\lambda_n = \left( \displaystyle\frac{n\pi}{2L}\right)^2$ and 
\[
  T_n(t) = \exp(-k \lambda_n t)
\]
Therefore, 
\[
  u(x,t) = \sum_{n=1,3,5, \hdots}^\infty A_n \sin(\sqrt{\lambda_n}x) \exp(-k \lambda t)
\]
Applying the initial conditions, we have 
\[
  A_n = \displaystyle\frac{2}{L} \int_0^L (f(x)-r(x)) \sin \displaystyle\frac{n\pi x}{L} dx
\]
and the solution is 
\[
  u(x,t) = v(x,t) + r(x)
\]
where $v(x,t)$ and $r(x)$ are mentioned above. 
\subsection*{b.}
Let $r(x,t)$ satisfy the $r(0,t) = A(t), r_x(L,t) = 0$, thus let
\[
  r(x,t) = c_1 x^2 + c_2x + c_3
\]
then applying the boudary conditions, we have 
\[
  r(x,t) = c_1 x^2 - 2c_1 Lx + A(t)
\]
Then let 
\[
  v(x,t) = u(x,t) - r(x,t)
\]
and 
\[
  r_{xx}(x,t) = - \displaystyle\frac{Q}{k} \implies c_1 = -\displaystyle\frac{Q}{k}
\]
and therefore, 
\[
  \begin{cases}
    v_t(x,t) = kv_{xx}(x,t) \\
    v(0,t) = u(0,t) - r(0) = 0 \\
    v_x(L,t) = u_x(L,t) - r'(L) = 0 \\
    v(x,0) = u(x,0) - r(x) = f(x) - r(x)
  \end{cases}
\]
Thus let $v(x,t) = X(x) T(t)$, we have the boundary conditions for $X$
\[
  \begin{cases}
    v(0,t) = X(0) T(t) =0 \implies X(0) = 0 \\
    v_x(L,t) = X'(L) T(t) = 0 \implies X'(L) = 0 
  \end{cases}
\]
Then we know that for $n=1,3,5, \hdots$
\[
  X_n(x) = \sin(\sqrt{\lambda_n}x)
\]
where $\lambda_n = \left( \displaystyle\frac{n\pi}{2L}\right)^2$ and 
\[
  T_n(t) = \exp(-k \lambda_n t)
\]
Therefore, 
\[
  u(x,t) = \sum_{n=1,3,5, \hdots}^\infty A_n \sin(\sqrt{\lambda_n}x) \exp(-k \lambda t)
\]
Applying the initial conditions, we have 
\[
  A_n = \displaystyle\frac{2}{L} \int_0^L (f(x)-r(x)) \sin \displaystyle\frac{n\pi x}{L} dx
\]
and the solution is 
\[
  u(x,t) = v(x,t) + r(x)
\]
where $v(x,t)$ and $r(x)$ are mentioned above.
\newpage
\section*{9.}
Let 
\[
  r(x) = - \displaystyle\frac{x}{\pi} + 1
\]
so that $r(0) = 1$ and $r(\pi) = 0$, then let 
\[
  v(x,t) = u(x,t) - r(x)
\]
so that we have 
\[
  \begin{cases}
    v_t = v_{xx} + e^{-2t} \sin(5x) \\
    v(0,t) = 0 \\
    v(\pi, t) = 0 \\
    v(x,0) = u(x,0) - r(x) = \displaystyle\frac{\pi}{x} - 1
  \end{cases}
\]
Let's first solve the homogenous part $v_t = v_{xx}$, let $v(x,t) = X(x)T(t)$, then we know $v$ has the solution 
\[
  v(x,t) = \sum_{n=1}^\infty  \sin(\sqrt{\lambda_n} x) A_n(t)
\]
where $\lambda_n = n^2$ for $n=1,2,3, \hdots$, plugging this back in the equations, 
\[
  \sum_{n=1}^\infty  A_n'(t) \sin(\sqrt{\lambda_n} x) = -\sum_{n=1}^\infty \lambda_n \sin(\sqrt{\lambda_n} x) A_n(t) + \exp(-2t) \sin(5x)
\]
Thus 
\[
  \sum_{n=1}^\infty \sin(n x) \left(A_n'(t) + n^2 A_n(t)\right) = \exp(-2t) \sin(5x)
\]
which means that 
\[
  A_n'(t) + n^2 A_n(t) = 
  \begin{cases}
    0, \indent& \text{ if } n\ne 5 \\
    \exp(-2t), \indent& \text{ if } n = 5
  \end{cases}
\]
In case $n \ne 5$, 
\[
  A_n'(t) + n^2 A_n(t) = 0 \implies A_n(t) = ce^{-n^2 t}
\]
and thus 
\[
  c = A_n(0)
\]
In case $n=5$, 
\[
  A_5'(t) + 25 A_5(t) = \exp(-2t) \implies A_5(t) = \displaystyle\frac{\exp(-2t)}{23} + ce^{-25t}
\]
and since $A_5(0) = \displaystyle\frac{1}{23} + c \implies c = A_5(0) - \displaystyle\frac{1}{23}$. \\
Now all we need to do is find $A_n(0)$, we have 
\[
  \sum_{n=1}^\infty \sin(nx) A_n(0) = v(x,0) = \displaystyle\frac{\pi}{x}-1
\]
Thus 
\[
  A_n(0) = \displaystyle\frac{2}{\pi} \int_0^\pi \left( \displaystyle\frac{\pi}{x}-1\right) \sin(nx) dx = \displaystyle\frac{-2}{n\pi}
\]
and therefore the solution is 
\[
  u(x,t) = \sum_{n=1}^\infty A_n(t)\sin(nx) + 1 - \displaystyle\frac{x}{\pi}
\]
where 
\[
  A_n(t) = 
  \begin{cases}
    \displaystyle\frac{1}{23}e^{-2t} - \left(\displaystyle\frac{2}{5}\pi + \displaystyle\frac{1}{23}\right)e^{-25t}, \indent &\text{ if } n = 5 \\
    -\displaystyle\frac{2}{n\pi}e^{-n^2t}, \indent &\text{ otherwise}
  \end{cases}
\]
\newpage
\section*{10.}
Let's first solve the homogenous part $u_{xx} = u_{yy}$, then we know $u$ has the solution 
\[
  u(x,y) = \sum_{n=1}^\infty A_n(y) \sin(xn)
\]
Plugging back in the equation, 
\[
  \sum_{n=1}^\infty - A_n(y) \sin(xn) + \sum_{n=1}^\infty A_n''(y) \sin(xn) = e^{2y}\sin(x)
\]
Thus 
\[
  \sum_{n=1}^\infty \sin(nx)(A_n''(y) - A_n(y)) = e^{2y}\sin(x)
\]
Therefore, 
\[
  A_n''(y) - A_n(y) = 
  \begin{cases}
    e^{2y} \sin(x), \indent &\text{ if } n = 1 \\
    0, \indent &\text{ if } n \ne 1
  \end{cases}
\]
In case $n=1$, 
\begin{align*}
  A_1''(y) - A_1(y) = e^{2y}
\end{align*}
and therefore, 
\[
  A_1(y) = C_1e^y + C_2e^{-y} + \displaystyle\frac{1}{3} e^{2y}
\]
We have $A_1(0) = 0$, thus $C_1 + C_2 +\displaystyle\frac{1}{3}=0$
and 
\[
  A_1(y) = K_1 e^y - (K_1 + 1/3)e^{-y} + \displaystyle\frac{1}{3}e^{2y}
\]
Then we can rewrite it as 
\[
  A_1(y) = C_n \sinh(y) + \displaystyle\frac{1}{3}(e^{-y} + e^{2y})
\]
In case $n\ne 1$, 
\[
  A_n(y) = K_1 e^y + K_2 e^{-y}
\]
Since $A_1(y) = 0$, $K_1 = - K_2$ and $A_n(y) = C_n \sinh(y)$.
Then all we need to do is find $C_n$, we have 
\[
  \sum_{n=1}^\infty \sin(nx) A_n(L) = f(x) 
\]
Thus for $n\ne 1$, 
\[
  A_n(L) = C_n \sinh(L) = \displaystyle\frac{2}{L} \int_0^L f(x) \sin(nx) dx  
\]
and 
\[
  C_n = \displaystyle\frac{2}{L\sinh(L)} \int_0^L f(x) \sin(nx) dx  
\]
and for $n = 1$, 
\[
  A_n(L) = C_1 \sinh(L) + \displaystyle\frac{1}{3} (e^{-L} + e^{2L}) = \displaystyle\frac{2}{L} \int_0^L f(x) \sin(nx) dx  
\]
and 
\[
  C_1 = \displaystyle\frac{2}{L} \left( \displaystyle\frac{2}{L}\int_0^L f(x) \sin(nx) dx  - \displaystyle\frac{e^{-L} + e^{2L}}{3}\right)
\]
\end{document}
