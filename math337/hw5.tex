\documentclass[11pt]{article}
    \title{\textbf{Math 217 Homework I}}
    \author{Khac Nguyen Nguyen}
    \date{}
    
    \addtolength{\topmargin}{-3cm}
    \addtolength{\textheight}{3cm}
    
\usepackage{amsmath}
\usepackage{mathtools}
\usepackage{amsthm}
\usepackage{amssymb}
\usepackage{pgfplots}
\usepackage{xfrac}  
\usepackage{hyperref}
\usepackage{xcolor}
\definecolor{Mybackground}{RGB}{40,49,51}
\pagecolor{Mybackground}
\color{white}


\newtheorem{definition}{Definition}[section]
\newtheoremstyle{mystyle}%                % Name
  {}%                                     % Space above
  {}%                                     % Space below
  {\itshape}%                                     % Body font
  {}%                                     % Indent amount
  {\bfseries}%                            % Theorem head font
  {}%                                    % Punctuation after theorem head
  { }%                                    % Space after theorem head, ' ', or \newline
  {\thmname{#1}\thmnumber{ #2}\thmnote{ (#3)}}%                                     % Theorem head spec (can be left empty, meaning `normal')

\theoremstyle{mystyle}
\newtheorem{theorem}{Theorem}[section]
\theoremstyle{definition}
\newtheorem*{exmp}{Example}
\begin{document}
\section*{1.}
\subsection*{a.}
\begin{align*}
  \mathcal{F}[c_1 f + c_2g](\xi) 
  &= \displaystyle\frac{1}{2\pi} \int_\mathbb{R} (c_1f + c_2g)(x) e^{ix \cdot \xi} dx \\
  &= \displaystyle\frac{1}{2\pi}c_1 \int_\mathbb{R} f(x) e^{ix \cdot \xi} dx + \displaystyle\frac{1}{2\pi}c_2 \int_\mathbb{R} g(x) e^{ix \cdot \xi} dx \\
  &= c_1\mathcal F f +c_2 \mathcal F g
\end{align*}
\subsection*{b.}
\[
  \mathcal F (fg) = \displaystyle\frac{1}{2\pi} \int_\mathbb{R} f(x)g(x) e^{-ix \cdot \xi} dx \ne \displaystyle\frac{1}{4\pi^2}\left(\int_\mathbb{R} f(x) e^{-ix \cdot \xi} \right)\left(\int_\mathbb{R} g(x) e^{-ix \cdot \xi} \right) = \mathcal F f \mathcal F g
\]
\newpage
\section*{2.}
\subsection*{a.}
\begin{align*}
  (\mathcal F f)(\xi) 
  &= \displaystyle\frac{1}{2\pi} \int_\mathbb{R} f(x) e^{ix \xi} dx \\
  &= \displaystyle\frac{1}{2\pi} \int_{-a}^a e^{ix \xi} dx \\
  &= \displaystyle\frac{1}{2\pi} \left.\displaystyle\frac{e^{ix \xi}}{i\xi} \right|_{x =-a}^a \\
  &= \displaystyle\frac{1}{2\pi} \displaystyle\frac{e^{ia \xi} - e^{-ia\xi}}{i\xi} \\
  &= \displaystyle\frac{2\sinh(ia\xi)}{2\pi i\xi} \\
  &= \displaystyle\frac{-i\sin(-a\xi)}{i\xi \pi} \\ 
  &= \displaystyle\frac{sin(a\xi)}{\xi \pi}
\end{align*}
\subsection*{b.}
\begin{align*}
  f(x) &= \int_\mathbb{R} e^{-|\xi| \alpha} e^{-i\xi x} d\xi \\
  &= \int_0^\infty e^{-\xi(\alpha + ix)} d\xi + \int_{-\infty}^0 e^{\xi(\alpha - ix)} d\xi \\
  &= -\left.\displaystyle\frac{e^{-\xi(\alpha + ix)}}{\alpha + ix}\right|^\infty_{\xi = 0} + \left.\displaystyle\frac{e^{\xi(\alpha - ix)}}{\alpha - ix}\right|^0_{\xi = -\infty} \\
  &= \displaystyle\frac{1}{\alpha + ix} + \displaystyle\frac{1}{\alpha - ix}\\
  &= \displaystyle\frac{\alpha - ix + \alpha + ix}{(\alpha+ix)(\alpha -ix)} \\
  &= \displaystyle\frac{2\alpha}{\alpha^2 + x^2}
\end{align*}
\subsection*{c.}
\begin{align*}
  &\int_\mathbb{R} -iF'(\xi) e^{-i\xi x} d\xi \\
  =& -ie^{-i\xi x}|_{-\infty}^\infty - \int_\mathbb{R} F(\xi) (-i \cdot (-i x) e^{-i \xi x}) d\xi \\
  =& x\int_\mathbb{R} F(\xi)e^{-i\xi x} d\xi \\
  =& \mathcal F [x f(x)]
\end{align*}
\newpage
\section*{3.}
We have that 
\begin{align*}
  \mathcal F [u_t] = U_t &= k \mathcal F [u_{xx}] + c \mathcal[u_x] \\
  &= -k\xi^2 U - ci \xi U 
\end{align*}
Thus we can find 
\[
  U(\xi, t) = C(\xi) e^{-k \xi^2 t -ci\xi t}
\]
and since $u(x,0) = f(x)$ and $U(\xi,0) = F(\xi)$, 
\[
  U(\xi, t) = F(\xi) e^{-k \xi^2 t-ci \xi t }
\]
Let $G(\xi) = e^{-k\xi^2 t}$, $H(\xi) = F(\xi) e^{-ci \xi t}$ we have 
\[
  U(\xi, t) = G(\xi) H(\xi)
\]
And the inverse fourier of $G,H$ are 
\[
  g(x) = \displaystyle\frac{1}{\sqrt{2kt}} e^{-x^2/4kt}
\]
\[
  h(x) = f(x-ct)
\]
Thus the solution is 
\[
  u(x,t) = \displaystyle\frac{1}{2\pi} \left( f(x-ct) * \displaystyle\frac{1}{\sqrt{2kt}} e^{-x^2/4kt}\right)
\]
\newpage
\section*{4.}
Apply the fourier transform, we have that 
\[
  \begin{cases}
    U_t = -k\xi^2 U - \gamma U \\
    U(\xi,0) = F(\xi) 
  \end{cases}
\]
Then, we can solve for 
\[
  U(\xi, t) = C(\xi) e^{-(k \xi^2 + \gamma)t}
\]
and using the initial condition, 
\[
  U(\xi, t) = F(\xi) e^{-(k \xi^2 + \gamma)t} = e^{-\gamma t} F(\xi) e^{-k\xi^2 t}
\]
And apply the inverse, we have 
\[
  u(x,t) = e^{-\gamma t} \left(f(x) * \displaystyle\frac{1}{\sqrt{2kt}} e^{-x^2/4kt} \right) 
\]
\newpage
\section*{5.}
Apply the fourier transform on $y$, we have that 
\[
  \begin{cases}
    U_{xx} -\xi^2 U = 0 \\
    U(0,\xi) = G_1(\xi) \\
    U(L, \xi) = G_2(\xi)
  \end{cases}
\]
Thus 
\[
  U(x, \xi) = C_1(\xi) e^{-\xi x} + C_2(\xi) e^{\xi x}
\]
To ensure the boundedness of the solution, we must have that 
\[
  C_1(\xi) = 0 \text{ if } \xi < 0 \text{ and } C_2(\xi) = 0 \text{ if } \xi > 0
\]
Thus, the solution can be rewrite as 
\[
  U(x,\xi) = C(\xi) e^{-|\xi|x}
\]
The initial conditions state that 
\[
  C(\xi) = G_1(\xi) 
\]
and 
\[
  U(L,\xi) = G_1(\xi) e^{-|\xi| L} = G_2(\xi)
\]
Thus, 
\[
  u(x,y) = \displaystyle\frac{1}{2\pi} \left(g_1(y) * \displaystyle\frac{2L}{y^2 + L^2}\right)
\]
\newpage
\section*{6.}
Apply fourier transform, we have 
\[
  \begin{cases}
    U_{tt}(\xi, t) =  - c^2 \xi^2 U(\xi, t) \\
    U(\xi, 0) = F(\xi) \\
    U_t(\xi, 0) = 0
  \end{cases}
\]
Thus 
\[
  U(\xi, t) = C_1(\xi) \cos(c\xi t) + C_2(\xi) \sin(c\xi t)
\]
Apply the boundary conditions, we have that 
\[
  C_2(\xi) c\xi \cos(c\xi 0) = 0 \implies C_2(\xi) = 0
\]
and 
\[
  C_1(\xi) = F(\xi)
\]
Thus 
\[
  U(\xi, t) = F(\xi) \cos(c\xi t)
\]
and therefore,
\begin{align*}  
  u(x,t)
  &= \int_\mathbb{R} F(\xi) \cos(c\xi t) e^{-i\xi x} d\xi \\
  &= \displaystyle\frac{1}{2} \int_\mathbb{R} F(\xi) (e^{-i\xi(x-ct)} + e^{-i\xi(x+ct)}) d\xi \\
  &= \displaystyle\frac{1}{2} \left(f(x-ct) + f(x+ct) \right)
\end{align*}
\newpage
\section*{7.}
Apply the fourier cosine transform on y and $u_y(x,0) = 0$,  
\[
  \begin{cases}
    U_{xx}(x, \xi) - \xi^2 U(x, \xi) = 0 \\
    U(0, \xi) = G_1(\xi) \\
    U_x(L, \xi) = 0 \\
  \end{cases}
\]
We have that 
\[
  U(x, \xi) = C_1(\xi) e^{-\xi x} + C_2(\xi) e^{\xi x}
\]
Apply the boundary conditions, 
\[
    -\xi C_1(\xi) e^{-\xi L} + \xi C_2(\xi) e^{\xi L} = 0 \implies C_1(\xi) =C_2(\xi) e^{-2 \xi L}
\]
Thus, we can rewrite 
\[
  U(x, \xi) = C(\xi) \cosh(\xi (L-x))
\]
Apply the other boundary conditions give us 
\[
  C(\xi) = \displaystyle\frac{G_1(\xi)}{\cosh(\xi L)}
\]
Thus, solution is 
\[
  u(x,y) = \displaystyle\frac{1}{\pi}\int_0^\infty g_1(\overline x) (f(x, y-\overline x) + f(x,y+\overline x)) dx
\]
where 
\[
  f(x,y) = \int_0^\infty \displaystyle\frac{\cosh(\xi(L-x))}{\cosh(\xi L)} \cos(\xi y) d\xi
\]
\newpage
\section*{8.}
Apply fourier transform on $y$, 
\[
  \begin{cases}
    U_{xx}(x, \xi) - \xi^2 U(x,\xi) = 0 \\
    U(0,\xi) = G(\xi)
  \end{cases}
\]
Then 
\[
  U(x, \xi) = C_1(\xi) e^{\xi x} + C_2(\xi) e^{-\xi x}
\]
To ensure $U$ is bounded for $x<0$, 
\[
  C_1(\xi) = 0 \text{ if } \xi < 0 \text{ and } C_2(\xi) = 0 \text{ if } \xi >0
\]
Thus, we can rewrite 
\[
  U(x, \xi) = C(\xi) e^{|\xi|x}
\]
and apply the boundary condition gives 
\[
  U(x,\xi) = G(\xi) e^{|\xi|x}
\]
Thus let $t = y-\overline x$ so that $dt = -d\overline x$, we have 
\begin{align*}
  u(x,y) &= \displaystyle\frac{1}{2\pi} \int_\mathbb{R} g(\overline x)  \displaystyle\frac{-2(y-\overline x)}{(y-\overline x)^2+x^2} d\overline x \\
  &= \displaystyle\frac{1}{2\pi} \int_{y+1}^{y-1} \displaystyle\frac{2t}{t^2 + x^2} dt \\
  &= \displaystyle\frac{1}{2\pi} \ln(x^2 + t^2)|_{t=y+1}^{y-1} \\
  &= \displaystyle\frac{1}{2\pi} \left(\ln\left(y^2-2y+x^2+1\right)-\ln\left(y^2+2y+x^2+1\right) \right)
\end{align*}
\newpage
\section*{9.}
Since both a,b have $u(0,y) = 0$, apply the fourier sine transform on $x$, we have 
\[
  U_{yy}(\xi, y) - \xi^2 U(\xi, y) = 0 
\]
Therefore, 
\[
  U(\xi, y) = C_1(\xi) e^{\xi y} + C_2(\xi) e^{-\xi y}
\]
To ensure the boundedness of $U$ on $y>0$, 
\[
  C_1(\xi) = 0
\]
Thus 
\[
  U(\xi, y) = C(\xi) e^{-\xi y}
\]
\subsection*{a.}
\[
  u_y(x,0) = f(x) \implies U_y(\xi, 0) = F(\xi)
\]
Hence, 
\[
  U_y(\xi, 0) = -\xi C(\xi) e^{-\xi 0} = F(\xi)
\]
and 
\[
  U(\xi, y) = - \displaystyle\frac{F(\xi)}{\xi} e^{-\xi y}
\]
and 
\[
  u(x, y) = \displaystyle\frac{1}{\pi} \int_0^\infty f(\overline x)[g(x-\overline x,y) + g(x+ \overline x,y)] d\overline{x}
\]
where $g(x,y)$ is the inverse fourier cosine of $-\displaystyle\frac{e^{-\xi y}}{\xi}$, which is 
\[
  g(x,y) = - \int_0^y \displaystyle\frac{x}{x^2 + \overline x^2} d\overline x = - \arctan \displaystyle\frac{y}{x}
\]
\subsection*{b.}
\[
  u(x,0) = f(x) \implies U(\xi, 0) = F(\xi)
\]
Hence, 
\[
  U(\xi, y) = F(\xi) e^{-\xi y}
\]
and 
\[
  u(x, y) = \displaystyle\frac{1}{\pi} \int_0^\infty f(\overline x)[g(x-\overline x,y) + g(x+ \overline x,y)] d\overline{x}
\]
where 
\[
  g(x,y) = \displaystyle\frac{y}{x^2 + y^2}
\]
\newpage
\section*{10.}
Apply the fourier transform on $x$,    
\[
  \begin{cases}
    U_{tt}(\xi, t) = - c^2 \xi^2 U(\xi, t) \\
    U(\xi, 0) = 0 \\
    U_t(\xi, 0) = G(\xi) 
  \end{cases}
\]
Thus 
\[
  U(\xi, t) = C_1(\xi) \cos(c\xi t) + C_2(\xi) \sin(c \xi t)
\]
Apply the boundary condition $C_1(\xi) = 0$ and 
\[
  U(\xi, t) = C(\xi) \sin(c\xi t)
\]
Apply the other boundary conditions, 
\[
  U_t(\xi, 0) = C(\xi) c\xi \cos(c\xi 0) = G(\xi) \implies C(\xi) = \displaystyle\frac{G(\xi)}{c \xi}
\]
Hence, 
\[
  U(\xi, t) = G(\xi) \displaystyle\frac{\sin(c\xi t)}{c\xi }
\]
Thus 
\[
  u(x,t) = \displaystyle\frac{1}{2\pi} \int_\mathbb{R} g(\overline x) f(x-\overline x, t) d\overline x
\]
where 
\[
  f(x, t) = 
  \begin{cases}
    \displaystyle\frac{\pi}{c} &\text{ if } |x| < ct \\
    0 &\text{ if } |x| > ct
  \end{cases}
\]
\end{document}
