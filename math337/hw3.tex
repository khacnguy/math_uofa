

\documentclass[11pt]{article}
    \title{\textbf{Math 217 Homework I}}
    \author{Khac Nguyen Nguyen}
    \date{}
    
    \addtolength{\topmargin}{-3cm}
    \addtolength{\textheight}{3cm}
    
\usepackage{amsmath}
\usepackage{mathtools}
\usepackage{amsthm}
\usepackage{amssymb}
\usepackage{pgfplots}
\usepackage{xfrac}  
\usepackage{hyperref}
\usepackage{xcolor}
\definecolor{Mybackground}{RGB}{40,49,51}
\pagecolor{Mybackground}
\color{white}


\newtheorem{definition}{Definition}[section]
\newtheoremstyle{mystyle}%                % Name
  {}%                                     % Space above
  {}%                                     % Space below
  {\itshape}%                                     % Body font
  {}%                                     % Indent amount
  {\bfseries}%                            % Theorem head font
  {}%                                    % Punctuation after theorem head
  { }%                                    % Space after theorem head, ' ', or \newline
  {\thmname{#1}\thmnumber{ #2}\thmnote{ (#3)}}%                                     % Theorem head spec (can be left empty, meaning `normal')

\theoremstyle{mystyle}
\newtheorem{theorem}{Theorem}[section]
\theoremstyle{definition}
\newtheorem*{exmp}{Example}
\begin{document}
\section*{1.}
Apply seperation of variables, $u(x,t) = X(x) T(t)$, we have 
\begin{align*}
  &\rho_0 X(x) T''(t) = T_0 X''(x)T(t) - \beta X(x)T'(t) \\ 
  \implies & X(x) (\rho_u T''(t) + \beta T'(t)) = T_0 X''(x)T(t) \\
  \implies & \displaystyle\frac{\rho_u T''(t) + \beta T'(t)}{T_0 T(t)} = \displaystyle\frac{X''(x)}{X(x)} = - \lambda
\end{align*}
and 
\[
  \begin{cases}
    u(0,t) = 0 \implies X(0) = 0 \\
    u(L,t) = 0 \implies X(L) = 0
  \end{cases}
\]
Thus for every integer $n\ge 1$,  
\[
  X_n(x) = \sin \displaystyle\frac{n\pi x}{L}
\]
and 
\[
  \lambda_n = \displaystyle\frac{n^2\pi^2}{L^2}
\]
Now we need to solve for 
\[
  \rho_u T''(t) + \beta T'(t) + T_0 \displaystyle\frac{n^2 \pi^2}{L^2} T(t) = 0
\]
which has no solution as 
\[
 \beta^2 - 4 \rho_u \displaystyle\frac{T_0 n^2 \pi^2}{L^2} < \beta^2 -\displaystyle\frac{4 \rho_u \pi^2 T_0}{L^2} < 0 
\]
\newpage
\section*{2.}
Apply seperation of variables, $u(x,t) = X(x) T(t)$, we have  \\
\[
  \begin{cases}
    u(0,t) = 0 \implies X(0) = 0 \\
    u(L,t) = 0 \implies X(L) = 0 \\
    u_t(x,0) = 0 \implies T'(0) = 0 \\
  \end{cases}
\]
and 
\begin{align*}
  &X(x) T''(t) = c^2 X''(x)T(t) \\
  \implies & \displaystyle\frac{T''(t)}{T(t)} = c^2 \displaystyle\frac{X''(x)}{X(x)} = -\lambda
\end{align*}
and
Thus for every integer $n\ge 1$,  
\[
  X_n(x) = \sin \displaystyle\frac{n\pi x}{L}
\]
and 
\[
  \lambda_n =  \displaystyle\frac{c^2 n^2\pi^2}{L^2}
\]
and thus 
\[
  T''(t) + \lambda T(t) = 0 
\]
and 
\[
  T(t) = c_1 \cos \displaystyle\frac{cn\pi t}{L} + c_2 \sin \displaystyle\frac{cn \pi t}{L}
\]
\[
  T'(t) = \displaystyle\frac{cn\pi}{L} \left( - c_1 \sin \displaystyle\frac{cn\pi t}{L} + c_2 \cos \displaystyle\frac{cn \pi t}{L} \right)
\]
Therefore, 
\[
  T'(0) = \displaystyle\frac{cn\pi }{L} c_2 \cos \displaystyle\frac{cn\pi 0}{L} = \displaystyle\frac{cn \pi c_2}{L} = 0 \implies c_2 = 0
\]
Thus $c_1 \ne 0$ to avoid trivial solution
\[
  T_n(t) = \cos \displaystyle\frac{cn\pi t}{L} 
\]
and 
\begin{align*}
  u(x,t) &= \sum_{n=1}^\infty A_n \sin \displaystyle\frac{n\pi x}{L} \cos \displaystyle\frac{cn\pi t}{L} \\
  &= \sum_{n=1}^\infty \displaystyle\frac{A_n}{2} \left(\sin \displaystyle\frac{n\pi(x+ct)}{L} + \sin \displaystyle\frac{n\pi(x-ct)}{L} \right) \\
  &= \displaystyle\frac{F(x+ct) - F(x-ct)}{2} 
\end{align*}
As 
\[
  A_n = \displaystyle\frac{2}{L} \int_0^L f(x) \sin \displaystyle\frac{n\pi x}{L} dx 
\]
from 
\[
  u(x,0) = \sum_{n=1}^\infty A_n \sin \displaystyle\frac{n\pi x}{L} = f(x)
\]
\newpage
\section*{3.}
Apply seperation of variables, $u(x,t) = X(x) T(t)$, we have  \\
\[
  \begin{cases}
    u(0,t) = 0 \implies X(0) = 0 \\
    u(L,t) = 0 \implies X(L) = 0 \\
    u(x,0) = 0 \implies T(0) = 0 \\
  \end{cases}
\]
and 
\begin{align*}
  &X(x) T''(t) = c^2 X''(x)T(t) \\
  \implies & \displaystyle\frac{T''(t)}{T(t)} = c^2 \displaystyle\frac{X''(x)}{X(x)} = -\lambda
\end{align*}
and
Thus for every integer $n\ge 1$,  
\[
  X_n(x) = \sin \displaystyle\frac{n\pi x}{L}
\]
and 
\[
  \lambda_n =  \displaystyle\frac{c^2 n^2\pi^2}{L^2}
\]
and thus 
\[
  T''(t) + \lambda T(t) = 0 
\]
and 
\[
  T(t) = c_1 \cos \displaystyle\frac{cn\pi t}{L} + c_2 \sin \displaystyle\frac{cn \pi t}{L}
\]
Therefore, 
\[
  T(0) = c_1 \cos \displaystyle\frac{cn\pi 0}{L} = 0 \implies c_1 = 0
\]
Thus $c_2 \ne 0$ to avoid trivial solution
\[
  T_n(t) = \sin \displaystyle\frac{cn\pi t}{L} 
\]
and 
\begin{align*}
  u(x,t) &= \sum_{n=1}^\infty A_n \sin \displaystyle\frac{n\pi x}{L} \sin \displaystyle\frac{cn\pi t}{L} \\
  &= \sum_{n=1}^\infty \displaystyle\frac{A_n}{2} \left( \cos \displaystyle\frac{n\pi(x-ct)}{L} - \cos \displaystyle\frac{n\pi (x+ct)}{L} \right) \\ 
  &= \sum_{n=1}^\infty \displaystyle\frac{A_n}{2} \left. \cos \displaystyle\frac{n\pi \overline{x}}{L} \right|_{x+ct}^{x-ct} \\
  &= \sum_{n=1}^\infty \displaystyle\frac{A_n}{2} \int_{x-ct}^{x+ct} \displaystyle\frac{n\pi}{L}\sin \displaystyle\frac{n\pi \overline{x}}{L} d \overline{x} \\
  &= \displaystyle\frac{1}{2c} \int_{x-ct}^{x+ct} \underbrace{\sum_{n=1}^\infty A_n \displaystyle\frac{n\pi c}{L} \sin \displaystyle\frac{n\pi \overline{x}}{L}}_{G(\overline{x})} d\overline{x}
\end{align*}
as we can get $A_n$ to match 
\[
  u_t(x,t) = \sum_{n=1}^\infty \displaystyle\frac{cn\pi}{L} A_n \sin \displaystyle\frac{n\pi x}{L} \cos \displaystyle\frac{cn\pi t}{L} d\overline{x}
\]
Thus as $u_t(x,0) = f(x)$, we have  
\[
  \displaystyle\frac{n \pi c}{L}A_n = \displaystyle\frac{2}{L} \int_0^L f(x) \sin \displaystyle\frac{n\pi x}{L} dx 
\]
\newpage
\section*{4.}
\subsection*{1.}
\begin{align*}
  E(t) &= \displaystyle\frac{1}{2} \int_0^L u_t^2(x,t)dx + \displaystyle\frac{c^2}{2} \int_0^L u_x^2(x,t) dx \\
  &= \displaystyle\frac{1}{2} \int_0^L u_t^2(x,t) + c^2 u_x^2(x,t)dx \\
\end{align*}
Thus 
\begin{align*} 
  E'(t) &= \displaystyle\frac{1}{2} \int_0^L 2u_t u_{tt} + 2c^2 u_x u_{xt} dx \\
  &= c^2 \int_0^L u_t u_{xx} + u_x u_{xt} dx \\
  &= c^2 \left. \left( u_x u_t \right)\right|_0^L
\end{align*}
\subsection*{2.}
\subsubsection*{a.}
Since $u(0,t) = u(L,t) = 0$, $u_t(0,t) = u_t(L,t) = 0$
\[
  E'(t) = c^2 u_x(L,t)u_t(L,t) - c^2u_x(0,t)u_t(0,t) = 0
\]
Thus energy is conserved.
\subsubsection*{b.}
Since $u(L,t) = 0$, $u_t(L,t) = 0$
\[
  E'(t) = c^2 u_x(L,t)u_t(L,t) - c^2u_x(0,t)u_t(0,t) = 0
\]
Thus energy is conserved.
\subsubsection*{c, d.}
Since $u(0,t) = 0$, $u_t(0,t) = 0$
\[
  E'(t) = -c^2 \gamma u(L,t) u_t(L,t) = -\displaystyle\frac{c^2\gamma}{2} \displaystyle\frac{d}{dt} u_t^2(L,t)
\]
Thus integrating
\[
  E(t) - E(0) =  -\displaystyle\frac{\gamma c^2}{2} u^2(L,t')|_{t' = 0}^t
\]
and therefore
\[
  E(t) = E(0) + \displaystyle\frac{\gamma c^2}{2} (u^2(L,0) - u^2(L,t))
\]
If $\gamma > 0$, we have that $E$ increase at time $t$ if $u(L,0) > u(L,t)$ and decrease if $u(L,0) < u(L,t)$. Similarly, for $\gamma <0$, $E$ decrease at time $t$ if $u(L,0) > u(L,t)$ and increase if $u(L,0) < u(L,t)$. 
\newpage
\section*{5.}
We have that 
\[
  H \phi'' + \alpha H \phi' + (H \lambda \beta+ H\gamma) \phi = 0
\]
We want to find $H$ so that the DE is of the form 
\[
  p'(x)\phi'(x) + p(x) \phi''(x) + (\lambda \sigma + q)\phi = 0 
\]
Thus we have the conditions
\[
  \begin{cases}
    p' = \alpha H \\
    p = H \\
    H \lambda \beta = \lambda \sigma \\
    H \gamma = q
  \end{cases}
\]
The first 2 equations give us 
\[
  H'(x) = \alpha(x) H(x)
\]
Thus 
\[
  H(x) = \exp \left(\int \alpha(x) dx \right)
\]
and 
\[
  p(x) = \exp \left(\int \alpha(x) \right), q(x) = \gamma(x) \exp \left(\int \alpha(x) \right), \sigma(x) = \beta(x) \exp \left(\int \alpha(x) \right) 
\]
\newpage
\section*{6.}
\subsection*{a.}
We know that the eigenfunction and eigenvalue are 
\[
  \phi_n(x) = \cos \displaystyle\frac{n\pi x}{L}
\]
\[
  \lambda_n = \left( \displaystyle\frac{n\pi}{L}\right)^2
\]
Thus the smallest eigenvalue is $\lambda_0 = 0$ and there is no largest eigenvalue as 
\[
  \lim_{n \to \infty} \lambda_n = \infty
\]
\subsection*{b.}
\[
  \cos \displaystyle\frac{n\pi x}{L} = 0 \implies \displaystyle\frac{n\pi x}{L} = \displaystyle\frac{\pi}{2} + k\pi
\]
where $k \in \mathbb{Z}$. Thus 
\[
  x = \left(\displaystyle\frac{\pi}{2} + k\pi\right) \displaystyle\frac{L}{n\pi} = \displaystyle\frac{L}{2n} + \displaystyle\frac{kL}{n} = L \left( \displaystyle\frac{1+2k}{2n}\right)
\]
which means that as $0<x<L$, 
\[
  0<\displaystyle\frac{1+2k}{2n} < 1 \implies 0 \le k \le n-1 
\]
But since the eigenfunction starts at $n=0$, we have that this is the $n+1$-th eigenfunction which has $n$ zeros.
\subsection*{c.}
Since, the domain is $0<x<L$, every function $f(x)$ can be represented by the Fourier cosine series thus the eigenfunctions are complete. It is also obivously orthogonal from the material in class. 
\subsection*{d.}
Since $\phi'(0) = \phi'(L) = 0$, and we can see that 
\[
  p(x) = 1, \indent q(x) = 0, \indent \sigma(x) = 1
\]
Thus the Rayleigh quotient is 
\[
  \lambda = \displaystyle\frac{\int_0^L \phi'(x)^2 dx}{\int_0^L \phi(x)^2 dx} \ge 0
\]
$0$ can be an eigenvalue as else $\phi''(x) = \phi'(x) = 0$ for all $x \in (0,L)$ and therefore $\phi(x) = C$ for some constant $C$ since this satisfies the equation 
\[
  \phi''(x) = - \lambda \phi(x)
\]
\newpage
\section*{7.}
This is a Sturm-Liouville problem where
\[
  p(x) = 1, \indent q(x) = -x^2, \indent \sigma(x) = 1
\]
Thus we have the Rayleigh quotient
\[
  \displaystyle\frac{- \phi(x) \phi'(x) |^1_0 + \int_0^1 \phi'(x)^2 +x^2 \phi^2(x) dx}{\int_0^1 \phi^2(x) dx} = \displaystyle\frac{\int_0^1 \phi'(x)^2 + x^2 \phi^2(x)dx}{\int_0^1 \phi^2(x) dx}
\]
If the quotient = 0 then $\phi'(x)^2 + x^2 \phi^2(x) = 0$ a.e. in (0,1), which means that $\phi'(x) = \phi(x) = 0$ in (0,1) as $\phi(x)$ and $\phi'(x)$ is continuous. But that is a trivial solution thus 0 is not an eigenvalue.
\newpage
\section*{8.}
\subsection*{a.}
Multiplying, we have 
\[
  x \phi''(x) + \phi'(x) + \displaystyle\frac{\lambda}{x}\phi(x) = (\phi'(x)x)' + \displaystyle\frac{\lambda}{x}\phi(x) = 0
\]
is indeed a Sturm-Liouville with 
\[
  p(x) = x, \indent q(x) = 0, \indent \sigma(x) = \displaystyle\frac{1}{x}
\]
\subsection*{b.}
The Rayleigh quotient thus is 
\[ 
  \displaystyle\frac{-x \phi(x) \phi'(x) |^b_1 + \int_1^b x\phi'(x)^2 dx}{\int_1^b \phi^2(x) \displaystyle\frac{1}{x} dx} = \displaystyle\frac{\int_1^b x\phi'(x)^2 dx}{\int_1^b \phi^2(x)\displaystyle\frac{1}{x} dx}
\]
We have that for all $1<x<b$
\[
  x > 0, \phi'(x)^2 \ge 0, \phi^2(x) \ge 0, \displaystyle\frac{1}{x} > 0
\]
thus 
\[
 \displaystyle\frac{\int_1^b x\phi'(x)^2 dx}{\int_1^b \phi^2(x)\displaystyle\frac{1}{x} dx} \ge 0 
\]
\subsection*{c.}
If 0 is an eigenvalue then there is $\phi(x)$ such that $\phi'(x) = 0$ for all $x \in (1,b)$ but since $\phi(1) = \phi(b) = 0$. This means that $\phi(x) = 0$ and is a trivial solution thus 0 is not an eigenvalue. 
Since it is equidimensional, let $\phi(x) = x^r$. Plugging in we have that
\[
  x^2 r(r-1)x^{r-2} + xrx^{r-1} + \lambda x^r = x^r (r(r-1) + r + \lambda)) = 0  
\]
which has the characterisic function 
\[
  r^2 + \lambda = 0
\]
Thus 
\[
  r = \pm i\sqrt{\lambda}
\]
Since $\lambda > 0$, we have that 
\[
  \phi(x) = x^{\pm i \sqrt{\lambda}} = \exp(\pm i\sqrt{\lambda}\ln(x))
\]
The solution is thus 
\[
  \phi(x) = c_1 \cos(\sqrt{\lambda} \ln x) + c_2 \sin(\sqrt{\lambda} \ln x)
\]
Applying the boundary conditions, we have $c_1 = 0$ and 
\[
  \sin(\sqrt{\lambda} \ln(b)) = 0
\]
Therefore, for all $n\ge 1$,  
\[
  \lambda_n = \left( \displaystyle\frac{n\pi}{\ln b} \right)^2 
\]
and 
\[
  \phi_n(x) = \sin \left(\displaystyle\frac{n\pi}{\ln b} \ln x\right)
\]
\subsection*{d.}
The weight is found to be 
\[
  \sigma(x) = \displaystyle\frac{1}{x}
\]
And let $y = \ln x$, we have $dy = \displaystyle\frac{1}{x} dx$ and thus  
\begin{align*}
  &\int_1^b \phi_n(x) \phi_m(x) \sigma(x) dx \\
  =& \int_1^b \sin\left(\displaystyle\frac{n\pi}{\ln b} \ln x\right) \sin\left(\displaystyle\frac{m\pi}{\ln b} \ln x\right) \displaystyle\frac{1}{x} dx \\
  =& \int_0^{\ln b} \sin\left(\displaystyle\frac{n\pi}{\ln b} y\right) \sin\left(\displaystyle\frac{m\pi}{\ln b} y\right) dy \\
  =& 0 
\end{align*}
if $n \ne m$. 
\subsection*{e.}
Let the $n$-th eigenfunction be 0
\[
  \phi_n(x) = \sin \left(\displaystyle\frac{n\pi}{\ln b} \ln x\right) = 0
\]
We have 
\[
  \displaystyle\frac{n \pi}{\ln b} \ln x = k \pi \implies x = \exp\left(\displaystyle\frac{k\ln b}{n}\right) = b^{k/n}
\]
where $k \in \mathbb{Z}$. Since $1<x<b$, we have that 
\[
  0< \displaystyle\frac{k}{n} < 1
\]
and 
\[
  1 \le k \le n-1
\]
Thus $n-1$ zeros.
\newpage
\section*{9.}
\subsection*{a.}
Assume solution of the form $u(x,t) = X(x) T(t)$ then 
\[
  c \rho X T' = (K_0 X')' T + \alpha X T \implies \displaystyle\frac{(K_0 X')'}{c\rho X} + \displaystyle\frac{\alpha}{c \rho} = \displaystyle\frac{T'}{T} = -\lambda
\]
Thus we have the ODE for $X$ to be 
\[
  (K_0X')' + (c\rho \lambda + \alpha)X=0 
\]
which is a Sturm-Liouville with 
\[
  p(x) = K_0, \indent q(x) = \alpha, \indent \sigma(x) = c\rho
\]
Thus, the Rayleigh quotient is 
\[
  \lambda = \displaystyle\frac{-K_0 \phi(x) \phi'(x) |^L_0 + \int_0^L K_0 \phi'(x)^2 - \alpha \phi(x)^2 dx}{\int_0^L \phi(x)^2 \rho c dx} = \displaystyle\frac{\int_0^L K_0 \phi'(x)^2 - \alpha \phi(x)^2 dx}{\int_0^L \phi(x)^2 \rho c dx}  
\]
which is $\ge 0$ if $\alpha < 0$. 
\subsection*{b.}
We have that the eigenfunction for $t$ is 
\[
  T_n(t) = \exp(-\lambda_n t)
\]
Assume the appropriate eigenfunctions for $x$ is $X_n$. Then 
\[
  u(x,t) = \sum_{n=1}^\infty A_n  \exp(-\lambda_n t) X_n(x)
\]
To solve for $A_n$, we have that 
\[
  f(x) = u(x,0) = \sum_{n=1}^\infty A_n X_n(x)
\]
Multiplying by $X_m(x) c \rho$ and integrating on $[0,L]$. Then 
\[
  \int_0^L f(x) X_m(x) c \rho dx = \sum_{n=1}^\infty A_n \underbrace{\int_0^L X_n(x) X_m(x) \sigma(x)}_{ = 0 \text{ if } m \ne n} dx 
\]
Thus 
\[
  A_n = \displaystyle\frac{\int_0^L f(x) \phi_n(x) c\rho dx}{\int_0^L \phi^2_n(x) c \rho dx}
\]
\newpage
\section*{10.}
Apply seperation of varibles $u(x,t) = X(x) T(t)$, we have that 
\[
  \begin{cases}
    u(0,t) = 0 \implies X(0) = 0 \\
    u(L,t) = 0 \implies X(L) = 0 \\
    u(x,0) = f(x) \implies X(x)T(0) = f(x) \\
    u_t(x,0) = g(x) \implies X(x)T'(0) = g(x)
  \end{cases}
\]
and 
\begin{align*}
  &\rho X(x)T''(t) = T_0 X''(x)T(t) + \alpha X(x)T(t) \\ 
  \implies & \displaystyle\frac{T''(t)}{T(t)} = \displaystyle\frac{T_0 X''(x) + \alpha X(x)}{\rho X(x)} = -\lambda 
\end{align*}
Let's first rewrite the equation for $x$, 
\[
  T_0 X''(x) + X(x) (\alpha + \lambda \rho) = 0
\]
Thus is a Sturm-Liouville equation with 
\[
  p(x) = T_0, \indent q(x) = \alpha, \indent \sigma(x) = \rho
\]
As $X(0) = X(L) = 0$, the Rayleigh quotient will then be 
\[
  \lambda = \displaystyle\frac{\int_0^L T_0 X'(x)^2 - \alpha X(x)^2 dx}{\int_0^L X(x)^2 \rho dx} \ge 0
\]
as $\alpha < 0, \rho >0$ and the assumption we made $T_0 > 0$. Also note that if $0$ is an eigenvalue then $X'(x) = X(x) = 0$ and thus is $0$ is not an eigenvalue. \\ 
Since $\lambda > 0$, 
\[
  T(t) = c_1 \sin(\sqrt{\lambda}t) + c_2 \cos(\sqrt{\lambda}t)
\]
Thus 
\[
  u(x,t) = \sum_{n=1}^\infty A_n X_n(x) \sin(\sqrt{\lambda_n}t) + \sum_{n=1}^\infty B_n X_n(x) \cos(\sqrt{\lambda_n}t) 
\]
and 
\[
  u_t(x,t) = \sum_{n=1}^\infty A_n X_n(x) \sqrt{\lambda_n} \cos(\sqrt{\lambda_n}t) - \sum_{n=1}^\infty B_n X_n(x) \sqrt{\lambda_n} \sin(\sqrt{\lambda_n}t) 
\]
Now, we need to solve for $A_n, B_n$
\[
  \begin{cases}
    u(x,0) = f(x) = \sum_{n=1}^\infty B_n X_n(x) \\
    u_t(x,0) = g(x) = \sum_{n=1}^\infty A_n X_n(x) \sqrt{\lambda_n}
  \end{cases}
\]
Thus we get 
\[
  \begin{cases}
    \int_0^L f(x)X_m(x) \rho(x) dx = \sum_{n=1}^\infty B_n \int_0^L X_n(x) X_m(x) \rho(x)dx \\
    \int_0^L g(x)X_m(x) \rho(x) dx = \sum_{n=1}^\infty A_n \sqrt{\lambda_n} \int_0^L X_n(x) X_m(x) \rho(x)dx \\
  \end{cases}
\]
and therefore, 
\[
  \begin{cases}
    A_n = \displaystyle\frac{\int_0^L g(x) X_n(x) \rho(x) dx}{\sqrt{\lambda_n} \int_0^L X_n(x)^2 \rho(x) dx} \\
    B_n = \displaystyle\frac{\int_0^L f(x) X_n(x) \rho(x) dx}{\int_0^L X_n(x)^2 \rho(x) dx}
  \end{cases}
\]
\end{document}
