

\documentclass[11pt]{article}
    \title{\textbf{Math 217 Homework I}}
    \author{Khac Nguyen Nguyen}
    \date{}
    
    \addtolength{\topmargin}{-3cm}
    \addtolength{\textheight}{3cm}
    
\usepackage{amsmath}
\usepackage{mathtools}
\usepackage{amsthm}
\usepackage{amssymb}
\usepackage{pgfplots}
\usepackage{xfrac}  
\usepackage{hyperref}
\usepackage{xcolor}
\definecolor{Mybackground}{RGB}{40,49,51}
\pagecolor{Mybackground}
\color{white}


\newtheorem{definition}{Definition}[section]
\newtheoremstyle{mystyle}%                % Name
  {}%                                     % Space above
  {}%                                     % Space below
  {\itshape}%                                     % Body font
  {}%                                     % Indent amount
  {\bfseries}%                            % Theorem head font
  {}%                                    % Punctuation after theorem head
  { }%                                    % Space after theorem head, ' ', or \newline
  {\thmname{#1}\thmnumber{ #2}\thmnote{ (#3)}}%                                     % Theorem head spec (can be left empty, meaning `normal')

\theoremstyle{mystyle}
\newtheorem{theorem}{Theorem}[section]
\theoremstyle{definition}
\newtheorem*{exmp}{Example}
\begin{document}
\section*{1.}
Apply speration of variables, $u(x,t) = X(x) T(t)$, we have 
\begin{align*}
  &\rho_0 X(x) T''(t) = T_0 X''(x)T(t) - \beta X(x)T'(t) \\ 
  \implies & X(x) (\rho_u T''(t) + \beta T'(t)) = T_0 X''(x)T(t) \\
  \implies & \displaystyle\frac{\rho_u T''(t) + \beta T'(t)}{T_0 T(t)} = \displaystyle\frac{X''(x)}{X(x)} = - \lambda
\end{align*}
and 
\[
  \begin{cases}
    u(0,t) = 0 \implies X(0) = 0 \\
    u(L,t) = 0 \implies X(L) = 0
  \end{cases}
\]
Thus for every integer $n\ge 1$,  
\[
  X_n(x) = \sin \displaystyle\frac{n\pi x}{L}
\]
and 
\[
  \lambda_n = \displaystyle\frac{n^2\pi^2}{L^2}
\]
Now we need to solve for 
\[
  \rho_u T''(t) + \beta T'(t) + T_0 \displaystyle\frac{n^2 \pi^2}{L^2} T(t) = 0
\]
which has no solution as 
\[
 \beta^2 - 4 \rho_u \displaystyle\frac{T_0 n^2 \pi^2}{L^2} < \beta^2 -\displaystyle\frac{4 \rho_u \pi^2 T_0}{L^2} < 0 
\]
\newpage
\section*{2.}
Apply speration of variables, $u(x,t) = X(x) T(t)$, we have  \\
\[
  \begin{cases}
    u(0,t) = 0 \implies X(0) = 0 \\
    u(L,t) = 0 \implies X(L) = 0 \\
    u_t(x,0) = 0 \implies T'(0) = 0 \\
  \end{cases}
\]
and 
\begin{align*}
  &X(x) T''(t) = c^2 X''(x)T(t) \\
  \implies & \displaystyle\frac{T''(t)}{T(t)} = c^2 \displaystyle\frac{X''(x)}{X(x)} = -\lambda
\end{align*}
and
Thus for every integer $n\ge 1$,  
\[
  X_n(x) = \sin \displaystyle\frac{n\pi x}{L}
\]
and 
\[
  \lambda_n =  \displaystyle\frac{c^2 n^2\pi^2}{L^2}
\]
and thus 
\[
  T''(t) + \lambda T(t) = 0 
\]
and 
\[
  T(t) = c_1 \cos \displaystyle\frac{cn\pi t}{L} + c_2 \sin \displaystyle\frac{cn \pi t}{L}
\]
\[
  T'(t) = \displaystyle\frac{cn\pi}{L} \left( - c_1 \sin \displaystyle\frac{cn\pi t}{L} + c_2 \cos \displaystyle\frac{cn \pi t}{L} \right)
\]
Therefore, 
\[
  T'(0) = \displaystyle\frac{cn\pi }{L} c_2 \cos \displaystyle\frac{cn\pi 0}{L} = \displaystyle\frac{cn \pi c_2}{L} = 0 \implies c_2 = 0
\]
Thus $c_1 \ne 0$ to avoid trivial solution
\[
  T_n(t) = \cos \displaystyle\frac{cn\pi t}{L} 
\]
and 
\begin{align*}
  u(x,t) &= \sum_{n=1}^\infty A_n \sin \displaystyle\frac{n\pi x}{L} \cos \displaystyle\frac{cn\pi t}{L} \\
  &= \sum_{n=1}^\infty \displaystyle\frac{A_n}{2} \left(\sin \displaystyle\frac{n\pi(x+ct)}{L} + \sin \displaystyle\frac{n\pi(x-ct)}{L} \right) \\
  &= \displaystyle\frac{F(x+ct) - F(x-ct)}{2} 
\end{align*}
As 
\[
  A_n = \displaystyle\frac{2}{L} \int_0^L f(x) \sin \displaystyle\frac{n\pi x}{L} dx 
\]
from 
\[
  u(x,0) = \sum_{n=1}^\infty A_n \sin \displaystyle\frac{n\pi x}{L} = f(x)
\]
\newpage
\section*{3.}
Apply speration of variables, $u(x,t) = X(x) T(t)$, we have  \\
\[
  \begin{cases}
    u(0,t) = 0 \implies X(0) = 0 \\
    u(L,t) = 0 \implies X(L) = 0 \\
    u(x,0) = 0 \implies T(0) = 0 \\
  \end{cases}
\]
and 
\begin{align*}
  &X(x) T''(t) = c^2 X''(x)T(t) \\
  \implies & \displaystyle\frac{T''(t)}{T(t)} = c^2 \displaystyle\frac{X''(x)}{X(x)} = -\lambda
\end{align*}
and
Thus for every integer $n\ge 1$,  
\[
  X_n(x) = \sin \displaystyle\frac{n\pi x}{L}
\]
and 
\[
  \lambda_n =  \displaystyle\frac{c^2 n^2\pi^2}{L^2}
\]
and thus 
\[
  T''(t) + \lambda T(t) = 0 
\]
and 
\[
  T(t) = c_1 \cos \displaystyle\frac{cn\pi t}{L} + c_2 \sin \displaystyle\frac{cn \pi t}{L}
\]
Therefore, 
\[
  T(0) = c_1 \cos \displaystyle\frac{cn\pi 0}{L} = 0 \implies c_1 = 0
\]
Thus $c_2 \ne 0$ to avoid trivial solution
\[
  T_n(t) = \sin \displaystyle\frac{cn\pi t}{L} 
\]
and 
\begin{align*}
  u(x,t) &= \sum_{n=1}^\infty A_n \sin \displaystyle\frac{n\pi x}{L} \sin \displaystyle\frac{cn\pi t}{L} \\
  &= \sum_{n=1}^\infty \displaystyle\frac{A_n}{2} \left( \cos \displaystyle\frac{n\pi(x-ct)}{L} - \cos \displaystyle\frac{n\pi (x+ct)}{L} \right) \\ 
  &= \sum_{n=1}^\infty \displaystyle\frac{A_n}{2} \left. \cos \displaystyle\frac{n\pi \overline{x}}{L} \right|_{x+ct}^{x-ct} \\
  &= \sum_{n=1}^\infty \displaystyle\frac{A_n}{2} \int_{x-ct}^{x+ct} \displaystyle\frac{n\pi}{L}\sin \displaystyle\frac{n\pi \overline{x}}{L} d \overline{x} \\
  &= \displaystyle\frac{1}{2c} \int_{x-ct}^{x+ct} \underbrace{\sum_{n=1}^\infty A_n \displaystyle\frac{n\pi c}{L} \sin \displaystyle\frac{n\pi \overline{x}}{L}}_{G(\overline{x})} d\overline{x}
\end{align*}
as we can get $A_n$ to match 
\[
  u_t(x,t) = \sum_{n=1}^\infty \displaystyle\frac{cn\pi}{L} A_n \sin \displaystyle\frac{n\pi x}{L} \cos \displaystyle\frac{cn\pi t}{L} d\overline{x}
\]
Thus as $u_t(x,0) = f(x)$, we have  
\[
  \displaystyle\frac{n \pi c}{L}A_n = \displaystyle\frac{2}{L} \int_0^L f(x) \sin \displaystyle\frac{n\pi x}{L} dx 
\]
\newpage
\section*{4.}
\subsection*{1.}
\begin{align*}
  E(t) &= \displaystyle\frac{1}{2} \int_0^L u_t^2(x,t)dx + \displaystyle\frac{c^2}{2} \int_0^L u_x^2(x,t) dx \\
  &= \displaystyle\frac{1}{2} \int_0^L u_t^2(x,t) + c^2 u_x^2(x,t)dx \\
\end{align*}
Thus 
\begin{align*} 
  E'(t) &= \displaystyle\frac{1}{2} \int_0^L 2u_t u_{tt} + 2c^2 u_x u_{xt} dx \\
  &= c^2 \int_0^L u_t u_{xx} + u_x u_{xt} dx \\
  &= c^2 \left. \left( u_x u_t \right)\right|_0^L
\end{align*}
\subsection*{2.}
\subsubsection*{a.}
\[
  E'(t) = 0
\]
Thus energy is conserved
\subsubsection*{b.}
\[
  E'(t) = 0
\]
Thus energy is conserved
\subsubsection*{c.}

\subsubsection*{d.}
\end{document}
