\documentclass[11pt]{article}
    \title{\textbf{Math 217 Homework I}}
    \author{Khac Nguyen Nguyen}
    \date{}
    
    \addtolength{\topmargin}{-3cm}
    \addtolength{\textheight}{3cm}
    
\usepackage{amsmath}
\usepackage{mathtools}
\usepackage{amsthm}
\usepackage{amssymb}
\usepackage{pgfplots}
\usepackage{xfrac}
\usepackage{hyperref}
\usepackage{wasysym}

\newtheorem{definition}{Definition}[section]
\newtheoremstyle{mystyle}%                % Name
  {}%                                     % Space above
  {}%                                     % Space below
  {\itshape}%                                     % Body font
  {}%                                     % Indent amount
  {\bfseries}%                            % Theorem head font
  {}%                                    % Punctuation after theorem head
  { }%                                    % Space after theorem head, ' ', or \newline
  {\thmname{#1}\thmnumber{ #2}\thmnote{ (#3)}}%                                     % Theorem head spec (can be left empty, meaning `normal')

\theoremstyle{mystyle}
\newtheorem{theorem}{Theorem}[section]
\theoremstyle{definition}
\newtheorem*{exmp}{Example}
\begin{document}
\section*{1.}
We first have that 
\[
  u_t = ku_{xx}
\]
where $k = \frac{K}{cp}$. Then because it is an equilibrium temperature distribution 
\[
  u_{xx} = 0
\]
which implies that 
\[
  u_x = C_1
\]
and 
\[
  u(x) = C_1 x + C_2
\]
Plugging in $u(0) = T$ and $u_x(L) = \alpha$, we have that 
\[
  u(x) = \alpha x + T
\]
\newpage
\section*{2.}
We have the heat equation 
\[
  cpu_t = K_0 u_{xx} + K_0 = K_0 (u_{xx} + 1)
\]
and thus 
\[
  \frac{cp}{K}u_t = u_{xx} + 1
\]
Since $u_t = 0$, we have that $u_{xx} = -1$, therefore 
\[
  u_{x}(x) = -x + C_1
\]
and 
\[
  u(x) = -\displaystyle\frac{x^2}{2} + C_1 x + C_2
\]
Thus, we have the system of equations
\[
  \begin{cases}
    u(0) = C_2 = T_1 \\
    u(L) = \displaystyle\frac{-L^2}{2} + C_1 L + C_2 = T_2
  \end{cases}
\]
Hence, 
\[
  \begin{cases}
    C_1 = \displaystyle\frac{1}{L}\left(T_2 - T_1 + \displaystyle\frac{L^2}{2} \right)\\
    C_2 = T_1
  \end{cases}
\]
and we obtain the solution 
\[
  u(x) = -\displaystyle\frac{x^2}{2} + \displaystyle\frac{x}{L}\left(T_1 - T_1 + \displaystyle\frac{L^2}{2} \right) + T_1
\]
\newpage
\section*{3.}
We have that 
\[
  K_0 u_{xx} + x^2 K_0 = K_0(u_{xx} + x^2) = 0 \implies u_{xx} = -x^2
\]
Thus $u_{x}(x) = -\displaystyle\frac{x^3}{3} + C_1$ where $C_1 = -\displaystyle\frac{L^3}{3}$ as $u_x(L) = 0$. Therefore, we have 
\[
  u(x) = -\displaystyle\frac{x^4}{12}  - \displaystyle\frac{L^3}{3}x + \underbrace{T_2}_{C_2}
\]
as $u(0) = T_2$
\newpage
\section*{4.}
We have that 
\[
  u_{xx} = 0
\]
Therefore, since $u_x(L) = \alpha$, $u_x = \alpha$, and thus we get 
\[
  \alpha - [u(0)-T] = 0 \implies u(0) = \alpha + T
\]
and 
\[
  u(x) = \alpha x + C 
\]
where $u(0) = \alpha + T$ thus 
\[
  u(x) = \alpha x + \alpha + T
\]
\newpage
\section*{5.}
\subsection*{a.}
Assuming there is a equilibrium, all the environment variables must be constant therefore 
\[
  u_{xx} = -\frac{Q_0}{K_0}
\]
Thus 
\[
  u_x(x) = -\frac{Q_0}{K_0}x + C_1
\]
Therefore, $u_x(0) = C_1 = 0$ and $u_x(L) = \underbrace{-\frac{Q_0}{K_0}L}_{\ne 0} + C_1 = 0$ which is a contradiction. Therefore, there is no equilibrium when the environmen has constant physical properties and both sides of the rod are insulated. 
\subsection*{b.}
First, notice that 
\[
  \frac{d}{dt} \int_V e dV = \int_V \displaystyle\frac{\partial e}{\partial t} dV = \int_0^L \displaystyle\frac{\partial e}{\partial t} A dx
\]
where $A$ is the area of the rod at the end. \\
We know that 
\[
  \displaystyle\frac{\partial e}{\partial t} = \displaystyle\frac{\partial (c\rho u)}{\partial t} = K_0 \displaystyle\frac{\partial^2 u}{\partial x^2} + Q_0
\]
and therefore, 
\begin{align*}
  \displaystyle\frac{d}{dt} \int_V edV 
  &= \int_0^L \left(K_0 \displaystyle\frac{\partial^2 u }{\partial x^2} + Q_0 \right) A dx \\
  &= A \left. \left(K_0 \displaystyle\frac{\partial u}{\partial x}(x,t) + Q_0x \right) \right|_0^L\\
  &= AK_0 \left(u_x(L, t) - u_x(0, t) \right) + AQ_0L \\
  &= AQ_0 L
\end{align*}
as both ends are insulated thus $u_x(L, t) = u_x(0,t) = 0$.
Thus we get
\[
  \int_V edV = AQ_0L t + C_2
\]
Thus, 
\[
  \rho c A \int_0^L u(x,0) dx = AQ_0L0 + C_2 = C_2
\]
\newpage
\section*{6.}
\subsection*{a.}
\begin{align*}
  \displaystyle\frac{d}{dt}\int_V e dV 
  &= \int_0^L A \rho c u_t(x,t) dx \\
  &= \int_0^L A \rho c (u_{xx}(x,t) + x ) dx\\
  &= A \rho c \left. \left( u_x(x,t) + \displaystyle\frac{x^2}{2} \right) \right|_0^L \\
  &= A \rho c \left(7 - \beta + \displaystyle\frac{L^2}{2}\right) 
\end{align*}
Thus, 
\[
  \int_V edV = A\rho c \left(7 - \beta + \displaystyle\frac{L^2}{2} \right) t + C
\]
where 
\[
  C = \int_0^L A\rho c \underbrace{u(x,0)}_{f(x)} dx
\]
\subsection*{b.}
Since $u_t = 0$, $\int_0^L \displaystyle\frac{\partial u}{\partial t} dx = 0$ and thus 
\begin{align*}
  \frac{d}{dt} \int_0^L u(x,t) dx = \underbrace{\int_0^L \frac{\partial u}{\partial t} dx}
  _{0}
  &= \int_0^L \frac{\partial^2 u}{\partial x^2} + x dx \\
  &= \left. \frac{\partial u}{\partial x} \right|_0^L + \frac{L^2}{2} \\
  &= 7 - \beta + \frac{L^2}{2} \\  
\end{align*}
Therefore, $\beta = 7 + \frac{L^2}{2}$ and $\int_0^L u(x,t) dx$ is constant over time which means that  
\[
  \lim_{t \to \infty} \int_0^L u(x,t) dx = \int_0^L u(x,0) dx = \int_0^L f(x) dx  
\]
We also have that $u_{xx} = -x$ thus $u_x = -\frac{x^2}{2} + 7 + \frac{L^2}{2}$ and hence
\[
  u(x) = -\frac{x^3}{6} + 7x + \frac{L^2x}{2} + C
\]
where
\[    
  C = \frac{1}{L} \left(\frac{L^4}{24} - \frac{7L^2}{2} - \frac{L^4}{4} + \int_0^L f(x)dx \right) 
\]
as
\[
  \int_0^L f(x) dx = \int_0^L \left(-\frac{x^3}{6} + 7x + \frac{L^2x}{2} + C\right) dx
\]
\newpage
\section*{7.}
\subsection*{a.}
Since there is no source, 
\[
  \displaystyle\frac{d}{dt} \int_0^L e(x,t) A dx = A [\phi(0, t) - \phi(L, t)] = A[\alpha - \beta]
\]
\subsection*{b.}
Thus, we have that the total amount of chemical based on time is 
\[
  \int_0^L c \rho u(x,t) A dx = A(\alpha - \beta) t + C
\]
where $C = \int_0^L c\rho f(x) A dx$ after substituting $t = 0$ in the equation. 
\subsection*{c.}
If there is an equilibrium then the total amount of chemical must be a constant thus from part a, $\alpha = \beta$. 
\newpage
\section*{8.}
\begin{align*}
  & \nabla^2 u = 0 \\
  \implies & \iiint_\mathbb{R} \nabla \cdot \nabla u = 0 \\
  \implies & \oiint \nabla u \cdot \hat{n} dS = 0\\
  \implies & \oiint \underbrace{-K_0 \nabla u}_{\phi} \cdot \hat{n} dS = 0\\
\end{align*}
Thus there is no heat flow through any surface in the object. 
\newpage
\section*{9.}
Assuming the angle does not affect the temperature, that is $u_{\theta} = 0 $, thus 
\[
  \frac{\partial}{\partial r} \left(r \frac{\partial u}{\partial r}\right) = 0
\]
and hence 
\[
  \frac{\partial u}{\partial r} = \frac{C_1}{r} 
\]
\subsection*{a.}
\[
  u(r) = C_1 \ln(r) + C_2
\]
\[
  \begin{cases}
    C_1 \ln(r_1) + C_2 = T_1 \\
    C_1 \ln(r_2) + C_2 = T_2 
  \end{cases}
\]
Thus 
\[
  \begin{cases}
    C_1 = \frac{T_1 - T_2}{\ln(r_1/r_2)} \\
    C_2 = T_1 - C_1 \ln(r_1)
  \end{cases}
\]
and therefore, 
\[
  u(r) = T_1 + (T_2 - T_1) \displaystyle\frac{\ln \displaystyle\frac{r}{r_1}}{\ln \displaystyle\frac{r_2}{r_1}} 
\]
\subsection*{b.}
Following the same steps as part a, since the outer radius is insulated, we get that 
\[
  \displaystyle\frac{\partial u}{\partial r} = 0
\]
Thus 
\[
  u(r) = C_3 
\]
where $C_3 =T_1$ as $u(r_1) = T_1$ thus 
\[
  u(r) = T_1
\]
\newpage
\section*{10.}
Since there is no source, we have that 
\[
  0 = \nabla^2 u = \frac{1}{r} \frac{\partial}{\partial r} \left( r \frac{\partial u}{\partial r} \right) + \frac{1}{r^2} \frac{\partial^2 u}{\partial \theta^2}
\]
But it is axially symmetric, and therefore, 
\[
  \frac{1}{r} \frac{\partial}{\partial r} \left( r \frac{\partial u}{\partial r} \right) = 0
\]
and thus
\[
  r \displaystyle\frac{\partial u}{\partial r} = C_1
\]
for some constant $C_1$, and therefore
\[
  r_1 \displaystyle\frac{\partial u}{\partial r}(r_1) = 0 = C_1
\]
and thus $C_1 = 0$. \\
Therefore, 
\[
  u(r, \theta) = u(r) = C_2 = T_0
\]
as $u(r_0) = T_0$





\end{document}
