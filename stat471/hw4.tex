\documentclass[11pt]{article}
    \title{\textbf{Math 217 Homework I}}
    \author{Khac Nguyen Nguyen}
    \date{}

    \addtolength{\topmargin}{-3cm}
    \addtolength{\textheight}{3cm}

\usepackage{amsmath}
\usepackage{mathtools}
\usepackage{amsthm}
\usepackage{amssymb}
\usepackage{pgfplots}
\usepackage{xfrac}
\usepackage{hyperref}
\usepackage{graphicx}
\long\def\comment#1{}

\usepgfplotslibrary{polar}
\usepgflibrary{shapes.geometric}
\usetikzlibrary{calc}
\pgfplotsset{compat = newest}
\pgfplotsset{my style/.append style = {axis x line = middle, axis y line = middle, xlabel={$x$}, ylabel={$y$}, axis equal}}
\begin{document}
\section*{1.}
From the Classical Girsanov Theorem, we have that 
\[
    dA_t = -X_tA_tdB_t    
\]
with $A_0 = 1$ is the likelihood martingale that change the standard brownian motion into the Ornstein-Uhlenbeck process.
\newpage
\section*{2.}
We have that 
\[
    \pi(i) = 
    \begin{pmatrix}
        n \\
        i
    \end{pmatrix}
    p^i (1-p)^{n-i} 
    = \frac{n!}{i!(n-i)!}
    p^i (1-p)^{n-i}
\]
Also
\[
    \frac{\pi(i)}{\pi(0)} = \frac{a_{i-1}a_{i-2}\hdots a_0}{s_i s_{i-1} \hdots s_1} = \frac{n!}{i!(n-i)!}p^i(1-p)^{-i}     
\]
However, 
\[
    s_i s_{i-1} \hdots s_1 = i! (1-p)^i    
\]
Hence, 
\[
    a_{i-1}a_{i-2} \hdots a_0 = \frac{n!}{(n-i)!}p^i    
\]
Hence, we can speculate 
\[
    a_i = p (n-i) 
\]
Since $\pi(i) = 0$ for $i > n$, $a(i) = 0$ for $i>=n$.
\newpage
\section*{3.}
We have that 
\[
    S_{100} - 30 \to \sqrt{0.3 \cdot 0.7} \cdot 10 B^1
\]
Also
\[
    S_{200} - S_{100} - 30 \to \sqrt{0.3 \cdot 0.7} \cdot 10 (B^2 - B^1)
\]
We also have that 
Thus 
\[
    P((S_{100}-30)^2 + (S_{200} - S_{100} -30)^2) = P((B^1)^2 + (B^2-B^1)^2 > 20/21) 
\]
$B^1, B^2-B^1$ are the standard normal distributions. Hence, the sum of those squared is the exponential distribution with mean 2.
Hence, the probability is 
\[
    1-e^{-2\frac{20}{21}} = 1-e^{-40/21}     
\]

\newpage
\section*{4.}
\end{document}