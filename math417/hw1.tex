
\documentclass[11pt]{article}
    \title{\textbf{Math 217 Homework I}}
    \author{Khac Nguyen Nguyen}
    \date{}
    
    \addtolength{\topmargin}{-3cm}
    \addtolength{\textheight}{3cm}
    
\usepackage{amsmath}
\usepackage{mathtools}
\usepackage{amsthm}
\usepackage{amssymb}
\usepackage{pgfplots}
\usepackage{xfrac}
\usepackage{hyperref}



\newtheorem{definition}{Definition}[section]
\newtheoremstyle{mystyle}%                % Name
  {}%                                     % Space above
  {}%                                     % Space below
  {\itshape}%                                     % Body font
  {}%                                     % Indent amount
  {\bfseries}%                            % Theorem head font
  {}%                                    % Punctuation after theorem head
  { }%                                    % Space after theorem head, ' ', or \newline
  {\thmname{#1}\thmnumber{ #2}\thmnote{ (#3)}}%                                     % Theorem head spec (can be left empty, meaning `normal')

\theoremstyle{mystyle}
\newtheorem{theorem}{Theorem}[section]
\theoremstyle{definition}
\newtheorem*{exmp}{Example}
\begin{document}
\section*{1.}
\subsection*{1.}
If $x \in \liminf_{n\to \infty} E_n$ then, there exists $j \in \mathbb{N}$ such that $\forall k \ge j, x \in E_k$ which means that $x \in \cup_{k=j}^\infty E_k \subseteq \cup_{k=j-1}^\infty E_k \subseteq \hdots \subseteq \cup_{k=1}^\infty E_k$i thus $x \in \cap_{j=1}^\infty \cup_{k=j}^\infty E_k =: \limsup_{n \to \infty} E_n$.
\subsection*{2.}
$x \in \liminf_{n \to \infty}(E_n \cap F_n)$ 
$\iff$ there exists $j \in \mathbb{N}$ such that $\forall k \ge j, x \in E_k \cap F_k$ which is equivalent to $x \in E_k$ and $x \in F_k$ 
$\iff$ $x \in \liminf_{n\to \infty} E_n$ and $x \in \liminf_{n \to \infty} F_n$ 
$\iff$ $x \in \liminf_{n\to \infty} E_n \cap \liminf_{n \to \infty} F_n$ 
\subsection*{3.}
Base case: n=1
\[
  \cup_{j=1}^1 E_j := E_1 = E_1 \backslash \underbrace{\cup_{k=1}^0 E_k}_{\varnothing} := F_1 
\]
Inductive steps: if the equation holds for $n$ then it also holds for $n+1$.
\[
  \cup_{j=1}^{n+1} E_j = \cup_{j=1}^n E_j \cup E_{n+1} = \cup_{j=1}^n F_j \cup E_{n+1} \stackrel{\text{(1)}}{=} \cup_{j=1}^n F_j \cup \underbrace{E_{n+1} \backslash (\cup_{k=1}^n E_k)}_{F_{n+1}} = \cup_{j=1}^{n+1} F_j  
\]
as $\cup_{k=1}^n E_k = \cup_{k=1}^n F_k \implies E_{n+1} \backslash (\cup_{k=1}^n E_k) = F_{n+1}$.
\newpage
\section*{2.}
Assume $X$ is non-empty, then there exists $z_0 \in X$, thus fixing $z_0 = (x_0,y_0)$, we can define an equivalence class, $\xi_r$ as follows: 
$p$ is in the class $\xi_r$ if 
\[
  |p-x_0| = r
\]
where $r$ is some rational number so that the set containing all classes $\xi_r$ is countable. Next, define the function 
\[
  f_r: [0, 2\pi] \to \mathbb{R}^2, \indent t \to (x_0 + r\cos(t), x_0 + r\sin(t))
\]
such that $\xi_r \subseteq \text{Img}(f_r)$. 

WLOG, assume that $f_r(0) \in X$, then 
as the function \\ 
$g(\theta) =  |f_r(\theta) - f_r(0)|$ is a continuous function that strictly increasing on $[0, \pi]$ from $0$ to $2r$ and then strictly decreasing on $[\pi,2\pi]$ back to 0, there is a bijective function that maps $[0,\pi] \to [0, 2r]$ and $[\pi, 2\pi] \to [0,2r]$. Thus, the set of all points in each $\xi_r$ is countable because $\mathbb{Q}_{[0,2r]}$ is countable. And since the set containing all classes $\xi_r$ is also countable, $X$ is countable.
\newpage
\section*{3.}
\subsection*{1.}
Since $A \subseteq A \cup B$, $\text{card} (A) \le \text{card}(A \cup B)$. Now suppose $A$ is countable, then we can write $A$ and $B$ as $\{a_1, a_2, \hdots\}$ and $\{b_1, b_2, \hdots\}$. Therefore, it is possible to construct a bijection function between $A$ and $A\cup B$
\[
  \phi(a_i) = 
  \begin{cases}
    b_{i/2}, & \text{ if } i \text{ is even} \\ 
    a_{(i+1)/2}, & \text{ if } i \text{ is odd} \\ 
  \end{cases}
\]
and thus having the same cardinality. We can extend that to the case where $A$ is not countableas there is an infinite countable subset $\{a_1, a_2, \hdots \} = \tilde{A} \subset A$ and therefore we can construct a bijection between $A$ and $A \cup B$ based on the previous bijection.
\[
  \psi(a) = 
  \begin{cases}
    \phi(a), & \text{ if } a \in \tilde{A} \\
    a, & \text{ if } a \in A \backslash \tilde{A}
  \end{cases}
\]
where in the case $a \in \tilde{A}$, there is $a_i$ such that $a = a_i$
\[
  \phi(a) = \phi(a_i) = \begin{cases}
    b_{i/2}, & \text{ if } i \text{ is even} \\ 
    a_{(i+1)/2}, & \text{ if } i \text{ is odd} \\ 
  \end{cases}
\]
\subsection*{2.}
For every $x \in E$, there exists $\delta_x > 0 $ such that $(x-\delta_x, x) \cap B = \varnothing$, and since $\mathbb{Q}$ is dense, 
we can create a function $f$ that maps $x$ to a rational number in $(x-\delta_x, x)$. We claim that $f$ is injective. \\
For any $x_1 \ne x_2 \in E$, if $f(x_1) = f(x_2)$, then $f(x_1) \in (x_1-\delta_{x_1}, x_1)$ and $f(x_2) \in (x_2 - \delta_{x_2}, x_2)$. 
But we have $(x_1 - \delta_{x_1}, x_1) \cap (x_2 - \delta_{x_2}, x_2) = \varnothing$ else $B \ni x_1 \in (x_2 - \delta_{x_2}, x_2)$ or $B \ni x_2 \in (x_1 - \delta_{x_1}, x_1)$. Therefore, $f(x_1) \ne f(x_2)$ and hence a contradiction. 
Thus $f$ is injective.
\newpage
\section*{4.}
\subsection*{a.}
For any set $E_k = \{x \in E: x \ge \frac{1}{k} \}$, we can see that $\sum_{x \in E}x \ge \sum_{x \in E_k} x \ge \frac{j}{k}$ where $j$ is the number of elements in $E_k$. Thus as $\sum_{x \in E} x < \infty$, there is finite countable elements in $E_k$. We also have that 
\[
  E = \bigcup_{k \in N} E_k
\]
Thus $E$ is at most countable.
\subsection*{b.}
Let $E_k = \{x_i: i \le k\}$ and since every element in $E$ is positive, the series strictly increasing and $\lim_{n \to \infty} \sum_{i=1}^n x_i = \infty$ or $\lim_{n \to \infty} \sum_{i=1}^n x_i = L$ for some $L >0 \in \mathbb{R}$. 

If $\lim_{n \to \infty} \sum_{i=1}^n x_i = \infty$, then for every $M > 0$, there exists $k_0$ such that for every $k>k_0$, $\sum_{x \in E_k} x= \sum_{i=1}^k x_i > M$ thus $\sup_{F \in \mathcal{F}} s_F = \sum_{x \in E} x = \infty = \lim_{n \to \infty} \sum_{i=1}^n x_i$.

If $\lim_{n \to \infty} \sum_{i=1}^n x_i = L$, then we consider two cases 
\begin{itemize}
  \item If $\sup_{F \in \mathcal{F}} s_F > L$, then there is a subset $F$ such that $s_F \ge L$ and since $E$ is infinitely countable, we can let $F' = F \cup \{x_0\}$ so that $s_{F'} > L$. Since $F'$ is finite, and the mapping which we called $f$ is bijective. We can find the largest index $k = \max\{f^{-1}(x): x \in F' \}$, and thus $\lim_{n \to \infty} \sum_{i=1}^n x_i > \sum_{i=1}^k x_i > s_{F'} > L$ which is a contradiction.
  \item If $\sup_{F \in \mathcal{F}} s_F < L$, then there exists an $\epsilon >0$ so that for every $F \in \mathcal{F}$, $L - s_F > \epsilon$ but this contradicts with $\lim_{n\to \infty} \sum_{i=1}^n x_i = L$ as for every $\epsilon >0$ there exists $k_0$ such that for every $k>k_0$, $L - \underbrace{\sum_{i=1}^k x_i}_{s_{E_k}}< \epsilon$. 
    Therefore, $\sup_{F \in \mathcal{F}} s_F = L = \lim_{n \to \infty} \sum_{i=1}^n x_i$
\end{itemize}



\newpage
\section*{5.}
1. $\varnothing$ is countable. \\
2. If $E \subseteq S$, then either $E$ or $E^c$ is countable, therefore $E^c \subseteq S$. \\
3. If $E_k \subseteq S$ are all countable then $\bigcup_{n=1}^\infty E_n \in S$. If one or more of $E_k$ are not countable then $\bigcap_{n=1}^\infty E_n$ is countable thus $\bigcup_{n=1}^\infty E_n \in S$
Therefore, $S$ is a $\sigma$-algebra.
Every singleton is contained in $S$. 


\end{document}
