\documentclass[11pt]{article}
    \title{\textbf{Math 217 Homework I}}
    \author{Khac Nguyen Nguyen}
    \date{}
    
    \addtolength{\topmargin}{-3cm}
    \addtolength{\textheight}{3cm}
    
\usepackage{amsmath}
\usepackage{mathtools}
\usepackage{amsthm}
\usepackage{amssymb}
\usepackage{pgfplots}
\usepackage{xfrac}  
\usepackage{hyperref}
\usepackage{xcolor}
\definecolor{Mybackground}{RGB}{40,49,51}
\pagecolor{Mybackground}
\color{white}


\newtheorem{definition}{Definition}[section]
\newtheoremstyle{mystyle}%                % Name
  {}%                                     % Space above
  {}%                                     % Space below
  {\itshape}%                                     % Body font
  {}%                                     % Indent amount
  {\bfseries}%                            % Theorem head font
  {}%                                    % Punctuation after theorem head
  { }%                                    % Space after theorem head, ' ', or \newline
  {\thmname{#1}\thmnumber{ #2}\thmnote{ (#3)}}%                                     % Theorem head spec (can be left empty, meaning `normal')

\theoremstyle{mystyle}
\newtheorem{theorem}{Theorem}[section]
\theoremstyle{definition}
\newtheorem*{exmp}{Example}
\begin{document}
\section*{1.}
Since $f$ is measurable, then we know that for any open interval $(x,y), f^{-1}(x,y)$ is measurable, then for an arbitary $a \in \mathbb{R}$
\begin{itemize}
  \item If $a \ge 0$, $\{0 < f < 1/a \}$ is measurable thus $\{g > a \}$ is also measurable. 
  \item Now if $a < 0$, we have that \\
    $\{g > a\} = \{g > 0\} \cup \{g = 0\} \cup \{a < g < 0\}$, but we have 
    \[
      \{g = 0 \} = f^{-1}(\{0, \infty, -\infty\})
    \]
    and
    \[
      \{a < g < 0\} = \{a < 1/f < 0\} = \{f < 1/a\}
    \]
    are measurable.  
\end{itemize}
Therefore, 
\[
  \{g>a\}
\]
is measurable for all $a \in \mathbb{R}$. Thus $g$ is measurable.
\newpage
\section*{2.}
Suppose $m(F) = 0$ then we can find $n_0$ such that $\cup_{k = n_0}^\infty E_k < \infty$ thus WLOG we assume that $\cup_{k=1}^\infty E_k < \infty$.  
\[
  0 = m(\limsup_{n \to \infty} E_n) \ge \limsup_{n \to \infty} m(E_n)
\]
Therefore, 
\[
  \limsup_{n \to \infty} m(E_n) = \lim_{n \to \infty} \sup_{m \ge n} m(E_n) = 0
\]
and thus $m(E_n) \to 0$ as $n \to \infty$ and 
\[
  \lim_{n \to \infty} \chi_{E_n}(x) = 0
\]
a.e. $x \in \mathbb{R}^d$. \\
In the other direction, first let $G_n = \cup_{k=n}^\infty E_k$. \\
Suppose $m(F) > 0$ then if $m\left(\cup_{k=j}^\infty E_k \right) = \infty$ for all $j \in \mathbb{N}$ then obviously, $m(G_n) = \infty > a$ for all $a \in \mathbb{R}$. \\ 
Suppose $m(F) > 0$ and $m\left(\cup_{k=j}^\infty E_k \right) < \infty$ for some $j$ then 
\[
  \lim_{j \to \infty} m(\cup_{k=j}^\infty E_k) = m(F) > 0
\]
Thus there is $\varepsilon > 0$ and $n_0$ such that for all $n>n_0$, $m(\cup_{k=n}^\infty E_k) > \varepsilon$. \\
Therefore, in both cases there is some $\varepsilon > 0$ and $n_0$ such that for all $n>n_0$, 
\[
  m(G_n) > \varepsilon
\]
But for every $x \in G_n$, there is some $j \ge n$ such that $x \in E_j$ and thus $\chi_{E_j}(x) = 1$. However, if  
\[
  \lim_{n \to \infty} \chi_{E_n}(x) = 0 
\]
for all $x \in \mathbb{R}^d \backslash G$ where $m(G) = 0$ which means that $\{x: \exists n' > n, \chi_{E_n'}(x) \ne 0\} \to 0$ as $n \to \infty$, which is a contradiction. \\
\newpage
\section*{3.}
\subsection*{a.}
We know from notes 2 there is a nonmeasurable set $\mathcal{N} \subset [0,1]$. Define 
\[
  g: \mathbb{R} \to \mathbb{R}, \indent x \to
  \begin{cases}
    x, \indent \text{ if } x \in \mathcal{N} \\
    -x, \indent \text{ if } x \notin \mathcal{N}
  \end{cases}
\]
$g^{-1}(x)$ has at most 2 elements thus is measurable. But $\{g \ge 0\} \backslash (-\infty, 0] = \mathcal{N}$ is nonmeasurable.   
\subsection*{b.}
We first have that 
\[
  g^{-1}(a,\infty) = 
  \begin{cases}
    f'^{-1}(a,\infty), \indent \text{ if } a \ge 0 \\ 
    f'^{-1}(a,\infty) \cup \mathbb{R} \backslash B , \indent \text{ if } a < 0 \\
  \end{cases}
\]
Thus, we only need to prove that $f'$ is measurable as $\mathbb{R} \backslash B$ is measurable.\\
We have that 
\[
  f' = \lim_{h \to 0} \displaystyle\frac{f(x+h) - f(x)}{h}
\]
Let $g_h = \displaystyle\frac{f(x+h) - f(x)}{h}$, we can see that since $f(x+h)$ and $f(x)$ are both measurable, $g_h$ is measurable and thus $f'$ is measurable.
\newpage
\section*{4.}
Since $\mu$ is $\sigma$-finite, there is some $X_n \in \mathcal{M}$ such that $X_n \subseteq X_{n+1}$ and $\mu(X_n) < \infty$ for all $n \in \mathbb{N}$. \\ 
Thus for every $X_m$ and every $k \in \mathbb{N}$, we can apply the erogov's theorem on the set $X_m$ to get there is a subset $E_{m,k}$ such that $\mu(X_m \backslash E_{m,k}) < \varepsilon / 2^{mk} $ and $f_n \to f$ uniformly on $E_m$.\\
Now we have 
\begin{align*}
  \mu((\cup_{n,k=1}^\infty E_{n,k})^c) &= \mu(\cup_{n=1}^\infty X_n \backslash \cup_{n,k=1}^\infty E_{j,k}) \\
  &= \mu( \cup_{n=1}^\infty (X_n \backslash \cup_{k=1}^\infty E_{n,k})) \\
  &\le \sum_{n=1}^\infty \mu(X_n \backslash \cup_{k=1}^\infty E_{n,k})) \\
  &= \sum_{n=1}^\infty \mu (\cap_{k=1}^\infty (X_n \backslash E_{n,k})) \\ 
  &\le \sum_{n=1}^\infty \sum_{k=1}^\infty \mu(X_n \backslash E_{n,k}) \\
  &< \varepsilon \sum_{n=1}^\infty \displaystyle\frac{1}{2^n} \sum_{k=1}^\infty \displaystyle\frac{1}{2^k} \\
  = \varepsilon
\end{align*}
\newpage
\section*{5.}
First, note that $f(x) < \infty$ as $\int_X f < \infty$. Let $Y_n = (1/n, n)$ and $X_n = f^{-1}(Y_n \cup \{0\})$ so that 
\[
  \cup_{n=1}^\infty X_n = \cup_{n=1}^\infty f^{-1}(Y_n \cup \{0\}) = f^{-1}(\{0\} \cup \cup_{n=1}^\infty Y_n ) = f^{-1}([0,\infty)) = X 
\]  
\[
  Y_n \subset Y_{n+1} \implies X_n \subset X_{n+1}
\]
and 
\[
  \displaystyle\frac{1}{n} \cdot \mu(X_n \backslash f^{-1}(0)) < \int_{X_n \backslash f^{-1}(0))} f \le \int_X f  \le \infty \implies \mu(X_n \backslash f^{-1}(0)) < \infty 
\]
Then we can define the sequence of function 
\[
  f_n = f \cdot \chi_{X_n}
\]
that is
\begin{itemize}
  \item non-negative as $f, \chi_{X_n} > 0$ 
  \item $f_n(x) \uparrow f(x)$ for all $x \in X$ because of 
    \begin{enumerate}
      \item $f_n(x) \le f_{n+1}(x)$ for all $x \in X$ as $X_n \subset X_{n+1}$ 
      \item For all $x \in X$, $f(x) < \infty$, thus there exists $N \in \mathbb{N}$ such that $f(x) \in Y_N$ and thus $x \in X_N \subseteq X_{N+1} \subseteq \hdots $. Therefore, $f_n(x) \to f(x)$ as $n \to \infty$. 
    \end{enumerate}
\end{itemize}
Therefore, the monotone convergence theorem states that 
\[
  \lim_{n \to \infty} \int_{X_n} f dx = \lim_{n \to \infty} \int_X f \cdot \chi_{X_n} dx = \int_X f dx
\]
and noted that it is monotone increasing as well, therefore, for all $\varepsilon > 0$, we can find $F = X_{n_0}$ such that  
\[
  \int_X f - \int_F f < \varepsilon
\]
and thus let $E = F \backslash f^{-1}(0))$, we have $\mu(E) < \infty$ and 
\[
  \int_X f - \int_E f < \varepsilon
\]
as 
\[
  \int_E f = \int_F f
\]
\end{document}
