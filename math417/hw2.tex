\documentclass[11pt]{article}
    \title{\textbf{Math 217 Homework I}}
    \author{Khac Nguyen Nguyen}
    \date{}
    
    \addtolength{\topmargin}{-3cm}
    \addtolength{\textheight}{3cm}
    
\usepackage{amsmath}
\usepackage{mathtools}
\usepackage{amsthm}
\usepackage{amssymb}
\usepackage{pgfplots}
\usepackage{xfrac}
\usepackage{hyperref}



\newtheorem{definition}{Definition}[section]
\newtheoremstyle{mystyle}%                % Name
  {}%                                     % Space above
  {}%                                     % Space below
  {\itshape}%                                     % Body font
  {}%                                     % Indent amount
  {\bfseries}%                            % Theorem head font
  {}%                                    % Punctuation after theorem head
  { }%                                    % Space after theorem head, ' ', or \newline
  {\thmname{#1}\thmnumber{ #2}\thmnote{ (#3)}}%                                     % Theorem head spec (can be left empty, meaning `normal')

\theoremstyle{mystyle}
\newtheorem{theorem}{Theorem}[section]
\theoremstyle{definition}
\newtheorem*{exmp}{Example}
\begin{document}
\section*{1.}
\subsection*{1.}
For any $\varepsilon = (\varepsilon_n) \ne \tilde{\varepsilon} = (\tilde{\varepsilon}_n)$, there is a $N \in \mathbb{N}$ such that $\varepsilon_n = \tilde{\varepsilon}_n$ for all $n < N$ and $\varepsilon_N \ne \tilde{\varepsilon}_N$. WLOG, let $\varepsilon_N = 2$ and $\tilde{\varepsilon}_N = 0$. Thus we have 
\[
  g(\varepsilon) = \sum_{n=1}^\infty \frac{\varepsilon_n}{3^n} \ge \sum_{n=1}^{N-1} \frac{\varepsilon_n}{3^n} + \displaystyle\frac{2}{3^N}
\]
and 
\[
  g(\tilde{\varepsilon}) 
  = \sum_{n=1}^\infty \displaystyle\frac{\tilde{\varepsilon}_n}{3^n} 
  \le \sum_{n=1}^{N-1} \displaystyle\frac{\tilde{\varepsilon}_n}{3^n} + \sum_{n=N+1}^\infty \displaystyle\frac{2}{3^n}
  = \sum_{n=1}^{N-1} \displaystyle\frac{\varepsilon_n}{3^n} + \underbrace{\displaystyle\frac{2}{3^{N+1}} \cdot \frac{1}{1-1/3}}_{\frac{1}{3^N}}
\]
Thus $g(\varepsilon) > g(\tilde{\varepsilon})$ and therefore $g$ is injective.
\subsection*{2.}
Consider the function 
\[
  p: \{0,1\}^\mathbb{N} \to [0,1], \indent (\epsilon_n)_{n=1}^\infty \to \sum_{n=1}^\infty \displaystyle\frac{\epsilon_n}{2^n}  
\]
Since there is a natural bijection $h: \{0,1\}^\mathbb{N} \to \{0,2\}^\mathbb{N}$, $p = p \circ h^{-1} \circ h = g \circ h$ is injective. We claim that $p$ is also surjective. For every $x \in [0,1]$, there exists a sequence $\varepsilon =  (\varepsilon_n)_{n=1}^\infty$ such that $g(\varepsilon) = x$ 
\[
  \left|\sum_{n=1}^N \frac{\varepsilon_n}{2^n} - x \right| < \epsilon
\]
To construct the sequence $\varepsilon$, start from $n=0$, 
\begin{itemize}
  \item if $\sum_{i=1}^n + \displaystyle\frac{1}{2^{n+1}} < x$, then let $\varepsilon_{n+1} = 1$
  \item if $\sum_{i=1}^n + \displaystyle\frac{1}{2^{n+1}} > x$, then let $\varepsilon_{n+1} = 0$ 
  \item if $\sum_{i=1}^n + \displaystyle\frac{1}{2^{n+1}} = x$, then let $\varepsilon_{n+1} = 1$ and $\varepsilon_{i} = 0$ for all $i>n+1$ then stop the process
\end{itemize}
Increase $n$ by 1 and start the process again. \\
Since $\displaystyle\frac{1}{2^n} \to 0$ as $n \to \infty$ and $\sum_{i=1}^n \frac{1}{2^i} \le x$ for all $n \in \mathbb{N}$. Thus for every $\varepsilon >0$, there exists $n_0 \in N$ such that for every $n>n_0$, $0 \le x - \sum_{i=1}^n \frac{1}{2^i}< \epsilon$, and therefore $p(\varepsilon) = \sum_{n=1}^\infty \frac{1}{2^n} = x$. Thus, $p$ is bijective. 
Now consider the function 
\[
  k: (0,1) \to \mathbb{R}, \indent x \to \tan\left(2 x \pi - \pi \right)
\]
We have that $k$ is bijective thus $(0,1) \sim \mathbb{R}$. $(0,1) \sim [0,1]$ as the map
\[
  \phi(x) = 
  \begin{cases}
    \frac{1}{n+1} & \text{, if } x = \frac{1}{n} \text{ for some } n \in \mathbb{N} \\
    1 & \text{, if } x = 0 \\
    x & \text{, otherwise }
  \end{cases}
\]
is bijective and thus $\{0,1 \}^\mathbb{N} \sim \mathbb{R}$.  
\subsection*{3.}
$\{0,2\}^\mathbb{N} \sim \{0,1\}^\mathbb{N}$, thus $\{0,2\}^\mathbb{N} \sim \mathbb{R}$ and $\mathbb{R} \sim C$. 
\subsection*{4.}
Also, if $N_x$ is finite, we set $a_{n, x} = 0$ for all $n>N_x$ so that $\sum_{n=1}^\infty \frac{a_{n,x}}{3^n} = x$ regardless of $N_x$. \\   
For any $x \ne y$, that is $\sum_{n=1}^{N_x} \frac{a_{n,x}}{3^n} \ne  \sum_{n=1}^{N_y} \frac{a_{n,y}}{3^n}$, then as we know the function from subquestion 1 is injective, we have that $(a_{n,x}) \ne (a_{n,y})$, that is there exists $N \in \mathbb{N}$ such that for all $n<N, a_{n,x} = a_{n,y}$ and $a_{N,x} \ne a_{N,y}$. 
\begin{itemize}
  \item In case $N_x > N, N_y > N$, $a_{N,x} > a_{N,y} \implies x>y$ as \\
    \[
      \sum_{n=1}^{\infty} \frac{a_{n,x}}{3^n} - \frac{a_{n,y}}{3^n} 
      = \frac{2}{3^N} + \sum_{n=N+1}^\infty \frac{a_{n,x} - a_{n,y}}{3^n} \\
      \ge \frac{2}{3^N} - \underbrace{\sum_{n=N+1}^\infty \frac{2}{3^n}}_{1/3^N} 
      > 0
    \]
    and thus because of $a_{N,x} \ne a_{N,y}, x \ne y$ by assumption and WLOG, we have $a_{N,x} < a_{N,y} \iff x < y$. 
  \item In case $N_y > N, N_x \le N$ which is $N_x = N$ then $a_{N,x} = 1$. If $a_{N,y} = 2$ then obviously $x<y$, if $a_{N,y} = 0$ then since $x \ne y$, there is $n_0$ such that $a_{n_0,y} - a_{n_0,x} < 2$ and hence 
    \begin{align*}
       \sum_{n=1}^\infty \frac{a_{n,y} - a_{n,x}}{3^n} 
       &= -\frac{1}{3^N} + \sum_{n=N+1}^\infty \frac{a_{n,y}- a_{n,x}}{3^n} \\ 
       &= -\frac{1}{3^N} + \sum_{\substack{n=N+1 \\ n\ne n_0}}^\infty \frac{a_{n,y} - a_{n,x}}{3^n} + \frac{a_{n_0,y} - a_{n_0,x}}{3^{n_0}} \\
       &<   -\frac{1}{3^N} + \sum_{n=N+1}^\infty \frac{2}{3^n} \\ 
       &= 0
    \end{align*}
\end{itemize}
Finally, we can conclude that if $x<y$, then there exists $N \in \mathbb{N}$ such that for all $n<N$, $a_{n,x} = a_{n,y}$ and there is three cases 
\begin{itemize}
  \item $a_{N,x} = 0, a_{N,y} = 2$, thus $b_{N,x} = 0 , b_{N,y} = 1$ 
    \begin{align*}
      f(y) - f(x) 
      &= \sum_{n=1}^\infty \frac{b_{n,y} - b_{n,x}}{2^n} \\
      &= \frac{1}{2^N}+\sum_{n=N+1}^\infty \frac{b_{n,y} - b_{n,x}}{2^n} \\
      &\ge \frac{1}{2^N} + \sum_{n=N+1}^\infty \frac{-1}{2^n} \\
      &= \frac{1}{2^N} - \frac{1}{2^N} = 0  
    \end{align*}
  \item $a_{N,x} = 1, a_{N,y} = 2$, thus $b_{N,x} = b_{N,y} = 1$ 
    \begin{align*}
      f(y) - f(x) 
      &= \sum_{n=1}^\infty \frac{b_{n,y} - b_{n,x}}{2^n} \\
      &= \sum_{n=N+1}^\infty \frac{b_{n,y} - b_{n,x}}{2^n} \\
      &= \sum_{n=N+1}^\infty \frac{b_{n,y}}{2^n} \ge 0
    \end{align*}
  \item $a_{N,x} = 0, a_{N,y} = 1$, thus $b_{N,x} = 0, b_{N,y} = 1$. 
    \begin{align*}
      f(y) - f(x) 
      &= \sum_{n=1}^\infty \frac{b_{n,y} - b_{n,x}}{2^n} \\
      &= \frac{1}{2^N}+\sum_{n=N+1}^\infty \frac{- b_{n,x}}{2^n} \\
      &\ge \frac{1}{2^N} - \sum_{n=N+1}^\infty \frac{1}{2^n} \\
      &= 0
    \end{align*}
\end{itemize}
\subsection*{5.}
We know that $C \sim \{0,2\}^\mathbb{N} \sim \{0,1\}^\mathbb{N} \sim \mathbb{R}$ therefore $C \sim \mathbb{R}$. 
\newpage
\section*{2.}
For any pairwise disjoint $E_i \in \mathcal{M}$, let $A_n = \sqcup_{i=1}^n E_i$ then we have that 
\[
  \mu(\sqcup_{i=1}^\infty E_i) = \mu(\cup_{i=1}^\infty A_n) = \lim_{n \to \infty}\mu(A_n) = \lim_{n \to \infty} \mu(\sqcup_{i=1}^n E_i) = \sum_{i=1}^\infty \mu(E_i) 
\]
as $A_n \subseteq A_{n+1}$ for all $n \in \mathbb{N}$. 
\newpage
\section*{3.}
\subsection*{a.}
If $E_j \in \mathcal{M}$ for all $j \in \mathbb{N}$ then $E_1 \backslash E_n \in \mathcal{M}$ for all $n \in \mathbb{N}$ and thus $\cup_{n=2}^\infty (E_1 \backslash E_n) \in \mathcal{M}$, then $E_1 \backslash (\cup_{n=2}^\infty E_1 \backslash E_n) = E_1 \backslash (E_1 \backslash \cap_{n=2}^\infty E_n) =\\ E_1 \cap (\cap_{n=2}^\infty A_n) = \cap_{n=1}^\infty A_n \in \mathcal{M}$
\subsection*{b.}
We know that $\mathcal{M}$ being a $\sigma$-algebra implies that $X \in \mathcal{M}$. \\
If $X \in \mathcal{M}$, then for every $A \in \mathcal{M}$, $X \backslash A = A^c \in \mathcal{M}$. 
\subsection*{c.}
If $A \in \mathcal{S}$, then $A \in \mathcal{M}$ or $A^c \in \mathcal{M}$. In both case $A^c \in \mathcal{S}$ as $(A^c)^c \in \mathcal{M}$ or $A \in \mathcal{M}$. \\
If $A_j \in \mathcal{S}$ for all $j \in \mathbb{N}$ then $A_j \in \mathcal{M}$ or $A_j^c \in \mathcal{M}$. Let $A_{j_1}, A_{j_2}$ be subsequence of $A$ such that $A_{j_1} \in \mathcal{M}$ and $A_{j_2}^c \in \mathcal{M}$. Then we know that $P:= \cap_{j_1 = 1}^\infty A_{j_1} \in \mathcal{M}$ and since $Q := \cup_{j_2=1}^\infty A_{j_2}^c = (\cap_{j_2=1}^\infty A_{j_2})^c \in \mathcal{M}$ and thus $P\backslash Q = \cap_{j_1=1}^\infty A_{j_1} \cap \cap_{j_2=1}^\infty A_{j_2} = \cap_{j=1}^\infty A_j \in \mathcal{M} \subseteq \mathcal{S}$. 
\newpage
\section*{4.}
\subsection*{a.}
If $E \in \mathcal{M}$, then $E \cap X_\lambda \in \mathcal{M}_\lambda$ for all $\lambda \in \Lambda$, \\
then $X_\lambda \backslash (E \cap X_\lambda) = X\backslash (E \cap X_\lambda) = (X\backslash E)\cap X_\lambda  \in \mathcal{M}_\lambda$ and thus $E^c \in \mathcal{M}$. \\ 
If $E_i \in \mathcal{M}$ for $i \in \mathbb{N}$, then $E_i \cap X_\lambda \in \mathcal{M}_\lambda$, and thus $(\cup_{i=1}^\infty E_i )\cap X_\lambda \in \mathcal{M}_\lambda$ for all $\lambda \in \Lambda$ and thus $\cup_{i=1}^\infty E_i \in \mathcal{M}$. 
\subsection*{b.}
\[ 
  \mu(\varnothing) = \sum_{\lambda \in \Lambda} \mu_\lambda(\varnothing \cap X_\lambda) = \sum_{\lambda \in \Lambda} \underbrace{\mu_\lambda(\varnothing)}_{0} = 0 
\]
For any $E_j \in \mathcal{M}$ for all $j \in \mathbb{N}$ such that $E_j$ are pairwise disjoint, we have that 
\begin{align*}
  \mu(\sqcup_{j=1}^\infty E_j) 
  &= \sum_{\lambda \in \Lambda} u_\lambda (\sqcup_{j=1}^\infty E_j \cap X_\lambda) \\
  &= \sum_{\lambda \in \Lambda} \sum_{j=1}^\infty u_\lambda(E_j \cap X_\lambda) \\ 
  &= \sum_{j=1}^\infty \sum_{\lambda \in \Lambda} \mu_\lambda (E_j \cap X_\lambda) \\
  &= \sum_{j=1}^\infty \mu(E_j)
\end{align*}
\subsection*{c.}
If $\mu$ is $\sigma$-finite, then there exists $X_n \subseteq X_{n+1} \in \mathcal{M}$ such that $\cup_{n=1}^\infty X_n = X$ and $\mu(X_n) < \infty$ for all $n \in \mathbb{N}$. Thus if we let $X_{n, \lambda} = X_n \cap X_\lambda$, we have that $X_{n, \lambda} \subseteq X_{n+1, \lambda}$, $X_{n, \lambda} \in \mathcal{M}_\lambda$,  
\[
  \mu(X_n \cap X_\lambda) = \mu(X_{n, \lambda}) < \infty 
\]
and 
\[
  X_\lambda = X \cap X_\lambda = \left(\cup_{n=1}^\infty X_n \right) \cap X_\lambda = \cup_{n=1}^\infty (X_n \cap X_\lambda) = \cup_{n=1}^\infty X_{n, \lambda} 
\]
for all $n \in \mathbb{N}$, which means that all but a countable measure $\mu_\lambda$ have $\mu_\lambda(X_\lambda) = 0$ and the rest are $\sigma$-finite. \\
Now suppose all but a countable measure $\mu_\lambda$ have $\mu_\lambda(X_\lambda) = 0$ and the rest are $\sigma$-finite, then for every $\lambda \in \Lambda$, there exists $X_{n, \lambda} \in \mathcal{M}_\lambda$ such that $X_{n, \lambda} \subseteq X_{n+1, \lambda}$, $\cup_{n=1}^\infty X_{n, \lambda} = X_\lambda$ and $\mu_\lambda(X_{n, \lambda}) < \infty$ for every $n \in \mathbb{N}$. Since $\Lambda$ is a collection of measure, there is a bijection $\mathbb{N} \sim \Lambda$   
\begin{itemize}
  \item $X = \cup_{\lambda \in \Lambda} X_\lambda = \cup_{\lambda \in \Lambda} \cup_{n=1}^\infty X_{n, \lambda} = \cup_{\substack{n=1 \\}} $  
\end{itemize}
\newpage
\section*{5.}
We have that the definition of the outer measure for both parts a and b
\[
  \mu^*(A) := \inf \left\{\sum_{j=1}^\infty \rho(E_j): A \subseteq \cup_{j=1}^\infty E_j, E_j \in \mathcal{S} \right\}
\]
\subsection*{a.}
Then for any nonempty set $A \subseteq \mathcal{S}$, we have that if $\mu^*(A) = 0$ then $\sum_{j=1}^\infty \rho(E_j) = 0$ and thus $\rho(E_j) = 0$ for every $j \in \mathbb{N}$ and thus $E_j = \varnothing$ and $\cup_{j=1}^\infty E_j = \varnothing$. Therefore, $A = \varnothing$ and thus a contradiction. Therefore, $\mu^*(\varnothing) = 0$ and $ \mu^*(A) > 0$. But since $\sum_{j=1}^\infty \rho(E_j)$ is either an integer or infinity, $A \subseteq X \cup (\cup_{j=2}^\infty \varnothing)$ and 
\[
  \rho(X) + \sum_{j=2}^\infty \rho(\varnothing) = 1
\]
we have that $\mu^*(A) = 1$ and $\mu^*(\varnothing) = 0$. 
\subsection*{b.}
From definition, we have that for any set $A$ such that $\mu^*(A) \ge \rho(A)$. \\ 
If $\rho(A) = N$ for some $N \in \mathbb{N}$ or $N = \infty$ , then we can let 
\[
  K = \{k \in A: \text{k is an integer} \}
\]
Then $A \subseteq \cup_{j=0}^\infty E_j$ where 
\[
  E_j = 
  \begin{cases}
    \left(\frac{j}{2},\frac{j}{2} + 1\right), \indent &\text{ if } 2|j \text{ and } \frac{j}{2} \notin K\\
    \left(-\frac{j-1}{2} -1,-\frac{j-1}{2}/2\right), \indent &\text{ if } 2\nmid j \text{ and } -\frac{j-1}{2} \notin K \\
    \left[\frac{j}{2},\frac{j}{2} + 1\right), \indent &\text{ if } 2|j \text{ and } \frac{j}{2} \in K \\
    \left(-\frac{j-1}{2} -1,-\frac{j-1}{2}/2\right], \indent &\text{ if } 2\nmid j \text{ and } -\frac{j-1}{2} \in K \\
  \end{cases}
\]
so that 
\[
  \rho(E_j) = 
  \begin{cases}
    1, &\text{ if } \frac{j}{2} \in K \text{ or } -\frac{j-1}{2} \in K \\
    0, &\text{ otherwise}
  \end{cases}
\]
and thus there is $N$ interval $E_j$ such that $\rho(E_j) = 1$. Therefore,
\begin{align*}  
  \mu^*(A) 
  &\le \sum_{j=0}^\infty \rho(E_j) \\
  &=  \sum_{\substack{j=0 \\ j/2 \in K \text{ or } \\ -(j-1)/2 \in K}}^\infty \rho(E_j) 
  +  \sum_{\substack{j=0 \\ j/2 \notin K \text{ and } \\ -(j-1)/2 \notin K}}^\infty \rho(E_j) \\
  &= N + 0 = N = \rho(A) 
\end{align*}
which concludes that $\mu^*(A) = \rho(A)$. 
\end{document}
