 
\documentclass[11pt]{article}
    \title{\textbf{Math 217 Homework I}}
    \author{Khac Nguyen Nguyen}
    \date{}
    
    \addtolength{\topmargin}{-3cm}
    \addtolength{\textheight}{3cm}
    
\usepackage{amsmath}
\usepackage{mathtools}
\usepackage{amsthm}
\usepackage{amssymb}
\usepackage{pgfplots}
\usepackage{xfrac}
\usepackage{hyperref}



\newtheorem{definition}{Definition}[section]
\newtheoremstyle{mystyle}%                % Name
  {}%                                     % Space above
  {}%                                     % Space below
  {\itshape}%                                     % Body font
  {}%                                     % Indent amount
  {\bfseries}%                            % Theorem head font
  {}%                                    % Punctuation after theorem head
  { }%                                    % Space after theorem head, ' ', or \newline
  {\thmname{#1}\thmnumber{ #2}\thmnote{ (#3)}}%                                     % Theorem head spec (can be left empty, meaning `normal')

\theoremstyle{mystyle}
\newtheorem{theorem}{Theorem}[section]
\theoremstyle{definition}
\newtheorem*{exmp}{Example}


\begin{document}
\section{Preliminary}
\subsection{Basic on sets}
\subsection{Countable sets}
\begin{itemize}
  \item Bernstein's theorem: if $\text{card}(X) \le \text{card}(Y)$ and $\text{card}(Y) \le \text{card}(X)$ then $\text{card}(X) = \text{card}(Y)$
  \item \text{card}(\mathcal{P}(\mathbb{N})) = \text{card}(\mathbb{R})
\end{itemize}
\subsection{Properties of the real line \textorpdfstring{$\mathbb{R}$}}
\begin{definition}
  The set of extended real $\overline{\mathbb{R}}:=  \mathbb{R} \cup \{-\infty, \infty\}$. \\
  For $x \in \mathbb{R}, x \pm \infty = \pm \infty$ and 
  \begin{displaymath}
    x \cdot \infty = 
    \begin{cases}
      \infty, & \text{if } x > 0 \\ 
      -\infty, & \text{if } x < 0 \\
      0, & \text{if } x = 0
    \end{cases}
    But $\infty - \infty$ is undefined.
  \end{displaymath} 
\end{definition}
\begin{theorem}[Representation of open sets in $\mathbb{R}$]
  Every nonempty open set $\mathcal{O}$ in $\mathbb{R}$ can be written as at most countable union of pairwise disjoin open intervals. That is $\mathcal{O} = \sqcup_{j=1}^\infty (a_j, b_j)$ such that $(a_i, b_i) \cap (a_j, b_j) = \varnothing$ for all $i \ne j$ (some of the intervals may be empty. If such are ignored, the representation is unique).
\end{theorem}
\begin{exmp}
  Let $f: \mathbb{R} \to \mathbb{R}$ be an increasing function. Let $D$ denote the set of all points $x \in \mathbb{R}$ such that $f$ is not continuous at $x$, that is $D = \{x \in \mathbb{R}: f(x-) \ne f(x+)}$. Then $D$ is countable.
\end{exmp}


\end{document}
