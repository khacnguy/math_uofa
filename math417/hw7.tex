\documentclass[11pt]{article}
    \title{\textbf{Math 217 Homework I}}
    \author{Khac Nguyen Nguyen}
    \date{}
    
    \addtolength{\topmargin}{-3cm}
    \addtolength{\textheight}{3cm}
    
\usepackage{amsmath}
\usepackage{mathtools}
\usepackage{amsthm}
\usepackage{amssymb}
\usepackage{pgfplots}
\usepackage{xfrac}  
\usepackage{hyperref}
\usepackage{xcolor}
\definecolor{Mybackground}{RGB}{40,49,51}
\pagecolor{Mybackground}
\color{white}


\newtheorem{definition}{Definition}[section]
\newtheoremstyle{mystyle}%                % Name
  {}%                                     % Space above
  {}%                                     % Space below
  {\itshape}%                                     % Body font
  {}%                                     % Indent amount
  {\bfseries}%                            % Theorem head font
  {}%                                    % Punctuation after theorem head
  { }%                                    % Space after theorem head, ' ', or \newline
  {\thmname{#1}\thmnumber{ #2}\thmnote{ (#3)}}%                                     % Theorem head spec (can be left empty, meaning `normal')

\theoremstyle{mystyle}
\newtheorem{theorem}{Theorem}[section]
\theoremstyle{definition}
\newtheorem*{exmp}{Example}
\begin{document}
\section*{1.}
\subsection*{a.}
Suppose $m(F) = 0$, then for any $x \notin F$, $x \notin E_n$ for finitely many $n \in \mathbb{N}$, thus 
\[
  \lim_{n \to \infty} \chi_{E_n} (x) = 0
\]
a.e. $x \in \mathbb{R}^d$. \\
Now suppose $m(F) > 0$, then let $x \in F$, thus $x \in E_n$ for infinitely many $n \in \mathbb{N}$, which means $\limsup_{n \to \infty} \chi_{E_n} (x) = 1$ for all $x \in F$. Thus 
\[
  \lim_{n \to \infty} \chi_{E_n} (x) = 0
\]
for some set $X \subseteq F^c$ thus contradiction as $m(F) > 0$.  
\subsection*{b.}
Apply fatou's lemma, we have that 
\[
  \int_{\mathbb{R}^d} \liminf_{n \to \infty} f \chi_{E_n} dm = 0
\]
which means that 
\[
  m(f \liminf \chi_{E_n} \ne 0) = 0
\]
hence
\[
  m(\liminf \chi_{E_n} \ne 0) = 0
\]
Therefore, $\liminf \chi_{E_n}(x) = 1$ on a set $X$ where $m(X) = 0$. But for every $x \in G$, $\liminf \chi_{E_n}(x) = 1$ thus $m(G) = 0$. 
\newpage
\section*{2.}
\subsection*{a.}
We have that $\displaystyle\frac{x}{n} \ge \sin\left( \displaystyle\frac{x}{n}\right)$ thus $x > n\sin\left(\displaystyle\frac{x}{n}\right)$, and let $t = x^2$, we have 
\[
  \int_0^\infty \displaystyle\frac{x}{x^4+1} = \int_0^\infty \displaystyle\frac{1}{2(t^2+1)}dt = \displaystyle\frac{\pi}{4} 
\]
Thus the solution is 
\[
  \int_0^\infty \displaystyle\frac{\lim_{n \to \infty} x \cdot n/x \sin(x/n)}{x^4+1} = \int_0^\infty \displaystyle\frac{x}{x^4+1} = \displaystyle\frac{\pi}{4}
\]
\subsection*{b.}
Since $n^2(1-\cos(x/n)) \le n^2(1 - \cos^2(x/n)) = n^2 \sin^2(x/n) \le x^2$ and let $t = x^3$, 
\[
  \int_{-\infty}^\infty \displaystyle\frac{x^2}{x^6 + 1} dx = \int_{-\infty}^\infty \displaystyle\frac{1}{3(t^2+1)} dt = \displaystyle\frac{\pi}{3}
\]
We have the solution through L'Hopital and let $y = x^2$ 
\[
  \int_{-\infty}^\infty \lim_{n \to \infty} \displaystyle\frac{n^2(1-\cos(x/n))}{1+x^6} = \int_{-\infty}^\infty \displaystyle\frac{x^2/2}{x^6 + 1} dx = \displaystyle\frac{\pi}{6}
\]
as 
\[
  \lim_{n \to \infty} n^2(1-\cos(x/n)) = \lim_{n \to \infty} \displaystyle\frac{-x\sin(x/n)/n^2}{-2/n^3} = \displaystyle\frac{x}{2} x \lim_{n \to \infty} \displaystyle\frac{sin(x/n)}{x/n} = \displaystyle\frac{x^2}{2} 
\]
\newpage
\section*{3.}
If $\lim_{n \to \infty} \int_E |f_n - f| = 0$ then for every $\varepsilon > 0$ there is $n_0$ such that for all $n > n_0$,  
\[
  \left|\int_E |f_n| - \int_E |f|\right| \le \left|\int_E (|f_n| - |f|) \right| \le  \int_E |f_n - f| \le \varepsilon
\]
Thus $\int_E |f_n| \to \int_E |f|$. \\
Now suppose $\int_E |f_n| \to \int_E |f|$, then we know that 
\[ 
  |f_n| + |f| \to 2|f|
\] 
a.e. $x \in E$  
\[
  \int_E |f_n - f| \le \int_E |f_n| + |f|
\] 
\[
  \int_E |f_n - f| = 0
\]
as $f_n \to f$ a.e. $x \in E$ and 
\[
  \lim_{n \to \infty} \int_E |f_n| + |f| = \int_E 2|f|
\]
Thus applying the Generalized Dominance Convergence Theorem on $|f_n - f|$ and $|f_n|+|f|$, we have that 
\[
  \lim_{n \to \infty} \int_E |f_n - f| = \int_E 0 = 0
\]
\newpage
\section*{4.}
\subsection*{a.}
For all $\varepsilon >0 $ we can find a uniformly continuous function $g$ such that $\int_{\mathbb{R}^d} |f-g| < \varepsilon/3$ and small enough $t>0$ such that $|g(x-t) - g(x)| < \varepsilon / 3m(E)$ for all $x \in \mathbb{R}^d$. Then 
\begin{align*}
  & \int_{\mathbb{R}^d} |f_t(x) - f(x)| \\
  \le& \int_{\mathbb{R}^d} |f(x-t) - g(x-t)| + |g(x-t) - g(x)| + |g(x) - f(x)| \\
  <& \displaystyle\frac{\varepsilon}{3} + \displaystyle\frac{\varepsilon}{3} + \displaystyle\frac{\varepsilon}{3} \\
  =& \varepsilon
\end{align*}
Thus 
\[
  \int_{\mathbb{R}^d} |f_t(x) - f(x)| = 0
\]
\subsection*{b.}
Since $\chi_E \in \mathcal{L}_1(\mathbb{R}^d)$, for all $\varepsilon > 0$, there is a uniformly continuous function $h$ such that $\int_{\mathbb{R}^d} |\chi_E - h| < \varepsilon / 3$, then let the sequence $x_n \to x$ and thus there is an $n_0$ such that for all $n > n_0$, $|h(x_n) - h(x)| < \varepsilon/3m(E)$. Then
\begin{align*}
  & |\phi(x) - \phi(x_n)| \\
  =& \left| \int_{\mathbb{R}^d} \chi_E(x+t) \chi_E(t) - \chi_E(x_n+t) \chi_E(t) dt \right|\\
  =& \left|\int_{\mathbb{R}^d} \chi_E(t)\left( \chi_E(x+t) - \chi_E(x_n+t)  \right) dt \right| \\
  \le & \int_{\mathbb{R}^d} |\chi_E(x+t) - \chi_E(x_n+t) | dt \\
  \le & \int_{\mathbb{R}^d} |\chi_E(x+t) - h(x+t)| + |h(x+t) - h(x_n+t)| + |h(x_n+t) - \chi_E (x_n+t) | \\
  < & \displaystyle\frac{\varepsilon}{3} + \displaystyle\frac{\varepsilon}{3} + \displaystyle\frac{\varepsilon}{3} \\
  = & \varepsilon
\end{align*}
\subsection*{c.}
We first have that for $x \in E$,  
\[
  \phi(x) = \int_{\mathbb{R}^d} \chi_E(x+t) \chi_E(t) dt = m(E \cap (E - x)) = m(E_x)
\]
where $E_x = \{y: y \in E, x+y \in E\}$. \\
Notice that if $y \in E_x$ then $y \in E$ and $x+y\in E$ thus $x \in E-E$. Thus if $m(E_x) > 0$ then $x \in E - E$. \\
Now since $m(E)> 0$, we have that there is $B_\varepsilon(x_0) \subseteq E$ and thus for any $ \delta < \varepsilon/2$, we have that $\phi(x) = m(E_x) > 0$ for all $x \in B_{\delta/2}(0)$. Thus $B_\delta(0) \subseteq E-E$
\newpage
\section*{5.}
For every $\varepsilon > 0$, we can find a respective integrable step function $\phi$ such that $\int_\mathbb{R} |f - \phi| < \varepsilon/2$, where 
\[
  \phi = \sum_{k=1}^N a_k \chi_{R_k}
\]
where $a_k \in \mathbb{R}$ and $R_k$ are bounded intervals. Thus we can find an interval $R$ such that $\cup_{k=1}^N R_k \subseteq R$. Let $M = \max_R \phi$  \\
Now for any $x \in R$, since $\sin(\lambda x) \to 0$ as $\lambda \to \infty$, there is $\delta > 0$ such that for all $\lambda < \delta$ , 
\[
  |\sin(\lambda x)| < \displaystyle\frac{m(R) \varepsilon}{2M}
\]
Then 
\begin{align*}
  & \left|\int_\mathbb{R} f(x) \sin(\lambda x) dx \right| \\ 
  \le & \int_\mathbb{R} |f(x) - \phi(x)| |\sin(\lambda x)| dx + \int_\mathbb{R} |\phi(x) \sin(\lambda x)| dx \\
  \le & \int_\mathbb{R} |f(x) - \phi(x)| dx + \int_R |\phi(x)| |\sin(\lambda x)| dx \\
  \le &  \displaystyle\frac{\varepsilon}{2} + \int_R \displaystyle\frac{m(R) \varepsilon}{2M} \cdot M dx \\
  \le & \displaystyle\frac{\varepsilon}{2} + \displaystyle\frac{\varepsilon}{2} \\
  = \varepsilon
\end{align*}
\end{document}
