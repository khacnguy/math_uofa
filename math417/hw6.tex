
\documentclass[11pt]{article}
    \title{\textbf{Math 217 Homework I}}
    \author{Khac Nguyen Nguyen}
    \date{}
    
    \addtolength{\topmargin}{-3cm}
    \addtolength{\textheight}{3cm}
    
\usepackage{amsmath}
\usepackage{mathtools}
\usepackage{amsthm}
\usepackage{amssymb}
\usepackage{pgfplots}
\usepackage{xfrac}  
\usepackage{hyperref}
\usepackage{xcolor}
\definecolor{Mybackground}{RGB}{40,49,51}
\pagecolor{Mybackground}
\color{white}


\newtheorem{definition}{Definition}[section]
\newtheoremstyle{mystyle}%                % Name
  {}%                                     % Space above
  {}%                                     % Space below
  {\itshape}%                                     % Body font
  {}%                                     % Indent amount
  {\bfseries}%                            % Theorem head font
  {}%                                    % Punctuation after theorem head
  { }%                                    % Space after theorem head, ' ', or \newline
  {\thmname{#1}\thmnumber{ #2}\thmnote{ (#3)}}%                                     % Theorem head spec (can be left empty, meaning `normal')

\theoremstyle{mystyle}
\newtheorem{theorem}{Theorem}[section]
\theoremstyle{definition}
\newtheorem*{exmp}{Example}
\begin{document}
\section*{1.}
\subsection*{b.}
Since $f_n(x) \uparrow f(x)$ for all $x \in X$, we have that 
\[
  \int_X f = \int_X \liminf_{n \to \infty} f_n \le \liminf_{n \to \infty} \int_X f_n 
\]
But we also know that $\sup_{n \ge k} \int_X f_n \le \int_X f$ for all $n \in \mathbb{N}$, thus 
\[
  \int_X f \ge \lim_{n \to \infty} \sup_{n \ge k} \int_X f_n 
\] 
Thus we have that 
\[
  \limsup_{n \to \infty} \int_X f_n = \liminf_{n \to \infty} \int_X f_n = \lim_{n \to \infty} \int_X f_n = \int_X f
\]
\subsection*{b.}
Define a sequence of function
\[
  f_n(x) = f(x) \cdot \chi_{x \le n}
\]
Thus $f_n(x) \le f_{n+1}(x)$ for all $n \in \mathbb{N}$ and is nonnegative as $f$ is nonnegative. Then from part a, we know that 
\[
  \int_N f d\mu = \lim_{n\to \infty} \int_N f_n = \lim_{n \to \infty} \int_{\{1,2,\hdots, n\}} f_n d\mu = \lim_{n \to \infty} \sum_{i=1}^n f(i) 
\]
\newpage
\section*{2.}
For any measurable subset $E$, we have that 
\[
  \int_E f = \int_E \liminf_{n \to \infty} f_n \le \liminf_{n \to \infty} \int_E f_n \le \limsup_{n \to \infty} \int_E f_n
\]
We also have that
\begin{align*}
  \int_E f &= \int_X f - \int_{E^c} f \\
  &= \int_X f - \int_{E^c} \liminf f_n \\
  &\ge \int_X f - \liminf \int_{E^c} f_n \\
  &= \int_X f + \limsup \int_{E^c} -f_n \\
  &= \limsup \left( \int_X f - \int_{E^c} f \right) \\ 
  &=\limsup \int_E f 
\end{align*}
Thus 
\[
  \int_E f = \limsup_{n \to \infty} \int_E f_n = \liminf_{n \to \infty} \int_E f_n = \lim_{n \to \infty} \int_E f_n
\]
\newpage
\section*{3.}
\subsection*{a.}
Let $\varphi = \sum_{j=0}^n c_j \chi_{E_j}$, where $c_0 = 0$ and $E_j$ are pairwise disjoint and $\cup_{j=0}^n E_j = X$.  
\[
  \int_X \varphi d\nu =\sum_{j=0}^n \int_{E_j} c_j f d\mu = \int_{X}  d\mu = \int_X \sum_{j=0}^n c_j \chi_{E_j} f d\mu = \int_X \varphi f d\mu
\]
\subsection*{b.}
Since $X$ is a nonnegative measurable function, there is a sequence of nonnegative simple function $\phi_n$ such that $\phi_n \uparrow g$ for all $x \in X$. Then 
\[
  \int_X \phi_n d\nu = \int_X \phi_n f d\mu
\]
and 
\[
  \lim_{n \to \infty} \int_X \phi_n d\nu = \lim_{n \to \infty} \int_X \phi_n f d\mu
\]
Since $\phi_n \uparrow g$ andthus $\phi_n f \uparrow gf$, we have that 
\[
  \int_X g d\nu = \int_X gf d\mu
\]
\newpage
\section*{4.}
Definition: If $f_n \to f$ in measure then for an arbitary $\varepsilon > 0$,  
\[
  \lim_{n \to \infty} \mu(\underbrace{\{x \in X: |f_n(x) - f(x)| \ge \varepsilon\}}_{X_n} = 0
\]
Therefore, for all $\delta > 0$, we can find $n_0$ such that for all $n>n_0$, $\mu(X_n) < \delta/2$ and $|f_n-f| < \varepsilon$ for all $x \in X \backslash X_n$.  
\begin{align*} 
  &\lim_{n \to \infty} \rho(f_n, f) \\ 
  =& \lim_{n \to \infty} \int_X \displaystyle\frac{|f_n-f|}{|f_n - f| + 1} d\mu \\
  \le& \lim_{n \to \infty} \int_{X_n} d\mu + \int_{X\backslash X_n} \displaystyle\frac{\varepsilon}{\varepsilon + 1} d\mu \\
  \le& \displaystyle\frac{\delta}{2} + \mu(X) \cdot \displaystyle\frac{\varepsilon}{\varepsilon + 1}  
\end{align*}
Since $\varepsilon$ is arbitary, choose $\varepsilon = \displaystyle\frac{\delta}{2\mu(X)-\delta}$ so that 
\[
  \lim_{n \to \infty} \rho(f_n, f) \le \displaystyle\frac{\delta}{2} + \mu(X) \cdot \displaystyle\frac{\delta}{\delta + 2\mu(X) - \delta} > \delta
\]
If $f_n \to f$ in measure is false then there is some $\varepsilon, \delta > 0$ such that for all $n_0 > 0$, there is $n> n_0$ such that 
\[ 
  \mu(\underbrace{\{x \in X: |f_n(x) - f(x)| \ge \varepsilon\}}_{X_n} > \delta
\]
and therefore
\begin{align*}
  &\rho(f_n, f) \\
  =& \int_X \displaystyle\frac{|f_n-f|}{|f_n - f| + 1} d\mu \\
  \ge& \int_{X_n} \displaystyle\frac{|f_n-f|}{|f_n-f|+ 1} d\mu \\
  > & \delta \displaystyle\frac{\varepsilon}{\varepsilon + 1} \\
\end{align*} 
since $|f_n - f| \ge \varepsilon$ and $\displaystyle\frac{x}{1+x} = 1 - \displaystyle\frac{1}{1+x}$ is an increasing function. Thus 
\[
  \lim_{n \to \infty} \rho(f_n, f) \ge \displaystyle\frac{\delta \varepsilon}{\varepsilon+1} > 0
\]
for some $\varepsilon, \delta > 0$, thus is a contradiction. 
\newpage
\section*{5.}
Let $F_1 = \{x: f(x) \ge 1\}$ and thus $F_2 = \{x: 0< f(x) < 1\}$. Therefore, we have that $f(x)^{1/n}$ monotonely increasing converges to 1 for $x \in F_2$ and monotonely decreasing converges to 1 for $x \in F_1$. Thus we can apply the monotone converging theorem and get 
\begin{align*}
  &\lim_{n \to \infty} \int_{F_2} f(x)^{1/n} dx \\
  &= \lim_{n \to \infty} \int_{F_2} f(x)^{1/n} dx \\
  &= \int_{F_2} \lim_{n \to \infty} f(x)^{1/n} dx \\
  &= \int_{F_2} dx \\
  &= \mu(F_2)
\end{align*}
Let the function $g$ be 
\[
  g: F_1 \to \mathbb{R}, \indent x \to f(x)
\]
so that $g \ge |f(x)^{1/n}|$ for all $x \in F_1$, therefore, 
\begin{align*}
  &\lim_{n \to \infty} \int_{F_1} f(x)^{1/n} dx \\
  &= \lim_{n \to \infty} \int_{F_1} f(x)^{1/n} dx \\
  &= \int_{F_1} \lim_{n \to \infty} f(x)^{1/n} dx \\
  &= \int_{F_1} dx \\
  &= \mu(F_1)
\end{align*}
Thus 
\[
  \lim_{n \to \infty} \int_E f(x)^{1/n} dx = \lim_{n \to \infty} \int_{F_1 \sqcup F_2 \sqcup E_0} f(x)^{1/n} dx = \lim_{n \to \infty} \int_{F_1 \sqcup F_2} f(x)^{1/n} dx = \mu(F_1) + \mu(F_2) = \mu(E \backslash E_0)
\]
\end{document}
