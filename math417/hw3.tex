
\documentclass[11pt]{article}
    \title{\textbf{Math 217 Homework I}}
    \author{Khac Nguyen Nguyen}
    \date{}
    
    \addtolength{\topmargin}{-3cm}
    \addtolength{\textheight}{3cm}
    
\usepackage{amsmath}
\usepackage{mathtools}
\usepackage{amsthm}
\usepackage{amssymb}
\usepackage{pgfplots}
\usepackage{xfrac}
\usepackage{hyperref}



\newtheorem{definition}{Definition}[section]
\newtheoremstyle{mystyle}%                % Name
  {}%                                     % Space above
  {}%                                     % Space below
  {\itshape}%                                     % Body font
  {}%                                     % Indent amount
  {\bfseries}%                            % Theorem head font
  {}%                                    % Punctuation after theorem head
  { }%                                    % Space after theorem head, ' ', or \newline
  {\thmname{#1}\thmnumber{ #2}\thmnote{ (#3)}}%                                     % Theorem head spec (can be left empty, meaning `normal')

\theoremstyle{mystyle}
\newtheorem{theorem}{Theorem}[section]
\theoremstyle{definition}
\newtheorem*{exmp}{Example}
\begin{document}
\section*{1.}
\subsection*{a.}
Since $\mathcal{B}(\mathbb{R})$ is the $\sigma$-algebra generated by all open sets in $\mathbb{R}$, and $Y$ is contained only open sets. We have that $\sigma(Y) \subseteq \mathcal{B}(\mathbb{R})$. Thus to prove that $\sigma(Y) = \mathbb{R}$, we need to prove that all open sets in $\mathbb{R} \in Y$. Since every open sets in $\mathbb{R}$ can be rewritten as at most countable of pairwise disjoint open intervals and there are 3 types of open intervals in $\mathbb{R}$. For $a, b \in \mathbb{R}$ 
\begin{itemize}
  \item $(a,\infty)$ \\
    Choose any $x \in \mathbb{R}$ such that $x>a$, since $D$ is dense, there is infinitely countable number $d \in \mathbb{D}$ such that $a<d<x$. Thus there is a stricly increasing $(d_n)_{n=1}^\infty$ such that $\lim d_n = a$. Therefore, $\cup_{n=1}^\infty (d, \infty) = (a, \infty) \in \sigma(\mathbb{Y})$.    
  \item $(-\infty, a)$ \\
    Choose any $x \in \mathbb{R}$ such that $x<a$, since $D$ is dense, there is infinitely countable number $d \in \mathbb{D}$ such that $a>d>x$. Thus there is a stricly decreasing $(d_n)_{n=1}^\infty$ such that $\lim d_n = a$. Therefore, $\cup_{n=1}^\infty (d, \infty)^c = \cup_{n=1}^\infty (-\infty, d] = (-\infty, a) \in \sigma(\mathbb{Y})$.    
  \item $(a, b)$ \\
    We have that $(a,b) = (a, \infty) \cap (-\infty, b) \in \sigma(Y)$
\end{itemize}
Thus, every open sets in $\mathbb{R}$ is in $\sigma(Y)$, thus $\sigma(Y) = \mathcal{B}(\mathbb{R})$. 
\subsection*{b.}
For any interval $(a, \infty) \in Y$,  notice that since $D$ is dense, we can construct a strictly decreasing $(a_n)_{n=1}^\infty \in D$ and strictly increasing sequence $(b_n)_{n=1}^\infty \in D$ such that $\lim (a_n, b_n) = (a, \infty)$. And thus for any $a \in D$, we have that $(a, \infty) = \cup_{n=1}^\infty (a_n, b_n) \in \sigma(Z)$. Therefore, $\sigma(Y) \subseteq \sigma(Z)$. \\
For any $a, b \in D$, we have that $(-\infty, b) \cap (-\infty, a)^c = [a, b) \in \sigma(Y)$. Thus $\sigma(Z) \subseteq \sigma(Y)$ and $\sigma(Y) = \sigma(Z) = \mathcal{B}(\mathbb{R})$.  
\subsection*{c.}
From definition, for any $E \in \mathcal{B}(\overline{\mathbb{R}})$, $E$ can be rewritten as $E = E' \cup B'$ or $E' \cup B'^c$ where $E' \in \mathcal{B}(\mathbb{R})$ and $B' \in \{(a, \infty], [-\infty, a), [-\infty, a) \cup (b, \infty]$. Thus we have that $E \backslash \{-\infty, \infty\} = (E' \backslash \{-\infty, \infty \}) \cup (B' \backslash \{-\infty, \infty \}) = E' \cup B''$, where $B'' \in \{(a, \infty), (-\infty, a), (-\infty, a) \cup (b, \infty)\} \subset \mathcal{B}(\mathbb{R})$ thus $E \backslash \{-\infty, \infty \} \in \mathcal{B}(\mathbb{R})$. 
\newpage
\section*{2.}
\subsection*{a.}
For any open sets $G = (a,b) \in \mathbb{R}$, we have that 
\[
  (a,b) = \cup_{n=1}^\infty \left[a + \displaystyle\frac{1}{n}, b - \displaystyle\frac{1}{n}\right]
\]
\subsection*{b.}
For each $x \in \mathbb{R} \backslash E$, $f$ is continuous at $x$ thus for all $n \in \mathbb{N}$, there is $\delta_{x, n}$ such that $\left| f(y) - f(x)\right| < \displaystyle\frac{1}{n}$ for all $y \in (x-\delta_{x,n}, x+\delta_{x,n})$. Thus let 
$\mathcal{O}_n = \cup_{x \in E^c} (x-\delta_{x,n}, x + \delta_{x,n})$. We can see that 
\[
  \bigcap_{n = 1}^\infty \mathcal{O}_n = \bigcap_{n=1}^\infty \bigcup_{x \in E^c} (x-\delta_{x,n}, x + \delta_{x,n}) = \bigcup_{x \in E^c} \underbrace{\bigcap_{n=1}^\infty (x-\delta_{x,n}, x+\delta_{x,n})}_{x} = E^c 
\]
And therefore, $E^c$ is a $G_\delta$ set and thus $E$ is a $F_\sigma$ set. 
\newpage
\section*{3.}
\subsection*{a.}
If $\mu$ is $\sigma$-finite, then there exists some $X_n \in \mathcal{M}$ such that $X = \cup_{n=1}^\infty X_n$, $X_n \subseteq X_{n+1}$ and $\mu(X_n) < \infty$. Thus for each $E \in \mathcal{M}$ with $\mu(E) = \infty$, there exists $N \in \mathbb{N}$ such that 
\[
  \mu(X_N \cap E)  > 0 
\]
as else 
$\mu(E \cap \bigcup_{j=1}^n) = \mu(E \cap A) = 0$. But since $\mu(X_N \cap E) \le \mu(X_N) < \infty$. $X_N$ satisfies $X_N \in \mathcal{M}, X_N \subseteq E, 0< \mu(X_N)< \infty $. 
\subsection*{b.}
Let 
\[
  S = \{ F \in \mathcal{M}: F \subseteq E, \mu(F) < \infty \}
\]
Then we know that $\sup_{F \in S} \mu(F)$ must exists.  Suppose it is less than $\infty$, that is $\sup_{F \in S} \mu(F) = L$ for some $L \in \mathbb{R}$, then we can choose $(F_n)_{n=1}^\infty$ such that $\lim \mu(F_n) = L$. Then, we have that $\mu(\cup_{F \in S} F) = \mu(\cup_{n =1}^\infty F_n) = L$. Therefore, $\mu(E \backslash \cup_{F \in S}F) = \infty$, thus there exists $F' \subset E \backslash F$, so that $0<\mu(F')<\infty$. But $F \cup F' \subset E$ and $\infty > \mu(F \cup F') = \mu(F) + \mu(F') > L$ which is a contradiction and therefore $\sup_{F \in S}\mu(F)= \infty$ and there is some set $F \subseteq E$ such that $C<\mu(F)< \infty$.  
\subsection*{c.}
First, 
\[
  u_0(\varnothing) = \sup \{\mu(F): F \subseteq \varnothing, \mu(F) < \infty \} = 0 
\]
as the only subset of empty set is itself. If $E_j \in \mathcal{M}$ for all $j \in \mathbb{N}$ and $E_j$ are pairwise disjoint then in case where there is $j$ such that $\mu(E_j) = \infty$ then from part b, we have  
\[
  \sum_{j=1}^\infty \mu_0(E_j) = \infty = \mu_0(\sqcup_{j=1}^\infty E_j)
\]
If $\mu_0(E_j) = L_j$ are finite for every $j \in \mathbb{N}$ then we can choose $(F_{j,n})_{n=1}^\infty \subseteq E_j$ such that $\lim \mu(F_{j,n}) = L_j$ and since $E_j$ are pairwise disjoint, $F_{j_1, n} \cap F_{j_2, n} = \varnothing$ for $j_1 \ne j_2$. Therefore, there is a sequence $F_n = \sqcup_{j=1}^\infty F_{j,n} \subset \sqcup_{j=1}^\infty E_j$ such that $\lim \mu_0(F_n) = \sum_{j=1}^\infty L_j$ thus $\mu_0(\sqcup_{j=1}^\infty E_j) = \sum_{j=1}^\infty L_j = \sum_{j=1}^\infty \mu_0(E_j)$. 
If $\mu_0(E) = \infty$ for some $E \in \mathcal{M}$, then for any constant $C$, there exists $F \subseteq E$ such that $\mu(F)<\infty$. Now if for every $F \subseteq E$ satisfying $\mu(F) < \infty$, $\mu(F) = 0$ then $\mu_0(E) = 0$ which is a contradiction. Thus there exists some $F$ such that $F \subseteq E$ and $0<\mu(F)<\infty$.  
\newpage
\section*{4.}
\subsection*{a.}
\begin{itemize}
  \item $\varnothing \in \mathcal{E}$. Choosing $a< b$ so that $([-\infty , a] \cap \mathbb{Q}) \cap (\mathbb{Q} \cap (b, \infty]) \in \mathcal{E}$. 
  \item If $E = (a_1, b_1] \cap \mathbb{Q}, F = (a_2, b_2] \cap \mathbb{Q} \in \mathcal{E}$ then in case $(a_1, b_1] \cap (a_2, b_2] = \varnothing, E \cap F = \varnothing \in \mathcal{E}$. In case $(a_1, b_1] \cap (a_2, b_2] \ne \varnothing$ then there exists $a_3, b_3$ such that $(a_1, b_1] \cap (a_2, b_2] = (a_3, b_3]$ thus $E \cap F = (a_3, b_3] \cap \mathbb{Q} \in \mathcal{E}$.
  \item If $E = (a, b] \cap \mathbb{Q} \in \mathcal{E}$ where $a<b$, then $E^c = ((-\infty, a] \cup (b, \infty]) \cap \mathbb{Q}\\ = \underbrace{((-\infty, a] \cap \mathbb{Q})}_{\in \mathcal{E}} \cap \underbrace{((b, \infty] \cup \mathbb{Q})}_{\in \mathcal{E}}$, which are disjoint as $a<b$
\end{itemize}
\subsection*{b.}
From definition, $\mathcal{A} \subseteq \mathbb{Q}$ thus $\sigma(\mathcal{A}) \subseteq \mathcal{P}(\mathbb{Q})$. \\
For any $E \subseteq \mathcal{P}(Q)$, we can write $E = \{x_1, x_2, \hdots, \}$ as rationals are countable. Then for any $x_j$, we can define  
\[ 
  E_{j,n} = \left(x_1 - \displaystyle\frac{1}{n}, x_1 \right]\cap \mathbb{Q} \in \mathcal{A} \subseteq \sigma(\mathcal{A})
\]
so that 
\[
  \bigcap_{n=1}^\infty E_{j,n} = x_j \cap \mathbb{Q} = x_j 
\]
Thus 
\[
  \bigcap_{k=1}^\infty \bigcap_{n=1}^\infty E_{j,n} = E
\]
\subsection*{c.}
We have that $u_0(\varnothing) = 0$ and for $E_j \in \mathcal{A}$, where $E_j$ are pairwise disjoint, there is 2 cases
\begin{itemize}
  \item if there is non-empty $E_k$ then $\sqcup_{j=1}^n E_j \ne \varnothing$ and thus 
    \[
      \mu_0(\sqcup_{j=1}^\infty E_j ) = \infty = \sum_{j=1}^\infty E_j = \mu_0(E_k) + \sum_{\substack{j=1 \\ j \ne k}}^\infty \mu_0(E_j) = \infty 
    \]
  \item if all of them are empty, then simply
    \[
      \mu_0(\sqcup_{j=1}^\infty E_j) = \mu_0(\varnothing) = 0 = \sum_{j=1}^n \mu_0(E_j) 
    \]
\end{itemize}
\newpage
\section*{5.}
\subsection*{a.}
Since $Q \cap [0,1] \subseteq \cup_{j=1}^\infty R^o_j$, taking closure, we have $[0,1] \subseteq \cup_{j=1}^\infty \overline{R_j^o}$. Thus 
\[
  \sum_{j=1}^\infty |R_j^o| = \sum_{j=1}^\infty |\overline{R_j^o}| \ge \left|\cup_{j=1}^\infty \overline{R_j^o} \right| \ge 1 
\]
\subsection*{b.}
Since $m(E_j) = 1$, we have that $m(E_j^c) = 0$ and thus  
\[ 
  m(\cap_{j=1}^\infty E_j) = 1 - m(\cup_{j=1}^\infty E_j^c) \ge 1 - \sum_{j=1}^\infty m(E_j^c) = 1 
\]
But since $\cap_{j=1}^\infty E_j \subseteq [0,1]$ thus $m(\cap_{j=1}^\infty E_j) \le 1$. Therefore, $m(\cap_{j=1}^\infty E_j) = 1$. 
\subsection*{c.}
Suppose that $m(\cap_{j=1}^n A_n) = 0$, then $m(\cup_{j=1}^n A_j^c) = 1$. However, we have that $m(\cup_{j=1}^n A_n^c) \le \sum_{j=1}^n m(A_j^c)$. 
Now we know that 
\[ 
  \sum_{j=1}^n m(A_j^c) + \sum_{j=1}^n m(A_j) > n-1 + 1 = n
\]
But 
\[
  \sum_{j=1}^n m(A_j^c) + \sum_{j=1}^n m(A_j) = \sum_{j=1}^n m(A_j^c) + m(A_j) = \sum_{j=1}^n m([0,1]) = n
\]
\end{document}
