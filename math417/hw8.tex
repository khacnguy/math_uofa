
\documentclass[11pt]{article}
    \title{\textbf{Math 217 Homework I}}
    \author{Khac Nguyen Nguyen}
    \date{}
    
    \addtolength{\topmargin}{-3cm}
    \addtolength{\textheight}{3cm}
    
\usepackage{amsmath}
\usepackage{mathtools}
\usepackage{amsthm}
\usepackage{amssymb}
\usepackage{pgfplots}
\usepackage{xfrac}  
\usepackage{hyperref}
\usepackage{xcolor}
\definecolor{Mybackground}{RGB}{40,49,51}
\pagecolor{Mybackground}
\color{white}


\newtheorem{definition}{Definition}[section]
\newtheoremstyle{mystyle}%                % Name
  {}%                                     % Space above
  {}%                                     % Space below
  {\itshape}%                                     % Body font
  {}%                                     % Indent amount
  {\bfseries}%                            % Theorem head font
  {}%                                    % Punctuation after theorem head
  { }%                                    % Space after theorem head, ' ', or \newline
  {\thmname{#1}\thmnumber{ #2}\thmnote{ (#3)}}%                                     % Theorem head spec (can be left empty, meaning `normal')

\theoremstyle{mystyle}
\newtheorem{theorem}{Theorem}[section]
\theoremstyle{definition}
\newtheorem*{exmp}{Example}
\begin{document}
\section*{1.}
Since $f$ is integrable on $\mathbb{R}$, there is a compactly supported continuous function $h$ on $H$ such that $\int_\mathbb{R} |f-h|< \varepsilon/2$. Since $\int_{\mathbb{R}} |f| < \infty$, we have that $h$ is bounded on $H$ thus $h$ is uniformly continuous on $H$ and hence on $\mathbb{R}$. Thus for any $\varepsilon > 0$, there is $\delta > 0$ such that for all $|x-y|< \delta$, $|h(x)-h(y)| < \varepsilon/2m(H)$   
\begin{align*}
  |F(x) - F(y)|  
  &= \left|\int_{-\infty}^x f(t) dt - \int_{-\infty}^y f(t) dt \right| \\
  &= \left|\int_y^x f(t) dt \right| \\
  &\le \int_y^x |f(t) - h(t)| dt + \int_y^x |h(t)| dt \\
  &\le \displaystyle\frac{\varepsilon}{2} + \displaystyle\frac{\varepsilon}{2m(H)} m(H) \\
  &= \varepsilon
\end{align*}
\newpage
\section*{2.}
\subsection*{a.}
For any $\{E_n\} \in \mathcal{M}$, $\{F_n\} \in \mathcal{N}$, if $D \subseteq \cup_{n=1}^\infty E_n \times F_n$, then for any $(x,x) \in D$ there is $n \in \mathbb{N}$ such that $(x,x) \in E_n \times F_n$ which implies $x \in E_n \cap F_n$. Therefore, $[0,1] \subseteq \cup_{n=1}^\infty (E_n \cap F_n)$. Thus $\mu(E_n \cap F_n) > 0$ for some $n \in \mathbb{N}$, which means that $\mu(E_n) > 0$ and $\nu(F_n) = \infty$. Therefore, $(\mu \times \nu)(D) = \infty$.
\subsection*{b.}
\[
  A_x = \{x\} \text{ and } A^y = \{y\}
\]
Thus 
\[
  \nu(A_x) = 1 \text{ and } \mu(A^y) = 0
\]
and hence
\[
  \int_Y \int_X \chi_D d\mu d\nu = \int_{[0,1]} \mu(A^y) d\nu = 0
\]
\[
  \int_X \int_Y \chi_D d\nu d\mu = \int_{[0,1]} \nu(A_x) d\mu = 1
\]
and 
\[
  \iint_{X \times Y} \chi_D d(\mu \times \nu) = \infty
\]
\newpage
\section*{3.}
Apply Theorem 5.5
\begin{align*}
  \int_0^a |g(x)| dx 
  &= \int_0^a \int_x^a |t^{-1}f(t)| dt dx \\ 
  &= \int_0^a \int_0^t \displaystyle\frac{|f(t)|}{t} dx dt \\
  &= \int_0^a |f(t)| dt \\
  &< \infty
\end{align*}
Thus $g$ is integrable. Therefore, 
\begin{align*}
  \int_0^a g(x) dx 
  &= \int_0^a \int_x^a t^{-1}f(t) dt dx \\ 
  &= \int_0^a \int_0^t \displaystyle\frac{f(t)}{t} dx dt \\
  &= \int_0^a f(t) dt \\
\end{align*}
\newpage
\section*{4.}
First, note that if $\lambda_f(\alpha) = \infty$ for some $\alpha > 0$ then 
\[
  \int_X |f(x)|^p d\mu(x) = \infty = p \int_0^\infty \alpha^{p-1} \lambda_f(\alpha) d\alpha
\]
Now suppose $\lambda_f(\alpha) < \infty$ for all $\alpha > 0$, then for any $x$, we have that 
\[
  \int_0^{|f(x)|} p\alpha^{p-1} d\alpha = \alpha^p |_{\alpha = 0}^{|f(x)|} = |f(x)|^p
\]
Thus 
\begin{align*}
  &\int_X |f(x)|^p d\mu(x) \\
  =& \int_X \int_0^{|f(x)|} p \alpha^{p-1} d\alpha d\mu(x) \\
  =& \int_X \int_0^\infty p\alpha^{p-1} 1_{|f(x)| > \alpha} d\alpha d\mu(x) \\
  =& \int_0^\infty p\alpha^{p-1} \int_X 1_{|f(x)| > \alpha} d\mu(x)d\alpha \\
  =& \int_0^\infty p \alpha^{p-1} \lambda_f(\alpha) d\alpha
\end{align*}
\newpage
\section*{5.}
\subsection*{a.}
Let $M = \int_{\mathbb{R}^d} |f(x)| dx$ and $N = \int_{\mathbb{R}^d} |g(y)| dy$, then from theorem 5.5,  
\begin{align*}
  &\int_{\mathbb{R}^{2d}} |H(x,y)| d(x\times y) \\
  =& \int_{\mathbb{R}^d} \int_{\mathbb{R}^d} |H(x,y)| dy dx \indent  \left( =\int_{\mathbb{R}^d} [f * g](x) dx \right) \\
  =& \int_{\mathbb{R}^d} |g(y)| \int_{\mathbb{R}^d} |f(x-y)|  dx dy \\
  =& M \int_{\mathbb{R}^d} |g(y)| dy \\ 
  =& MN < \infty
\end{align*}
We also get from the above equations that 
\[
  \int_{\mathbb{R}^d} |f(x-y) g(y)| dy < \infty
\]
for a.e. $x \in \mathbb{R}^d$. Thus $[f*g]$ is well-defined a.e. $x \in \mathbb{R}^d$. 
\subsection*{b.}
Let $\xi_n \to \xi$, then for every $\varepsilon > 0$, we can find a uniformly continuous compact supported function $h$ on $X$ such that $\int_{\mathbb{R}^d} |f-h| < \varepsilon/4$.  
Now, for every $x \in X$ we have 
\begin{align*}
  &\lim_{n \to \infty} |\exp(-ix \cdot (\xi - \xi_n)) - 1| \\
  =& \lim_{n \to \infty}\sqrt{\left(\cos(- x \cdot (\xi - \xi_n)) -1 \right)^2 + \left(\sin(-x \cdot (\xi-\xi_n))\right)^2 } \\
  =& \lim_{n \to \infty} \sqrt{2 - 2 \cos(-x \cdot (\xi-\xi_n)) } \\
  =& 0
\end{align*}
Thus, there is $n_0$ such that for all $n > n_0$, 
\[
  |e^{-ix \cdot (\xi - \xi_n)} - 1| < \displaystyle\frac{\varepsilon}{2Mm(X)}
\]
where 
\[
  M = \sup_{x \in X} |x| < \infty
\]
Therefore, we have
\begin{align*}
  &| \widehat f(\xi) - \widehat f(\xi_n)| \\
  \le &\int_{\mathbb{R}^d} |f(x)| |e^{-ix \cdot \xi} - e^{-ix \cdot \xi_n} | dx \\
  \le &\int_{\mathbb{R}^d} |f(x) - h(x)| |e^{-ix \cdot \xi} - e^{-ix \cdot \xi_n} | dx \\ 
  &+ \int_{\mathbb{R}^d} |h(x)| |e^{-ix \cdot \xi} - e^{-ix \cdot \xi_n} | dx \\
  \le &2\int_{\mathbb{R}^d} |f(x) - h(x)| dx \\ 
  &+ \int_X |h(x)|  |e^{-ix \cdot (\xi-\xi_n)} - 1 |  dx \\
  < & \displaystyle\frac{\varepsilon}{2} + m(X) M \displaystyle\frac{\varepsilon}{2M m(X)} \\
  =& \varepsilon
\end{align*}
and $\widehat f$ is continuous.
Now since we know that $H$ is integrable, $H(x) e^{-ix \cdot \xi}$ is also integrable, therefore 
\begin{align*}
  &\widehat{f * g}(\xi) \\
  =& \int_{\mathbb{R}^d} [f*g](x) e^{-ix \cdot \xi} dx \\
  =& \int_{\mathbb{R}^d} \int_{\mathbb{R}^d} f(x-y)g(y) e^{-ix \cdot \xi} dy dx \\
  =& \int_{\mathbb{R}^d} \int_{\mathbb{R}^d} f(x-y)g(y) e^{-ix \cdot \xi} dx dy \\
  =& \widehat f(\xi)\int_{\mathbb{R}^d} g(y) e^{-i(y-x) \cdot \xi} e^{-ix \cdot \xi} dy \\
  =& \widehat f(\xi) \widehat g(\xi)
\end{align*}

\end{document}
