\documentclass[11pt]{article}
    \title{\textbf{Math 217 Homework I}}
    \author{Khac Nguyen Nguyen}
    \date{}
    
    \addtolength{\topmargin}{-3cm}
    \addtolength{\textheight}{3cm}
    
\usepackage{amsmath}
\usepackage{mathtools}
\usepackage{amsthm}
\usepackage{amssymb}
\usepackage{pgfplots}
\usepackage{xfrac}
\usepgfplotslibrary{polar}
\usepgflibrary{shapes.geometric}
\usetikzlibrary{calc}
\pgfplotsset{compat = newest}
\pgfplotsset{my style/.append style = {axis x line = middle, axis y line = middle, xlabel={$x$}, ylabel={$y$}, axis equal}}
\begin{document}
\section*{1.}
Since the set is compact with content zero, we have that there exists $I_1, I_2, \ldots, I_n$ such that
\[
    I_1 \cup I_2 \cup \ldots \cup I_n \subset U \text{ and } \sum_{j=1}^n \mu(I_j) < \frac{\epsilon}{2} 
\]
Consider the interval $I_i = (a_{i,1}, b_{i,1}) \times \ldots \times (a_{i,N}, b_{i,N})$. 
Since the set of rational number is dense, we can find $b_1'$ such that 
\[
    \mathbb{Q} \ni b_{i,1}' - b_{i,1} < \frac{\epsilon (b_{i,1}- a_{i,1})}{2nN\mu(I_i)}
\]
Therefore, $O_{i,1} = [a_{i,1}, b_{i,1}'] \times \ldots \times [a_{i,N}, b_{i,N}]$ satisfies
\[
    \mu(O_{i,1}) - \mu(I_i) 
    = \mu(I_i) \frac{b_{i_1}' - a_{i,1}}{b_{i,1} - a_{i,1}} -\mu(I_i) 
    = \mu(I_i)\frac{b_{i_1}' - b_{i,1}}{b_{i,1} - a_{i,1}} 
    < \frac{\mu(I_i)}{b_{i,1} - a_{i,1}} \cdot \frac{\epsilon (b_{i,1}- a_{i,1})}{2nN\mu(I_i)} 
    = \frac{\epsilon}{2nN}
\]
Then using $O_{i,1}$, we can find $O_{i,2}$ such that $\mu(O_{i,2}) - \mu(O_{i,1}) < \cfrac{\epsilon}{2nN}$ using the same process. \\
Do this process for the rest $n-1$ subintervals, 
we have that $O_i = O_{i,n} = [a_{i,1}, b_{i,1}'] \times \ldots \times [a_{i,N}, b_{i,N}']$ satisfies $I_i \subset O_i$ and 
\[
    \mu(O_i) - \mu(I_i) < \frac{\epsilon}{2nN} \cdot N = \frac{\epsilon}{2n}
\]
and hence
\[
    \sum_{j=1}^n \mu(O_i) - \sum_{j=1}^n \mu(I_i) < \frac{\epsilon}{2n} \cdot n = \frac{\epsilon}{2}    
\]
Therefore, 
\[
    \sum_{j=1}^n \mu(O_i) < \sum_{j=1}^n \mu(I_i) + \frac{\epsilon}{2} < \epsilon    
\]
Since each intervals in $O_i$ has a rational length, we can split it into cubes. \\
Suppose we have a interval $I = [a_1,b_1] \times \ldots \times [a_N,b_N]$ such that $b_i - a_i = \cfrac{x_i}{y_i}$ where $x_i$ and $y_i$ are integers. Then
since for every subintervals $[a_i,b_i]$, we have that $\cfrac{\frac{x_i}{y_i}}{\frac{1}{\prod_{j=1}^N y_j}} = x_i \cdot \prod_{\substack{j=1 \\ i\ne j}}^N y_j$ is also an integer and hence
we can split the interval $I$ into cubes with sides of length $\frac{1}{\prod_{j=1}^N y_j}$. Therefore, 
we get the results. 
\pagebreak
\section*{2.}
Let $I$ be a compact interval such that  $D \subset I_1 \cup I_2 \cup \ldots \cup I_n \subset I$, we have that 
\[
    \mu(D) = \int_I \chi_D \le \int_I \chi_{\bigcup_{j=1}^n I_j} \le \int_I \sum_{j=1}^n \chi_{I_j} = \sum_{j=1}^n \int_I \chi_{I_j} = \sum_{j=1}^n \mu(I_j)    
\]
and hence $\mu(D) \le \inf \sum_{j=1}^n \mu(I_j)$. \\
For all $\epsilon >0$ then there is a partition $P$ of $I$ such that 
\[
    \left| \mu(D) - \sum_v \chi_D(x_v)\mu(I_v) \right| < \epsilon    
\] 
Next, we choose $I_1, I_2, \ldots, I_n$ satisfy the condition $D \cap I_v \ne \varnothing$ 
and choose $x_v \in D \cap I_v$. 
Therefore, 
\begin{equation*}
    \begin{aligned}
        \sum_{j=1}^n \mu(I_j) &= \sum_v \chi_D(x_v)\mu(I_v) \\
        &\le \mu(D) + \left| \mu(D) - \sum_v \chi_D(x_v) \mu(I_v) \right| \\
        &< \mu(D) + \epsilon     
    \end{aligned}
\end{equation*}
Therefore, 
\[
    \mu(D) = \inf \sum_{j=1}^n \mu(I_j)    
\]
\pagebreak
\section*{3.}
Since $f$ and $g$ are continuous and $I$ is compact. $f$ and $g$ are Riemann integrable on $I$ and 
hence $fg$ is also Riemann integrable, which means that there exists a partition $P$ such that 
for all refinement $P_\epsilon$ of $P$
\[
    \left| \int_I fg - \sum_v f(x_v) g(x_v) \mu(I_v)\right| < \frac{\epsilon}{2} 
\]
for arbitary $x_v \in I_v$. We know that $f(I)$ are compact 
since $f$ is continuous and $I$ is compact. 
Therefore, there exists $M = \sup_{x \in I}f(x)$.
Then since $g$ is uniformly continuous on $I$, we can find a refinement $Q$ of $P$ such that 
for all subdivision, $|g(x_v) - g(y_v)| < \frac{\epsilon}{2 M \mu(I)}$. 
Hence, for all refinement $Q_\epsilon$ of $Q$, we have 
\begin{equation*}
    \begin{aligned}
        \left| \int_I fg - \sum_v f(x_v) g(y_v) \mu(I_v) \right| 
        \le & \left| \int_I fg - \sum_v f(x_v) g(x_v) \mu(I_v) \right| \\
         +& \left| \sum_v f(x_v) g(x_v) \mu(I_v) - \sum_v f(x_v) g(y_v) \mu(I_v) \right| \\
        < & \frac{\epsilon}{2} 
        + \left| \sum_v f(x_v) \mu(I_v) (g(x_v) - g(y_v) )\right| \\
        < & \frac{\epsilon}{2} + \sum_v M \mu(I_v) \frac{\epsilon}{2 M \mu(I)}  \\
        = & \frac{\epsilon}{2} + \frac{\epsilon}{2 \mu(I)} \sum_v \mu(I_v) 
        = \frac{\epsilon}{2} + \frac{\epsilon}{2} = \epsilon
    \end{aligned}
\end{equation*}
\pagebreak
\section*{4.}
\subsection*{a.}
Since $D$ has content, $\partial D$ has content zero, and hence
\[
    \mu(\overline{D}) = \mu(D \cup \partial D) \le \mu(D) + \mu(\partial D) = \mu(D)    
\]
Since $D \subseteq \overline{D}, \mu(D) \le \mu(\overline{D})$ and hence $\mu(D) = \mu(\overline{D})$
\subsection*{b.}
$Z$ is a set of content zero, therefore $Z$ is bounded and 
hence $\overline{Z}$ is compact and has content zero. 
Therefore, $\phi(\overline{Z})$ has content zero, which means that 
$\phi(Z) \subset \phi(Z) \cup \phi(\partial Z) = \phi(\overline{Z})$ also has content zero.
\pagebreak
\section*{5.}
\subsection*{a.}
Consider the function 
\[
    \phi: \mathbb{R}^2 \to \mathbb{R}^2, \indent (x,y) \to \left(\frac{zx}{h}, \frac{zy}{h} \right)    
\]
Then 
\[
    \det \phi(x,y) =
    \det 
    \begin{pmatrix}
        \frac{z}{h} & 0 \\
        0 & \frac{z}{h}
    \end{pmatrix}
    = \frac{z^2}{h^2}  
\]
\begin{equation*}
    \begin{aligned}
        \mu_3 
        &= \int_{0+}^h \int_{\mathbb{R}^2} \chi_D  \left(\frac{hx}{z}, \frac{hy}{z} \right) dxdydz \\
        &= \int_{0+}^h \int_{\mathbb{R}^2} \chi_D (x,y) \cdot \frac{z^2}{h^2} dxdydz \\
        &= \mu_2(D) \int_{0+}^h \frac{z^2}{h^2} dz \\
        &= \mu_2(D) \left.\frac{z^3}{3h^2} \right|_{0+}^{h} \\
        &= \mu_2(D) \cdot \frac{h}{3}
    \end{aligned}
\end{equation*}
which is the volume of a “cone” formed by projecting $D$ to the origin of $\mathbb{R}^3$
\subsection*{b.}
\begin{equation*}
    \begin{aligned}
        \mu_n 
        &= \int_{0+}^h \int_{\mathbb{R}^{n-1}} \chi_D  \left(x, y \right) dxdydz \\
        &= \mu_{n-1}(D) \int_{0+}^h 1 dz \\
        &= \mu_{n-1}(D) \left.z \right|_{0+}^{h} \\
        &= \mu_{n-1}(D) \cdot h
    \end{aligned}
\end{equation*}
\subsection*{c.}
When $n=1$, the object generated is a line with $\mu_1 = h$
When $n=2$, the object generated is a rectangle with $\mu_2 = h \cdot \mu_1$
\pagebreak
\section*{6.}
Suppose $x_0 \in K$ such that $\phi(x_0) \in \partial \phi(K)$, then $x_0 \in K\backslash Z$ or $x_0 \in Z$. \\
In the case where $x_0 \in K\backslash Z$. Suppose that $x_0 \notin \partial K$. 
Since $Z$ has content zero, 
there exists an open neighborhood around $x_0$: $B(x_0)$ such that $B(x_0) \subset K$ and $\det J_\phi(x) \ne 0$ for all $x \in B(x_0)$.
Hence, $\phi(B(x_0)) \subset \phi(K)$ is open. \\
However, we know that $\phi(x_0) \in \partial \phi(K)$, 
which means that for every open neighborhood around $\phi(x_0)$, there exists a point not in $K$. 
Therefore, this is a contradiction and $x_0 \in \partial K$. \\
Finally, since if $x_0 \in K$ satisifes $\phi(x_0) \in \partial \phi(K)$ then $x_0 \in \partial K$ or $x_0 \in Z$, 
we have that $\partial\phi(K) \subset \phi(\partial K) \cup \phi(Z)$. \\
Since $\partial K$ has content zero because $K$ has content, and $Z$ has content zero, 
$\phi(\partial K)$ and $\phi(Z)$ has content zero and hence $\partial \phi(K)$ also has content zero. Therefore, $\phi(K)$ has content.
\end{document}