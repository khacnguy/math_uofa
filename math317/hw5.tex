\documentclass[11pt]{article}
    \title{\textbf{Math 217 Homework I}}
    \author{Khac Nguyen Nguyen}
    \date{}
    
    \addtolength{\topmargin}{-3cm}
    \addtolength{\textheight}{3cm}
    
\usepackage{amsmath}
\usepackage{mathtools}
\usepackage{amsthm}
\usepackage{amssymb}
\usepackage{pgfplots}
\usepackage{xfrac}
\usepgfplotslibrary{polar}
\usepgflibrary{shapes.geometric}
\usetikzlibrary{calc}
\pgfplotsset{compat = newest}
\pgfplotsset{my style/.append style = {axis x line = middle, axis y line = middle, xlabel={$x$}, ylabel={$y$}, axis equal}}
\begin{document}
\section*{1.}
Consider the partition $t_{2n+1} = 0$ and $t_j = \cfrac{1}{j}$ where $1 \le j \le 2n$. Then we have that 
\begin{equation*}
    \begin{aligned}
        \sum_{j=1}^{2n} \| t_{j+1} - t_j \| 
        &= \| t_{2n+1} - t_{2n} \| + \sum_{j=1}^{2n-1} \sqrt{(1-1)^2 + \left(\cfrac{\cos((j+1) \pi)}{j+1} - \cfrac{\cos(j \pi)}{j} \right)^2} \\
        &= \cfrac{\cos(2n \pi)}{2n} + \sum_{j=1}^{2n-1} \left| \cfrac{1}{j+1} + \cfrac{1}{j} \right| \cdot | \cos((j+1)\pi) | \\
        &> \cfrac{1}{2n} + \sum_{j=1}^{2n-1} \cfrac{1}{j} \\
        &= \sum_{j=1}^{2n} \cfrac{1}{j}
    \end{aligned}
\end{equation*}
However, we know that the harmonic series $\sum_{j=1}^\infty \cfrac{1}{j}$ diverges. Therefore, the supremum
of $\sum_{j=1}^{2n} \|t_{j+1} - t_j \|$ of all partitions does not exists and hence the curve is not rectifiable.
\pagebreak
\section*{2.}
\begin{equation*}
    \begin{aligned}
        \alpha'(t) 
        &= J_{\Phi \circ \gamma}(t) \\
        &= J_\Phi(\gamma(t)) J_\gamma(t) \\
        &= 
        \begin{bmatrix}
            \cfrac{\partial \Phi_1}{\partial x}(\gamma(t)) & \cfrac{\partial \Phi_1}{\partial y}(\gamma(t)) \\
            \cfrac{\partial \Phi_2}{\partial x}(\gamma(t)) & \cfrac{\partial \Phi_2}{\partial y}(\gamma(t)) \\
            \cfrac{\partial \Phi_3}{\partial x}(\gamma(t)) & \cfrac{\partial \Phi_3}{\partial y}(\gamma(t)) 
        \end{bmatrix}
        \begin{bmatrix}
            \cfrac{\partial x}{\partial t}(t) \\
            \cfrac{\partial y}{\partial t}(t) 
        \end{bmatrix} \\
        &= 
        \begin{bmatrix}
            \cfrac{\partial \Phi_1}{\partial x}(\gamma(t)) \cfrac{\partial x}{\partial t}(t) + \cfrac{\partial \Phi_1}{\partial y}(\gamma(t)) \cfrac{\partial y}{\partial t}(t) \\
            \cfrac{\partial \Phi_2}{\partial x}(\gamma(t)) \cfrac{\partial x}{\partial t}(t) + \cfrac{\partial \Phi_2}{\partial y}(\gamma(t)) \cfrac{\partial y}{\partial t}(t) \\
            \cfrac{\partial \Phi_3}{\partial x}(\gamma(t)) \cfrac{\partial x}{\partial t}(t) + \cfrac{\partial \Phi_3}{\partial y}(\gamma(t)) \cfrac{\partial y}{\partial t}(t) 
        \end{bmatrix}
    \end{aligned}
\end{equation*}
Let $
x_i = \cfrac{\partial \Phi_i}{\partial x}(\gamma(t)), 
y_i = \cfrac{\partial \Phi_i}{\partial y}(\gamma(t)), 
a = \cfrac{\partial x}{\partial t}(t), 
b =\cfrac{\partial y}{\partial t}(t)$, then 
\begin{equation*}
    \begin{aligned}
        \alpha'(t) \cdot N(\gamma(t)) 
        &= \alpha'(t) \cdot \left( \frac{\partial \Phi}{\partial x}(\gamma(t)) \times \frac{\partial \Phi}{\partial y}(\gamma(t)) \right) \\
        &= 
        \det 
        \begin{bmatrix}
            x_1 \cdot a + y_1 \cdot b & x_2 \cdot a + y_2 \cdot b & x_3 \cdot a + y_3 \cdot b \\
            x_1 & x_2 & x_3 \\
            y_1 & y_2 & y_3 
        \end{bmatrix} \\
        &= 0
    \end{aligned}
\end{equation*}
as the first row is a linear combination of the second and third row. Therefore, the two vectors are orthogonal.
\pagebreak
\section*{3.}
Since $K$ is a normal domain, 
there exists a piecewise $\mathcal{C}^1$ curve 
$\gamma: [a,b] \to \mathbb{R}^2$ on the positive oriented boundary of $K$, and since $\Phi|_{\partial K} = \Psi|_{\partial K}$. \\
\[
    \int_{\Phi} \text{curl} f \cdot n d\sigma = \int_{\Phi \circ \gamma} Pdx + Qdy + Rdz = \int_{\Psi \circ \gamma} Pdx + Qdy + Rdz = \int_{\Psi} \text{curl} f \cdot n d\sigma 
\]
\pagebreak
\section*{4.}
\subsection*{a.}
Apply the Gauss theorem to the vector field $f\nabla g$ then
\begin{equation*}
    \begin{aligned}
        \int_S f D_n g d\sigma &=  \int_S f \nabla g \cdot n d\sigma \\
        & = \int_V \nabla (f \nabla g) \\
        & = \int_V f \nabla^2 g + \nabla f \cdot \nabla g \\
        & = \int_V f \Delta g + \int_V \nabla f \cdot \nabla g
    \end{aligned}
\end{equation*}
\subsection*{b.}
From part a, we have that 
\[
    \int_S f D_n g d\sigma = \int_V f \Delta g + \int_V \nabla f \cdot \nabla g    
\]
\[
    \int_S g D_n f d\sigma = \int_V g \Delta f + \int_V \nabla g \cdot \nabla f
\]
Therefore, 
\[
    \int_S (f D_n g - g D_n f) d\sigma = \int_V (f \Delta g - g \Delta f)
\]
\pagebreak
\section*{5.}
\subsection*{a.}
From 4a, let $g=1$ then $\nabla g = \Delta g = 0$, we have that 
\[
    0 = \int_V (\nabla g) \cdot (\nabla f) + \int_V g \nabla f = \int_S g D_n f d\sigma
\]
\subsection*{b.}
From 4a, let $g = f$, we have that 
\[
    \int_S f D_n f d\sigma = \int_V f \Delta f + \int_V \nabla f \cdot \nabla f = \int_V |\nabla f|^2
\]
\end{document}