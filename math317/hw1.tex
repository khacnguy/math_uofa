\documentclass[11pt]{article}
    \title{\textbf{Math 217 Homework I}}
    \author{Khac Nguyen Nguyen}
    \date{}
    
    \addtolength{\topmargin}{-3cm}
    \addtolength{\textheight}{3cm}
    
\usepackage{amsmath}
\usepackage{mathtools}
\usepackage{amsthm}
\usepackage{amssymb}
\usepackage{pgfplots}
\usepackage{xfrac}
\usepgfplotslibrary{polar}
\usepgflibrary{shapes.geometric}
\usetikzlibrary{calc}
\pgfplotsset{compat = newest}
\pgfplotsset{my style/.append style = {axis x line = middle, axis y line = middle, xlabel={$x$}, ylabel={$y$}, axis equal}}
\begin{document}
\section*{1.}
\subsection*{a.}
Let $x, y \in U$ such that $f(x) = f(y)$ and $\xi =  y-x$. 
From the Taylor's theorem and the fact that $x+\theta \xi$ is in the convex set $U$ for $\theta \in [0,1]$, we have that for each $j= 1,\ldots, N$, there exists a number $\theta_j \in [0,1]$ such that
\[
    f_j(y) = f_j(x+\xi) = f_j(x) + \sum_{k=1}^N \frac{\partial f_j}{\partial x_k}(x+\theta_j \xi)\xi_k = f_j(x)   
\]
It follows that 
\[
    \sum_{k=1}^N \frac{\partial f_j}{\partial x_k}(x+\theta_j \xi)\xi_k = 0
\]
for $j = 1, \ldots, N$. Let 
\[
    A:=
    \begin{bmatrix}
        \frac{\partial f_1}{\partial x_1}(x+\theta_1 \xi) & \hdots &\frac{\partial f_1}{\partial x_N}(x+\theta_1 \xi) \\
        \vdots & \ddots & \vdots \\
        \frac{\partial f_N}{\partial x_1}(x+\theta_N \xi) & \hdots &\frac{\partial f_1}{\partial x_N}(x+\theta_N \xi)
    \end{bmatrix}    
\]
so that $A\xi = 0$. However, since the points in the set $\{x + \theta\xi | \theta \in [0,1]\}$ are collinear points.
We have that $\det(A) \ne 0$ and therefore $\xi=0$ which  means that $x=y$.
\subsection*{b.}
For any $(x_1,y_1), (x_2, y_2) \in \mathbb{R}^2$, we have that
\[
    \det 
    \begin{bmatrix}
        \cfrac{\partial f_1}{\partial x_1}(x_1,y_1) & \cfrac{\partial f_1}{\partial x_2}(x_1,y_1)  \\
        \cfrac{\partial f_2}{\partial x_1}(x_1,y_1) & \cfrac{\partial f_2}{\partial x_2}(x_1,y_1)  
    \end{bmatrix}   
    = \det 
    \begin{bmatrix}
        3x_1^2 & -1 \\
        e^{x_1+y_1} & e^{x_2+y_2}  
    \end{bmatrix}
    = 3x_1^2 \cdot e^{x_2+y_2} + e^{x_1 + y_1}
\]
which is $>0$ for all $(x_1,y_1), (x_2, y_2)$ and hence $f$ is injective.
\pagebreak
\section*{2.}
\subsection*{a.}
\begin{equation*}
    \begin{aligned}
        \det J_f(x,y) &= 
        \det 
        \begin{bmatrix}
            \cfrac{\sqrt{x^2+y^2} - x \cdot \cfrac{2x}{2\sqrt{x^2+y^2}}}{x^2+y^2} & \cfrac{-xy}{(x^2+y^2)^{3/2}}\\
            \cfrac{-xy}{x^2+y^2} & \cfrac{\sqrt{(x^2+y^2)^{3/2}} - y \cdot \cfrac{2y}{2\sqrt{x^2+y^2}}}{x^2+y^2}
        \end{bmatrix}  \\
        &=  
        \det 
        \begin{bmatrix}
            \cfrac{y^2}{(x^2+y^2)^{3/2}} & \cfrac{-xy}{(x^2+y^2)^{3/2}} \\
            \cfrac{-xy}{(x^2+y^2)^{3/2}} & \cfrac{x^2}{(x^2+y^2)^{3/2}}
        \end{bmatrix} 
        =0 \\
    \end{aligned}
\end{equation*}
\subsection*{b.}
For all $(x,y)$, we have that 
\[
    |f(x,y)| = \sqrt{\frac{x^2}{x^2 + y^2} + \frac{y^2}{x^2 + y^2}} = 1    
\]
Therefore, $f(U)$ is the circle around the origin with radius 1. It does not contain any non-empty open 
subset as for all point $(x,y) \in f(U): \forall \epsilon >0: (x+\epsilon, y) \notin f(U)$
\pagebreak
\section*{3.}
Consider 
\[
    f: \mathbb{R} \to \mathbb{R}, \indent x \to x^3
\]
Then $f \in \mathcal{C}^1(U,\mathbb{R}^N)$ where $U = \mathbb{R}$ and $N=1$. \\
$\det J_f(0) = 0$ but every neighborhood $(-\epsilon , \epsilon)$ is mapped to $(-\epsilon^3, \epsilon^3)$ which is open. 
\pagebreak
\section*{4.}
\subsection*{a.}
Consider the function 
\[
    h: \mathbb{R}^N \to \mathbb{R}, \indent x \to |x|    
\]
It is obvious that for each $j=1, \ldots ,N$, $\cfrac{\partial h}{\partial x_j}(x) = \cfrac{x_j}{|x|}$. \\
If there is a local maximum at $f(x_0) \ne (0,\ldots, 0)$, then applying the chain rule, we have that for each $j = 1, \ldots ,N$
\[
    0 = \frac{\partial g}{\partial x_j}(x_0) = 
    \frac{\partial (h \circ f)}{\partial x_j}(x_0) = \frac{\partial h}{\partial x_j}(f(x_0)) 
    \cdot \underbrace{\frac{\partial f}{\partial x_j}(x_0)}_{\ne 0}
\]
and hence $f(x_0) = (0,\ldots ,0)$, which is a contradiction. At $f(x_0) = (0,\ldots,0)$, 
$g(x_0)= h(f(x_0))=0$ and hence it is clearly the global minimum as norm is always $\ge 0$. Therefore, there is no local maximum.  
\subsection*{b.}
Since $\overline{U}$ is compact $g(\overline{U})$ is also compact and hence $g(\overline{U})$ must attain its maximum since every converges sequence converges to a point in the set and hence the supremum is the maximum. \\
Since $\tilde{f}$ does not attain its local maximum and hence global maximum on $\text{int}$ $\overline{U}$, 
the global maximum must be attained on the boundary $\partial U = \partial \overline{U}$.  
\end{document}