\documentclass[11pt]{article}
    \title{\textbf{Math 217 Homework I}}
    \author{Khac Nguyen Nguyen}
    \date{}
    
    \addtolength{\topmargin}{-3cm}
    \addtolength{\textheight}{3cm}
    
\usepackage{amsmath}
\usepackage{mathtools}
\usepackage{amsthm}
\usepackage{amssymb}
\usepackage{pgfplots}
\usepgfplotslibrary{polar}
\usepgflibrary{shapes.geometric}
\usetikzlibrary{calc}
\pgfplotsset{compat = newest}
\pgfplotsset{my style/.append style = {axis x line = middle, axis y line = middle, xlabel={$x$}, ylabel={$y$}, axis equal}}
\begin{document}
\section*{1.}
a.
we have that 
\[
\begin{cases}
\cfrac{c}{2} + \cfrac{c}{3} + \cfrac{c}{4} + \cfrac{c}{5} + p \cdot 2 = \cfrac{77}{60} \, c + 2p = 1 \\
\cfrac{c}{2} \cdot 0 + \cfrac{c}{3} \cdot 1 + \cfrac{c}{4} \cdot 2 + \cfrac{c}{5} \cdot 3 + p \cdot (4+5) = \cfrac{43}{30} \, c + 9p =1.5
\end{cases}
\implies \begin{cases}
c = \cfrac{360}{521} \\
p = \cfrac{59}{1042}\\
\end{cases}
\]
b.
Given that there is at least 2 injuries, then the probability that there is 5 injuries is 
\[
\cfrac{p}{1-\cfrac{c}{2}-\cfrac{c}{3}} = \frac{59}{442}
\]
Hence the probability that there is less than 5 injuries is
\[
1 - \cfrac{59}{442} = 0.8665
\]
\pagebreak
\section*{2.}
Bowl 1: 3 white and 4 red \\
Bowl 2: 4 white and 3 red \\
Bowl 3 : 5 white and 2 red \\
$P(B_1) = 0.2, P(B_2) = 0.5, P(B_3) = 0.3$ \\
The probability 2 red balls is chosen is
\[
\cfrac{C^4_2}{C^7_2} \cdot 0.2 + \cfrac{C^3_2}{C^7_2} \cdot 0.5 + \cfrac{C^2_2}{C^7_2} \cdot 0.3 = \cfrac{1}{7} 
\]
The probability 1 red ball is chosen is 
\[
\cfrac{C^4_1 \cdot C^3_1}{C^7_2} \cdot 0.2 + \cfrac{C^3_1 \cdot C^4_1}{C^7_2} \cdot 0.5 + \cfrac{C^2_1 \cdot C^5_1}{C^7_2} \cdot 0.3 = \cfrac{19}{35} 
\]
Hence, the expected value of W is
\[
\cfrac{1}{7} \cdot (2^1-1) + \cfrac{19}{35} \cdot (2^2-1) = 0.9714
\]
\pagebreak
\section*{3.}
a.
Let $Y_i$ be the amount of money we get/lose in the i-th turn. Then the expected value and variance of any $Y_i$ is 
\[
E[Y_i] = 5 \cdot \cfrac{1}{10} + 3 \cdot \frac{2}{10} + 1 \cdot \cfrac{3}{10} - 2 \cdot \cfrac{4}{10} = \frac{3}{5}
\]
\[
V[Y_i] = E[Y_i^2] - (E[Y_i])^2 =  5^2 \cdot \cfrac{1}{10} + 3^2 \cdot \frac{2}{10} + 1^2 \cdot \cfrac{3}{10} + (-2)^2 \cdot \cfrac{4}{10} - \frac{9}{25} = 5.84
\]
Therefore, the expected value and variance of $Y$ is 
\[
E[Y] = E[Y_1+Y_2+Y_3+Y_4] = E[Y_1] + E[Y_2] + E[Y_3] + E[Y_4] = 4 \cdot \frac{3}{5} = \frac{12}{5}
\]
\[
V[Y] =  V[Y_1+Y_2+Y_3+Y_4] =  V[Y_1] + V[Y_2] + V[Y_3] + V[Y_4] = 4 \cdot 5.84 = 23.36
\]
b. since $X$ follows a hypergeometric distribution (10,5,4), we have that 
We have that the expected value and variance of $X$ is
\[
E[X] = 4 \cdot 5 / 10 = 2
\]
\[
V[X] = 4 \cdot \frac{5}{10} \cdot \frac{5}{10} \cdot \frac{6}{9} = \frac{2}{3}
\]
Hence the expected value and variance of $W$ is 
\[
E[W] = E[9-5X] = 9 - 5E[X] = -1
\]
\[
V[W] = V[9-5X] = (-5)^2 \cdot V[X] = \frac{50}{3}
\]
\pagebreak
\section*{4.}
Let $S$ be the amount you have to pay in the original grid. \\
Let $A'$ be the amount you have to pay if there is only point $A$ on the grid. \\
Let $B'$ be the amount you have to pay if there is only point $B$ on the grid. \\
Let $C'$ be the amount you have to pay if there is only point $C$ on the grid. \\
Let $D'$ be the amount you have to pay if there is only point $D$ on the grid. \\
Then let $y$ be a random path. It is obvious that 
\[S(y) = A'(y) + B'(y) + C'(y) + D'(y)\]
Because if $y$ passes through point $A$ in $S$, then $y$ also passes through point $A$ in $A'$. Since $A$ only exists in $A'$ and $S$, the amount you have to pay for point $A$ going through $S$ and going through all of $A', B',C',D' $ is the same. The same happens for $B,C,D$. \\
Hence, 
\begin{equation*}
\begin{aligned}
E[Y] = E[S(y)] &= E[A'(Y) + B'(Y) + C'(Y) + D'(Y)] \\
&= E[A'(Y)] + E[B'(Y)] + E[C'(Y)] + E[D'(Y)] \\
&= 10 \cdot \cfrac{C^4_1 \cdot C^9_4}{C^{13}_5} + 5 \cdot \cfrac{C^8_2 \cdot C^5_2}{C^{13}_5} + 3 \cdot \cfrac{C^7_4 \cdot C^6_2}{C^{13}_5}+ 6 \cdot \cfrac{C^{11}_3 \cdot C^2_2}{C^{13}_5} \\
&\approx 6.997
\end{aligned}
\end{equation*}
\pagebreak
\section*{5.}
The total number of ways the hand can be dealt is $C_5^{52} = 2598960$. \\
Let $A(X)$ mean the number of ways we are dealt $X$ amount of hearts. \\
The number of ways our hand have 0 and 1 hearts are
\[
A(X = 1) = C_4^{39} \cdot C_1^{13} = 1069263
\]
\[
A(X = 0) = C_5^{39} = 575757
\]
Hence the number of ways our hand have at least two hearts is 
\[
A(X\ge 2) = 2598960 - A(X=0) - A(X=1) = 953940
\]
The number of ways our hand have three hearts is 
\[
C_2^{39} \cdot C_3^{13} = 211926
\]
Hence, the probability that the hand have exactly there hearts given that it has at least two hearts is:
\[\frac{A(X = 3)}{A(X \ge 2)} = \frac{211926}{953940}\]
\pagebreak
\section*{6.}
The probability that the team has there fourth win before therefore 4-th loss is the probability that the team has 4 wins and less than or equal to 3 losses, which means that the team has their fourth win before or at the 7-th turn. Let $Y$ be the number of trial until the r-th success, then $Y \sim \text{NegativeBinomial}(4,0.6)$. Hence the probability that the teams has their 4-th win before their 4-th loss is
\[
\sum_{y=4}^7 P(Y=y) = \sum_{y=4}^7 \begin{pmatrix} y-1 \\ 3 \end{pmatrix} \cdot 0.6^4 \cdot 0.4^{y-4} = \frac{11097}{15625}
\]
\pagebreak
\section*{7.}
a. Let $X$ be the number of attempts an applicants need to complete the first task. Then $X \sim$ Geometric($0.19$) \\
Hence the probability that any applicants complete their first task on the first, second, third, fourth and fifth attempts are
\[
P(X=1) = 0.81^0 \cdot 0.19 = 0.19
\]
\[
P(X=2) = 0.81^1 \cdot 0.19 = 0.1539
\]
\[
P(X=3) = 0.81^2 \cdot 0.19 = 0.1247
\]
\[
P(X=4) = 0.81^3 \cdot 0.19 \approx 0.10097379
\]
\[
P(X=5) = 0.81^4 \cdot 0.19 \approx 0.0817887699
\]
Hence the probability that a random applicant get the job is 
\[
P(X\le 5) \approx 0.6513
\]
b.
The probability that an applicant does not complete the task in the first 5 tries is 
\[
P(X>5) = 1 - P(X\le 5) \approx 0.3487
\] 
Therefore, the expected value for $X, X^2$ is 
\begin{equation*}
\begin{aligned}
E[Y] &= E[Y | X = 1] P(X = 1) + E[Y | X = 2] P(X = 2) + E[Y | X = 3] P(X = 3) \\
&\indent + E[Y | X = 4] P(X = 4) + E[Y | X = 5] P(X = 5) + E[Y| X >5] P(X>5) \\
&\approx 356542.5 \\
E[Y^2] &= E[Y^2 | X = 1] P(X = 1) + E[Y^2 | X = 2] P(X = 2) + E[Y^2 | X = 3] P(X = 3) \\
&\indent + E[Y^2 | X = 4] P(X = 4) + E[Y^2 | X = 5] P(X = 5) + E[Y^2| X >5] P(X>5) \\
&\approx 232845062500
\end{aligned}
\end{equation*}
\[
\sigma_Y = |\sqrt{V(Y)}| = |\sqrt{E[(Y-\mu_Y)^2]}| = \sqrt{E[Y^2] - (E[Y])^2} \approx 325149.978
\]
\pagebreak
\section*{8.}
Since, the computer does not crash on the first 2 days, the probability that it has its third crash on the seventh day is the same as the probability that it has its third crash on the fifth-day, given that the computer can crash any days with the same probability 0.42. \\
Let $Y$ be the numuber of days until the third crash happens, then $Y \sim $ NegativeBinomial($3,0.42$), which means that the probability for the computer to crash given that it did not crash on the first 2 days is 
\[
P(Y=5) = \begin{pmatrix} 5-1 \\ 2 \end{pmatrix} \cdot 0.42^3 \cdot 0.58^2 = 0.149539
\]
\pagebreak
\section*{9.}
Since the number of defects per foot on each beam follows a Poisson distribution with mean 0.52:
$\mu_Y = E[Y] = \lambda = 0.52$. Hence, the expected profits on one product is 
\begin{equation*}
\begin{aligned}
E[X] = E[30-2Y-3Y^2] &= 30 - 2E[Y] - 3E[Y^2] + 3\mu_Y^2 - 3\mu_Y^2 \\
&= 30 - 2E[Y] - 3 \sigma_Y^2 - 3\mu_Y^2 \\
&= 30 - 5 \cdot 0.52 - 3 \cdot 0.52^2 \\
&= 26.5888
\end{aligned}
\end{equation*}
And therefore, the expected profits on three random products is
\[
26.5888 \cdot 3 = 79.7664
\]
\pagebreak
\section*{10.}
Let $X$ be the expected value of the difference between the amount you would be reimbursed in the case of hail damage to your car and the amount you paid for this policy.\\
the number of times per year you will have hail damage to your car (call this the random variable Y ) follows a Poisson distribution with a mean of 1.45: 
$\mu_Y = E[Y] = \lambda = 1.45$. Let $n$ be the number of times you hail damage to your car, then $\forall n \ge 1,$ the number of times you would be reimbruised is $n-1$. In case, the number of times you hail damage is 0, the number of times you would be reimbruised is $0$. \\ 
\[P(Y=0) = P(X=-750) = \cfrac{1.45^0}{0!} \cdot e^{-1.45}\]
For $y\ge 1$:
\[
P(Y=y) = P(X = -750+1000(y-1)) 
\]
We also have,
\[ 
\]
Hence,
\begin{equation*}
\begin{aligned}
&E[Y] = E[Y|Y\ge 1] P(Y\ge 1) + E[Y=0] P(Y=0) \\
\implies & 1.45 = E[Y| Y\ge 1](1-P(Y=0)) + 0 = E[Y|Y\ge 1] \cdot (1- e^{-1.45})\\
\implies & E[Y|Y\ge 1] = \frac{1.45}{1-e^{-1.45}} = 1.8944
\end{aligned}
\end{equation*}
Therefore,
\begin{equation*}
\begin{aligned}
E[X] &= E[X|Y\ge 1] \cdot P(Y\ge 1) + E[X|Y=0] \cdot P(Y=0) \\
&= E[-750+1000(Y-1) | Y \ge 1] \cdot P(Y\ge 1) + E[X|Y=0] \cdot P(Y=0) \\
&= E[-1750+1000Y|Y \ge 1]  \cdot P(Y\ge 1) + E[-750|Y=0] \cdot P(Y=0)\\
&= (-1750 + 1000E[Y| Y \ge 1]) \cdot 0.7654 + (-750) \cdot 0.23457 + = -65.404
\end{aligned}
\end{equation*}
Therefore, we should not buy it.
\pagebreak
\section*{11.}
The number of points the player score per game follows a distribution with a mean of 1.43.
Hence, the probability that he scores exactly 3 points in a game is
\[
P(Y=3) = \frac{1.43^3}{3!} \cdot e^{-1.43} = 0.11663
\] 
Therefore, the number of game the player scores exactly 3 points $\sim$ Binomial$(10,0.11663)$. Let $X$ be the number of games he scores exactly 3 points. The probability that he has 0 and 1 games in which he scores exactly 3 points are
\[
P(X=0) = C^{10}_0 \cdot 0.11663^0 \cdot (1-0.11663)^{10-0} = 0.28935
\]
\[
P(X=1) = C^{10}_1 \cdot 0.11663^1 \cdot (1-0.11663)^{10-1} = 0.38203
\]
Therefore, the probability that the player scores at least 2 games with exactly 3 points is
\[
P(X\ge 2) = 1 - P(X=0) - P(X=1) = 0.32862
\]
Also, the probability that he scores less than 3 points is the sum of the probability he scores 0,1 and 2 points. Therefore, the probability that he scores less than 3 points is 
\[
P(Y\le 2) = \sum_{k=0}^2 \frac{1.43^k}{k!} \cdot e^{-1.43} = 0.8262
\]
Which means that the probability that he scores at least 3 points is 
\[
P(Y \ge 3) = 1 - P(Y\le 2) = 0.1738 
\]
Therefore, the number of games the player need to score at least 3 points the first time follows a geometric series with $p=0.1738$. As a result, the expected value and variance of $U$ is 
\[
E[U] = \frac{1}{0.1738} = 5.7537
\]
\[
V[U] = \frac{1-p}{p^2} = 27.35178
\]
\end{document}