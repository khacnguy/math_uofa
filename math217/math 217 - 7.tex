\documentclass[11pt]{article}
    \title{\textbf{Math 217 Homework I}}
    \author{Khac Nguyen Nguyen}
    \date{}
    
    \addtolength{\topmargin}{-3cm}
    \addtolength{\textheight}{3cm}
    
\usepackage{amsmath}
\usepackage{mathtools}
\usepackage{amsthm}
\usepackage{amssymb}
\usepackage{pgfplots}
\usepgfplotslibrary{polar}
\usepgflibrary{shapes.geometric}
\usetikzlibrary{calc}
\pgfplotsset{compat = newest}
\pgfplotsset{my style/.append style = {axis x line = middle, axis y line = middle, xlabel={$x$}, ylabel={$y$}, axis equal}}
\begin{document}
\section*{1.}
For each $k$, we can consider the matrix $A$ as a block matrix as follows: 
\[
\begin{pmatrix}
    A_k & B \\
    C & D
\end{pmatrix}
\]
Then we have that 
\[
    A
    \begin{pmatrix}
    x_1 \\
    x_2
    \end{pmatrix}
    \cdot 
     \begin{pmatrix}
    x_1 \\
    x_2
    \end{pmatrix} = \lambda \|x^2\|
    > 0 \text { for all } 
    x = 
    \begin{pmatrix}
        x_1 \\
        x_2
    \end{pmatrix}
\]
Hence, 
\[
    \begin{pmatrix}
        A_k & B \\
        C & D 
    \end{pmatrix}
    \cdot 
    \begin{pmatrix}
        x_1 \\
        0
    \end{pmatrix}
    \cdot 
    \begin{pmatrix}
        x_1 \\
        0
    \end{pmatrix}
    = A_k x_1 \cdot x_1 > 0
\]
which means that upper left submatrices are all positive definite.
\pagebreak
\section*{2.}
As $a >0$ and $ad > b^2>0$, $d$ is also larger then 0. \\
If $\lambda$ is an eigenvalue of $A$, then\\
\[\chi_A(\lambda) = (a-\lambda)(d- \lambda) - b^2 = ad-b^2 - \lambda(a+d) + \lambda^2 = 0\]
And hence, 
\[
ad-b^2+\lambda^2 = \lambda(a+d)
\]
We know that $ad-b^2 >0$, hence $ad-b^2 + \lambda^2 >0$. Therefore, $\lambda > 0 $ as $(a+d) > 0$. \\
Therefore, $A$ is positive definite. \\
\pagebreak
\section*{3.}
\[
\frac{\partial f}{\partial y}(x,y) = (3+2\cos x)(-\sin y) = 0 \iff \sin y = 0 \iff y \in \{0, \pi \}
\]
\[
\frac{\partial f}{\partial x}(x,y) = -2 \sin x \cos y = 0 \iff \cos y = 0 \lor \sin x = 0 \iff x = 0 \lor \cos y = 0\]
As $\forall y: \sin y \ne \cos y$, we can conclude that the stationary points are $(0,0), (0,\pi)$
\[
\frac{\partial^2 f}{\partial x \partial y}(x,y) = 2\sin x\sin y 
\]
\[
\frac{\partial^2 f}{\partial y \partial x}(x,y) = 2\sin x \sin y 
\]
\[
\frac{\partial^2 f}{\partial y^2}(x,y) = (3+2\cos x)(-\cos y) 
\]
\[
\frac{\partial^2 f}{\partial x^2}(x,y) = -2\cos x\cos y
\]
det(Hess $f$)(0,$\pi$) = $(3+2\cos 0)(-\cos \pi)(-2\cos 0\cos \pi)-(2\sin 0 \sin \pi)^2 = 10 > 0$ \\
where $(3+2\cos 0)(-\cos \pi)= 5 > 0$. \\
Therefore (0,$\pi$) is a local minimum.
$f(0,\pi) = -5$ \\
det(Hess $f$)(0,0) = $(3+2\cos 0)(-\cos 0)(-2\cos 0\cos 0)-(2\sin 0 \sin 0)^2 = 10 > 0$ \\
where $(3+2\cos 0)(-\cos \pi)= -5 < 0$. \\
Therefore (0,0) is a local maximum. $f(0,0) = 5$ 
\pagebreak
\section*{4.}
\[
\frac{\partial f}{\partial y}(x,y,z) = 3x^2 - 3 = 0 \iff x \in \{1,-1\}
\]
\[
\frac{\partial f}{\partial y}(x,y,z) = -3y^2 + 9 = 0\iff y \in \{\sqrt{3}, \sqrt{-3}\} 
\]
\[
\frac{\partial f}{\partial z}(x,y,z) = 2z = 0 \iff z = 0 
\]
As $\forall y: \sin y \ne \cos y$, we can conclude that the stationary points must be in the set $\{1,-1\} \times \{\sqrt{3}, \sqrt{-3}\} \times \{0\}$.
$\forall a \ne b \in \{x,y,z\}:$
\[
\frac{\partial^2 f}{\partial a \partial b} = 0 
\]
\[
\frac{\partial^2 f}{\partial x^2}(x,y,z) = 6x
\]
\[
\frac{\partial^2 f}{\partial y^2}(x,y,z) = -6y
\]
\[
\frac{\partial^2 f}{\partial z^2}(x,y,z) = 2
\]
Hence, as (Hess $f$)($x,y,z$) is a diagonal matrix, its eigenvalue are $\{6x, 6y, 2\}$. Which means that (Hess $f$)($x,y,z$) be definite, and positive definite in this case because $2 > 0$. $6x,6y > 0$. Hence, $(1,\sqrt{3},2\}$ is the only local minimum point where the others are saddle points. $f(1,\sqrt{3},2) = 6\sqrt{3} + 2$
\pagebreak
\section*{5.}
\begin{equation*}
\begin{aligned}
&\frac{\partial f}{\partial x}(x,y) = 2x \cdot e^{-(x^2+y^2)} + (x^2+2y^2) \cdot (-2x) \cdot e^{-(x^2+y^2)} = 0 \\
\implies & e^{-(x^2+y^2)}(2x)(1-x^2-2y^2) = 0 \\
\implies & x = 0 \lor x^2+2y^2 = 1
\end{aligned}
\end{equation*}
\begin{equation*}
\begin{aligned}
&\frac{\partial f}{\partial y}(x,y) = 4y \cdot e^{-(x^2+y^2)} + (x^2+2y^2) \cdot (-2y) \cdot e^{-(x^2+y^2)} = 0 \\
\implies & e^{-(x^2+y^2)}(2y)(2-x^2-2y^2) = 0 \\
\implies & y = 0 \lor x^2+2y^2 = 2
\end{aligned}
\end{equation*}
Hence, we have that $\nabla f(x,y) = 0$ if and only if \\
1. $x=y=0$ \\
2. $x=0$ and $x^2+2y^2 = 2$. Which means that $x=0$ and $y \in \{1,-1\}$ \\
3. $y=0$ and $x^2+2y^2 = 1$. Which means that $x \in \{1,-1\}$ and $y=0$ \\
4. $x^2+2y^2 = 2$ and $x^2+2y^2=1$ which is impossible.
Therefore, the set of stationary points is $(\{0\}\times \{0,1,-1\}) \cup (\{0,1,-1\} \times \{0\})$.
\begin{equation*}
\begin{aligned}
\frac{\partial^2 f}{\partial x^2}(x,y) &= (2-6x^2-4y^2) \cdot e^{-(x^2+y^2)} + (2x-2x^3-4xy^2) \cdot (-2x) \cdot e^{-(x^2+y^2)}\\
&= e^{-(x^2+y^2)}(2-10x^2-4y^2+4x^4+8x^2y^2) 
\end{aligned}
\end{equation*}
\begin{equation*}
\begin{aligned}
\frac{\partial^2 f}{\partial y^2}(x,y) &= (4-12y^2-2x^2) \cdot e^{-(x^2+y^2)} + (4y-4y^3-2yx^2) \cdot (-2y) \cdot e^{-(x^2+y^2)}\\
&= e^{-(x^2+y^2)}(4-20y^2-2x^2+8y^4+4x^2y^2) 
\end{aligned}
\end{equation*}
\begin{equation*}
\begin{aligned}
\frac{\partial^2 f}{\partial y \partial x}(x,y) &= (2x)(-4y)e^{-(x^2+y^2)} + (2x)(-2y)(1-x^2-2y^2)e^{-(x^2+y^2)}\\
&= -4xye^{-(x^2+y^2)}(3-x^2-2y^2)
\end{aligned}
\end{equation*}
\begin{equation*}
\begin{aligned}
\frac{\partial^2 f}{\partial x \partial y}(x,y) &=(2y)(-2x)e^{-(x^2+y^2)} + (2y)(2-x^2-2y^2)(-2x)e^{-(x^2+y^2)}\\
&= -4xye^{-(x^2+y^2)}(1 +2 -x^2-2y^2)
\end{aligned}
\end{equation*}
We have that 
\[
\frac{\partial^2 f}{\partial x^2} (0,0) > 0, \frac{\partial^2 f}{\partial x^2}(1,0) = \frac{\partial^2 f}{\partial x^2} (-1,0) = -4e^{-1}< 0, \frac{\partial^2 f}{\partial x^2}(0,1) = \frac{\partial^2 f}{\partial x^2}(0,-1) < 0 
\]
\[
\frac{\partial^2 f}{\partial y^2} (0,0) > 0, \frac{\partial^2 f}{\partial y^2}(1,0) = \frac{\partial^2 f}{\partial y^2} (-1,0) = 2e^{-1}> 0, \frac{\partial^2 f}{\partial x^2}(0,1) = \frac{\partial^2 f}{\partial x^2}(0,-1) < 0 
\]
For all stationary point $(x,y)$, we have
\[
\frac{\partial^2 f}{\partial x \partial y}(x,y) = \frac{\partial^2 f}{\partial y \partial x}(x,y) = 0
\]
Every (Hess $f$)($x,y$) of a stationary points are in the form $\begin{pmatrix} a & 0 \\ 0 &d \end{pmatrix}$. Hence, (0,0) is a local minimum and (0,1), (0,-1) are local maximums while (1,0) and (-1,0) are saddle points. 
$f(0,0) = 0, f(0,1) = f(0,-1) = \frac{2}{e}$
\pagebreak
\section*{6.}
\[
\frac{\partial f}{\partial x}(x,y) = \cos(x) + \cos(x+y)
\]
\[
\frac{\partial f}{\partial y}(x,y) = \cos(y) + \cos(x+y)
\]
Therefore, $\nabla f = 0 \iff \cos x = \cos y$ which means that $x=y$ as it is bounded between $0$ and $\frac{\pi}{2}$. \\
We also have that $\cos x + \cos (x+x) = 0 \iff \cos x + 2\cos^2(x) -  1 = 0 \iff x = \frac{\pi}{3}$. 
\[
\frac{\partial^2 f}{\partial x^2}(\frac{\pi}{3},\frac{\pi}{3}) = -\sin(\frac{\pi}{3}) - \sin(\frac{2\pi}{3}) = -\sqrt{3}
\]
\[
\frac{\partial^2 f}{\partial y^2}(\frac{\pi}{3},\frac{\pi}{3}) = -\sin(\frac{\pi}{3}) - \sin(\frac{2\pi}{3}) = -\sqrt{3}
\]
\[
\frac{\partial^2 f}{\partial x \partial y} = \frac{\partial^2 f}{\partial x \partial y} = -\sin(x+y)
\]
Hence, $\frac{\partial^2 f }{\partial x \partial y} (\frac{\pi}{3}, \frac{\pi}{3}) = \frac{\partial^2 f}{\parti al x \partial y} (\frac{\pi}{3}, \frac{\pi}{3}) =  \frac{\sqrt{3}}{2}$
Therefore, $\frac{\partial^2 f }{\partial x^2} \frac{\partial^2 f }{\partial y^2}-\left(\frac{\partial^2 f }{\partial x \partial y} \right)^2 > 0$ and hence $(\frac{\pi}{3}, \frac{\pi}{3})$ is a local (and thus global) maximum. 
Since there is no local minimum, the global mimumum is on the boundary of the domain, that is $x=0$ or 
$y=0$ or $x = \frac{\pi}{2}$ or $y = \frac{\pi}{2}$. \\
If $x=0$ then $\frac{\partial f}{\partial y}(0,y) = 2\cos(y) = 0 \iff y = \frac{\pi}{2}$
\\
If $y=0$ then $\frac{\partial f}{\partial x}(x,0) = 2\cos(y) = 0 \iff x = \frac{\pi}{2}$
\\
If $x = \frac{\pi}{2}$ then $\frac{\partial f}{\partial y}(\frac{\pi}{2},y) = \cos(y) + \cos(\frac{\pi}{2} + y) = 0 \iff y = \frac{\pi}{4}$
\\
If $y = \frac{\pi}{2}$ then $\frac{\partial f}{\partial x}(x,\frac{\pi}{2}) = \cos(x) + \cos(\frac{\pi}{2} + x) = 0 \iff x = \frac{\pi}{4}$
Therefore, the minimum point must be at one of these points: $(0, \frac{\pi}{2}), (\frac{\pi}{2},0), (0,0), (\frac{\pi}{2}, \frac{\pi}{2}), (\frac{\pi}{4}, \frac{\pi}{2}), (\frac{\pi}{2}, \frac{\pi}{4})$
\[
f\left(0,0\right) = 0, f\left(0,\frac{\pi}{2}\right) = f\left(\frac{\pi}{2},0\right) = 2, f\left(\frac{\pi}{4}, \frac{\pi}{2}\right) = f\left(\frac{\pi}{2}, \frac{\pi}{4}\right) = 1 + \sqrt{2}, f\left(\frac{\pi}{2}, \frac{\pi}{2}\right) = 2
\]
Hence, the global minimum is at $\left(0, 0\right)$ with the value of 0    
\end{document}