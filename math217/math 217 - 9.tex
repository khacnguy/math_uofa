\documentclass[11pt]{article}
    \title{\textbf{Math 217 Homework I}}
    \author{Khac Nguyen Nguyen}
    \date{}

    \addtolength{\topmargin}{-3cm}
    \addtolength{\textheight}{3cm}

\usepackage{amsmath}
\usepackage{mathtools}
\usepackage{amsthm}
\usepackage{amssymb}
\usepackage{pgfplots}
\usepgfplotslibrary{polar}
\usepgflibrary{shapes.geometric}
\usetikzlibrary{calc}
\pgfplotsset{compat = newest}
\pgfplotsset{my style/.append style = {axis x line = middle, axis y line = middle, xlabel={$x$}, ylabel={$y$}, axis equal}}
\begin{document}
\section*{1.}
(a) It is obvious from definition that $\varnothing$ and the compact set $I$ has content
and is a subset of $I$. Hence, $\varnothing, I \in \mathcal{A}$ \\
(b) 
If $A \in \mathcal{A}$ then both $I$ and $A$ has a zero measure boundary,
\[
\partial(I\backslash A) \subset \partial (A \cup I) \subset \partial A \cup \partial I    
\]
which means that $I\backslash A \subset I$ also have zero measure boundary and hence has content. 
Hence $I\backslash A \in \mathcal{A}$.\\
(c) If $A_1, \ldots ,A_n \in \mathcal{A}$ then $A_1, \ldots A_n \subset I$ and has measure zero boundary.
Hence $A_1 \cup \ldots \cup A_n \subset I$ also have boundary and therefore $A_1 \cup \ldots \cup A_n \in \mathcal{A}$.
\pagebreak
\section*{2.}
We create a sequence of partition $P_n = P_{x_n} \times P_{y_n} $ as follows:
\[
P_{x_n} = \left\{\frac{1}{n}, \frac{2}{n}, \ldots \frac{n-1}{n}, 1 \right\}
\]
\[
P_{y_n} = \left\{\frac{1}{n}, \frac{2}{n}, \ldots \frac{n-1}{n}, 1 \right\}
\]
For each $I_v$, we choose $(x_v,y_v)$ such that
\[\|(x_v, y_v)\| = \text{max}\{\|(x_v,y_v)\|: (x_v,y_v) \in I_v} \]
Hence,
\begin{equation*}
    \begin{align}
        S(P_n,f) &= \sum_v \mu(I_v) \cdot f(x_v,y_v) \\
        &= \frac{1}{n^2} \cdot \left(\frac{1}{n} \cdot \frac{1}{n} + \frac{2}{n} \cdot \frac{1}{n} + \ldots + 1 \cdot \frac{1}{n} \right) \\
        &+ \frac{1}{n^2} \cdot \left(\frac{1}{n} \cdot \frac{2}{n} + \frac{2}{n} \cdot \frac{2}{n} + \ldots + 1 \cdot \frac{2}{n} \right) \\
        &+ \ldots \\
        &+ \frac{1}{n^2} \cdot \left(\frac{1}{n} \cdot \frac{n}{n} + \frac{2}{n} \cdot \frac{n}{n} + \ldots + 1 \cdot \frac{n}{n} \right) \\
        &= \frac{1}{n^2} \cdot \left(\frac{1}{n} + \frac{2}{n} + \ldots + \frac{n}{n} \right)^2 \\
        &= \frac{1}{n^2} \cdot \left(\frac{n(n-1)}{2n}\right)^2 \\
        &= \frac{n^4 - 2n^3 + n^2}{4n^4} \\
        &= \frac{1}{4} \text { as } n \to \infty
    \end{align}
\end{equation*}
Hence, $\int_{[0,1]^2} f = \frac{1}{4}$
\pagebreak
\section*{3.}
\begin{equation*}
    \begin{align}
        \int_{[0,1]^3} f &= \int_0^1 \int_0^1 \int_0^1 f(x,y,z) dz dx dy \\
        &= \int_0^1 \int_0^1 \int_0^{xy} f(x,y,z) dz dx dy + \int_0^1 \int_0^1 \int_{xy}^1 f(x,y,z) dz dx dy \\
        &= \int_0^1 \int_0^1 \int_0^{xy} xy dz dx dy + \int_0^1 \int_0^1 \int_{xy}^1 z dz dx dy \\
        &= \int_0^1 \int_0^1 x^2y^2 dx dy + \int_0^1 \int_0^1 \frac{1-x^2y^2}{2}  dx dy \\
        &= \int_0^1 \frac{y^2}{3} dy + \int_0^1 \frac{1}{2}- \frac{y^2}{6} dy \\
        &= \frac{5}{9}
    \end{align}
\end{equation*}
\pagebreak
\section*{4.}
\begin{equation*}
    \begin{align}
        \int_D f &= \int_0^1 \int_0^{\sqrt{1-x^2}} \cfrac{4y^3}{(x+1)^2} dy dx \\
        &= \int_0^1 \int_0^{\sqrt{1-x^2}} \cfrac{4y^3}{(x+1)^2} dy dx \\
        &= \int_0^1 \frac{(1-x^2)^2}{(x+1)^2} dx \\
        &= \int_0^1 (1-2x+x^2) dx \\
        &= \frac{1}{3}
    \end{align}    
\end{equation*}

\pagebreak
\section*{5.}
\begin{equation*}
    \begin{align}
        \mu(D) &= \int_D \chi_D \\
        &= \int_a^b \int_0^{f(x)} 1 dydx \\
        &= \int_a^b (f(x)-0)dx \\
        &= \int_a^b f(x)dx
    \end{align}
\end{equation*}
\pagebreak
\section*{6.}
We have that
\begin{equation*}
    \begin{aligned}
        &\int_0^1 \left( \int_{1/2}^1 f(x,y) dx \right) dy \\
        &= \left(1-\frac{1}{2}\right)^2 \cdot 2^2 = 1
    \end{aligned}
\end{equation*}
and for all $n \in \mathbb{N}$ such that $n>1$:
\begin{equation*}
    \begin{aligned}
        &\int_0^1 \left( \int_{2^{-n}}^{2^{-n+1}} f(x,y) dx \right) dy \\
        &= (2^{-n+1} - 2^{-n})^2 \cdot 2^{2n} + (2^{-n} - 2^{-n-1}) \cdot (2^{-n+1} - 2^{-n}) \cdot (-2^{2n+1}) \\
        &= 2^{-2n} \cdot 2^{2n} + 2^{-n-1} \cdot 2^{-n} \cdot (-2^{2n+1}) \\
        &= 1 - 1 = 0
    \end{aligned}
\end{equation*}
which means that 
\begin{equation*}
    \begin{align}
        &\int_0^1 \left(\int_0^1 f(x,y) dx \right) dy \\
        &= \int_0^1 \left(\int_{1/2}^1 f(x,y) dx \right) dy + \int_0^1 \left(\int_{1/4}^{1/2} f(x,y) dx \right) dy \\
        &+ \int_0^1 \left(\int_{1/8}^{1/4} f(x,y) dx \right) dy + \ldots \\
        &= 1 + \sum_{n=2}^\infty \int_0^1 \left(\int_{2^{-n}}^{2^{-n+1}} f(x,y) dx \right) dy \\
        &= 1 + 0 = 1
    \end{align}
\end{equation*}
We also have that for all natural number $n$:
\begin{equation*}
    \begin{aligned}
        &\int_0^1 \left( \int_{2^{-n}}^{2^{-n+1}} f(x,y) dy \right) dx \\
        &= (2^{-n+1} - 2^{-n})^2 \cdot 2^{2n} + (2^{-n} - 2^{-n-1}) \cdot (2^{-n+1} - 2^{-n}) \cdot (-2^{2n+1}) \\
        &= 2^{-2n} \cdot 2^{2n} + 2^{-n-1} \cdot 2^{-n} \cdot (-2^{2n+1}) \\
        &= 1 - 1 = 0
    \end{aligned}
\end{equation*}
which means that 
\begin{equation*}
    \begin{align}
        &\int_0^1 \left(\int_0^1 f(x,y) dy \right) dx \\
        &= \int_0^1 \left(\int_{1/2}^1 f(x,y) dy \right) dx + \int_0^1 \left(\int_{1/4}^{1/2} f(x,y) dy \right) dx \\
        &+ \int_0^1 \left(\int_{1/8}^{1/4} f(x,y) dy \right) dx + \ldots \\
        &= \sum_{n=2}^\infty \int_0^1 \left(\int_{2^{-n}}^{2^{-n+1}} f(x,y) dy \right) dx \\
        &= 0 
    \end{align}
\end{equation*}
Therefore, 
\[
    \int_0^1 \left( \int_0^1 f(x,y) dy \right) dx \ne \int_0^1 \left( \int_0^1 f(x,y) dx \right) dy 
\]
However, it does not contradicts the Fubini's theorem because when we write out the whole sequence
\begin{equation*}
    \begin{align}
        &\int_0^1 \left( \int_0^1 f(x,y) dy \right) dx \\
        &= (1-1) + (1-1) + (1-1) + \ldots \\
        &= 1 - 1 + 1 - 1 + 1 -1 + \ldots 
    \end{align}
\end{equation*}
and 
\begin{equation*}
    \begin{align}
        &\int_0^1 \left( \int_0^1 f(x,y) dx \right) dy \\
        &= 1 + \left((1-1) + (1-1) + (1-1) + \ldots \right)\\
        &= 1 + \left(1-1+1-1+1-1+ \ldots \right) \\
    \end{align}
\end{equation*}
We can see that both $1-1 + 1-1 + 1-1 + \ldots$ and $1 + (1-1+1-1+\ldots)$
diverges. And hence we cannot rearrange the terms while we should be able to rearrange terms in the riemann integral.
\end{document}