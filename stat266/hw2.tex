\documentclass[11pt]{article}
    \title{\textbf{Math 217 Homework I}}
    \author{Khac Nguyen Nguyen}
    \date{}
    
    \addtolength{\topmargin}{-3cm}
    \addtolength{\textheight}{3cm}
    
\usepackage{amsmath}
\usepackage{mathtools}
\usepackage{amsthm}
\usepackage{amssymb}
\usepackage{pgfplots}
\usepackage{xfrac}
\usepgfplotslibrary{polar}
\usepgflibrary{shapes.geometric}
\usetikzlibrary{calc}
\pgfplotsset{compat = newest}
\pgfplotsset{my style/.append style = {axis x line = middle, axis y line = middle, xlabel={$x$}, ylabel={$y$}, axis equal}}
\begin{document}
\section*{1.}
We know that $\frac{1}{4} \left(Z_1 + Z_2 + Z_3 + Z_4\right) \sim Normal(0,\frac{1}{4})$ \\
and hence $\frac{1}{2} \left(Z_1 + Z_2 + Z_3 + Z_4\right) \sim Normal(0,1)$ \\
We also have that $W=Z_5^2 + Z_6^2 + Z_7^2 + Z_8^2 + Z_9^2 + Z_{10}^2 \sim \chi_6^2$ \\
Therefore, with $c=\sqrt{6}/2$
\[
    \frac{\sqrt{6}}{2}\cdot \frac{Z_1 + Z_2 + Z_3 + Z_4}{\sqrt{Z_5^2 + Z_6^2 + Z_7^2 + Z_8^2 + Z_9^2 + Z_{10}^2}} \sim t_6    
\]
\pagebreak
\section*{2.}
We have that 
\[
    Z_1^2 + Z_2^2 + \ldots + Z_n^2 \sim \chi^2_{n}
\]
and 
\[
    Z_{n+1}^2 + Z_{n+2}^2 + \ldots + Z_{3n}^2 \sim \chi^2_{2n}
\]
Therefore, with $c=2$
\[
    2 \cdot \frac{Z_1^2 + Z_2^2 + \ldots + Z_n^2}{Z_{n+1}^2 + Z_{n+2}^2 + \ldots + Z_{3n}^2 \sim \chi^2_{2n}} \sim F_{2n}^n
\]
\pagebreak
\section*{3.}
\[
    \overline{X} = 2\overline{Y} + 35 > 60 \iff \overline{Y} > 12.5    
\]
\[
    \sum_{i=1}^{52} Y_i \sim Gamma(\alpha = 3 \cdot 52 = 160, \beta = 5)
\]
\[
    \overline{Y} = \frac{1}{52} \sum_{i=1}^{52} Y_i \sim Gamma(\alpha = 160/52=3, \beta = 5\cdot52 = 260)
\]
and hence
\pagebreak
\section*{4.}
Consider $Y = \sum_{i=1}^{100} Y_i \sim Normal(\mu_Y = 100 \cdot 2540 = 254000, \sigma_Y = 100 \cdot 2100 = 210000)$ 
Then $Z = \cfrac{300000-254000}{210000}  = \cfrac{23}{105}$ and hence the proability that the total of 100 claims will be over 300000 dollars is 0.4129
\pagebreak
\section*{5.}
For each bulb, the probability that it is not a dud is 
\[
    1 - \int_0^{2.5} 11 \cdot e^{-11x} dx  = e^{- \frac{55}{2}}
\]
Then the probability that there is less than 45 duds follows a normal distribution with $\mu = 200 \cdot e^{- \frac{55}{2}}$
and $\sigma = \sqrt{200 \cdot e^{- \frac{55}{2}} \cdot (1-e^{- \frac{55}{2}})}$, which hence is 
\[
    \frac{45-200 \cdot e^{- \frac{55}{2}}}{\sqrt{200 \cdot e^{- \frac{55}{2}} \cdot (1-e^{- \frac{55}{2}})}}
\]
\pagebreak
\section*{6.}
We have that 
\begin{equation*}
    \begin{aligned}
        E[\overline{Y}]^2 
        & = E[\overline{Y}^2] - V[\overline{Y}] \\
        & = E[\overline{Y}^2] -  \frac{\beta^2}{m} \\
        & = E[\overline{Y}^2] -  \frac{E[\overline{Y}]^2}{m} \\
    \end{aligned}
\end{equation*}
Therefore, 
\[
    E[\overline{Y}]^2 = E[\overline{Y}^2] \cdot \frac{m}{m+1}   
\]
Hence, 
\begin{equation*}
    \begin{aligned}
        E[C] &= E[2Y^2 - 4Y] \\
        & = 2E[Y^2] - 4E[Y] \\
        & = 2(V[Y] +E[Y]^2) - 4E[Y] \\
        & = 2(\beta^2 + \beta^2) - 4 \beta \\
        & = 4\beta^2 - 4\beta \\
        & = 4E[\overline{Y}]^2 - 4 E[\overline{Y}] \\
        & = 4 \frac{m}{m+1} E[\overline{Y}^2] - 4E[\overline{Y}]
    \end{aligned}
\end{equation*}
Therefore, an unbiased estimator is $\cfrac{4m \overline{Y}^2}{m+1} - 4\overline{Y}$
\pagebreak
\section*{7.}
$X_{(n)} = \max\{X_1, X_2, \ldots, X_n\}$. Therefore, 
\begin{equation*}
    \begin{aligned}
        F_{X_{(n)}}(x) &= P(X_{(n)} \le x) = P(X_1, X_2, \ldots, X_n < x) \\
        &= \left(\frac{x}{\theta}\right)^n \\
        f_{X_{(n)}}(x) &= n \cdot \frac{1}{\theta} \cdot \left(\frac{x}{\theta}\right)^{n-1}\\
    \end{aligned}
\end{equation*}
Therefore, 
\[
    E[X_{(n)}] = \int_0^{\theta} x \frac{n}{\theta}  \left(\frac{x}{\theta} \right)^{n-1} dx 
    = \frac{n}{\theta^n} \int_0^\theta x^n dx = \frac{\theta n }{n+1}    
\]
and similarly
\[
    E[X_{(n)}^2] = \int_0^{\theta} x^2 \frac{n}{\theta}  \left(\frac{x}{\theta} \right)^{n-1} dx 
    = \frac{n}{\theta^n} \int_0^\theta x^{n+1} dx = \frac{\theta^2 n }{n+2}
\]
We have that 
\begin{equation*}
    \begin{aligned}
        V[Y] &= V[E[Y|X]] + E[V[Y|X]] \\
        &= V\left[\frac{X}{3} \right] + E\left[\frac{X^2}{9} \right] \\
        &= \frac{\theta^2}{108}  + \frac{1}{9} (E[X]^2 - V[X]) \\
        &= \frac{\theta^2}{108}  + \frac{1}{9} \left(\frac{\theta^2}{4} - \frac{\theta^2}{12} \right) \\
        &= \frac{\theta^{2}}{36} \\
        &= \frac{n+2}{36n}E[X_{(n)}^2]
    \end{aligned}
\end{equation*}
\pagebreak
\section*{8.}
\[
    F_{Y_{(n)}}(y) = \left(\frac{5y^4}{(\beta+1)^5} \right)^n    
\]
\[
    f_{Y_{(n)}}(y) = n \left(\frac{5y^4}{(\beta+1)^5} \right)^{n-1} \cdot \frac{20y^3}{(\beta+1)^5}
\]
\begin{equation*}
    \begin{aligned}
        E[Y_{(n)}] 
        &= \int_0^{\beta+1} y \cdot n \left(\frac{5y^4}{(\beta+1)^5} \right)^{n-1} \cdot \frac{20y^3}{(\beta+1)^5} \\
        &=
    \end{aligned}
\end{equation*}

\[
    E[\hat{\beta_1}] =     
\]
\end{document}