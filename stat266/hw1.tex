\documentclass[11pt]{article}
    \title{\textbf{Math 217 Homework I}}
    \author{Khac Nguyen Nguyen}
    \date{}
    
    \addtolength{\topmargin}{-3cm}
    \addtolength{\textheight}{3cm}
    
\usepackage{amsmath}
\usepackage{mathtools}
\usepackage{amsthm}
\usepackage{amssymb}
\usepackage{pgfplots}
\usepackage{xfrac}
\usepgfplotslibrary{polar}
\usepgflibrary{shapes.geometric}
\usetikzlibrary{calc}
\pgfplotsset{compat = newest}
\pgfplotsset{my style/.append style = {axis x line = middle, axis y line = middle, xlabel={$x$}, ylabel={$y$}, axis equal}}
\begin{document}
\section*{1.}
\subsection*{a.}
\[
    f_Y(y) = \int_0^2 \frac{1}{6}dx = \frac{1}{3} 
\]
Since $0<y<3$, $2<y^2 +2<11$, and hence if $u\le 2, F_U(u) = 0$ and if $u \ge 11, F_U(u) = 1$. For $2<u<11$, we have that
\begin{equation*}
    \begin{aligned}
        F_U(u) &= P(U\le u) = P(Y^2+2 \le u) \\
        &= P(Y^2 \le u-2) = P(Y \le \sqrt{U-2}) \\
        &= \int^{\sqrt{U-2}}_0 \frac{1}{3}dy = \frac{\sqrt{U-2}}{3}
    \end{aligned}
\end{equation*}
\subsection*{b.}
Since $0<x<2, 0<y<3, -3<x-y<2$. Therefore, if $v \le -3, F_V(v) = 0$ and if $v\ge 2, F_V(v) = 1$. 
For $-3<v<2$, we have that
\begin{equation*}
    \begin{aligned}
        F_V(v) &= P(V \le v) = P(X-Y\le v) \\ 
        &= P(Y\ge -v+X)         
    \end{aligned}
\end{equation*}
For $-3<v<-1$,
\begin{equation*}
    \begin{aligned}
        F_V(v) &= \int_{-v}^3 \int_{0}^{y+v} \frac{1}{6} dxdy \\
        &= \int_{-v}^3 \frac{y+v}{6} dy = \frac{9}{12} + \frac{3v}{6} -\frac{v^2}{12} + \frac{v^2}{6} \\
        &= \frac{(3+v)^2}{12} \\
        \implies & f_V(v) = \frac{2v+6}{12}
    \end{aligned}
\end{equation*} 
For $0<v<2$, 
\begin{equation*}
    \begin{aligned}
        F_V(v) &= 1- \int_v^2 \int_0^{-v+x} \frac{1}{6} dy dx \\
        &= 1- \int_v^2 \frac{1}{6}(-v+x) dx \\
        &= 1 +\frac{2v}{6} - \frac{1}{3} - \frac{v^2}{6} + \frac{v^2}{12}\\
        &= 1- \frac{(v-2)^2}{12} \\
        \implies & f_V(v) = \frac{-2v+4}{12}
    \end{aligned}
\end{equation*}
For $-1<v<0$
\begin{equation*}
    \begin{aligned}
        F_V(v) &= \int_0^2 \int_{-v+x}^3 \frac{1}{6} dydx \\
        &= \frac{4v+8}{12} \\
        \implies & f_V(v) = \frac{1}{3}
    \end{aligned}
\end{equation*}
\pagebreak
\section*{2.}
Since $ 16 \ge U = Y^4 \ge 0$. We have that if $0\le u, F_U(u) = 0$ and if $u\ge 16, F_U(u) = 1$.
For $0<u<16$, we have that
\begin{equation*}
    \begin{aligned}
        F_U(u) &= P(U\le u) = P(Y^4 \le u) = P(-\sqrt[4]{u} \le Y \le \sqrt[4]{u}) \\
        &=  \int_0^{\sqrt[4]{u}} \frac{y}{6} dy + \left|\int_{-\sqrt[4]{u}}^0 \frac{y^2}{4} dy \right|\\
        &= \frac{\sqrt{u}}{12} + \frac{\sqrt[4]{u^3}}{12}
    \end{aligned}
\end{equation*} 
\pagebreak
\section*{3.}
\[
    f_Y(y) = \int_y^\infty \frac{1}{9}e^{-x/3} dx = \frac{1}{3}e^{-y/3}    
\]
\begin{equation*}
    \begin{aligned}
        F_U(u) &= P(U\le u) = P(1-Y^2 \le u) = P(Y^2 \ge 1-u) = P(Y\ge \sqrt{1-u}) \\
        &= \int_{\sqrt{1-u}}^\infty \frac{1}{3}e^{-y/3} dy  = e^{-\sqrt{1-u}/3} 
    \end{aligned}
\end{equation*}
\pagebreak
\section*{4.}
\subsection*{a.}
\[
    x = h^{-1}(u) = \frac{u+3}{2} \implies \frac{d}{du}h^{-1}(u) = \frac{1}{2}
\]
If $1>\cfrac{u+3}{2} > 0 \iff -1>u>-3$
\[
    f_U(u) = f_X\left(\frac{u+3}{2}\right) \cdot \frac{1}{2} = \frac{-u-1}{2}  
\]
If $u\le -3$ or $u \ge -1$ then $f_U(u) = 0$
\subsection*{b.}
\[
    x = h^{-1}(v) = \sqrt[3]{v} \implies \frac{d}{dv}h^{-1}(v) = \frac{1}{3} v^{-2/3}     
\]
If $1<\sqrt[3]{v}<0 \iff 1<v<0$
\[
    f_V(v) = f_X(\sqrt[3]{v}) \cdot \frac{1}{3}x^{-2/3} = \frac{2}{3}v^{-2/3}(1-\sqrt[3]{v}) 
\]
If $v\ge 1$ or $v \le 0$ then $f_V(v) = 0$.
\pagebreak
\section*{5.}
\[
    y = h^{-1}(u) = \frac{2-u}{4} \implies \frac{d}{du}h^{-1}(u) = -\frac{1}{4}    
\]
If $-1<\cfrac{2-u}{4}< 1 \iff -2<u<6$
\[
    f_U(u) = f_X\left(\frac{2-u}{4}\right) \cdot \left|-\frac{1}{4}\right| = \frac{6-u}{32} 
\]
\pagebreak
\section*{6.}
Consider $V=Y$
\[
    x = h_1^{-1}(u,v) = \frac{v}{u}, \indent  y = h_2^{-1}(u,v) = v
\]
\[
    \det 
    \begin{vmatrix}
        \cfrac{\partial h_1^{-1}}{\partial u} & \cfrac{\partial h_1^{-1}}{\partial v} \\
        \cfrac{\partial h_2^{-1}}{\partial u} & \cfrac{\partial h_2^{-1}}{\partial v}
    \end{vmatrix}    
    = 
    \det 
    \begin{vmatrix}
        -\cfrac{v}{u^2} & \cfrac{1}{u} \\
        0 & 1 
    \end{vmatrix}
    = -\frac{v}{u^2}
\]
As $y>0$, we have that $v>0$ and hence 
\[
    f_{U,V}(u,v) = f_{X,Y}\left(\frac{v}{u}, v\right) \cdot \left| -\frac{v}{u^2} \right| =  
    \frac{v}{u} e^{-(v/u + v)} \cdot \frac{v}{u^2} = \frac{v^2}{u^3}e^{-(v+uv)/u} 
\]
\pagebreak
\section*{7.}
\subsection*{a.}
For $0<\cfrac{u-v}{2}<y, x<\cfrac{u+v}{2}<1 \iff 2>u>v>0, u<2-v, v<1$.
\[
    x = h_1^{-1}(u,v) = \frac{u-v}{2}, \indent y = h_2^{-1}(u,v) = \frac{u+v}{2}    
\]
\[
    \det 
    \begin{vmatrix}
        \cfrac{\partial h_1^{-1}}{\partial u} & \cfrac{\partial h_1^{-1}}{\partial v} \\
        \cfrac{\partial h_2^{-1}}{\partial u} & \cfrac{\partial h_2^{-1}}{\partial v}
    \end{vmatrix}    
    = 
    \det 
    \begin{vmatrix}
        \cfrac{1}{2} & -\cfrac{1}{2} \\
        \cfrac{1}{2} & \cfrac{1}{2} 
    \end{vmatrix}
    = \frac{1}{2}
\]
\[
    f_{U,V}(u,v) = f_{X,Y}\left( \frac{u-v}{2}, \frac{u+v}{2}\right) \cdot \frac{1}{2} 
    = 6 \cdot \frac{u-v}{2} \cdot \frac{1}{2} = \frac{3(u-v)}{2}     
\]
\subsection*{b.}
For $0<u<1$
\[
    f_U(u) =  \int_0^{u} \frac{3(u-v)}{2} dv
\]
For $1<u<2$
\[
    f_U(u) =  \int_0^{2-u} \frac{3(u-v)}{2} dv
\]
\[
    f_V(v) = \int_u^{2-v} \frac{3(u-v)}{2} du =
\]
\pagebreak
\section*{8.}
\[
    x = h_1^{-1}(u,v) = \frac{uv}{u-1}, \indent y = h_2^{-1}(u,v) = \frac{v}{u-1}
\]
For $0<\cfrac{uv}{u-1}<2, 0<\cfrac{v}{u-1}<3$ 
\[
    \det 
    \begin{vmatrix}
        \cfrac{\partial h_1^{-1}}{\partial u} & \cfrac{\partial h_1^{-1}}{\partial v} \\
        \cfrac{\partial h_2^{-1}}{\partial u} & \cfrac{\partial h_2^{-1}}{\partial v}
    \end{vmatrix}    
    = 
    \det 
    \begin{vmatrix}
        \cfrac{-v}{(u-1)^2} & \cfrac{u}{u-1} \\
        \cfrac{-v}{(u-1)^2} & \cfrac{1}{u-1} 
    \end{vmatrix}
    = \cfrac{v}{(u-1)^2}
\]
If $u>1$ then for $v<\cfrac{2u-2}{u}, v < 3u-3, v>0$ 
\[
    f_{U,V}(u,v) = f_{X,Y}\left(\frac{uv}{u-1}, \frac{v}{u-1}\right) \cdot \frac{v}{(u-1)^2} = \frac{v}{6(u-1)^2}
\]
If $0<u<1$ then for $v>\cfrac{2u-2}{u}, v>3u-3, v<0$
\[
    f_{U,V}(u,v) = f_{X,Y}\left(\frac{uv}{u-1}, \frac{v}{u-1}\right) \cdot \frac{v}{(u-1)^2} = \frac{v}{6(u-1)^2}
\]
For $0<v<2$, 
\[
    f_V(v) = \int_{-2/(v-2)}^\infty \cfrac{v}{6(u-1)^2} du  = -\frac{v-2}{6}
\]
For $-1<v<0$,
\[
    f_V(v) = \int_{0}^{-2/(v-2)} -\cfrac{v}{6(u-1)^2} du = \frac{1}{3}
\]
For $-3<v<-1$,
\[
    f_V(v) = \int_{0}^{(v+3)/3} -\cfrac{v}{6(u-1)^2} du = \frac{v+3}{6}
\]
which is the same as the answer in question 1b
\pagebreak
\section*{9.}
We have that
\[
    m_U(t) = E[e^{tu}] = E[e^{tcY}] = m_Y(tc) = (1-\beta tc)^{-n/2}
\]
If $U \sim \chi_n^2$ then
\[
    m_U(t) = m_\chi(t) \implies (1-\beta tc)^{-n/2} = (1-2t)^{-n/2} \implies c = \frac{2}{\beta}  
\]
\pagebreak
\section*{10.}
\begin{equation*}
    \begin{aligned}
        m_U(t) &= E[e^{t(3y^2-2)}] = e^{-2t}E[e^{3ty^2}] \\
        &= e^{-2t} \int_0^3 e^{3ty^2} \cdot \frac{2y}{9} dy \\
        &= \frac{e^{-2t} (e^{27t}-1)}{27t} \\
        &= \frac{e^{25t} - e^{-2t}}{27t}          
    \end{aligned}
\end{equation*}
which is the moment generating function of Uniform($25t, -2t$) if $t<0$ or Uniform($-2t, 25t$) if $t>0$.
\pagebreak
\section*{11.}
\begin{equation*}
    \begin{aligned}
        m_U(t) &= E[e^{-at\ln(y/x)}] = E\left[\left(\frac{y}{x}\right)^{-at}\right] \\
        &= \int_0^\infty \int_0^x \left(\frac{y}{x}\right)^{-at} \cdot \frac{e^{-x/2}}{4} dydx \\
        &= \int_0^\infty \frac{x^{-at+1}}{-at+1} \cdot \frac{1}{x^{-at}} \cdot \frac{e^{-x/2}}{4} dx \\
        &= \frac{1}{4(-at+1)}\int_0^\infty x \cdot e^{-x/2} \\
        &= \frac{1}{4(-at+1)} \cdot 4 \\
        &= \frac{1}{-at+1}
    \end{aligned}
\end{equation*} 
which is the moment generating function of Exponential($a$)
\pagebreak
\section*{12.}
We have that
\[
    F_Y(y) = y^5    
\]
and hence
\[
    f_{Y_7}(y) = 7(F_Y(y))^6 f_Y(y) = 35y^{34}
\]
Therefore, the expected value of the biggest bonus is 
\[
    \int_0^1 y \cdot 35y^{34} dy = \frac{35}{36}y^{36} |^1_0 = \frac{35}{36}  
\]
\pagebreak
\section*{13.}
\[
    f_X(x) = \int_0^\infty \cfrac{1}{\beta \theta} e^{-\left(x/\beta +y/\theta\right)} dy = \frac{e^{-x/\beta}}{\beta}
\]
\[
    f_Y(y) = \int_0^\infty \cfrac{1}{\beta \theta} e^{-\left(x/\beta +y/\theta\right)} dx = \frac{e^{-y/\theta}}{\theta}
\]
and hence $X$ and $Y$ is independent.
Let $U = \min(X,Y)$, then
\begin{equation*}
    \begin{aligned}
        F_U(u) &= P(U\le u) = 1- P(X\ge u, Y \ge u) \\
        &= 1- P(X\ge u)P(Y\ge u) \\
        &= 1- \int_u^\infty \frac{e^{-x/\beta}}{\beta} dx \cdot \int_u^\infty \frac{e^{-y/\theta}}{\theta} dy  \\
        &= 1 - e^{-u/\beta} \cdot  e^{-u/\theta}
    \end{aligned}
\end{equation*}
Therefore, 
\[
    f_U(u) = -e^{-u/\beta-u/\theta}\left(-\frac{1}{\beta}-\frac{1}{\theta}\right)
\]
and hence
\begin{equation*}
    \begin{aligned}
        E[U] &= \int_0^\infty u \cdot -e^{-u/\beta-u/\theta}\left(-\frac{1}{\beta}-\frac{1}{\theta}\right) du \\
        &= \left(-\frac{1}{\beta}- \frac{1}{\theta}\right) \cdot \left(-\frac{1}{\left(\frac{1}{\beta} + \frac{1}{\theta}\right)^2} \right) \\ 
        = \frac{1}{\frac{1}{\beta} + \frac{1}{\theta}} = \frac{\beta \theta}{\beta + \theta}
    \end{aligned}
\end{equation*}
\pagebreak
\section*{14.}
Let $U = \min(X,Y)$, then
\begin{equation*}
    \begin{aligned}
        F_U(u) &= P(U\le u) = 1- P(X\ge u, Y\ge u) \\ 
        &= 1 - \int_u^1 \int_u^1 \frac{4}{3}x + \frac{2}{3}y dxdy \\
        &=  -u^3+u^2+u
    \end{aligned}
\end{equation*}
and hence
\[
    f_U(u) = -3u^2 + 2u + 1 
\]
Therefore, 
\[
    E[U] = \int_0^1 u(-3u^2+2u+1) du = \frac{5}{12} 
\]
\pagebreak
\section*{15.}
We have that 
\begin{equation*}
    \begin{aligned}
        f_{Y_1, Y_2}(y_1, y_2) &= \frac{2!}{0!} [F_{Y_2}(y_2) - F_{Y_1}(y_1)]^0 f_Y(Y_2) \cdot f_Y(y_1) \\
        &= 2 \cdot \frac{1}{\beta}e^{-y_1/\beta} \cdot \frac{1}{\beta}e^{-y_2/\beta}  \\
        &= 2 \cdot \frac{1}{\beta^2}e^{-(y_1+y_2)/\beta}
    \end{aligned}
\end{equation*}
Therefore, 
\begin{equation*}
    \begin{aligned}
        P(3Y_1 > Y_1 > Y_2) 
        &= \int_0^\infty \int_{y_1}^{3y_1} 2 \cdot \frac{1}{\beta^2}e^{-(y_1+y_2)/\beta} dy_2 dy_1 \\
        P(3Y_1 > Y_2>Y_1) 
        &= \int_0^\infty \int_{0}^{y_1} 2 \cdot \frac{1}{\beta^2}e^{-(y_1+y_2)/\beta} dy_2 dy_1 \\
        P(3Y_1 > Y_2) 
        &= P(3Y_1 > Y_2 | Y_1>Y_2) \cdot P(Y_1>Y_2) \\
        &+ P(3Y_1>Y_2|Y_2>Y_1)\cdot P(Y_2>Y_1) \\
        &= 0.5 \int_0^\infty \int_{y_1}^{3y_1} 2 \cdot \frac{1}{\beta^2}e^{-(y_1+y_2)/\beta} dy_2 dy_1 \\
        &= \int_0^\infty -\frac{\left(e^{-\frac{4y_1}{\beta}}-e^{-\frac{y_1}{\beta}}\right)}{\beta} dy_1 \\
        &= -\frac{1}{\beta}\left(\frac{\beta}{4}-\beta\right) \\
        &= \frac{3}{4}
    \end{aligned}
\end{equation*}
\end{document}