\documentclass[11pt]{article}
    \title{\textbf{Math 217 Homework I}}
    \author{Khac Nguyen Nguyen}
    \date{}
    
    \addtolength{\topmargin}{-3cm}
    \addtolength{\textheight}{3cm}
    
\usepackage{amsmath}
\usepackage{mathtools}
\usepackage{amsthm}
\usepackage{amssymb}
\usepackage{pgfplots}
\usepackage{xfrac}
\usepackage{hyperref}



\newtheorem{definition}{Definition}[section]
\newtheoremstyle{mystyle}%                % Name
  {}%                                     % Space above
  {}%                                     % Space below
  {\itshape}%                                     % Body font
  {}%                                     % Indent amount
  {\bfseries}%                            % Theorem head font
  {}%                                    % Punctuation after theorem head
  { }%                                    % Space after theorem head, ' ', or \newline
  {\thmname{#1}\thmnumber{ #2}\thmnote{ (#3)}}%                                     % Theorem head spec (can be left empty, meaning `normal')

\theoremstyle{mystyle}
\newtheorem{theorem}{Theorem}[section]
\theoremstyle{definition}
\newtheorem*{exmp}{Example}
\begin{document}
\section*{1.}
\begin{align*}
  (I-P_X)_{i,i} SS_{\text{res}} = [e_i^T (I-P_X) e_i] [y^T (I-P_X)y] \ge [e_i^T(I-P_X)y]^2 = r^2_i
\end{align*}
from the hints and
\[
  SS_{\text{res}} = r^T r = (y^T - \hat y^T) r = y^T r - \underbrace{0}_{\text{since } x^T r = 0} = y^T(I-P_X)y
\]
Thus we have 
\[
  |s_i| \le \displaystyle\frac{|r_i|}{\sqrt{r_i^2/(n-p-1)}} = \sqrt{n-p-1}
\]
\clearpage
\section*{2.}
\subsection*{a.}
\[
  V[Y] = \mu^2 \implies s(\mu) = \mu
\]
\[
  T(\mu) = \int \displaystyle\frac{1}{\mu} d\mu = \ln(\mu)
\]
\subsection*{b.}
\[
  V[Y] = \mu^k \implies s(\mu) = \mu^{k/2}
\]
\[
  T(\mu) = \int \displaystyle\frac{1}{\mu^{k/2}} d\mu = \frac{\mu^{-k/2 + 1}}{-k/2 + 1}
\]
\subsection*{c.}
\[
  V[Y] =e^\mu \implies s(\mu) = e^{\mu/2}
\]
\[
  T(\mu) = \int \displaystyle\frac{1}{e^{\mu/2}} d\mu = -\displaystyle\frac{2}{e^{\mu/2}}
\]
\subsection*{d.}
\[
  V[Y] = \mu^{-2} \implies s(\mu) = \mu^{-1}
\]
\[
  T(\mu) = \int \mu d\mu = \mu^2
\]
\subsection*{e.}
\[
  V[Y] = \displaystyle\frac{\mu^2}{e^{2\mu}} \implies s(\mu) = \displaystyle\frac{\mu}{e^\mu}
\]
\[
  T(\mu) = \int \displaystyle\frac{e^\mu}{\mu} d\mu 
\]
which is not integrable, though we can express $e^\mu$ as a Maclaurin series and take the antiderivative terms by terms to get 
\[
  T(\mu) = \int \displaystyle\frac{e^\mu}{\mu} d\mu =\ln|x| + \sum_{n=1}^\infty \displaystyle\frac{x^n}{n \cdot n!}
\]
\clearpage
\section*{3.}
\subsection*{a.}
We have that 
\[
  \displaystyle\frac{1}{n} \sum_{i=1}^n V(\varepsilon_i) = \sigma_1^2
\]
and
\[
  \displaystyle\frac{1}{m} \sum_{i=m+1}^{m+n} V(\varepsilon_i) = \sigma_2^2
\]
Thus 
\[
  \displaystyle\frac{1}{n+m} \sum_{i=1}^{m+n} V(\varepsilon_i) = \displaystyle\frac{n}{m+n} \sigma_1^2 + \displaystyle\frac{m}{m+n} \sigma_2^2
\]
and we can find 
\[
  \delta = \displaystyle\frac{n}{m+n} < 1
\]
\subsection*{b.}
We have that 
\begin{align*} 
  E[SS_{\text{res}}] &= E[Y^T (I-P_X)Y]  \\
  &= \text{Tr}(E[Y^T (I-P_X)Y]) \\
  &= \text{Tr}(I-P_X) E[YY^T] \\
  &= \text{Tr}((I-P_X)YY^T) + \text{Tr}((I-P_X)\Sigma)  \\
  &= E[Y^T] (I-P_X)E[Y] + \text{Tr} ((I-P_X) \Sigma) \\
  &= 0 + \sigma_1^2 n + \sigma_2^2 m - \sigma_1^2 \sum_{i=1}^n P_{X_{i,i}} - \sigma_2^2 \sum_{i=n+1}^{m+n} P_{X_{i,i}}
\end{align*}
since $(I-P_X)E[Y]= 0$. 
Thus 
\[
  E\left[\frac{SS_{\text{res}}}{(n+m-p-1)}\right] = \displaystyle\frac{n- \sum_{i=1}^n P_{X_{i,i}}}{n+m-p-1}\sigma_1^2 + \displaystyle\frac{m-\sum_{i=n+1}^{m+n} P_{X_{i,i}}}{n+m-p-1} \sigma_2^2  
\]
Thus since $\sum_{i=1}^{m+n} P_{X_{i,i}} = p+1$, we have 
\[
  E\left[\frac{SS_{\text{res}}}{(n+m-p-1)}\right] - \sigma_1^2 =   \displaystyle\frac{m-\sum_{i=n+1}^{m+n} P_{X_{i,i}}}{n+m-p-1} (\sigma_2^2 -\sigma_1^2) < 0
\]
and similarly, 
\[
  E\left[\frac{SS_{\text{res}}}{(n+m-p-1)}\right] - \sigma_2^2 =   \displaystyle\frac{m-\sum_{i=1}^{n} P_{X_{i,i}}}{n+m-p-1} (\sigma_1^2 -\sigma_2^2) < 0
\]
\subsection*{c.}
We first have additionally, 
\[
  \displaystyle\frac{1}{k} \sum_{m+n+1}^{m+n+k} V(\varepsilon_i) = \sigma_3^2
\]
and thus 
\[
  \displaystyle\frac{1}{m+n+k} \sum_{i=1}^{m+n+k}V(\varepsilon_i) = \displaystyle\frac{n}{m+n+k} \sigma_1^2 + \displaystyle\frac{m}{m+n+k}\sigma_2^2 + \displaystyle\frac{k}{m+n+k}\sigma_3^2
\]
\end{document}
