\documentclass[11pt]{article}
    \title{\textbf{Math 217 Homework I}}
    \author{Khac Nguyen Nguyen}
    \date{}
    
    \addtolength{\topmargin}{-3cm}
    \addtolength{\textheight}{3cm}
    
\usepackage{amsmath}
\usepackage{mathtools}
\usepackage{amsthm}
\usepackage{amssymb}
\usepackage{pgfplots}
\usepackage{xfrac}
\usepackage{hyperref}



\newtheorem{definition}{Definition}[section]
\newtheoremstyle{mystyle}%                % Name
  {}%                                     % Space above
  {}%                                     % Space below
  {\itshape}%                                     % Body font
  {}%                                     % Indent amount
  {\bfseries}%                            % Theorem head font
  {}%                                    % Punctuation after theorem head
  { }%                                    % Space after theorem head, ' ', or \newline
  {\thmname{#1}\thmnumber{ #2}\thmnote{ (#3)}}%                                     % Theorem head spec (can be left empty, meaning `normal')

\theoremstyle{mystyle}
\newtheorem{theorem}{Theorem}[section]
\theoremstyle{definition}
\newtheorem*{exmp}{Example}
\begin{document}
\section*{1.}
We first have, 
\[
  F = \displaystyle\frac{SS_{\exp}/p}{SS_{\text{res}}/(n-p-1)}
\]
hence
\[
  \displaystyle\frac{F}{F+c} = \displaystyle\frac{1}{1 + c/F}
\]
and  
\[
  1 - \displaystyle\frac{SS_{\text{res}}}{SS_{\text{total}}} = \displaystyle\frac{SS_{\exp}}{SS_{\text{total}}}
\]
Therefore,
\begin{align*} 
  &1 + \displaystyle\frac{c}{F} = \frac{SS_\text{total}}{SS_{\exp}}\\
  \implies &1 + \displaystyle\frac{c(n-p-1)}{p} \displaystyle\frac{SS_\text{res}}{SS_{\exp}} = \frac{SS_\text{total}}{SS_{\exp}}\\
  \implies &SS_{\exp}+ \displaystyle\frac{c(n-p-1)}{p} SS_\text{res} = SS_\text{total} \\
  \implies &c = \displaystyle\frac{p}{n-p-1}
\end{align*}
We then have 
\[
  5 = \displaystyle\frac{8}{n-8-1} \implies n = 10.6 
\]
\pagebreak
\section*{2.}
We first have that 
\[
  X^T X = 
  \begin{pmatrix}
    n & \sum x_i \\
    \sum x_i & \sum x_i^2
  \end{pmatrix}
  = 
  \begin{pmatrix}
    45 & 0 \\
    0 & \sum x_i^2
  \end{pmatrix}
\]
and thus the inverse is 
\[
  \begin{pmatrix}
    \displaystyle\frac{1}{45} & 0 \\
    0 & \displaystyle\frac{1}{\sum x_i^2}
  \end{pmatrix}
\]
then we can calculate 
\[
  \text{se}(c) = \sqrt{\displaystyle\frac{1}{45} SS_\text{res} / 43}
\]
where
\[
  SS_\text{res} = \sum_{i=1}^{45} (Y_i - \hat Y_i)^2
\]
Thus, we can find the confidence interval for $c$
\[
  \hat c - t_{\alpha/2, 43} \sqrt{\displaystyle\frac{SS_\text{res}}{1935}} < c < \hat c + t_{\alpha/2, 43} \sqrt{\displaystyle\frac{SS_\text{res}}{1935}}
\]
Similarly, 
\[
  \text{se}(c) = \sqrt{\displaystyle\frac{1}{\sum x_i^2} SS_\text{res} / 43}
\]
and the confidence for $d$ is 
\[
  \hat d - t_{\alpha/2, 43} \sqrt{\displaystyle\frac{SS_\text{res}}{43 \sum_{i=1}^45 x_i^2}} < d < \hat d + t_{\alpha/2, 43} \sqrt{\displaystyle\frac{SS_\text{res}}{43 \sum_{i=1}^45 x_i^2}}
\]
\end{document}
