\documentclass[11pt]{article}
    \title{\textbf{Math 217 Homework I}}
    \author{Khac Nguyen Nguyen}
    \date{}
    
    \addtolength{\topmargin}{-3cm}
    \addtolength{\textheight}{3cm}
    
\usepackage{amsmath}
\usepackage{mathtools}
\usepackage{amsthm}
\usepackage{amssymb}
\usepackage{pgfplots}
\usepackage{xfrac}
\usepackage{hyperref}


\usepackage{tikz}
\usetikzlibrary{plotmarks}


\newtheorem{definition}{Definition}[section]
\newtheoremstyle{mystyle}%                % Name
  {}%                                     % Space above
  {}%                                     % Space below
  {\itshape}%                                     % Body font
  {}%                                     % Indent amount
  {\bfseries}%                            % Theorem head font
  {}%                                    % Punctuation after theorem head
  { }%                                    % Space after theorem head, ' ', or \newline
  {\thmname{#1}\thmnumber{ #2}\thmnote{ (#3)}}%                                     % Theorem head spec (can be left empty, meaning `normal')

\theoremstyle{mystyle}
\newtheorem{theorem}{Theorem}[section]
\theoremstyle{definition}
\newtheorem*{exmp}{Example}



\begin{document}
\section*{3.2}
\subsection*{iii.}
\subsubsection*{a.}
\[
  A = 
  \begin{pmatrix}
    1 & 1 \\
    -1 & 0
  \end{pmatrix}
\]
\[
  |A-\lambda I |= 
  \begin{vmatrix}
    1 - \lambda & 1 \\
    -1 & -\lambda
  \end{vmatrix}
  = \lambda^2 - \lambda + 1 
  \implies \lambda = \displaystyle\frac{1 \pm \sqrt{3}i}{2}
\]
For $\lambda = \displaystyle\frac{1 + \sqrt{3}i}{2}$, the eigenvector is 
$\begin{pmatrix}
  \displaystyle\frac{-\sqrt{3}i-1}{2} \\
  1
\end{pmatrix}
$. \\
For $\lambda = \displaystyle\frac{1 - \sqrt{3}i}{2}$, the eigenvector is 
$\begin{pmatrix}
  \displaystyle\frac{\sqrt{3}i-1}{2} \\
  1
\end{pmatrix}
$
\subsection*{b.}
Thus 
\[
  T = 
  \begin{pmatrix}
    \frac{\sqrt{3}i - 1}{2} & \frac{-\sqrt{3}i -1}{2} \\
    1 & 1
  \end{pmatrix}
\]
\subsection*{c.}
First, let's look at the solution of 
\[
  X' = AX
\]
Based on the eigenvector the general solution is 
\[
  X(t) = c_1 e^{t/2} 
  \begin{pmatrix}
    \cos (\sqrt{3}t/2) \\
    - \sin (\sqrt{3}t/2) \\
  \end{pmatrix}
  + c_2 e^{t/2}
  \begin{pmatrix}
    \sin (\sqrt{3}t/2) \\
    \cos (\sqrt{3}t/2) \\
  \end{pmatrix}
\]
and we have 
\[
  T^{-1} A T = 
  \begin{pmatrix}
    \frac{1-i\sqrt{3}}{2} & 0 \\
    0 & \frac{1+i\sqrt{3}}{2} 
  \end{pmatrix}
\]
which yields 
\[
  Y(t) = c_1 e^{t/2} 
  \begin{pmatrix}
    \cos (\sqrt{3}t/2) \\
    - \sin (\sqrt{3}t/2) \\
  \end{pmatrix}
  + c_2 e^{t/2}
  \begin{pmatrix}
    \sin (\sqrt{3}t/2) \\
    \cos (\sqrt{3}t/2) \\
  \end{pmatrix}
\]
\subsubsection*{d.}
Spiral source for both system (taken from book):
\begin{figure}[h]
  \centering
  \includegraphics[width=0.8\textwidth]{/home/dani/Desktop/2024-10-06_17-23.png}
  \caption{}
  \label{fig:2024-10-06_17-23}
\end{figure}
\subsection*{iv.}
\subsubsection*{a.}
\[
  A = 
  \begin{pmatrix}
    1 & 1 \\
    -1 & 3
  \end{pmatrix}
\]
\[
  |A-\lambda I |= 
  \begin{vmatrix}
    1 - \lambda & 1 \\
    -1 & 3-\lambda
  \end{vmatrix}
  = \lambda^2 - 4\lambda + 4 
  \implies \lambda = 2
\]
For $\lambda = 2$, the eigenvector is 
$\begin{pmatrix}
  -1 \\
  0
\end{pmatrix}
$
and 
$ 
\begin{pmatrix}
  1 \\
  1
\end{pmatrix}
$
\subsection*{b.}
Thus 
\[
  T = 
  \begin{pmatrix}
    1 & -1\\
    1 & 0
  \end{pmatrix}
\]
\subsection*{c.}
First, let's look at the solution of 
\[
  X' = AX
\]
Based on the eigenvector the general solution is 
\[
  X(t) = c_1 e^{2t} 
  \begin{pmatrix}
    -1 \\
    0 
  \end{pmatrix}
  + c_2 \left(te^{2t} 
  \begin{pmatrix}
    -1 \\
    0
  \end{pmatrix}
  + e^{2t}
  \begin{pmatrix}
    1 \\
    1 
  \end{pmatrix}
    \right)
\]
and we have 
\[
  T^{-1} A T = 
  \begin{pmatrix}
    2 & 1 \\
    0 & 2 
  \end{pmatrix}
\]
which has eigenvalue $2$ with eigenvector $ \begin{pmatrix}
  1 \\ 0
\end{pmatrix}$ and $ \begin{pmatrix}
  0 \\ 1
\end{pmatrix}$,  thus 
\[
  Y(t) = c_1 e^{2t} 
  \begin{pmatrix}
    1 \\
    0 
  \end{pmatrix}
  + c_2 \left(te^{2t} 
  \begin{pmatrix}
    1 \\
    0
  \end{pmatrix}
  + e^{2t}
  \begin{pmatrix}
    0 \\
    1 
  \end{pmatrix}
    \right)
\]
\subsubsection*{d.}
Portrait graphs for both system, (graph is taken from book and is wrong, to make it right, change the direction of the arrow since it is a source instead of sink as the eigenvalue is positive).
\begin{figure}[h]
  \centering
  \includegraphics[width=0.8\textwidth]{/home/dani/Desktop/2024-10-06_17-30.png}
  \caption{}
  \label{fig:2024-10-06_17-30}
\end{figure}
\subsection*{v.}
\subsubsection*{a.}
\[
  A = 
  \begin{pmatrix}
    1 & 1 \\
    -1 & -3
  \end{pmatrix}
\]
\[
  |A-\lambda I |= 
  \begin{vmatrix}
    1 - \lambda & 1 \\
    -1 & -3-\lambda
  \end{vmatrix}
  = \lambda^2 + 2\lambda - 4 
  \implies \lambda = -1 \pm \sqrt{3} 
\]
For $\lambda = -1 -\sqrt{3}$, the eigenvector is 
$\begin{pmatrix}
  -2 + \sqrt{3} \\
  1
\end{pmatrix}
$. \\
For $\lambda = -1 + \sqrt{3}$, the eigenvector is 
$\begin{pmatrix}
  -2 - \sqrt{3} \\
  1
\end{pmatrix}
$
\subsection*{b.}
Thus 
\[
  T = 
  \begin{pmatrix}
    -2 + \sqrt{3} & - 2 - \sqrt{3} \\
    1 & 1
  \end{pmatrix}
\]
\subsection*{c.}
First, let's look at the solution of 
\[
  X' = AX
\]
Based on the eigenvector the general solution is 
\[
  X(t) = c_1 e^{t(-1-\sqrt{3})} 
  \begin{pmatrix}
    -2 + \sqrt{3} \\
    1 \\
  \end{pmatrix}
  + c_2 e^{t(-1 + \sqrt{3})}
  \begin{pmatrix}
    -2 - \sqrt{3} \\
    1 \\
  \end{pmatrix}
\]
and we have 
\[
  T^{-1} A T = 
  \begin{pmatrix}
    -1 - \sqrt{3} & 0 \\
    0 & -1 + \sqrt{3} 
  \end{pmatrix}
\]
which yields 
\[
  Y(t) = c_1 e^{t(-1-\sqrt{3})} 
  \begin{pmatrix}
    -1 + \sqrt{3} \\
    1 \\
  \end{pmatrix}
  + c_2 e^{t(-1 + \sqrt{3})}
  \begin{pmatrix}
    -1 - \sqrt{3} \\
    1 \\
  \end{pmatrix}
\]
\subsubsection*{d.}
For both systems, 
\begin{figure}[h]
  \centering
  \includegraphics[width=0.8\textwidth]{/home/dani/Desktop/IMG_1991.png}
  \caption{}
  \label{fig:img_1991}
\end{figure}
\clearpage
\section*{3.3}
\subsection*{a.}
\[
  x'' + x' + x = 0 
\]
The charateristics equation is $r^2 + r + 1 = 0$ thus 
\[
  r = \displaystyle\frac{-1 \pm \sqrt{3}i}{2}
\]
and the general solution is 
\[
  x(t) = c_1 e^{-t/2}\cos(\sqrt{3}t/2) + c_2 e^{-t/2} \sin(\sqrt{3}t/2)
\]
\subsection*{b.}
\[
  x'' + 2x' + x = 0 
\]
The charateristics equation is $r^2 + 2r + 1 = 0$ thus 
\[
  r = -1
\]
and the general solution is 
\[
  x(t) = c_1 e^{-t} + c_2 te^{-t}
\]
\clearpage
\section*{3.5}
The charateristics polynomial is 
\[
  \lambda^2 -\lambda(2+a) = 0 \implies \lambda = 0,2+a
\]
For eigenvalue 0, the eigenvector is 
\[
  \begin{pmatrix}
    -1 \\
    a
  \end{pmatrix}
\]
For eigenvalue $2+a$, the eigenvector is 
\[
  \begin{pmatrix}
    1 \\
    2
  \end{pmatrix}
\]
Thus the bifurcation point is $a=-2$. \\
\begin{itemize}
  \item $a > -2$
    As the eigenvector $(-1,a)$ is a constant in the solution. 

\begin{figure}[h]
  \centering
  \includegraphics[width=0.8\textwidth]{/home/dani/Desktop/IMG_1993.png}
  \caption{}
  \label{fig:img_1993}
\end{figure}
  \item $a = -2$
    Let $(1,0)$ be the other eigenvector, then the solution is just 
    $c_1 (1,0) + c_2 (1,2)$ which is just constant everywhere. 
\begin{figure}[h]
  \centering
  \includegraphics[width=0.8\textwidth]{/home/dani/Desktop/IMG_1994.png}
  \caption{}
  \label{fig:img_1994}
\end{figure}
\clearpage
  \item $a < -2$
\begin{figure}[h]
  \centering
  \includegraphics[width=0.8\textwidth]{/home/dani/Desktop/IMG_1995.png}
  \caption{}
  \label{fig:img_1995}
\end{figure}
\end{itemize}
\clearpage
\section*{3.11}
Let $A$ be the matrix in the equation
\[
  |A-\lambda I | = (a-\lambda)(d-\lambda) - bc = -\lambda(a+d) + \lambda^2 = 0
\]
Thus 
\[
  \lambda \in \{0, a+d\}
\]
For $\lambda = 0$, the eigenvector will be 
$\begin{pmatrix}
  d \\
  -c
\end{pmatrix}
$ \\
For $\lambda = a+d$, the eigenvector will be 
$
\begin{pmatrix}
  b \\
  d
\end{pmatrix}
$
which we can use to obtain the general solution 
\[
  X(t) = c_1 \begin{pmatrix}
    d \\
    -c
  \end{pmatrix} + c_2 e^{a+d} 
  \begin{pmatrix}
    b \\ 
    d
  \end{pmatrix}
\]
Since $c_1 
\begin{pmatrix}
  d \\
  -c
\end{pmatrix}
$ is constant, the curves should eventually converge to a line parallel with 
$ \text{sign}(c_2) \cdot 
\begin{pmatrix}
  b \\
  d
\end{pmatrix}$. \\ 
Thus, if $a+d > 0$, it is a source, and every curves should eventually converges to said vector. \\
If $a+d = 0$, then it is a source and every curves is a line. \\
If $a+d < 0$, then it is a sink and every curves should eventually converges to said vector.
\clearpage
\section*{3.13}
Let the matrix be 
\[
  A = 
  \begin{pmatrix}
    a & b \\
    c & d 
  \end{pmatrix}
\]
Since we know the charateristics polynomial is 
\[
  \lambda^2 - (a+d) \lambda + ad - bc = 0
\]
We can substitute $\alpha = -a-d, \beta = ad-bc$ in and have 
\begin{align*}
  &A^2 - (a+d)A + (ad-bc)I \\
  =& 
  \begin{pmatrix}
    a^2 +bc & ab + bd \\
    ac + cd & bc + d^2
  \end{pmatrix}
  - 
  \begin{pmatrix}
    a(a+d) & b(a+d) \\
    c(a+d) & d(a+d) 
  \end{pmatrix}
  +
  (ad-bc)I \\
  =&
  \begin{pmatrix}
    0 & 0 \\
    0 & 0
  \end{pmatrix}
\end{align*}
\clearpage 
\section*{4.1}
$T = a$, $D = \displaystyle\frac{a^2}{4} - 2$ and the charateristics polynomial is 
\[
  -a\lambda + \lambda^2 - 2 + \displaystyle\frac{a^2}{4} = 0
\]
Thus, there is 2 real eigenvalue as 
\[
  a^2 - 4\left(-2 + \displaystyle\frac{a^2}{4}\right) = 8 > 0  
\]
Now the eigenvalue are 
\[
  \lambda_{1,2} = \displaystyle\frac{a \pm \sqrt{8}}{2}
\]
where $\lambda_1 > \lambda_2$. Thus there are three section. 
\begin{itemize}
  \item $\lambda_1 > \lambda_2 > 0$, $a > \sqrt{8}$. 
  \item $\lambda_1 >0 > \lambda_2 $, $-\sqrt{8} < a < \sqrt{8}$ 
  \item $0 > \lambda_1 > \lambda_2 > 0$, $a < -\sqrt{8}$
\end{itemize}
and 2 subsection 
\begin{itemize}
  \item $\lambda_1 > \lambda_2 = 0, a = \sqrt{8}$, then the solution has a constant vector and the all curves should converges to parallel with the other one.
  \item $\lambda_2 < \lambda_1 = 0, a = -\sqrt{8}$, then the solution has a constant vector and the all curves should converges to parallel with the other one.
\end{itemize}
Hence, the trace determinant plane is 
\begin{figure}[h]
  \centering
  \includegraphics[width=0.8\textwidth]{/home/dani/Desktop/IMG_2004.png}
  \caption{}
  \label{fig:img_2004}
\end{figure}
\clearpage 
\section*{4.2}
\[
  T = 2a, D = a^2-b^2
\]
\[
  T^2 - 4D = 4a^2 -4a^2 + 4b^2 = 4b^2\ge 0
\]
and the charateristics polynomial is 
\[
  a^2 - 2a\lambda + \lambda^2 -b^2 = 0 
\]
Thus if $b = 0$, then there is only 1 eigenvalue which is $a$ (which the corresponding eigenvector is $(1,0),(0,1)$. \\ 
If $b \ne 0$, then there is 2 eigenvalue 
\[
  \lambda_{1,2} = a \pm |b|
\]
If $b > 0$,  
\[
  \lambda_{1} = a + b > 
  \lambda_{2} = a - b
\]
If $b < 0$, 
\[
  \lambda_{1} = a - b > \lambda_2 = a+b 
\]
The eigenvector for eigenvalue $ a + b$ is
\[
  \begin{pmatrix}
    1 \\ 1
  \end{pmatrix}
\]
and the eigenvector for eigenvalue $ a - b$ is \[
  \begin{pmatrix}
    1 \\ -1
  \end{pmatrix}
\]
Thus we have 4 regions in the ab plane 

\begin{figure}[h]
  \centering
  \includegraphics[width=0.8\textwidth]{/home/dani/Desktop/IMG_2005.png}
  \caption{}
  \label{fig:img_2005}
\end{figure}

\clearpage
\section*{4.3}
The charateristics equation is $r^2 +br + k$ and thus 
have  
\begin{itemize}
  \item 2 real solutions if $b^2 > 4k$ \\
    We know that $r_{1,2} = -b \pm \sqrt{b^2 -4k}$
    , $r_1 = -b - \sqrt{b^2-4k} < 0$ and $r_2 = -b + \sqrt{b^2 - 4k} < 0$. Thus $b^2 > 4k$ has similar portraits. 
  \item 1 real duplicated solution if $b = 2\sqrt{k}$, and. There is obviously 1 portraits here.  
  \item 2 complex solution if $b^2 < 4k$. 
    Since $b>0$, the region $b^2 < 4k$ also has similar portraits as $b>0$. The real parts of the solution of the charateristics polynomial is always positive.
\end{itemize}
\clearpage
\section*{4.5}
\subsection*{a.}
We first put them into canonical form. \\
The first matrix $A$ has eigenvalue $2$ with eigenvector 
\[
  \begin{pmatrix}
    1 \\
    3
  \end{pmatrix}
\]
and eigenvalue $-1$ with eigenvector 
\[
  \begin{pmatrix}
    1 \\ 
    0
  \end{pmatrix}
\]
Thus we can find the canonical form 
\[
 T_1^{-1}AT = 
 \begin{pmatrix}
  -1 & 0 \\
  0 & 2
\end{pmatrix}  
\]
where
\[
  T_1 = 
  \begin{pmatrix}
    1 & 1 \\
    0 & 3
  \end{pmatrix}
\]
The second matrix $B$ has eigenvalue $-2$ with eigenvector 
\[
  \begin{pmatrix}
    0 \\
    1
  \end{pmatrix}
\]
and eigenvalue $1$ with eigenvector 
\[
  \begin{pmatrix}
    3 \\ 
    1
  \end{pmatrix}
\]
Thus we can find the canonical form 
\[
 T_2^{-1} A T_2 = \begin{pmatrix}
  -2 & 0 \\
  0 & 1
\end{pmatrix}  
\]
where 
\[
  T_2 = 
  \begin{pmatrix}
    0 & 3 \\
    1 & 1
  \end{pmatrix}
\]
Thus we can obtain the conjugacy
\[
  H(x,y) = T_2 T_1^{-1} H'(x,y)
\]
with $H'$ being the conjugacy between the 2 canonical form of the 2 matrices and 
\[
  H'(x,y) = (h_1(x), h_2(y))
\]
where 
\[
  h_1(x) = 
  \begin{cases}
    x^2, &\text{ if } x \ge 0\\
    -x^2, &\text{ if } x < 0\\
  \end{cases}
\]
and 
\[
  h_2(y) = 
  \begin{cases}
    y^2, &\text{ if } y \ge 0 \\
    -y^2, &\text{ if } y < 0
  \end{cases}
\]
The reason this works is explained in first 6 lines in 4.6.
\subsection*{b.}
We first put them into canonical form. \\
The first matrix $A$ has eigenvalue $2i$ with eigenvector 
\[
  \begin{pmatrix}
    -i \\
    2
  \end{pmatrix}
\]
and eigenvalue $-2i$ with eigenvector 
\[
  \begin{pmatrix}
    i \\ 
    2
  \end{pmatrix}
\]
Thus we can find the canonical form 
\[
 T_1^{-1}A T_1 = \begin{pmatrix}
  -2i & 0 \\
  0 & 2i
\end{pmatrix}  
\]
where 
\[
  T_1 = 
  \begin{pmatrix}
    i & -i \\
    2 & 2
  \end{pmatrix}
\]
The second matrix $B$ has eigenvalue $2i$ with eigenvector 
\[
  \begin{pmatrix}
    -i \\
    1
  \end{pmatrix}
\]
and eigenvalue $-2i$ with eigenvector 
\[
  \begin{pmatrix}
    i \\ 
    1
  \end{pmatrix}
\]
Thus we can find the canonical form 
\[
  T_2^{-1} A T_2 = \begin{pmatrix}
  -2i & 0 \\
  0 & 2i
\end{pmatrix}  
\]
where 
\[
  T_2 = 
  \begin{pmatrix}
    i & -i \\
    1 & 1
  \end{pmatrix}
\]
Thus we can obtain the conjugacy
\[
  H(x,y) = T_2 \circ T_1^{-1}
\]
since they have the same Jordan canonical form. 
\clearpage
\section*{4.6}
If 2 linear system $X' = AX$ and $Y' = BY$ have the same eigenvalue $\pm i\beta \ne 0$, then we know that both matrix have the same canonical form 
\[
  C = \begin{pmatrix}
    0 & \beta \\
    -\beta & 0 
  \end{pmatrix}
\]
Thus the conjugacy is $T_2 T_1^{-1}$ since $T_1^{-1}$ send the solution of $X' =AX$ to $Z' = CZ$ and $T_2$ send the solution of $Z' = CZ$ to $Y' = BY$ (this can also be applied to 4.5). \\
If they have different eigenvalue $\pm i\beta$ and $\pm i\gamma$ then WLOG assume $|\gamma| > |\beta|$ then the solution of both consists of $\sin$ and $\cos$ with period $\displaystyle\frac{2\pi}{|\gamma|}$ or $\displaystyle\frac{2\pi}{|\beta|}$  
\[
  \phi^A(t,X_0) = \phi^A(t + \displaystyle\frac{2\pi}{\beta}, X_0)
\]
\[
  \phi^B(t,X_0) = \phi^B(t + \displaystyle\frac{2\pi}{\gamma}, X_0)
\]
Thus there is no conjugacy. \\
$\gamma = - \beta$ means that the systems have the same eigenvalue which is the first scenario. 
\end{document}
