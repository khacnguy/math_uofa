	\documentclass[11pt]{article}
    \title{\textbf{Math 328 Homework I}}
    \author{Khac Nguyen Nguyen}
    \date{}
    
    \addtolength{\topmargin}{-3cm}
    \addtolength{\textheight}{3cm}
    
\usepackage{amsmath}
\usepackage{mathtools}
\usepackage{amsthm}
\usepackage{amssymb}


\begin{document}	
\section*{1.}
We use induction to prove that for a group $G, g \in G,$ and two integers $a, b$,
\begin{equation*}
g^{a+b} = g^a \cdot g^b 
\end{equation*}
$\forall g \in G \; \forall a \in \mathbb{Z} \; \forall b \in \mathbb{Z}^+ \cup \{0\}$, we have \\~\\
$\textbf{Base case: } b=0$
\[
g^{a+0} = g^a = g^a \cdot 1 = g^a \cdot g^0 
\]
$\textbf{Inductive steps:}$ Suppose $g^{a+b} = g^a \cdot g^b,$ we have  
\[
g^{a+(b+1)} = g^{(a+b)+1} = g \cdot g^{(a-1)+(b+1)} = g \cdot g^{a-1} \cdot g^{b+1} = g^a \cdot g^{b+1}.
\]
Therefore,
\begin{equation*}
\forall g \in G \; \forall a \in \mathbb{Z} \; \forall b \in \mathbb{Z}^+ \cup \{0\}: g^{a+b} = g^a \cdot g^b
\end{equation*}
$\forall g \in G \; \forall a \in \mathbb{Z} \; \forall b \in \mathbb{Z}^-$.
Since $b \in \mathbb{Z}^-, -b \in \mathbb{Z}^+$,
\[
g^{a+b} = (g^{-1})^{-a-b} \stackrel{\text{(2)}}{=} (g^{-1})^{-a} \cdot (g^{-1})^{-b} = g^a \cdot g^b
\]




We also use induction to prove that for a group $G, g \in G,$ and two integers $a, b$,
\begin{equation*}
g^{ab} = (g^a)^b
\end{equation*}
$\forall g \in G \; \forall a \in \mathbb{Z} \; \forall b \in \mathbb{Z}^+ \cup \{0\}$, we have \\~\\
$\textbf{Base case: } b=0$
\[
g^{a \cdot 0} = g^0 = 1 = (g^a)^0 = (g^a)^b
\]
$\textbf{Inductive steps:}$ Suppose $g^{a \cdot b} = (g^a)^b,$ then
\[
g^{a \cdot (b+1)} = g^{a \cdot b + a} = g^{a \cdot b} \cdot g^a = (g^a)^b \cdot (g^a)^1 = (g^a)^{b+1}
\]
Therefore,
\begin{equation*}
\forall g \in G \; \forall a \in \mathbb{Z} \; \forall b \in \mathbb{Z}^+ \cup \{0\}: g^{ab} = (g^a)^b
\end{equation*}
$\forall g \in G \; \forall a \in \mathbb{Z} \; \forall b \in \mathbb{Z}^-$.
Since $b \in \mathbb{Z}^-, -b \in \mathbb{Z}^+$,
\[
g^{ab} = g^{(-a)(-b)} = (g^{-a})^{(-b)} = ((g^a)^{-1})^{-b} = (g^a)^{(-1)(-b)} = (g^a)^b
\]




We also use induction to prove that for a group $G, g, h \in G$ such that $g \cdot h = h \cdot g$ and an integer $a$,
\[
(gh)^a = g^a \cdot h^a
\]
First, we want to prove that for positive $a$
\begin{equation}
h \cdot g^a = g^a \cdot h
\end{equation}
 also using induction, which will be included along the main proof \\
$\forall g, h \in G \, \forall a \in \mathbb{Z}^+ \cup \{0\}$, we have \\
$\textbf{Base case: } a=0$
\[
h \cdot g^0 = h \cdot 1 = 1 \cdot h = g^0 \cdot h
\]
\[
(gh)^0 = 1 = 1 \cdot 1 = g^0 \cdot h^0
\]
$\textbf{Inductive steps:}$ Suppose $h \cdot g^a = g^a \cdot h$ and $(gh)^a = g^a \cdot h^a$, then
\[
h \cdot g^{a+1} = h \cdot g \cdot g^a = h \cdot g^a \cdot g = g^a \cdot h \cdot g = g^a \cdot g \cdot h = g^{a+1} \cdot h \text{, which prove (1) } 
\]
\[
(gh)^{(a+1)} = (gh) \cdot (gh)^a = g \cdot h \cdot g^a \cdot h^a  = g \cdot g^a \cdot h \cdot h^a = g^{a+1} \cdot h^{a+1}
\]
$\forall g, h \in G \, \forall a \in \mathbb{Z}^-$. Since $a \in \mathbb{Z}^-, -a \in \mathbb{Z}^+$,
\[
(gh)^a = (gh)^{(-1)(-a)} = ((hg)^{-1})^{-a} = (g^{-1} \cdot h^{-1})^{-a} = (g^{-1})^{-a} \cdot (h^{-1})^{-a} = g^a \cdot h^a 	 
\]






\pagebreak
\section*{2.}
To prove $a$ and $b$ has the same order, it is sufficient to prove that $a^n = 1 \iff b^n = 1$ where $n$ is finite because: \\
\indent a. If there don't exist finite n such that $a^n = 1$ then there also don't exist finite $m$ such that $b^m = 1$ or else $a^m = 1$ which is a contradiction \\
\indent b. If ord($a$) = $n$ then $b^n = 1$ which means that ord($b$) $\le$ n. If there exist $m \in \mathbb{N}$ such that $m<n$ satisfies thenq $a^m = 1$ which is a contradiction. \\
1. Suppose $g$ has order $n$ and $n$ is finite, then
\begin{equation*}
\begin{aligned}
&g^n = 1 \\ 
\implies & g^n \cdot g^{-n} = 1 \\
\implies & (g^{-1})^n = 1
\end{aligned}
\end{equation*}
If $g^{-1}$ has order $n$ where n is finite then $(g^{-1})^{-1} = g$ has order n as well. \\
2. Suppose $g$ has order $n$ and $n$ is finite that is $g^n = 1$ then
\begin{equation*}
\begin{aligned}
&h \cdot h^{-1} = 1 \\
\implies & h \cdot g^n \cdot h^{-1} = 1 \\
\implies & (hgh^{-1})^n = 1
\end{aligned}
\end{equation*}

Suppose $hgh^{-1}$ has order $n$ and $n$ is finite then
\begin{equation*}
\begin{aligned}
& (hgh^{-1})^n = 1 \\
\implies & h \cdot g^n \cdot h^{-1} = 1 \\
\implies & h^{-1} \cdot h \cdot g^n \cdot h^{-1} \cdot h = h^{-1} \cdot h \\
\implies &g^n = 1
\end{aligned}
\end{equation*}
3. Suppose $gh$ has order $n$ and n is finite, then
\begin{equation*}
\begin{aligned}
&(gh)^n = 1 \\
\implies &(gh)^{-n} (gh)^n = (gh)^{-n} \\
\implies & 1 = (gh)^{-n} \\
\implies & 1 = h^{-n} \cdot g^{-n} \\
\implies & h^n \cdot g^n = h^n \cdot h^{-n} \cdot g^{-n} \cdot g^n \\
\implies & (hg)^n = 1
\end{aligned}
\end{equation*}
WLOG, $hg$ has order $n$ and $n$ is finite implies $gh$ has order $n$. \\
4. If $g = g^{-1}$ then $g^2 = 1$ which means that $g$ must have an order that is less than $2$, which means $g$ must have order 1 or 2.











\pagebreak
\section*{3.}
Let $g,h$ be arbitary elements in group $G$. Then $g\cdot h$ is also an element in $G$, which means that
\begin{equation*}
\begin{aligned}
&(gh)^2 = 1 \\
\implies & ghgh = 1 \\
\implies & ghghhg = hg \\
\implies & ghgg = hg \\
\implies & gh = hg 
\end{aligned}
\end{equation*}
Consider the dihedral group of a hexagon. Which means that for each rotation $R$ on the hexagon, $R^6 = 1$ and for all reflection $r$ on the hexagon, $r^2 = 1 \implies r^6 = 1$. But dihedral group is not abelian, for example: \\
Consider the dihefral group of a hexagon. \\
Let $R_i$ denotes the rotation by angle $i \cdot \cfrac{2\pi}{6}$ and $r_i$ denotes the reflection in the line at angle $i \cdot \cfrac{\pi}{6}$ with respect to a fixed line passing through the center and one vertex of the hexagon. We have: 
\[r_1 \cdot R_1 = r_2 \ne R_1 \cdot r_1 = r_0\] 







\pagebreak
\section*{4.}
1. If there is $0 \le a < b <n \in \mathbb{N}$ such that $g^a = g^b$ then $g^{b-a} = 1$ which has $0 <b-a<n$ is a contradiction because $g$ has order $n$ \\
2. Similarly, if there is $a<b \in \mathbb{Z}$ such that $g^a = g^b$ then $g^{b-a} = 1$, which means that g has order $b-a$ which is a contradiction. \\
3. From the definition, we have
\[
\exists t \in \mathbb{Z}: a = tn + r
\]
\[
g^n = 1
\]
\begin{equation*}
\begin{aligned}
&\implies g^{tn} = 1 \\
&\implies g^{a-r} = 1 \\
&\implies g^a = g^r
\end{aligned}
\end{equation*}
If there is $a<b$ such that $g^a = g^b$ and $a$, $b$ has different remainder when divided by n then
\[\exists x,y \in \mathbb{Z}: b-a = xn + y\] where $0<y<n$
\begin{equation*}
\begin{aligned}
&g^a = g^b \\
\implies &g^{b-a} = 1 \\
\implies &g^{xn+y} = 1 \\
\implies &g^{xn} \cdot g^y = 1 \\
\implies &g^y = 1 \\
\end{aligned}
\end{equation*}
which is a contradiction because $g$ has order $n > y$. Therefore, the function is injective. \\
It is obviously well-defined because for each $a,b \in \mathbb{Z}$ such that $a-b$ is divisible by $n: g^a = g^b$. \\
$\forall a \in \mathbb{Z}: a$ has a remainder when being divided by $n$, therefore $g^a$ is in the image of the function. Hence $\{g \in \mathbb{Z} \, | \, g^a\}$ is the image of the function. \\
4.
From part 1 the size of subset $\{g^a \, | \, a \in \mathbb{Z}\}$ is larger or equal to n because $\{1,g,g^2, \cdots, g^{n-1}\}\subset \{g^a \, | \, a \in \mathbb{Z}\}$. But since $\forall b \in \mathbb{Z}: \exists x,y \in \mathbb{Z} : b = xn + y$ where $0 \le y < n$ which means that $g^b = g^y$. Since $g^y$ is already an element in the subset, the subset cannot have another element in $\mathbb{Z}$, hence has size n.

\pagebreak
\section*{5.}
Consider the set $\{g \in G \, | \, g \ne g^{-1}\}$. Since $g$ and $g^{-1}$ are uniquely determined by each other and if $g$ is in the set then because $g^{-1} \ne (g^{-1})^{-1} = g$, $g^{-1}$ is also in the set. Therefore, the set always has even size. \\
Since $G$ has even order, the set contains elements that have $g = g^{-1}$ also has even size. But $1 = 1^{-1}$, therefore, the set contains elements that have $g = g^{-1}$ must have more than 1 elements. Hence, there is another element other than $1$ satisfies $g= g^{-1}$ which is equivalent to $g^2 = 1$. Therefore, that element has order 2.














\pagebreak
\section*{6.}
Since the operation is already associative, what we need to prove are \\
1. If $e \cdot a = a$ then $a \cdot e = a = e \cdot a$ \\
2. If $h \cdot g = e$ then $g \cdot h = e = h \cdot g$\\
Since $h \in G$, there exists $t$ such that $t \cdot h = e$ \\
Because the operation is already operative, if $h \cdot g = e$, then 
\begin{equation*}
\begin{aligned}
&h \cdot g = e \\
\implies & h \cdot g \cdot h = e \cdot h \\
\implies & t \cdot h \cdot g \cdot h = t \cdot h \\
\implies & e \cdot g \cdot h  = t \cdot h \\
\implies & g \cdot h = e
\end{aligned}
\end{equation*}
which prove 2. \\
Since $a \in G$, there exists $b$ such that $a \cdot b = b \cdot a = e$ \\
Because the operation is already operative, if $e \cdot a = a$, then 
\begin{equation*}
\begin{aligned}
&e \cdot a = a \\
\implies & b \cdot e \cdot a = b \cdot a \\
\implies & b \cdot e \cdot a = e \\
\implies & a \cdot b \cdot e \cdot a  = a \cdot e \\
\implies & e \cdot e \cdot a  = a \cdot e \\
\implies & e \cdot a  = a \cdot e \\
\end{aligned}
\end{equation*}
which prove 1. \\

\end{document}