\documentclass[11pt]{article}
    \title{\textbf{Math 217 Homework I}}
    \author{Khac Nguyen Nguyen}
    \date{}
    
    \addtolength{\topmargin}{-3cm}
    \addtolength{\textheight}{3cm}
    
\usepackage{amsmath}
\usepackage{mathtools}
\usepackage{amsthm}
\usepackage{amssymb}
\usepackage{pgfplots}
\usepgfplotslibrary{polar}
\usepgflibrary{shapes.geometric}
\usetikzlibrary{calc}
\pgfplotsset{compat = newest}
\pgfplotsset{my style/.append style = {axis x line = middle, axis y line = middle, xlabel={$x$}, ylabel={$y$}, axis equal}}
\begin{document}
\section*{1.}
\subsection*{1.}
$G$ is cyclic. Hence, there is a generator $g$ such that $G = \{ g^k: k \in \mathbb{Z} \}$. \\
Since $f$ is homomorphism, we have that $g' \in G: \exists k \in \mathbb{Z}: g' = g^k$ and therefore, $f(g') = f(g^k) = (f(g))^k$. $f(g)$. $f$ is surjective therefore, $f(g)$ is the genarator of $H$. Hence, $H$ is also a cyclic group. 
\subsection*{2.}
If any group $G$ is isomorphic to a cyclic group $H$, there is a surjective isormorphism function that maps the cyclic group $H$ to $G$. Hence, it is also a cyclic group. \\
\subsection*{3.}
Supppose $N$ is a normal subgroup of $G$. Then $(g^k \cdot N) = (g \cdot N)^k$, hence $g \cdot N$ is the genarator of $G/N$. Which means that any quotient of a cyclic group is again cyclic. \\
\pagebreak
\section*{2.}
\subsection*{1.}
From the Langrange's Theorem, we know that every subgroup of $D_8$ must have 1,2,4 or 8 elements. \\
1 elements: $\{1\}$ \\
2 elements: since 1 must be included, the other element inverse is itself. \\
Every element in $D_8$ can be written in the form $s^ir^j$ where $i\in \{0,1\}$ and $j \in \{0,1,2,3\}$. \\
If $s=0$ then the only element not 1 that is an inverse of itself is $r^2$. \\
If $s=1$ then $sr^j sr^j = sr^j r^{4-j} s=sr^4s = s^2 = 1$. \\
Hence, the subgroup having 2 elements are $\{1,r^2\}, \{1, s\}, \{1, sr\}, \{1, sr^2\}, \{1,sr^3\}$.
4 elements: 1 is an element of the subgroup. \\
If there is zero elements of the form $sr^i$ then we have that $\{1,r,r^2,r^3\}$ is a subgroup.
If there is one elements of the form $sr^i$ then there is an element of the form $r^j$ where $1\le j \le 3$, which means that $sr^i \cdot r^j = sr^{i+j} \ne sr^i$ and hence not in the set which means there is no subgroup in this case. \\
If there is two or three elements of the form $sr^i$ then let 2 of them be $sr^i$ and $sr^j$, we have that $sr^i sr^j = sr^i r^{4-j} s =sr^{4-j+i}s = ssr^{4-(4-j+i)} = r^{j-i}$. $r^{j-i}$ must also be an inverse of itself which means that $j-i=2$. \\Hence, the subgroup in this case is $\{1, s, r^2, sr^2\}, \{1, sr, r^2, sr^3\}$. \\
Therefore the subgroup having 4 elements are $\{1,r,r^2,r^3\}, \{1, s, r^2, sr^2\}, \{1, sr, r^2, sr^3\}$.
8 elements: itself
\subsection*{2.}
From the Langrange's Theorem, we know that every subgroup of $S_3$ must have 1,2,3 or 6 elements. \\
1 elements: $\{()\}$
2 elements: since 1 is in any subgroup, the othere element is an inverse of itself. Hence, the subgroups are $\{(),(12)\}, \{(),(23)\}, \{(), (13)\}$ \\
3 elements: length 2 length 3 cant, 2 length 2 cant, 2 length 3 only 1 case
6 elements: itself
\subsection*{3.}
1 elements: $\{1\}$ \\
2 elements: since 1 is in any subgroup, the othere element is an inverse of itself. Hence, the subgroups is $\{1,-1\}$
4 elements: since 1 is in any subgroup, order 4 so one elements must be an inverse of itself so -1 must also be in the subgroup. As -1 is in the subgroup if i is in the subgroup then so does -i, similar to j and -j, k and -k. Hence the subgroups are $\{1,-1,i,-i\}, \{1,-1,j,-j\}, \{1,-1, k, -k\}$
8 elements: itself

\pagebreak
\section*{3.}
\subsection*{1.}
Since $\varphi_k(g) = g^k$, $\forall g^i \in G: \varphi_k(g^i) = (\varphi_k(g))^i = (g^k)^i = g^ki$.
Hence, the function is unique as every element is mapped to a predeterimend element in $G$. 
$\forall g^i, g^j \in G: \varphi_k(g^i) \cdot \varphi_k(g^j) = g^{ik} \cdot g^{kj} = g^{k(j+i)} = \varphi_k(g^{i+j})$. Hence, $\varphi_k$ is a homomorphism.
\subsection*{2.}
If gcd($k,n) \ne 1$, then $\exists m \in \mathbb{Z}: mn = k$ or $mk = n$. \\
If $mn = k$, then $\forall g^i \in G:\varphi_k(g^i) = g^{ik} = g^{imn} = 1$. Which means the function is not surjective and therefore does not have an inverse hence not an automorphism. \\
If $mk = n$, then $\varphi_k(g^m) = g^{mk} = 1 = \varphi_k(1)$. Hence, $\varphi_k$ is not injective and therefore do not have an inverrse hence not an automorphism. \\
If gcd($k,n$) = 1, then let $k = an + b$ where $a,b \in \mathbb{Z}, 0\le b < n$. \\
$\forall g^i \in G: \varphi_{n-b} (g^i) \cdot \varphi_k(g^i) = g^{i(n-b)} \cdot g^{i(an+b)} = 1$. Therefore $\varphi_k$ has an inverse and is automorphism. 
\subsection*{3.}
It is obvious that $\forall g^i \in G:$
\[
(\varphi_a \circ \varphi_b) (g^i) = ((g^i)^b)^a = g^{iba} = (g^i)^{ab} = \varphi_{a\cdot b}(g^i) 
\]
\subsection*{4.}
$\forall a,b \in \mathbb{Z}$ such that $ a \equiv b (\text{mod } n): \exists t \in \mathbb{Z}: a = b + tn$ \\
$\forall g^i \in G: \varphi_a(g^i) = g^{i(b+tn)} = g^{ib} = \varphi_b(g^i)$. Hence, $\varphi_a = \varphi_b$ and therefore the function is well-defined. \\
If we restrict $\{a$ mod $n \}$ to $\{a \text{ mod } n | \text{gcd}(a,n) = 1 \}$. Then we know that $\varphi_a$ is an automorphism hence we can restrict the original function to a bijection 
\[
f: (\mathbb{Z} / n)^\times \cong \text{Aut}(G), \indent (a \text{ mod }  n ) \to \varphi_a \text{ with gcd} (a,n) = 1
\] 
because every function $\varphi_k = f(k)$ and if $\varphi_k = \varphi_l$ then $k \equiv l$ (mod $n$) 	
\subsection*{5.}
\[
f(ab) = \varphi_{ab} = \varphi_a \cdot \varphi_b = f(a) \cdot f(b)
\]
Hence it is isormorphic.
\subsection*{6.}
If $H$ is an infinite cyclic group with generator $g$, then consider
\[
\varphi: \mathbb{Z} \to H, \indent a \to g^a
\]
The function is surjective. It is injective as if $g^a = g^b$ then $g$ has order $a-b$ or $b-a$ which is finite. We also have that \\
\[
\forall a,b \in \mathbb{Z}: \varphi(a\cdot b) = g^{a\cdot b} = g^a \cdot g^b = \varphi(a) \cdot \varphi(b)
\]
Hence, $\varphi$ is in isomorphic. Hence, Aut$(\mathbb{Z})$ is isoromorphic to Aut$(H$).
Consider the group action:
\[
\psi: \mathbb{Z} \to \text{Aut}(\mathbb{Z})
\]
$\forall f \in \text{Aut}(\mathbb{Z})$ $f(0) = 0, f(n) = f(-n), f(n) = nf(1)$. \\
If $|f(1)| \ge 2$ then $\nexists n: f(n) = 1$, which is a contradiction. \\
If $f(1) = 1$ then let $g \in \text{Aut}(\mathbb{Z})$ such that $g(1) = -1$. Then $\forall a \in \mathbb{Z}: f(g(a)) = f(-a) = a$. \\
If $f(1) = 0$ then $\forall a \in \mathbb{Z}: f(a) = 0$ and hence leads to a contradiction.
Hence, $H$ also have two automorphisms, one maps $g^a$ to $g^a$ and the other maps $g^a$ to $g^{-a}$
\pagebreak
\section*{4.}
\subsection*{1.}
If $\forall g \in \text{ker}(\pi): \pi(g) = 1$ therefore $\psi(g) = \delta(\pi(g)) = \delta(1) = 1$. Hence, $g \in \text{ker}(\psi)$ and hence ker$(\pi) \subset $ ker($\psi$).
\subsection*{2.}
Since, ker$(\pi) \subseteq \text{ker}(\psi)$. We can create a function $\delta$ such that $\forall g \in G:\pi(g) = h, \psi(g) = k$, then let $\delta(h) = k$ hence $\psi = \delta \circ \pi$. \\
$\forall g_1,g_2 \in G$: let $\pi(g_1) = h_1$, $\pi(g_2) = h_2$, $\psi(g_1) = k_1$, $\psi(g_2) = k_2$, $\delta(h_1) = k_1$, $\delta(h_2) = k_2$. We have: \\
$\delta(h_1 \cdot h_2) = \delta(\pi(g_1) \cdot \pi(g_2)) = \delta(\pi(g_1 \cdot g_2)) = \psi(g_1 \cdot g_2) = \psi(g_1) \cdot \psi(g_2) = \delta(\pi(g_1)) \cdot \delta(\pi(g_2)) = \delta(h_1) \cdot \delta(h_2)$. 
Hence, $\delta$ is homomorphic.\\ 
If there is a function $\delta$ such that $\delta(h_1) = k_2$ then $k_1 = \psi(g_1) = \delta(\pi(g_1)) = \delta(h_1) = k_2$. Therefore, the function is unique.
\subsection*{3.}
Let $\pi: G \to G/\text{ker}(\varphi)$ \\
First, let restrict $\varphi$ to a surjective function: $\varphi': G \to \text{im}(\varphi)$. \\ 
We know that ker($\varphi'$) = ker($\varphi$) are normal subgroups of $G$. Hence, from the universal property of quotients, \\
there exsists a unique homomorphism $\overline{\varphi}: G/\text{ker}(\varphi') \to \text{im}(\varphi')$ satisfies $\overline{\varphi} \circ \pi = \varphi'$
and since $\forall g \in G:\varphi'(g) = \varphi(g)$, ker($\varphi'$) = ker($\varphi$), im($\varphi'$) = im($\varphi$). \\
We can rewrite it as $\overline{\varphi}: G/\text{ker}(\varphi) \to \text{im}(\varphi)$ satisfies $\overline{\varphi} \circ \pi = \varphi$. \\
The kernel of $\overline{\varphi}$ consists of cosets of the form $g \cdot \text{ker}(\varphi)$ with $\varphi(g) = 1$, equivalently, $g \in \text{ker}(\varphi)$. Since $g \cdot \text{ker} (\varphi) = \text{ker}(\varphi) \iff g \in \text{ker}(\varphi)$. Hence, $\overline{\varphi}$ is injective. \\
$\varphi$ is surjective, hence so does $\overline{\varphi}$. \\
Therefore, $G/\text{ker}(\varphi) \cong \text{im}(\varphi)$.\\
As a result, if $\varphi$ is surjective, that is if $H = \text{im}(\varphi)$ then $G/\text{ker}(\varphi)$ is isomorphic to $H$. 
\pagebreak
\section*{5.}
Since the size of is 2. One coset is $1 \cdot H = H$ and the other is $G \backslash H$. \\
Pick $x \in G$\\
If $x \in H$ then obviously, $\forall h \in H: xhx^{-1} \in H$. \\
If $x \notin H$ then as right cosets are also $H \cdot 1 = H$ and hence $G \backslash H$. Hence,
as $xH \ne H, xH = G\backslash H$ and $Hx \ne H, Hx = G\backslash H$ and therefore $xH = Hx$. \\
In both cases, $H$ is proven normal. \\
Consider $G = S_3$ and its subgroup $H = \{ (), (12) \}$. \\ 
$[G:H] = 3$ as 
$() \cdot H = H$, $(23) \cdot H = \{(23), (123)\}$, $(13) \cdot H = \{(13), (321)\}$ \\
but (23)(12)(23) = (13) $\notin H$.
\pagebreak
\section*{6.}	
\subsection*{1.}
From definition $Z(G)$ is normal in $G$ as it is a subgroup of $G$ and $\forall g' \in Z(G): \forall g \in G: gg' = g'g$.  
\subsection*{2.}
(a) $\iff$ (b): already proven in homework 3 \\
(b) $\iff$ (d): directly, we have that $G/Z(G)$ is trivial if and only if $Z(G) = 1 \cdot Z(G) = G$ \\
(c) $\implies$ (a): $\exists G' \in G/Z(G): G/Z(G) = \langle G' \rangle$. Since $G'$ is a cosets, $\exists g \in G: G' = tZ(G)$. Hence each cosets is equals to $(G')^n = (tZ(G))^n = t^n Z(G)$. \\
For arbitary $x,y \in G: \exists i,j: x \in t^iZ(G)$ and $y \in t^jZ(G)$ and hence $\exists z_1, z_2 \in Z(G):x = t^iz_1$ and $y = t^jz_2$. \\
\begin{equation*}
\begin{aligned}
xy &= t^iz_1 t^jz_2 \\
&= t^i t^j z_1 z_2 \\
&= t^j t^i z_2 z_1 \\
&= t^j z_2 t^i z_1 \\
&= yx
\end{aligned}
\end{equation*}
and hence $G$ is abelian. \\
(d) $\implies$ (c): $G/Z(G)$ is trivial hence cyclic.
\subsection*{3.}
$G$ is a finite group of order $p \cdot q$ where $p$ and $q$ are primes. Hence $Z(G)$, subgroup of $G$, must have $1,p,q$ or $pq$ elements. \\
If the number of elements in $Z(G)$ is either $p$ or $q$ then $Z(G)$ is cyclic hence $G$ is abelian. \\
If $Z(G)$ has $pq$ elements then $Z(G) = G$ and hence $G$ is abelian. \\
If $Z(G)$ has 1 element then it is trivial.
\subsection*{4.}
Consider the map 
\[
\rho: G \to \text{Aut}(G), \indent  \rho(g)(h) = g \cdot h \cdot g^{-1}
\]
We have that:\\
ker$(\rho) = \{g: \forall h \in G: \rho(g)(h) = h\} = \{g: \forall h \in G: gh = hg\} = Z(G)$.
Hence, $G/\text{ker}(\rho) = G/Z(G) \cong \text{im}(\rho) = \text{Inn}(G)$. As Inn($G$) is a normal subgroup of $Z(G)$ and $Z(G)$ is cyclic, $G/Z(G)$ is also cyclic and hence $G$ is abelian.
\end{document}