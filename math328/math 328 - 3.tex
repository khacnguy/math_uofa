\documentclass[11pt]{article}
    \title{\textbf{Math 217 Homework I}}
    \author{Khac Nguyen Nguyen}
    \date{}
    
    \addtolength{\topmargin}{-3cm}
    \addtolength{\textheight}{3cm}
    
\usepackage{amsmath}
\usepackage{mathtools}
\usepackage{amsthm}
\usepackage{amssymb}
\usepackage{pgfplots}
\usepgfplotslibrary{polar}
\usepgflibrary{shapes.geometric}
\usetikzlibrary{calc}
\pgfplotsset{compat = newest}
\pgfplotsset{my style/.append style = {axis x line = middle, axis y line = middle, xlabel={$x$}, ylabel={$y$}, axis equal}}
\begin{document}
\section*{1.}
$\forall (x_1,y_1), (x_2,y_2) \in G \times G$
\begin{equation*}
\begin{aligned}
\sigma((x_1,y_1)) \cdot (x_2,y_2)) &= \sigma(x_1x_2,y_1y_2) \\
&= (y_1y_2,x_1x_2) = (y_1,x_1) \cdot (y_2,x_2) \\
&=\sigma(x_1,y_1) \cdot \sigma(x_2,y_2) \\ 
\end{aligned}
\end{equation*}
Therefore, $\sigma$ is an homomorphism. \\
We have that 
\[
\forall (x,y) \in G \times G: (\sigma \circ \sigma)(x,y) = \sigma(y,x) = (x,y)
\]
Therefore, $\sigma$ is an automorphism.
\begin{equation*}
\begin{aligned}
\forall g_1,g_2 \in G: \forall x \in X: &(g_1,g_2) \cdot x = (g_2,g_1) \cdot x \\
\implies & ((g_1,1) \cdot (1, g_2) )\cdot x = ((g_2,1) \cdot (1, g_1) )\cdot x \\
\implies & ((1,g_1) \cdot (1, g_2) )\cdot x = ((1,g_2) \cdot (1, g_1) )\cdot x \\
\implies & (1,g_1g_2) \cdot x = (1,g_2g_1) \cdot x \\
\text {Similarly,} &(g_1g_2,1)(x) = (g_2g_1,1)(x) 
\end{aligned}
\end{equation*}
$\forall g_1 = (g_{1_1}, g_{1_2}), g_2 = (g_{2_1}, g_{2_2}) \in G \times G, \forall x \in X:$
\begin{equation*}
\begin{aligned}
(\rho(g_1) \circ \rho(g_2)) (x) &=(\rho(g_{1_1}, g_{1_2}) \circ \rho(g_{2_1}, g_{2_2}))(x) \\
&= (g_{1_1}, g_{1_2}) \cdot (g_{2_1}, g_{2_2}) \cdot x \\
&= (g_{1_1} \cdot g_{2_1}, g_{1_2} \cdot g_{2_2}) \cdot x \\
&= (g_{1_1} \cdot g_{2_1},1) (1, g_{1_2} \cdot g_{2_2}) \cdot x \\
&= (g_{2_1} \cdot g_{1_1},1) (1, g_{2_2} \cdot g_{1_2}) \cdot x \\
&= (g_{2_1} \cdot g_{1_1}, g_{2_2} \cdot g_{1_2}) \cdot x \\
&= (g_{2_1}, g_{2_2}) \cdot (g_{1_1}, g_{1_2}) \cdot x \\
&=(\rho(g_{2_1}, g_{2_2}) \circ \rho(g_{1_1}, g_{1_2}))(x) \\
&= (\rho(g_2) \circ \rho(g_1)) (x)
\end{aligned}
\end{equation*}
\pagebreak
\section*{2.}
$G$ is a group, hence $1 \in G$, which means that $\forall x \in X: 1 \cdot x = x \implies x \sim x$
If $x \sim y$, then $\exists g \in G: g \cdot x = y$. \\
Hence, $g^{-1} \in G$ and $g^{-1} \cdot y = g^{-1} \cdot g \cdot x = 1 \cdot x = x$ which means that $y \sim x$
If $x \sim y$ and $y \sim z$ then $\exists g_1,g_2 \in G: g_1 \cdot x = y$ and $g_2 \cdot y = z$. \\
Let $g = g_2 g_1$, then $g \cdot x = g_2 \cdot g_1 \cdot x = g_2 \cdot y = z$, hence $x \sim z$.
Therefore, $\sim$ is an equivalence relation on $X$.
\pagebreak
\section*{3.}
Since Aut($G)$ contains all isomorphism function that maps $G \to G$ which means it is bijective and therefore is an element of Per($G$). Therefore, Aut($G ) \subset$ Per($G$). Since Aut($G$) is a group (hw2), it is a subgroup of Per($G$) \\ 
$\forall g \in G: \exists g^{-1} \in G: (\rho(g) \circ \rho(g^{-1}))(h) = g \cdot g^{-1} \cdot h \cdot (g^{-1})^{-1} \cdot g^{-1} =  h$.
Therefore, im($\rho ) \subset$ Aut($G$). \\
Hence, we can have an induced homomorphisms $\gamma$
such that range($\gamma$) $\subset$ range($\rho$) and $\forall g \in G: \rho(g) = \gamma(g)$. \\
Obviously, Ker($\gamma$) = Ker($\rho$) since $\forall g \in G: \rho(g) = \gamma(g)$. \\
If $g \in $ker$(\gamma)$ then $\gamma(g) = 1$, which means that $\forall g_1 \in G:$ 
\[g \cdot g_1 \cdot g^{-1} = g_1 \iff g \cdot g_1 \cdot g^{-1} \cdot g = g_1 \cdot g \iff g \cdot g_1 = g_1 \cdot g\] 
Hence ker($\gamma$) = $Z(G)$\\
$\forall \sigma \in$ im($\gamma$) $ \implies \exists g_1 \in G: \gamma(g_1) = \sigma_1$.
$g_1 \in G$. Let $g_2 = \tau(g_1)$, then 
\[
g_1 \tau^{-1}(h) g_1^{-1} = \tau^{-1}(g_2) \tau^{-1}(h) \tau^{-1} (g_2^{-1}) = \tau^{-1}(g_2hg_2^{-1}) 
\]
and hence 
\[
\rho(g_1) (\tau^{-1}(h)) = \tau^{-1}(\rho(g_2) (h)) \implies (\rho(g_1) \circ \tau^{-1})(h) = (\tau^{-1} \circ \rho(g_2)) (h)
\]
which means that 
\[
(\tau \circ \rho \circ \tau^{-1})(g_1) = \tau(\rho(g_1) (\tau^{-1}(h))) = \tau(\tau^{-1} \circ \rho(g_2))(h) = \rho(g_2)(h)
\]
Therefore, $\tau \circ \sigma \circ \tau^{-1} \in $im($\gamma$)
\pagebreak
\section*{4.}
1. 
Since $g_1,g_2,\ldots, g_n$ are generators of $G$. We have that, $\forall g \in G: \exists i_1,i_2, \ldots i_n \text{such that }: \forall j \in \{1,2,\ldots,n\}: 0 \le i_j < \text{ord}(g_j) = k_j$
\[
g = g_1^{i_1} \cdot g_2^{i_2} \cdot \ldots \cdot g_n^{i_n}
\]
Hence, there exists at most $k_1 k_2 \ldots k_n$ elements in $G$. \\ 
Quaternion group has 8 elements. But we have ord($i$) = 4, ord($-i$) = 4 hence the products of order of all elements in $Q_8 \ge 16>8$. \\
2. Suppose $g_1$ is an element in $G$ such that the order of it is finite: ord($g_1) = k_1$. Then, ord($g_1$) = ord($g_1^{-1}) = k_1$ which means that each element has an inverse. \\
Suppose $g_1,g_2$ are element in $G$ of finite order, then $\exists m,n \in \mathbb{N}: g_1^m = 1$ and $ g_2^n=1$, then $(g_1g_2)^{mn} = 1$ where $mn$ is also finite.\\
1 has order 1 hence also in the set. Hence, the set of all elements of finite order in $G$ is a subgroup of $G$.
Consider the set of all 2 by 2 matrices. 
\[
\begin{pmatrix}
-1 & 0 \\
0 & 1 
\end{pmatrix} \cdot 
\begin{pmatrix}
-1 & 1 \\
0 & 1 
\end{pmatrix} = 
\begin{pmatrix}
1 & -1 \\
0 & -1 
\end{pmatrix}
\]
\[
\begin{pmatrix}
-1 & 0 \\
0 & 1 
\end{pmatrix}^2 = \begin{pmatrix}
1 & 0 \\
0 & 1 
\end{pmatrix}
\]
\[\begin{pmatrix}
-1 & 1 \\
0 & 1 
\end{pmatrix}^2 = \begin{pmatrix}
1 & 0 \\
0 & 1 
\end{pmatrix}
\]
But as $\begin{pmatrix}
1 & -1 \\
0 & -1 
\end{pmatrix}
$ has eigenvalue 1 which means that 
$\forall x: \begin{pmatrix}
1 & -1 \\
0 & -1 
\end{pmatrix}x = x$, which means that for all naturanl number $k$ 
\[
\begin{pmatrix}
1 & -1 \\
0 & -1 
\end{pmatrix}^kx = x
\]
As $x$ is arbitary, $\begin{pmatrix}
1 & -1 \\
0 & -1 
\end{pmatrix}^k \ne 0$ and hence has infinite order which means it is not an element of the set of finite order elements. \\
3. 
If $\ (\mathbb{Q}, + )$ is a cyclic group then $\exists g>0$ be the generator since if $g<0$ then $\exists g' = -g > 0$ and $g \ne 0$. \\
Therefore, we have $\forall n \in \mathbb{Z}: g\cdot n < g\cdot (n+1)$ \\
Hence, $\nexists h \in (g^n, g^{n+1}) \cap \mathbb{Q}$. But since $Q$ is dense, that is there exists a rational number in all open interval, hence contradiction and $\mathbb{Q}$ is a non-cyclic group. \\
Every subgroup of a cyclic group is cyclic. Suppose, there is a finite generated subgroup generated by $\{g_1,g_2,\ldots, g_n\}$. Since, each $g_i$ is a generators of $\mathbb{Q}$, it is also an element of $\mathbb{Q}$. Let $g_i = \frac{p_i}{q_i}$ then we can find a cyclic group generated by $\{\prod_{i=1}^n\frac{1}{q_i}\}$. \\
Hence, we have $\forall g \in G: \exists j_1, j_2,\ldots, j_n \in \mathbb{Z}:$
\[
g = \sum_{i=1}^{n} j_i \cdot \frac{p_i}{q_i} 
= \sum_{i=1}^{n} \left(j_i \cdot p_i \cdot \prod_{\substack{k=1 \\ k\ne i}}^n q_k \prod_{k=1}^n\frac{1}{q_k} \right) 
= \left( \prod_{k=1}^n\frac{1}{q_k}\right) \sum_{i=1}^{n} \left(j_i \cdot p_i \cdot \prod_{\substack{k=1 \\ k\ne i}}^n q_k  \right)
\] which means that the subgroup generated by $\{g_1,g_2,\ldots, g_n\}$ is a subgroup of the subgroup generated by $\{\prod_{i=1}^n\frac{1}{q_i}\}$, which is a cyclic group. Hence, the subgroup generated by $\{g_1,g_2,\ldots, g_n\}$ is cyclic. \\
Consider the subgroup $H$ generated by $\{\frac{1}{2^n} | n \in \mathbb{N}\}$. If it is cyclic then there is an element $h$ which can generated the whole subgroup. We have that $\nexists h_1 \in (0,h_1) \cap H$, but $\exists n_0: \frac{1}{2^{n_0}} \in (0,h_1)$ which leads a contradiction. Hence, it is not cyclic.
\pagebreak
\section*{5.}
1.
If $h \in H$, then $ghg^{-1} = h$. Which means that $H \subseteq N_G(H)$  \\ 
$H \subseteq C_G(H) \iff \forall h_1,h_2 \in H: h_1 h_2 h_1^{-1} = h_2 \iff h_1 h_2 = h_2 h_1 \iff H$ is abelian. \\
2.
Since $H$ has order 2, let it elements be 1 and $h$.
Since has order 2, one element is 1 and each elements inverse is itself. Then we have that $H \subseteq N_G(H)$, which means that $\forall g \in N_G(H)$, either
\[ghg^{-1} = h \text{ and } g1g^{-1} = 1\]
or 
\[ghg^{-1} = 1 \text{ and } g1g^{-1} = h\]
In the first case, it is obvious that $N_G(H) = C_G(H)$. \\
In the second case, $g1g^{-1} = h \implies 1 = h$ which is a contradiction. Hence, $N_G(H) = C_G(H)$. \\
If $C_G(H) = N_G(H) = G$ then $\forall g \in G: ghg^{-1} = h$ which means that $h$ is a central element, $1$ is also a central element. Hence $H \subset Z(G)$.
\pagebreak
\section*{6.}
1. 
Every matrix in $H(\mathbb{R})$ is upper triangular hence is invertible.\\
The identity matrix $\begin{pmatrix}
1 & 0 & 0 \\
0 & 1 & 0 \\
0 & 0 & 1
\end{pmatrix} \in H(\mathbb{R})$ \\
$\forall a_1, b_1, c_1, a_2, b_2, c_2 \in \mathbb{R}:
\begin{pmatrix}
1 & a_1 & c_1 \\
0 & 1 & b_1 \\
0 & 0 & 1
\end{pmatrix} \cdot 
\begin{pmatrix}
1 & a_2 & c_2 \\
0 & 1 & b_2 \\
0 & 0 & 1
\end{pmatrix} 
=
\begin{pmatrix}
1 & a_1 + a_2 & c_1 + c_2 + a_1b_2 \\
0 & 1 & b_1 + b_2 \\
0 & 0 & 1
\end{pmatrix}$ which is also an element of $H(\mathbb{R})$ \\
$\forall a,b,c \in \mathbb{R}: 
\begin{pmatrix}
1 & a & c \\
0 & 1 & b \\
0 & 0 & 1
\end{pmatrix} \cdot 
\begin{pmatrix}
1 & -a & -c+ab \\
0 & 1 & -b \\
0 & 0 & 1
\end{pmatrix} 
= \begin{pmatrix}
1 & 0 & 0 \\
0 & 1 & 0 \\
0 & 0 & 1
\end{pmatrix}$ \\
which means that every matrix in $H(\mathbb{R})$ has an inverse.
Therefore, $H(\mathbb{R})$ is a subgroup of $GL_3(\mathbb{R})$ \\
2. Suppose $\begin{pmatrix}
1 & a_1 & c_1 \\
0 & 1 & b_1 \\
0 & 0 & 1
\end{pmatrix}$ is an element of $Z(H(\mathbb{R}))$ then $\forall a_2,b_2,c_2 \in \mathbb{R}:$ 
\[
\begin{pmatrix}
1 & a_1 & c_1 \\
0 & 1 & b_1 \\
0 & 0 & 1
\end{pmatrix} 
\cdot 
\begin{pmatrix}
1 & a_2 & c_2 \\
0 & 1 & b_2 \\
0 & 0 & 1
\end{pmatrix} 
= 
\begin{pmatrix}
1 & a_2 & c_2 \\
0 & 1 & b_2 \\
0 & 0 & 1
\end{pmatrix} 
\cdot 
\begin{pmatrix}
1 & a_1 & c_1 \\
0 & 1 & b_1 \\
0 & 0 & 1
\end{pmatrix}
\]
Hence,
\[
\begin{pmatrix}
1 & a_1 + a_2 & c_1 + c_2 + a_1b_2 \\
0 & 1 & b_1 + b_2 \\
0 & 0 & 1
\end{pmatrix}
=
\begin{pmatrix}
1 & a_1 + a_2 & c_1 + c_2 + a_2b_1 \\
0 & 1 & b_1 + b_2 \\
0 & 0 & 1
\end{pmatrix}
\]
Therefore, $\forall a_2,b_2 \in \mathbb{R}: a_1b_2 = a_2b_1 \implies a_2 = b_2 = 0$. \\
And hence, $Z(H(\mathbb{R}))$ consists of matrixes in the form: $\begin{pmatrix}
1 & 0 & x\\
0 & 1 & 0 \\
0 & 0 & 1
\end{pmatrix}$.
We can construct a function 
\[
f: (\mathbb{R},+) \to Z(H(\mathbb{R})), \indent x \to \begin{pmatrix}
1 & 0 & x \\
0 & 1 & 0 \\
0 & 0 & 1
\end{pmatrix}
\]
$f$ is surjective: $\forall 
\begin{pmatrix}
1 & 0 & x \\
0 & 1 & 0 \\
0 & 0 & 1
\end{pmatrix} 
\in Z(H(\mathbb{R})): f(x) = 
\begin{pmatrix}
1 & 0 & x \\
0 & 1 & 0 \\
0 & 0 & 1
\end{pmatrix}$\\
$f$ is injective: $\forall  x_1, x_2 \in \mathbb{R}$ such that $f(x_1) = f(x_2)$:
\[
\begin{pmatrix}
1 & 0 & x_1 \\
0 & 1 & 0 \\
0 & 0 & 1
\end{pmatrix}=
\begin{pmatrix}
1 & 0 & x_1 \\
0 & 1 & 0 \\
0 & 0 & 1
\end{pmatrix}
\implies x_1 = x_2
\]
$f$ is hommorphic: \\
$f(x_1 + x_2) = 
\begin{pmatrix}
1 & 0 & x_1+x_2 \\
0 & 1 & 0 \\
0 & 0 & 1
\end{pmatrix} = 
\begin{pmatrix}
1 & 0 & x_1 \\
0 & 1 & 0 \\
0 & 0 & 1
\end{pmatrix} \cdot
\begin{pmatrix}
1 & 0 & x_2 \\
0 & 1 & 0 \\
0 & 0 & 1
\end{pmatrix} =
f(x_1) \cdot f(x_2)$
Therefore, $Z(H(\mathbb{R}))$ is isomorphic to $(\mathbb{R}, +)$ \\
3. Let the map be $g$\\ 
$\forall (a,b) \in \mathbb{R}: \forall c \in \mathbb{R}: 
g\begin{pmatrix}
1 & a & c \\
0 & 1 & b \\
0 & 0 & 1
\end{pmatrix} = (a,b)
$ which means it is surjective. \\
$\forall a_1,b_1,c_1,a_2,b_2,c_2 \in \mathbb{R}:$ \\
\begin{equation*}
\begin{aligned}
g \left( 
\begin{pmatrix}
1 & a_1 & c_1 \\
0 & 1 & b_1 \\
0 & 0 & 1
\end{pmatrix} 
\cdot 
\begin{pmatrix}
1 & a_2 & c_2 \\
0 & 1 & b_2 \\
0 & 0 & 1
\end{pmatrix}\right) &= (a_1 + a_2, b_1 + b_2) \\
&= (a_1,b_1) + (a_2 + b_2) \\
&= g 
\begin{pmatrix}
1 & a_1 & c_1 \\
0 & 1 & b_1 \\
0 & 0 & 1
\end{pmatrix} 
+ g 
\begin{pmatrix}
1 & a_2 & c_2 \\
0 & 1 & b_2 \\
0 & 0 & 1
\end{pmatrix}
\end{aligned}
\end{equation*}
Which proves that $g$ is homomrphic and hence is a surjective group homomorphism.
The identity element of $(\mathbb{R}^2, +)$ is $(0,0)$ hence \\
\[
\text{ker}(g) = \left\{ \left. \begin{pmatrix}
1 & 0 & c \\
0 & 1 & 0 \\
0 & 0 & 1
\end{pmatrix}\right|c \in \mathbb{R}\right\}
\]
\end{document}