\documentclass[11pt]{article}
    \title{\textbf{Stat 265 Homework I}}
    \author{Khac Nguyen Nguyen}
    \date{}
    
    \addtolength{\topmargin}{-3cm}
    \addtolength{\textheight}{3cm}
    
\usepackage{amsmath}
\usepackage{mathtools}
\usepackage{amsthm}
\usepackage{amssymb}
\usepackage{pgfplots}
\usepackage{dcolumn}
\newcolumntype{2}{D{.}{}{2.0}}
\usepgfplotslibrary{polar}
\usepgflibrary{shapes.geometric}
\usetikzlibrary{calc}
\pgfplotsset{compat = newest}
\pgfplotsset{my style/.append style = {axis x line = middle, axis y line = middle, xlabel={$x$}, ylabel={$y$}, axis equal}}
\begin{document}
\section*{1.}
\subsection*{1.}
Consider $D_{2k}$: \[\forall x \in D_{2k}: \exists i \in \{1,2,\ldots,n\}, j \in \{0,1\}: x = r^is^j\]
If $j=0$ then 
\[
x^m = (r^i)^m = r^{im} = 1 = r^{kn} \text{ where } k \in \mathbb{N} \iff im \text{ is a multiply of } n  
\]
If $j=1$ then
\begin{equation*}
\begin{aligned}
x^2 &= (r^is)^2 = r^{i-1} \cdot r \cdot s \cdot r^is = r^{i-1} \cdot s \cdot r^{-1} \cdot r^is = (r^{i-1})^2\\
&= r^{i-2} \cdot r \cdot s \cdot r^{i-1} \cdot s = r^{i-2} \cdot s \cdot r^{-1} \cdot r^{i-1} \cdot s = (r^{i-2})^2\\
&\hdots \\
&= (rs)^2 = 1
\end{aligned}
\end{equation*}
Therefore, \\
In $D_6$:
\[
\text{ord}(1) = 1, \text{ord}(r) = 3, \text{ord}(r^2) = 3
\]
\[
\text{ord}(rs) = 2, \text{ord}(r^2s) = 2, \text{ord}(r^3s) = 2
\]
In $D_8$:
\[
\text{ord}(1) = 1, \text{ord}(r) = 4, \text{ord}(r^2) = 2,\text{ord}(r^3) = 4
\]
\[
\text{ord}(s) = 2, \text{ord}(rs) = 2, \text{ord}(r^2s) = 2, \text{ord}(r^3s) = 2
\]
In $D_{10}$:
\[
\text{ord}(1) = 1, \text{ord}(r) = 5, \text{ord}(r^2) = 5, \text{ord}(r^3) = 5, \text{ord}(r^4) = 5
\]
\[
\text{ord}(s) = 2, \text{ord}(sr) = 2, \text{ord}(sr^2) = 2, \text{ord}(sr^3) = 2, \text{ord}(sr^4) = 2
\]
\subsection*{2.}
Since $i$ is an integer, there exists $t \in \mathbb{Z}$ such that: $i = mt + j$ where $0\le j < m$, therefore
\[
\sigma = (a_1, a_2, \ldots, a_m) 
\]
\[\implies \sigma^i = \sigma^{mt} \cdot \sigma^{j} = \sigma^j =
(a_{j+1}, a_{j+2}, \ldots, a_m, a_1, a_2, \ldots, a_i)
\]
As $j \equiv i \text{ mod }  m, j+k \equiv i+k \text{ mod } m$ and therefore, $r \equiv j+k \text{ mod } m$
\[
\implies \sigma^i(a_k) = a_{j+k} = a_r
\]
\subsection*{3.}
For $S_3$, all the cycles are \\ 
order 1: $()$  \\ 
order 2: $(1,2), (1,3), (2,3)$ \\
order 3: $(1,2,3), (1.3,2)$ \\
For $S_4$, all the cycles are \\
order 1: $()$ \\
order 2: $(1,2),(2,3),(3,4),(1,4)$ \\
order 3: $(2,3,4), (2,4,3), (1,3,4), (1,4,3), (1,2,4), (1,4,2), (1,2,4), (1,4,2)$ \\
order 4: $(1,2,3,4), (1,2,4,3), (1,3,4,2), (1,3,2,4), (1,4,2,3), (1,4,2,3)$ \\
order 2: $(1,2)(3,4), (1,3)(2,4), (1,4)(2,3) $
\pagebreak
\section*{2.}
\subsection*{1.}
If $a$ and $b$ are central elements of a group $G$, then \\
\[
\forall h \in G: h \cdot (a \cdot b) = (h \cdot a) \cdot b = (a \cdot h ) \cdot b = a \cdot (h \cdot b) = a \cdot (b \cdot h) = (a \cdot b ) \cdot h
\]
which proves that $ a\cdot b$ is also a central element.
\subsection*{2.}
If $a$ is a central elements of a group $G$, then \\
\[
\forall h \in G: h \cdot a = a \cdot h \implies a^{-1} \cdot h \cdot a \cdot a^{-1} = a^{-1} \cdot a \cdot h \cdot a^{-1} \implies a^{-1} \cdot h = h \cdot a^{-1}
\]
which proves that $a^{-1}$ is also a central element. \\
\subsection*{3.}
1 is obviously a central element: $\forall g \in G: 1 \cdot g = g \cdot 1$
the product of central elements are also central element. 
the inverse of a central element is a central elements. \\
Therefore, the centre of $G$ is a subgroup of G. \\
\subsection*{4.}
The centre of $S_4$ is $\{()\}$:
Consider an arbitary cycle $\sigma \in Z(S_4)$, then $\sigma(a,b) = (a,b) \sigma \implies \sigma^{-1} (a,b) \sigma  = (a,b)$ but $\sigma^{-1}(a,b)\sigma= (\sigma(a), \sigma(b))$ which means that $\sigma$ either keep the location of a and b or swap the location of a and b. \\
If it swaps the location of a and b then $\sigma^{-1} (abc) \sigma = (\sigma(b) \sigma(c) \sigma(a)) = (a\sigma(c)b) \ne (abc)$ which is a contradiction, therefore $\sigma$ keeps the location of a and b. 
Since a and b are arbitary, $\sigma$ must keep the location of everything and every cycle can be break up into a. Which means that () must be the only identity. \\~\\
Since $\forall a \in Q_8: (-1) \cdot a = a \cdot (-1)$, -1 is a central element. \\
We have, $a \in Q_8 \iff -a \in Q_8$ and $\forall a \in Q_8: (-1)(-1) \cdot a \cdot (-1) = (-1) \cdot a \cdot (-1)(-1) \implies 1 \cdot (-a) = (-a) \cdot 1$, which means that 1 is also a central element. \\
 Since $i\cdot j = -j \cdot i = (-1) \cdot j \cdot i \ne j \cdot i$, i and j is not central element and since -i and -j is the inverse of i and j because $-i^2 = -j^2 = 1$.
 Also, $k \cdot (-j) = i\cdot j \cdot (-j) \implies k \cdot (-j) = i$ and $(-j)\cdot k = (-j) \cdot (-j) \cdot i = (-1) \cdot j \cdot (-j) \cdot i = (-1) \cdot i = -i \ne i$, which means that $k$ is not a central element and as -$k$ is the inverse of $k: -k^2 = 1$. \\
Let $x = r^i s^j \in D_{2n}$ where $i \in \{0,1,\ldots,n-1\}$ and $j \in \{0,1\}$ be an element in the central group $Z(D_{2n})$. Then 
\begin{equation*}
\begin{aligned}
&r^i \cdot s^j \cdot r = r \cdot r^i \cdot s^j \\
\implies &r^i \cdot s^j \cdot r = r^{i+1} \cdot s^j \\
\implies &r^{n-i} \cdot r^i \cdot s^j \cdot r = r^{n-i} \cdot r^{i+1} \cdot s^j \\
\implies &s^j \cdot r = r \cdot s^j \\
\end{aligned}
\end{equation*}
If $j = 1$ then $sr = rs = sr^{-1} $ which means that $ssr =ssr^{-1}$ and hence $r = r^{-1}$ which means that $r$ has order 2 and hence is a contradiction in $D_{2n} \, \forall n \ge 3$. \\
Thenefore $j=0$ and we get $x = r^i$ and get the following
\begin{equation*}
\begin{aligned}
&r^i \cdot s = s \cdot r^i \\
\implies &r^i \cdot s = r^{n-i} \cdot s \\
\implies &r^i \cdot s \cdot s = r^{n-i} \cdot s \cdot s \\
\implies &r^i = r^{n-i} \\
\implies &r^i \cdot r^i = r^{n-i} \cdot r^i \\
\implies &r^{2i} = 1
\end{aligned}
\end{equation*}
If $r^i s = sr^i$, then consider an arbitary element $y =s^n r^m  \in D_{2n}$ \\
If $n=0$, then obviously $r^i\cdot r^m = r^m\cdot r^i$ \\
If $n=1$, then $r^i \cdot s \cdot r^m = s \cdot r^i \cdot r^m = s \cdot r^m \cdot r^i$ \\
Therefore, any element $r^i$ satisfy $r^{2i}$ is a central element, which means that for odd $n$, the only central element is 1 and for even $n$, there is 2 central elments  $1$ and $r^{n/2}$.
\pagebreak
\section*{3.}
Since $\varphi$ is homomorphism, we have 
\[
\varphi(1) = \varphi(1 \cdot 1) = \varphi^2(1) \implies \varphi(1) = 1 \lor \varphi(1) = 0
\]
If $\varphi(1) = 0$, then $\forall g \in G: \varphi(g) = \varphi(g) \cdot \varphi(1) = 0$. 
Therefore, $\forall a \in \mathbb{Z}: \varphi(g^a) = 0 = \varphi^a(g)$ \\
If $\varphi(1) = 1$, then we use induction to prove that 
\[
\forall g \in G \, \forall a \in \mathbb{Z}^+ \cup \{0\} :\varphi(g^a) = \varphi(g)^a
\]
$\textbf{Base case:  }a = 0$ \\
\[
\varphi(g^0) = \varphi(1) = 1 = (\varphi(g))^0
\]
$\textbf{Inductive step:}$ Suppose $\varphi(g^a) = \varphi(g)^a$, prove that $\varphi(g^{a+1}) = \varphi(g)^{a+1}$
\[
\varphi(g^{a+1}) = \varphi(g^a \cdot g) = \varphi(g^a) \cdot \varphi(g) = \varphi(g)^a \cdot \varphi(g) = \varphi(g)^{a+1}
\]
We also have that since $G$ is a group, 
\[
1 =\varphi(1) = \varphi(g \cdot g^{-1}) = \varphi(g) \cdot \varphi(g^{-1}) \implies \varphi(g)^{-1} = \varphi(g^{-1})
\]
And 
\[
\forall g \in G \, \forall a \in \mathbb{Z}^- :\varphi(g^a) = \varphi((g^{-1})^{-a}) = \varphi(g^{-1})^{-a} = (\varphi(g)^{-1})^{-a} = \varphi(g)^a
\]
Therefore, $\forall g \in G \, \forall a \in \mathbb{Z}:\varphi(g^a) = \varphi(g)^a$
\pagebreak
\section*{4.}
\subsection*{1.} 
Since $S_3$ and $D_6$ has the same order(6), they are isomorphic
\subsection*{2.}
Since $S_4$ has order 8 and $D_{24}$ has order 24, they are not isomorphic
\subsection*{3.}
Consider 
$f: G \times H \to H \times G, \indent  (g,h) \to (h,g) $ \\
$\forall (h,g) \in H \times G: g \in G \land h \in H \\
\implies (g,h) \in G \times H \\
\implies f(g,h) = (h,g)$ \\
which proves that f is surjective.
If $exists (g_1,h_1), (g_2,h_2) \in G \times H$ such that $f(g_1,h_1) = f(g_2,h_2)$ then
$(h_1,g_1) = (h_2,g_2)$ which means that $h_1 = h_2$ and $g_1 = g_2$ and hence $(g_1,h_1) = (g_2,h_2)$. Therefore, f is injective and there $G \times H$ is isomorphic to $H \times G$.
\subsection*{4.}
Consider the function $id: G \to G, \indent g \to g$ \\
$\forall g \in G: id(g) = g$. \\ 
$\forall g_1, g_2 \in G$ such that $id(g_1) = id(g_2) \implies g_1 = g_2$. \\
Hence, $id$ is bijective. It is also obvious that $\forall g_1,g_2 \in G: id(g_1 \cdot g_2) = g_1 \cdot g_2 =id(g_1) \cdot id(g_2)$. Therefore, $id$ is isomorphic. \\
$\forall f \in \text{Aut}(G): G \to G, \indent g \to h$:\\
$f \circ id (g) = f(g) = id(f(g)) = id \circ f(g)$. \\
Therefore, $id$ is the identity. \\
$\forall f \in \text{Aut}(G): G \to G, \indent g \to h$\\
$\exists f^{-1}: G \to G, \indent h \to g$ such that $f\circ f^{-1} = f^{-1}\circ f = id$ \\
$\forall h_1,h_2 \in G$ such that $f^{-1}(h_1) = f^{-1}(h_2)$ then \\
$f \circ f^{-1} (h_1) = f \circ f^{-1}(h_2) \implies h_1 = h_2$ \\
$\forall g \in G:f(g) = h$ and therefore, $g = f^{-1} \circ f(g) = f^{-1}(h)$. \\
Therefore, $f^{-1}$ is bijective. \\
$\forall h_1,h_2 \in G: \exists g_1,g_2 \in G$ such that $f^{-1}(g_1) = h_1, f^{-1}(g_2) = h_2$, then \\
\[
f^{-1} (h_1 \cdot h_2) = f^{-1} (f(g_1) \cdot f(g_2)) = f^{-1}(f(g_1+g_2)) = g_1 + g_2 = f^{-1}(h_1) + f^{-2}(h_2)
\]
therefore $f^{-1}$ is an isomorphism and hence each function in Aut$(G)$ has an inverse also in Aut$(G)$
Function composition is associative.
Therefore, Aut$(G)$ is a group. 
\subsection*{5.}
$G$ and $H$ are isomorphic, there exists $g: G \to H$ and $g^{-1}: H \to G$	. \\
Consider $F:$ Aut($G) \to $ Aut ($H), \indent f \to g \circ f \circ g^{-1}$ \\
Since there is an obvious inverse of $F:$ Aut($H) \to $ Aut ($G), \indent h \to g^{-1} \circ h \circ g$:\\
\[
(F\circ F^{-1})(f) =F(g^{-1} \circ f  \circ g) = g \circ g^{-1} \circ f \circ g \circ g^{-1} = f
\]
Therefore, $F$ is bijective. $F(f_1 \circ f_2) = g \circ f_1 \circ f_2 \circ g^{-1} = g \circ f_1 \circ g^{-1} \circ g \circ f_2 \circ g^{-1} = F(f_1) \circ F(f_2)$. Hence, F is isomorphic, which means that Aut($G$) is isomorphic to Aut($H$)  
 
\pagebreak
\section*{6.}
\subsection*{1.}
\[
\forall a,b \in \mathbb{R} : (a + ia)(b + ib) = ab -ab + 2abi = 2abi \ne 1
\]
Therefore, there don't exists an inverse for all elements in the set \\$\{a + i\cdot a \, | a \in \mathbb{R}\}$ and hence is not a subgroup.
\subsection*{2.}
Consider $x = 1 , y = -1 \in \{z \in \mathbb{C} \land \|z\| = 1\}$
$x+y = 0 \notin \{z \in \mathbb{C} \land \|z\| = 1\}$ 
Hence $\{z \in \mathbb{C} \land \|z\| = 1\}$  is not a subgroup
\subsection*{3.}
Magnitude stay the same after $\cdot, g \cdot h \in$ the set\\
$\forall x \in \mathbb{C} \backslash \{0\}: \exists !z= \cfrac{x_1 - ix_2}{x_1^2+x_2^2}$ (since $x_1^2+x_2^2=0$ if and only if $x_1 = x_2 = 0$ which means that $x = 0+0i$) such that 
\[
x \cdot z = \left(x_1 \cdot \cfrac{x_1}{x_1^2 + x_2^2} - x_2 \cdot \frac{-x_2}{x_1^2 + x_2^2} \right) + i\left( x_1 \cdot \cfrac{-x_2}{x_1^2 + x_2^2} + x_2 \cdot \frac{x_1}{x_1^2 + x_2^2}\right) = 1+0i
\]
inverse exists as:
is a subgroup
\subsection*{4.}
Let the set be A. \\
$(1,2), (1,3) \in A$, but $(1,2)(1,3) = (1,3,2) \notin A$. 
Therefore, the set is not closed and not a subgroup.
\subsection*{5.}
Let the set be A. \\
$\forall a = sr^i \in A: sr \cdot sr^i = r^{n-1} s \cdot s r^i = r^{n-1+i} \notin A$.
Therefore, the set is not closed and not a subgroup.
\subsection*{6.}
Let the set be A. \\
If it is a subgroup, then as 0 is the identity in $\mathbb{Z}$ is an element of A, but 0 is even, which is a contradiction.
\subsection*{7.}
For $n \in \mathbb{N}$, let $A_n \subset \mathbb{Z}$ be the set contains integers which are divisible by $n$ that is $a \equiv 0 \text{ mod } n$. We have \\
If $a,b \in A_n: a \equiv 0 \text{ mod } n \land b \equiv 0 \text{ mod } n \implies a+b \equiv 0 \text{ mod } n$ and hence $a+b \in A_n$ \\
0 is a multiply of n, and hence $0 \in A_n$. Also $\forall a \in A_n: a + 0 = a$, which means that 0 is the identity.
$\forall a \in A_n: a \equiv 0 \text{ mod } n \implies -a \equiv 0 \text { mod } n \implies -a \in A_n $ and a+(-a) = 0 
Therefore, for any natural number $n, A_n$ satisfies all the conditions to be a group, that is closed, there is an identity and every element has an inverse.
\subsection*{8.}
Because of $r^4 = 1, s^2 = 1, rs^2 = sr^2, sr^2 = r^2s$ and 1 is the identity, we have the following table:
\begin{center}
\renewcommand\arraystretch{1.3}
\setlength\doublerulesep{0pt}
\begin{tabular}{r|*{4}{2|}}
$\cdot$ & 1 & r^2 & s & sr^2 \\
\hline
1 & 1 & r^2 & s & sr^2 \\ 
\hline
$r^2$ & r^2 & 1 & sr^2 & s \\ 
\hline
$s$ & s & sr^2 & 1 & r^2 \\ 
\hline
$sr^2$ & sr^2 & sr & r^2 & 1 \\ 
\hline
\end{tabular}
\end{center}
From the table, we can see that it satisfies all the conditions to be a group, that is closed, there is an identity and every element has an inverse.
\subsection*{9.}
Because of $r^4 = 1, s^2 = 1, rs = sr^3, sr = r^3s$ and 1 is the identity, we have the following table:
\begin{center}
\renewcommand\arraystretch{1.3}
\setlength\doublerulesep{0pt}
\begin{tabular}{r|*{4}{2|}}
$\cdot$ & 1 & r^2 & sr & sr^3 \\
\hline
1 & 1 & r^2 & sr & sr^3 \\ 
\hline
$r^2$ & 1 & 1 & sr^3 & sr \\ 
\hline
$sr$ & sr & sr^3 & 1 & r^2 \\ 
\hline
$sr^3$ & sr^3 & sr & r^2 & 1 \\ 
\hline
\end{tabular}
\end{center}
From the table, we can see that it satisfies all the conditions to be a group, that is closed, there is an identity and every element has an inverse.
\subsection*{10.}
Let $A$ be the set. $sr^2 \cdot s = r^3s \cdot s = r^3 \cdot s^2 = r^3 \notin A$. Hence A is not closed and not a subgroup.
\subsection*{11.}
Let $A$ be the set. $sr^3 \cdot r^2 = sr^5 = s \notin A$. Hence A is not closed and not a subgroup.
\subsection*{12.}
Let the set be A.
We have that $i \cdot i = -1 \notin A$
Hence $A$ is not closed and not a subgroup.
\subsection*{13.}
Because of $i^2 = 1, -i = -1 \cdot i, -1 \cdot (-1) = 1$ and 1 is the identity, we have the following table:
\begin{center}
\renewcommand\arraystretch{1.3}
\setlength\doublerulesep{0pt}
\begin{tabular}{r|*{4}{2|}}
$\cdot$ & 1 & -1 & i & -i \\
\hline
1 & 1 & -1 & i & -i \\ 
\hline
-1 & -1 & 1 & -i & i \\ 
\hline
i & i & -i & 1 & 1 \\ 
\hline
-i & -i & i & 1 & -1 \\ 
\hline
\end{tabular}
\end{center}
From the table, we can see that it satisfies all the conditions to be a group, that is closed, there is an identity and every element has an inverse.
\subsection*{14.}
Let the set be $A$. $A$ contains all cycles of length 2 (as stated in the first question of the assignment).
Since $\forall \sigma \in A, \exists a,b,c,d \in \{1,2,3,4\}: \sigma = (a,b)(c,d)$. Therefore, all cycles in $A$ has even order as the identity can be written as 0 transposition. And since $\forall \sigma_1, \sigma_2 \in A \subset S_4:$
\[
\sigma_1 \sigma_2 \text{ is even and } \sigma_1 \sigma_2 \in S_4 \text{ which has order} \le 4 
\]
Therefore, $\forall \sigma_1, \sigma_2 \in A: \sigma_1 \sigma_2 \in A$ \\
$A$ has the identity element. \\
$\forall \sigma \in A \subset S_4: \exists \sigma^{-1} \in S_4: \sigma \sigma^{-1} = 1$
As $\sigma $and 1 are two known even cycle, $\sigma^{-1} \in S_4$ also has even cycle and therefore is an element of $A$ \\
Therefore, $A$ is a subgroup.

\end{document}