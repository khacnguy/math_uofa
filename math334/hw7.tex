\documentclass[11pt]{article}
    \title{\textbf{Stat 265 Homework I}}
    \author{Khac Nguyen Nguyen}
    \date{}
    
    \addtolength{\topmargin}{-3cm}
    \addtolength{\textheight}{3cm}
    
\usepackage{amsmath}
\usepackage{mathtools}
\usepackage{amsthm}
\usepackage{amssymb}
\usepackage{pgfplots}
\usepackage{dcolumn}
\newcolumntype{2}{D{.}{}{2.0}}
\usepgfplotslibrary{polar}
\usepgflibrary{shapes.geometric}
\usetikzlibrary{calc}
\pgfplotsset{compat = newest}
\pgfplotsset{my style/.append style = {axis x line = middle, axis y line = middle, xlabel={$x$}, ylabel={$y$}, axis equal}}
\begin{document}
\section*{1 p.188}
\[
    y'''-y''-y'+y = 2e^{-t} + 3    
\]
Then as 
\[
    r^3 - r^2-r+1= (r-1)^2(r+1)=0  \iff r\in \{1,-1\}
\]
we have that 
\[
    y_c(t) = c_1 e^t + c_2 te^t + c_3 e^{-t}    
\]
as 
\[  W(0) = 
    \left.
    \begin{vmatrix}
        e^t & e^{-t} & te^t \\
        e^t & -e^{-t} & e^t + te^t\\
        e^t & e^{-t} & 2e^t + te^t
    \end{vmatrix} \right._{t=0}
    = 
    \begin{vmatrix}
        1 & 1 & 0 \\
        1 & -1 & 1\\
        1 & 1 & 2
    \end{vmatrix}
    = -4 \ne 0
\]
Let 
\[
    y_p(t) = Ate^{-t} + B
\]
Then 
\begin{equation*}
    \begin{aligned}
        y' &= Ae^{-t} - Ate^{-t} \\
        y''&= -2Ae^{-t} + Ate^{-t} \\
        y''' &= 3Ae^{-t} - Ate^{-t}
    \end{aligned}
\end{equation*}
Hence, 
\[
    y''' - y'' - y' + y = 4Ae^{-t} + B = 2e^{-t} + 3
\]
Hence, 
\[
    y(t) = \frac{1}{2}te^{-t} + 3 + c_1 e^t + c_2 te^t + c_3 e^{-t} 
\]
\newpage
\section*{6 p.188}
\[
    y^{(6)} + y''' = t    
\]
Then as 
\[
    r^6 + r^3 = r^3 (r+1)(r^2-r+1)
\]
Hence, 
\[
    y_c(t) = c_1 + c_2t + c_3t^2 + c_4e^{-t} + c_5 e^{t/2}\sin(\sqrt{3}t/2) + c_6 e^{t/2}\cos(\sqrt{3}t/2)     
\]
Then let 
\[
    y = At^4
\]
We have that 
\[
    y''' = 24At
\]
and 
\[
    y^{(6)} = 0
\]
which means that $A = 1/24$ and hence 
\[
    y(t) = \frac{1}{24}t^4  c_1 + c_2t + c_3t^2 + c_4e^{-t} + c_5 e^{t/2}\sin(\sqrt{3}t/2) + c_6 e^{t/2}\cos(\sqrt{3}t/2)     
\]
\newpage
\section*{2 p.192}
\[
    y'''-y' =t    
\]
Then as 
\[
    r^3-r = r(r-1)(r+1)     
\]
we have that 
\[
    y_c(t) = c_1 + c_2e^t + c_3e^{-t}
\]
Then we can calculate
\[
    W(t) = 
    \begin{vmatrix}
        1 & e^t & e^{-t} \\
        0 & e^t & -e^{-t} \\
        0 & e^t & e^{-t}     
    \end{vmatrix} 
    = 2 \ne 0
\]
\[
    W^{(1)}(t) = 
    \begin{vmatrix}
        0 & e^t & e^{-t} \\
        0 & e^t & -e^{-t} \\
        t & e^t & e^{-t}     
    \end{vmatrix} 
    = -2t
\]
\[
    W^{(2)}(t) = 
    \begin{vmatrix}
        1 & 0 & e^{-t} \\
        0 & 0 & -e^{-t} \\
        0 & t & e^{-t} 
    \end{vmatrix} 
    = te^{-t}
\]
\[
    W^{(3)}(t) = 
    \begin{vmatrix}
        1 & e^t & 0  \\
        0 & e^t & 0 \\
        0 & e^t & t     
    \end{vmatrix} 
    = te^t
\]
and then 
\[
    u_1(t) = \int \frac{-2t}{2} dt = -t^2/2
\]
\[
    u_2(t) = \int \frac{te^{-t}}{2} dt = -\frac{te^{-t} + e^{-t}}{2} 
\]
\[
    u_3(t) = \int \frac{te^t}{2} dt = \frac{te^t - e^t}{2}
\]
and therefore
\[
    y_p(t) = - \frac{t^2}{2} - \frac{te^{-t} + e^{-t}}{2} e^t + \frac{te^t -e^t}{2} \cdot e^{-t} = \frac{-t^2}{2} + 1    
\]
and hence
\[
    y(t) = -t^2/2 + c_1 + c_2e^t + c_3e^{-t}
\]
\newpage
\section*{2 p.192}
We have that $a_n = \frac{n}{2^n}$. Then as 
\[
    \left| \frac{a_n}{a_{n+1}} \right| =\left| \frac{2n}{(n+1)}\right|  
\]
which converges to $2$ as $n \to \infty$. Hence, $R=2$
\section*{4 p.192}
We have that $a_n = 2^n$. Then as 
\[
    \left| \frac{a_n}{a_{n+1}} \right| =\left| \frac{1}{2}\right|  
\]
which converges to $1/2$ as $n \to \infty$. Hence, $R=1/2$
\section*{6 p.192}
We have that $a_n = \frac{(-1)^n n^2}{3^n}$. 
Then as 
\[
    \left| \frac{a_n}{a_{n+1}} \right| =\left| \frac{3n^2}{(n+1)^2}\right|  
\]
which converges to $3$ as $n \to \infty$. Hence, $R=3$
\end{document}