\documentclass[11pt]{article}
    \title{\textbf{Math 217 Homework I}}
    \author{Khac Nguyen Nguyen}
    \date{}

    \addtolength{\topmargin}{-3cm}
    \addtolength{\textheight}{3cm}

\usepackage{amsmath}
\usepackage{mathtools}
\usepackage{amsthm}
\usepackage{amssymb}
\usepackage{pgfplots}
\usepackage{xfrac}
\usepackage{hyperref}
\usepackage{graphicx}
\long\def\comment#1{}

\usepgfplotslibrary{polar}
\usepgflibrary{shapes.geometric}
\usetikzlibrary{calc}
\pgfplotsset{compat = newest}
\pgfplotsset{my style/.append style = {axis x line = middle, axis y line = middle, xlabel={$x$}, ylabel={$y$}, axis equal}}
\begin{document}
\section*{1 p.261}
\[
    \mathcal{L}(\sin(2t)) = \cfrac{2}{s^2+4}
\]
Hence, 
\[
    \mathcal{L}^{-1}\left(\frac{3}{s^2+4}\right) = \frac{3}{2}\sin(2t)    
\]
\newpage
\section*{3 p.261}
\[
    \mathcal{L}(\frac{1}{s-a}) = e^{at}
\]
Hence, 
\[
    \mathcal{L}^{-1}\left(\frac{2}{s^2+3s-4}\right) = \frac{2}{5}\mathcal{L}^{-1}\left( \frac{1}{s-1} - \frac{1}{s+4} \right) = \frac{2}{5}e^t - \frac{2}{5}e^{-4t}
\]
\newpage
\section*{5 p.261}
\[
    \mathcal{L}(\frac{1}{s-a}) = e^{at}
\]
Hence, 
\[
    \mathcal{L}^{-1}\left(\frac{2s-3}{s^2-4}\right) = \mathcal{L}^{-1}\left( \frac{1}{4(s-2)} + \frac{7}{4(s+2)} \right) = \frac{1}{4}e^{2t} + \frac{7}{4}e^{-2t}   
\]
\newpage
\section*{7 p.261}
\[
    \mathcal{L}^{-1}\left(\frac{1-2s}{s^2+4s+5} \right) = \mathcal{L}^{-1} \left(\frac{-2(s+2) + 5}{(s+2)^2 + 1} \right) = e^{-2t} \mathcal{L}^{-1} \left( \frac{-2s+5}{s^2+1} \right) = e^{-2t} \left(-2 \cos(t) + 5\sin(t)\right)
\]
\newpage
\section*{8 p.261}
Let $Y(s) = \mathcal{L}[y(t)]$. Then 
\begin{equation*}
    \begin{aligned}
        &y'' - y' -6y = 0\\
        \implies &s^2 Y(s) - sy(0) - 6y'(0) - (sY(s) - y(0)) - 6Y(s)  = 0 \\
        \implies &s^2 Y(s) - s +6 - sY(s) +1 -6Y(s) = 0 \\
        \implies &Y(s) = \frac{s-7}{s^2-s-6} = \frac{9}{5(s+2)} - \frac{4}{5(s-3)}
    \end{aligned}
\end{equation*}
Hence, the unique solution is 
\[
    y(t) = \frac{9}{5}e^{-2t} - \frac{4}{5}e^{3t}    
\]
\newpage
\section*{10 p.261}
Let $Y(s) = \mathcal{L}[y(t)]$. Then 
\begin{equation*}
    \begin{aligned}
        &y'' - 2y' +2y = 0\\
        \implies &s^2 Y(s) - sy(0) - 6y'(0) - 2(sY(s) - y(0)) + 2Y(s)  = 0 \\
        \implies &s^2 Y(s)  - 6 - 2sY(s) + 2Y(s) = 0 \\
        \implies &Y(s) = \frac{6}{s^2-s+2} = \frac{6}{(s-1/2)^2 +7/4}
    \end{aligned}
\end{equation*}
Hence, the unique solution is 
\[
    y(t) = 6e^{t/2} \sin(\sqrt{7}t/2) \frac{2}{\sqrt{7}} = \frac{12\sqrt{7}}{7} e^{t/2} \sin\left( \frac{\sqrt{7}t}{2}\right)
\]
\newpage
\section*{20a p.261}
\begin{equation*}
    \begin{aligned}
        \mathcal{L}(\sin(t)) 
        &= \mathcal{L}\left(\sum_{n=0}^\infty \frac{(-1)^n t^{2n+1}}{(2n+1)!}\right)  \\
        &= \sum_{n=0}^\infty \mathcal{L}\left(\frac{(-1)^n t^{2n+1}}{(2n+1)!}\right)  \\
        &= \sum_{n=0}^\infty \int_0^\infty e^{-st} \frac{(-1)^n t^{2n+1}}{(2n+1)!} dt \\
        &= \sum_{n=0}^\infty \frac{(-1)^n}{(2n+1)!} \int_0^\infty e^{-st} t^{2n+1} dt \\
        &= \sum_{n=0}^\infty \frac{(-1)^n}{(2n+1)!} \mathcal{L}(t^{2n+1}) \\
        &= \sum_{n=0}^\infty \frac{(-1)^n}{(2n+1)!} \frac{(2n+1)!}{s^{2n+2}} \\
        &= \sum_{n=0}^\infty \frac{(-1)^n}{s^{2n+2}} \\
        &= \frac{1}{s^2} \sum_{n=0}^\infty \frac{(-1)^n}{(s^2)^n} \\
        &= \frac{1}{s^2} \frac{1}{1-\frac{-1}{s^2}} \\
        &= \frac{1}{s^2+1}
    \end{aligned}
\end{equation*}
\newpage
\section*{21a p.262}
\begin{equation*}
    \begin{aligned}
        F(s) &= \int_0^\infty e^{-st} f(t) dt \\
        \implies F'(s) &= \int_0^\infty \frac{\partial}{\partial s} (e^{-st} f(t)) \\
        &= \int_0^\infty -te^{-st}f(t)dt \\
        &= \mathcal{L}(-tf(t))
    \end{aligned}
\end{equation*}
\newpage
\section*{23 p.262}
Let $g(t) = \sin(bt)$ 
\[
    F(s) = \mathcal{L}(g(t)) = \frac{b}{s^2+b^2}
\]
Hence, 
\[
    F''(s) = \mathcal{L}(t^2g(t)) = \mathcal{L}(f(t)) = \frac{\partial^2}{\partial^2 s}\frac{b}{s^2+b^2} = \dfrac{2b\left(3s^2-b^2\right)}{\left(s^2+b^2\right)^3}   
\]

\end{document}