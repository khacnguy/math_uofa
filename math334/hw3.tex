\documentclass[11pt]{report}
    \title{\textbf{Math 217 Homework I}}
    \author{Khac Nguyen Nguyen}
    \date{}

    \addtolength{\topmargin}{-3cm}
    \addtolength{\textheight}{3cm}

\usepackage{amsmath}
\usepackage{mathtools}
\usepackage{amsthm}
\usepackage{amssymb}
\usepackage{pgfplots}
\usepackage{xfrac}
\usepackage{hyperref}

\usepgfplotslibrary{polar}
\usepgflibrary{shapes.geometric}
\usetikzlibrary{calc}
\pgfplotsset{compat = newest}
\pgfplotsset{my style/.append style = {axis x line = middle, axis y line = middle, xlabel={$x$}, ylabel={$y$}, axis equal}}
\begin{document}
\section*{3 p.77}
\begin{equation*}
    \begin{aligned}
        & (3x^2 - 2xy + 2) + (6y^2-x^2+3)y' = 0 \\
        \implies &(6y^2 - x^2 + 3)dy + (3x^2-2xy+2)dx = 0 \\
    \end{aligned}
\end{equation*}
The DE is exact since
\[
    \frac{\partial }{\partial x}(6y^2-x^2+3) = -2x = \frac{\partial}{\partial y}  (3x^2-2xy+2)
\]
Then we have that
\[
    \varphi (x,y) = \int (3x^2-2xy+2) dx = x^3-x^2y +2x + c(y)
\]
\[
    \frac{\partial \varphi}{\partial y} = \frac{\partial}{\partial y}(x^3-x^2y+2x+c(y)) = 6y^2 - x^2 + 3
\]
Hence,
\[
    c(y) = 2y^3 + 3y
\]
and therefore
\[
    \varphi(x,y) = x^3-x^2y +2x + 2y^3+3y = c
\]
is the solution.
\newpage
\section*{4 p.77}
\begin{equation*}
    \begin{aligned}
        &\frac{dy}{dx} = - \frac{ax+by}{bx+cy} \\
        \implies & (bx+cy)dy + (ax+by)dx = 0
    \end{aligned}
\end{equation*}
The DE is exact as
\[
    \frac{\partial }{\partial x}(bx+cy) = b = \frac{\partial }{\partial y}(ax+by)
\]
Then we have that
\[
    \varphi(x,y) = \int (ax+by) dx = \frac{ax^2}{2} + bxy + c(y)
\]
\[
    \frac{\partial \varphi}{\partial y} = \frac{\partial}{\partial y}(\frac{ax^2}{2}+bxy+c(y)) = bx+cy
\]
Hence,
\[
    c(y) = \frac{cy^2}{2}
\]
and therefore
\[
    \varphi(x,y) = \frac{ax^2}{2} + bxy + \frac{cy^2}{2} = c
\]
is the solution to the DE.
\newpage
\section*{7 p.77}
\begin{equation*}
    \begin{aligned}
        &\left( \frac{y}{x} +6x \right) + ( \ln x - 2) \frac{dy}{dx} = 0 \\
        \implies &\left( \frac{y}{x} +6x \right)dx + ( \ln x - 2) dy = 0
    \end{aligned}
\end{equation*}
The DE is exact as
\[
    \frac{\partial}{\partial x}(\ln x-2) = \frac{1}{x} = \frac{\partial}{\partial y}\left(\frac{y}{x} +6x\right)
\]
Then we have that
\[
    \varphi(x,y) = \int \left(\frac{y}{x} + 6x \right) dx = y \ln x + 3x^2 + c(y)
\]
\[
    \frac{\partial \varphi}{\partial y} = \frac{\partial }{\partial y}\left(y \ln x + 3x^2 + c(y) \right) = \ln x - 2
\]
Hence,
\[
    c(x) =  -2y
\]
and therefore,
\[
    \varphi(x,y) = y\ln x + 3x^2 -2y = c
\]
is the solution to the DE.
\newpage
\section*{10 p.77}
\begin{equation*}
    \begin{aligned}
        &\left( 9x^2+y-1\right) + (x-4y)\frac{dy}{dx} = 0 \\
        \implies &\left( 9x^2+y-1\right)dx + (x-4y) dy = 0
    \end{aligned}
\end{equation*}
The DE is exact as
\[
    \frac{\partial}{\partial x}(x-4y) = 1 = \frac{\partial}{\partial y}\left(9x^2 + y -1\right)
\]
Then we have that
\[
    \varphi(x,y) = \int \left(9x^2 +y -1 \right) dx = 3x^3 + yx- x + c(y)
\]
\[
    \frac{\partial \varphi}{\partial y} = \frac{\partial }{\partial y}\left(3x^3 +yx-x+ c(y) \right) = x-4y
\]
Hence,
\[
    c(x) =  -2y^2
\]
and therefore,
\[
    \varphi(x,y) = 3x^3 +yx-x-2y^2 = c
\]
is the solution to the DE.
Since $y(1) = 0, c = 2$. Therefore,
\begin{equation*}
    \begin{aligned}
        &3x^3 + yx -x - 2y^2 -2= 0 \\
        \implies & -2\left(y - \frac{x}{4}\right)^2 + \frac{x^2}{8} + 3x^3 -2 = 0 \\
        \implies & \left( y -\frac{x}{4} \right)^2 = \frac{x^2}{16} + \frac{3x^3}{2} -1 \\
    \end{aligned}
\end{equation*}
Hence, 
\begin{equation*}
    \begin{aligned}
        &\frac{x^2}{16} + \frac{3x^3}{2} - 1> 0 \\
        &24x^3 + x^2 - 16 > 0
    \end{aligned}
\end{equation*}
need to be true so that the solution is valid.
\newpage
\section*{1 p.112}
\begin{equation*}
    \begin{aligned}
        &y'' +2y' - 3y = 0 \\
        \implies &r^2 e^{rt} + 2re^{rt} - 3e^{rt} = 0 \\
        \implies &r \in \{1,-3\}
    \end{aligned}
\end{equation*}
Therefore,
\[
    y(t) = c_1e^{t} + c_2e^{-3t}
\]
\newpage
\section*{3 p.112}
\begin{equation*}
    \begin{aligned}
        & 6y'' - y'-y = 0 \\
        \implies & 6r^2 e^{rt} - re^{rt} - e^{rt} = 0 \\
        \implies & r \in \left\{ \frac{1}{2}, \frac{-1}{3} \right\} \\
    \end{aligned}
\end{equation*}
Therefore,
\[
    y(t) = c_1e^{t/2} + c_2e^{-t/3}
\]
\newpage
\section*{10 p.112}
\begin{equation*}
    \begin{aligned}
        & 2y'' + y'-4y = 0 \\
        \implies & 2r^2 e^{rt} + 1re^{rt} - 4e^{rt} = 0 \\
        \implies & r \in \left\{\frac{-1 + \sqrt{33}}{4},  \frac{-1 - \sqrt{33}}{4}\right\} \\
    \end{aligned}
\end{equation*}
Therefore,
\[
    y(t) = c_1e^{\frac{-1 + \sqrt{33}}{4}t} + c_2e^{\frac{-1 - \sqrt{33}}{4}t}
\]
as $t \to \infty$, $y(t) \to c_1^{\frac{-1 + \sqrt{33}}{4}t}$,
which diverges to $\infty$ if $c_1>0$, diverges to $-\infty$ if $c_1 < 0$, and converges to $0$ if $c_1 = 0$
\newpage
\section*{12 p.112}
\begin{equation*}
    \begin{aligned}
        & 4y'' - y = 0 \\
        \implies & 4r^2 e^{rt} - e^{rt} = 0 \\
        \implies & r \in \left\{\frac{1}{2},  -\frac{1}{2}\right\} \\
    \end{aligned}
\end{equation*}
Therefore,
\[
    y(t) = c_1e^{t/2} + c_2e^{-t/2}
\]
as $t \to \infty$, $y(t) \to c_1e^{t/2}$,
which diverges to $\infty$ if $c_1>0$, diverges to $-\infty$ if $c_1 < 0$, and converges to $0$ if $c_1 = 0$
\end{document}