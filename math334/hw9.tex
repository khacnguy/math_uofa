\documentclass[11pt]{article}
    \title{\textbf{Math 217 Homework I}}
    \author{Khac Nguyen Nguyen}
    \date{}

    \addtolength{\topmargin}{-3cm}
    \addtolength{\textheight}{3cm}

\usepackage{amsmath}
\usepackage{mathtools}
\usepackage{amsthm}
\usepackage{amssymb}
\usepackage{pgfplots}
\usepackage{xfrac}
\usepackage{hyperref}
\usepackage{graphicx}
\long\def\comment#1{}

\usepgfplotslibrary{polar}
\usepgflibrary{shapes.geometric}
\usetikzlibrary{calc}
\pgfplotsset{compat = newest}
\pgfplotsset{my style/.append style = {axis x line = middle, axis y line = middle, xlabel={$x$}, ylabel={$y$}, axis equal}}
\begin{document}
\section*{4 p.223}
$x=0$ is the singular point, hence let $y = x^r$, we have that 
\begin{equation*}
    \begin{aligned}
        &x^2y'' -xy' + y = 0  \\
        \implies & r(r-1)x^r - rx^r + x^r = 0 \\
        \implies & r^2x^r - 2rx^r + x^r = 0 \\
        \implies & x^r(r^2-2r+1) = 0 \\
        \implies & r = 1
    \end{aligned}
\end{equation*}
Hence, the solution is 
\[
    y = c_1x + c_2x \ln(x)    
\]
for $x>0$
\newpage
\section*{6 p.223}
$x=0$ is the singular point, hence let $y = x^r$, we have that 
\begin{equation*}
    \begin{aligned}
        &2x^2y'' -4xy' + 6y = 0  \\
        \implies & 2r(r-1)x^r - 4rx^r + 6x^r = 0 \\
        \implies & 2r^2x^r - 6rx^r + 6x^r = 0 \\
        \implies & x^r(2r^2-6r+6) = 0 \\
        \implies &r = \frac{6 \pm \sqrt{6^2-4\cdot 2 \cdot 6}}{4} = \frac{3 \pm i\sqrt{3}}{2}
    \end{aligned}
\end{equation*}
Hence, the solution is 
\[
    y = c_1x^{3/2}\cos\left(\frac{\sqrt{3}}{2}\ln(x)\right) + c_2x^{3/2}\sin\left(\frac{\sqrt{3}}{2}\ln(x)\right)   
\]
\newpage
\section*{10 p.223}
$x=0$ is the singular point, hence let $y = x^r$, we have that 
\begin{equation*}
    \begin{aligned}
        & 4x^2y'' + 8xy' + 17y = 0  \\
        \implies & 4r(r-1)x^r + 8rx^r + 17x^r = 0 \\
        \implies & 4r^2x^r + 4rx^r + 17x^r = 0 \\
        \implies & x^r(4r^2+4r+17) = 0 \\
        \implies &r = \frac{-4 \pm \sqrt{4^2-4\cdot 4 \cdot 17}}{4} =  -1 \pm 4i
    \end{aligned}
\end{equation*}
Hence, the solution is 
\[
    y = c_1\frac{1}{x}\cos\left(4\ln(x)\right) + c_2\frac{1}{x}\sin\left(4\ln(x)\right)   
\]
Then we know that 
\[
    y(1) = c_1 = 2
\] 
and 
\[
    y'(1) = \left. c_1 \frac{-1}{x^2}\cos(4\ln(x)) + c_2 \frac{1}{x} \cos(4\ln(x)) \cdot \frac{4}{x} \right|_{x=1} = -c_1 + 4c_2 = -3    
\]
Solving the equations we have that 
\[
    y = \frac{2}{x}\cos\left(4\ln(x)\right) -\frac{1}{4x}\sin\left(4\ln(x)\right)   
\]
As $x\to 0$, $y(x)$ fluctuates around 0 where the oscillation diverges to $\infty$. 
\newpage
\section*{1 p.228}
we have that 
\[
    \lim_{x\to 0} x\frac{1}{2x} = \frac{1}{2}     
\]
and 
\[
    \lim_{x\to 0} x\frac{x}{2x} = 0    
\]
Hence, $x=0$ is a regular singular point.
Assume $y(x) = \sum_{n=0}^\infty a_nx^{n+r}$. Then 
\begin{equation*}
    \begin{aligned}
        &2xy'' + y' + xy = 0 \\
        \implies &  \sum_{n=0}^\infty 2a_n(r+n)(r+n-1)x^{n+r-1} + a_n (r+n)x^{r+n-1} +  a_nx^{r+n+1} = 0 \\
        \implies &  \sum_{n=0}^\infty 2a_n(r+n)(r+n-1)x^{n+r-1} + a_n (r+n)x^{r+n-1} +  \sum_{n=2}^\infty a_{n-2}x^{r+n-1} = 0 \\
    \end{aligned}
\end{equation*}
Therefore, 
\[
    2a_0 r(r-1)x^{r-1} + a_0 rx^{r-1} + 2a_1(r+1)rx^r + a_1(r+1)x^r = 0  
\]
which can be simplify to
\[
    a_0x^{r-1}(2r^2 - r) + a_1x^r(2r^2 + 3r + 1) = 0    
\]
Hence, the indicial equation is
\[
    2r^2-r=0
\]
Thus $r=1/2$ is the larger root and the smaller root is $0$.
Hence, $a_1 = 0$ and for $n\ge 2$, the recurrence relation is 
\[
    2a_n(r+n)(r+n-1) + a_n(r+n) = a_n(1+2n)n= -a_{n-2}
\]
which means that 
\[
    a_n = \frac{-1}{n(2n+1)}a_{n-2}    
\]
Thus, for every $n$, 
\[
    a_{2n} = a_0  \frac{(-1)^n}{ \prod_{i=1}^n 2i (4i + 1)}    
\]
and $a_{2n+1} = 0$
so we can write the power series based on $a_n$. \\
If we choose $r = 0$, we have that $a_1 = 0$ and the recurrente relation is 
\[
    2a_n(r+n)(r+n-1) + a_n(r+n) = a_n n(2n-1)= -a_{n-2}
\]
which means that 
\[
    a_n = \frac{-1}{n(2n-1)}a_{n-2}    
\]
Thus, for every $n$, 
\[
    a_{2n} = a_0  \frac{(-1)^n}{ \prod_{i=1}^n 2i (4i - 1)}    
\]
and $a_{2n+1} = 0$
so we can write the power series based on $a_n$.
\end{document}