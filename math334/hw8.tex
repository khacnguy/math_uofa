\documentclass[11pt]{article}
    \title{\textbf{Math 217 Homework I}}
    \author{Khac Nguyen Nguyen}
    \date{}

    \addtolength{\topmargin}{-3cm}
    \addtolength{\textheight}{3cm}

\usepackage{amsmath}
\usepackage{mathtools}
\usepackage{amsthm}
\usepackage{amssymb}
\usepackage{pgfplots}
\usepackage{xfrac}
\usepackage{hyperref}
\usepackage{graphicx}
\long\def\comment#1{}

\usepgfplotslibrary{polar}
\usepgflibrary{shapes.geometric}
\usetikzlibrary{calc}
\pgfplotsset{compat = newest}
\pgfplotsset{my style/.append style = {axis x line = middle, axis y line = middle, xlabel={$x$}, ylabel={$y$}, axis equal}}
\begin{document}
\section*{1 p.208}
\begin{equation*}
    \begin{aligned}
        &y''-y=0 \\
        \implies &\sum_{k=2}^\infty k(k-1)a_kx^{k-2} - \sum_{k=0}^\infty a_kx^k = 0 \\
        \implies &\sum_{k=0}^\infty (k+1)(k+2)a_{k+2}x^k  - a_kx^k = 0 \\
        \implies &(k+1)(k+2)a_{k+2}-a_k = 0 \\
        \implies &a_{k+2} = \frac{a_k}{(k+1)(k+2)}
    \end{aligned}
\end{equation*}
and hence 
\[
    a_{2k} = a_0 \frac{1}{(2k)!} \text{ and } a_{2k+1} = a_1 \frac{1}{(2k+1)!}
\]
and hence 
\[
    y(x) = a_0 \underbrace{\sum_{k=0}^\infty \frac{1}{(2k)!}x^{2k}}_{y_1} + a_1 \underbrace{\sum_{k=0}^\infty \frac{1}{(2k+1)!}x^{2k+1}}_{y_2}
\]
Hence, 
\[
    W(y_1, y_2, 0) = 
    \begin{vmatrix}
        1 & 0 \\
        0 & 1
    \end{vmatrix} = 1
\]
\newpage
\section*{3 p.208}
\begin{equation*}
    \begin{aligned}
        & y''-xy' -y = 0 \\
        \implies & \sum_{k=0}^\infty (k+1)(k+2)a_{k+2}x^k - x \sum_{k=0}^\infty (k+1)a_{k+1}x^k - \sum_{k=0}^\infty a_k x^k = 0 \\
        \implies & \sum_{k=0}^\infty (k+1)(k+2)a_{k+2}x^k -  \sum_{k=1}^\infty ka_kx^k - \sum_{k=0}^\infty a_k x^k = 0 \\
        \implies & 2a_2 - a_0 + \sum_{k=1}^\infty ((k+1)(k+2)a_{k+2} - ka_k - a_k)x^k = 0 \\
        \implies & 
        \begin{cases}
            a_0 = 2a_2 \\
            (k+1)(k+2)a_{k+2} - ka_k - a_k = 0 \forall k > 0
        \end{cases} \\
        \implies & 
        \begin{cases}
            a_0 = 2a_2 \\
            a_{k+2} = \cfrac{a_k}{k+2} \forall k > 0
        \end{cases}
    \end{aligned}
\end{equation*}
Hence, 
\[
    a_{2k} = \frac{a_{2k-2}}{2k} = \frac{a_2}{2^{k-1} \cdot k!} = \frac{a_0}{2^k \cdot k!}
\]
for $k >0$, note that it also works for $k=1$. 
\[
    a_{2k+1} = \frac{a_1}{(2k+1)(2k-1)\hdots 1} = \frac{a_1 \prod_{i=1}^k 2i}{\prod_{i=0}^k (2i+1) \prod_{i=1}^k 2i} = \frac{a_1 2^{k+1} k!}{(2k+1)!} 
\]
Hence, 
\[
    y(x) = a_0 \sum_{k=0}^\infty \frac{1}{2^k \cdot k!} x^k + a_1 \sum_{k=1} \frac{2^{k+1}k!}{(2k+1)!}x^k     
\]
\newpage
\section*{4 p.214}
There is no singularity hence the radius of convergence is $\infty$.
\newpage
\section*{5 p.214}
We can rewrite the equations as 
\[
    y'' + \frac{x}{x^2-2x-3}y' + \frac{4}{x^2-2x-3}y = 0     
\]
The singularities are $3.-1$. Hence, the radius of convergence is 
\[
    R = 
    \begin{cases}
        1, \text{ if } x_0 = 4 \\
        3, \text{ if } x_0 = -4 \\
        1, \text{ if } x_0 = 0 \\
    \end{cases}    
\]
\section*{6 p.214}
We can rewrite the equation as 
\[
    y'' + \frac{4x}{1+x^3}y' + \frac{1}{1+x^3} y = 0   
\]
and hence the singularities can be found by letting $1+x^3=0$ which yields $x = -1, e^{i\pi/3}, e^{2i\pi/3}$, which are the three points dividing the unit circle into 3 equal sections. Hence, for $x_0=0$, the radius of convergence is $1$.
For $x_0 = 2$, the radius of convergence will be 
\[
    \left| 2 - \left(\frac{1}{2} + i\frac{\sqrt{3}}{2}\right)\right| = \sqrt{3}   
\] 
\end{document}