\documentclass[11pt]{article}
    \title{\textbf{Math 217 Homework I}}
    \author{Khac Nguyen Nguyen}
    \date{}
    
    \addtolength{\topmargin}{-3cm}
    \addtolength{\textheight}{3cm}
    
\usepackage{amsmath}
\usepackage{mathtools}
\usepackage{amsthm}
\usepackage{amssymb}
\usepackage{pgfplots}
\usepackage{xfrac}
\usepackage{hyperref}



\newtheorem{definition}{Definition}[section]
\newtheoremstyle{mystyle}%                % Name
  {}%                                     % Space above
  {}%                                     % Space below
  {\itshape}%                                     % Body font
  {}%                                     % Indent amount
  {\bfseries}%                            % Theorem head font
  {}%                                    % Punctuation after theorem head
  { }%                                    % Space after theorem head, ' ', or \newline
  {\thmname{#1}\thmnumber{ #2}\thmnote{ (#3)}}%                                     % Theorem head spec (can be left empty, meaning `normal')

\theoremstyle{mystyle}
\newtheorem{theorem}{Theorem}[section]
\theoremstyle{definition}
\newtheorem*{exmp}{Example}
\begin{document}
\section*{1.}
Solve $u_t + 3tu_x = u$ with $u(0,x) = h(x)$ for $x \in \mathbb{R}$.
Follow the method, we have 
\[
  \displaystyle\frac{dT(\sigma; s)}{ds} = 1, 
  \displaystyle\frac{dX(\sigma; s)}{ds} = 3t, 
  \displaystyle\frac{dU(\sigma; s)}{ds} = U
\]
with 
\[
  T(\sigma; 0) = 0, 
  X(\sigma; 0) = \sigma,
  U(\sigma; 0) = h(\sigma)
\]
Thus, 
\[
  T(\sigma; s) = s, X(\sigma;s) = \sigma + s, U(\sigma; s) = e^{s + \ln(h(\sigma))} 
\]
Next, we can obtain the inverse function, 
\[
  \begin{cases}
    t = T(\sigma;s) = s \\
    x = X(\sigma;s) = \sigma + s
  \end{cases}
  \implies 
  \begin{cases}
    \sigma = \Sigma(t,x) = x - t \\
    s = S(t,x) = t
  \end{cases}
\]
Finally, the solution is obtained as 
\[
  u(t,x) = U(\Sigma(t,x), S(t,x)) = e^{t + \ln(h(x-t))} 
\]
\pagebreak
\section*{2.}
Solve $u_t + u u_x = 2t$ with $u(0,x) = x$. 
Follow the method, we have 
\[
  \displaystyle\frac{dT(\sigma; s)}{ds} = 1, 
  \displaystyle\frac{dX(\sigma; s)}{ds} = U, 
  \displaystyle\frac{dU(\sigma; s)}{ds} = 2t
\]
with 
\[
  T(\sigma;0) = 0, X(\sigma;0) = \sigma, U(\sigma;0) = \sigma
\]
Thus, 
\[
  T(\sigma; s) = s, X(\sigma;s) = ts^2+ \sigma s + \sigma , U(\sigma;s) = 2ts + \sigma
\]
Next, we can obtain the inverse function, 
\[
  \begin{cases}
    t = T(\sigma;s) = s \\
    x = X(\sigma;s) = ts^2 + \sigma s + \sigma
  \end{cases}
  \implies 
  \begin{cases}
    \sigma = \Sigma(t,x) = \displaystyle\frac{x-t^3}{t+1} \\
    s = S(t,x) = t
  \end{cases}
\]
Finally, the solution is obtained as 
\[
  u(t,x) = U(\Sigma(t,x), S(t,x)) = 2t^2 + \displaystyle\frac{x-t^3}{t+1} 
\]
\pagebreak
\section*{3.}
Solve $yu_x + xu_y = u$ with $u(x,0) = x^3, u(0,y) = y^3$ for $x,y > 0$. \\ 
First, let's solve for $u(x,0) = x^3$. We have 
\[
  \displaystyle\frac{dX(\sigma; s)}{ds} = Y, 
  \displaystyle\frac{dY(\sigma; s)}{ds} = X, 
  \displaystyle\frac{dU(\sigma; s)}{ds} = U
\]
with 
\[
  X(\sigma;0) = \sigma, Y(\sigma;0) = 0, U(\sigma;0) = \sigma^3
\]
Thus, 
\[
  X(\sigma; s) = \displaystyle\frac{\sigma}{2} (e^{-s} + e^s), Y(\sigma;s) = \displaystyle\frac{\sigma}{2} (e^s - e^{-s}), U(\sigma;s) = e^{s + 3\ln(\sigma)}
\]
Next, we can obtain the inverse function, 
\[
  \begin{cases}
    x = X(\sigma;s) =  \displaystyle\frac{\sigma}{2} (e^{-s} + e^s) \\
    y = Y(\sigma;s) =  \displaystyle\frac{\sigma}{2} (- e^{-s} + e^s)
  \end{cases}
  \implies
  \begin{cases}
    x^2 = \displaystyle\frac{\sigma^2}{4} (e^{-2s} + 2 + e^{2s}) \\
    y^2 = \displaystyle\frac{\sigma^2}{4} (e^{-2s} - 2 + e^{2s}) \\
    x + y = \sigma e^s \\
    x - y = \sigma e^{-s}
  \end{cases}
\]
Hence,  
\[
  \begin{cases}
    \sigma = \Sigma(x,y) = \sqrt{x^2 - y^2} \\
    s = S(x,y) =  \displaystyle\frac{1}{2} \ln \left(\displaystyle\frac{x+y}{x-y} \right)
  \end{cases}
\]
Finally, the solution is obtained as 
\[
  u_1(x,y) = U(\Sigma(x,y), S(x,y)) = e^{\frac{1}{2} \ln\left( \frac{x+y}{x-y}\right) + \frac{3}{2} \ln(x^2-y^2)} = (x+y)^2 + x -y 
\]
and similarly, on $u_2(0,y) = y^3$, we have
\[
  u_2(x,y) = (x+y)^2 + y-x
\]
which means that $u_1(x,y) = u_2(x,y)$ on $x=y$ thus 
\[
  u(x,y) = 
  \begin{cases}
    (x+y)^2 + x - y, &\text{ if } x > y\\
    (x+y)^2 + y - x, &\text{ if } y > x\\ 
  \end{cases}
\]
\pagebreak
\section*{4.}
Solve $u_x^2 + u_y = u$ with $u(x,0) = x^2$. \\
First, let $F(x,y,u,p,q) = p^2 + q - u$, we further have
\begin{equation}
  X_0(\sigma) = \sigma, Y_0(\sigma) = 0, U_0(\sigma) = \sigma^2 
\end{equation}
We have
\[
  \begin{cases}
    P_0^2 + Q_0^2 - U_0 = P_0^2 + Q_0^2 - \sigma^2 = 0 \\
    2\sigma = U'_0(\sigma) = P_0
  \end{cases}
  \implies 
  \begin{cases}
    P_0 = 2\sigma \\
    Q_0 = \sigma \sqrt{5} 
  \end{cases}
\]
Then we have the ODE system 
\[
  \begin{cases}
    \displaystyle\frac{dX}{ds} = F_p = 2P \\
    \displaystyle\frac{dY}{ds} = F_q = 1 \\
    \displaystyle\frac{dU}{ds} = PF_p + QF_q = 2P^2 + Q \\
    \displaystyle\frac{dP}{ds} = -F_x - F_z P  =  P = 2 \sigma\\
    \displaystyle\frac{dQ}{ds} = -F_y - F_z Q = Q = \sqrt{5} \sigma 
  \end{cases}
\]
Therefore, 
\[
  \begin{cases}  
    P(\sigma, s) = 2\sigma s + 2 \sigma \\
    Q(\sigma, s) = \sqrt{5}\sigma s + \sqrt{5}\sigma
  \end{cases}
\]
We have 
\[
  \begin{cases}
    \displaystyle\frac{dX}{ds} = 4 \sigma s  + 4 \sigma \\ 
    \displaystyle\frac{dY}{ds} = 1 \\
    \displaystyle\frac{dU}{ds} = 2P^2 + Q = 8\sigma^2 (s+1)^2 + \sqrt{5} \sigma (s+1)
  \end{cases}
\]
and with the initial conditions $(1)$, we have 
\[
  \begin{cases}
    X(\sigma,s)= 2 \sigma s^2 + 4 \sigma s + \sigma \\
    Y(\sigma, s)= s \\
    U(\sigma, s)= \displaystyle\frac{8\sigma^2}{3}(s+1)^3 + \displaystyle\frac{\sqrt{5}\sigma}{2} (s+1)^2 - \displaystyle\frac{5\sigma^2}{3} - \displaystyle\frac{\sqrt{5}\sigma}{2} 
  \end{cases}
\]
Hence, we can obtain the inverse function, 
\[
  \begin{cases}
    \sigma = \displaystyle\frac{x}{4y+1}\\
    s = y
  \end{cases}
\]
and the solution  
\[
  u(x,y) = U(\sigma, s) = \left(\frac{x}{4y+1}\right)^2 \left(\displaystyle\frac{8}{3} (y+1)^3  - \displaystyle\frac{5}{3}\right) + \displaystyle\frac{\sqrt{5}x}{8y+2} ((y+1)^2 - 1) 
\]
\pagebreak
\section*{5.}
Solve $u_x u_y = xy$ with $u(x,0) = x$. \\
First, let $F(x,y,u,p,q) = pq - xy$, we further have that 
\begin{equation}
  X_0(\sigma) = \sigma, Y_0(\sigma) = 0, U_0(\sigma) = \sigma 
\end{equation}
We have 
\[
  \begin{cases}
    P_0Q_0 - X_0Y_0 = P_0 Q_0 = 0 \\
    1 = U_0'(\sigma) = X_0'(\sigma) P_0 + Y_0'(\sigma) Q_0 = P_0 
  \end{cases}
  \implies 
  \begin{cases}
    P_0 = 1 \\
    Q_0 = 0
  \end{cases}
\]
Then we have the ODE system 
\[
  \begin{cases}
    \displaystyle\frac{dX}{ds} = F_p = Q \\
    \displaystyle\frac{dY}{ds} = F_q = P \\
    \displaystyle\frac{dU}{ds} = PF_p + QF_q = 2PQ  \\
    \displaystyle\frac{dP}{ds} = -F_x - F_z P  =  Y  \\
    \displaystyle\frac{dQ}{ds} = -F_y - F_z Q = X  
  \end{cases}
\]
From which, we have
\[
  \displaystyle\frac{d^2Q}{ds^2} = \displaystyle\frac{dX}{ds} = Q 
\]
thus 
\[
  Q(\sigma, s) = C_1(\sigma) e^s + C_2(\sigma) e^{-s} \text{ with } Q(\sigma, 0) = 0
\]
and 
\[
  Q(\sigma, s) = C_5(\sigma)(e^s - e^{-s}) = C_5(\sigma) \sinh(s) 
\]
and similarly, 
\[
  P(\sigma, s) = C_3(\sigma) e^s + C_4(\sigma) e^{-s} \text{ with } P(\sigma, 0) = 1
\]
and hence $C_3 + C_4 = 1$. \\ 
Finally, we can obtain 
\[
  X = C_5\cosh(s)
\]
and 
\[
  Y = C_3(\sigma) e^s - C_4(\sigma) e^{-s} = C_3(\sigma) e^s + (C_3(\sigma) - 1) e^{-s} = C_3(\sigma)\cosh(s) - e^{-s}
\]
Now, using the initial conditions, we can get 
\[
  X = \sigma \cosh(s) \text{ and } Y = \cosh(s) - e^{-s} = \sinh(s)
\]
and the inverse functions
\[
  \begin{cases}
    \sigma = \displaystyle\frac{x}{\cosh(\ln(y + \sqrt{1+y^2}))} = \displaystyle\frac{x}{\sqrt{y^2+1}}\\
    s = \sinh^{-1}(y) = \ln(y + \sqrt{1 + y^2})
  \end{cases}
\]
We can also obtain that 
\begin{align*}
  \displaystyle\frac{dU}{ds} &= 2C_5(\sigma) \sinh(s) \left(C_3(\sigma) e^s + C_4(\sigma) e^{-s}\right) \\
  &= 2 C_5(\sigma) \sinh(s) (C_3(\sigma) e^s + (1 - C_3(\sigma)) e^{-s}) \\
  &= 2C_5(\sigma) \sinh(s) (e^{-s} + C_3(\sigma) 2\sinh(s)) \\
  &= C_6(\sigma) \sinh(s) e^{-s} + C_7(\sigma) \sinh^2(s) 
\end{align*}
Thus 
\[
  U = C_6(\sigma) \frac{\mathrm{e}^{-2s} + 2s}{4} + C_7(\sigma)\frac{\sinh\left(2s\right) - 2s}{4}
\]
with the initial conditions
\[
  \sigma = U_0(\sigma) = C_6(\sigma) \displaystyle\frac{3}{4} 
\]
Thus 
\[
  U(\sigma, s) = \displaystyle\frac{\sigma (e^{-2s} + 2s)}{3} + C_7(\sigma) \displaystyle\frac{\sinh(2s) - 2s}{4} 
\]
Therefore, the solution is 
\begin{align*}
  u(x,y) &= \displaystyle\frac{x((y+\sqrt{1+y^2})^{-2} + 2 \ln(y+ \sqrt{1+y^2}))}{3\sqrt{y^2+1}} \\
  & + C\left( \displaystyle\frac{x}{\sqrt{y^2+1}}\right) \displaystyle\frac{\sinh(2\ln(y+ \sqrt{1+y^2}))-2\ln(y + \sqrt{1+y^2})}{4}
\end{align*}
\end{document}
