\documentclass[11pt]{article}
    \title{\textbf{Math 217 Homework I}}
    \author{Khac Nguyen Nguyen}
    \date{}
    
    \addtolength{\topmargin}{-3cm}
    \addtolength{\textheight}{3cm}
    
\usepackage{amsmath}
\usepackage{mathtools}
\usepackage{amsthm}
\usepackage{amssymb}
\usepackage{pgfplots}
\usepackage{xfrac}
\usepackage{hyperref}



\newtheorem{definition}{Definition}[section]
\newtheoremstyle{mystyle}%                % Name
  {}%                                     % Space above
  {}%                                     % Space below
  {\itshape}%                                     % Body font
  {}%                                     % Indent amount
  {\bfseries}%                            % Theorem head font
  {}%                                    % Punctuation after theorem head
  { }%                                    % Space after theorem head, ' ', or \newline
  {\thmname{#1}\thmnumber{ #2}\thmnote{ (#3)}}%                                     % Theorem head spec (can be left empty, meaning `normal')

\theoremstyle{mystyle}
\newtheorem{theorem}{Theorem}[section]
\theoremstyle{definition}
\newtheorem*{exmp}{Example}
\begin{document}
\section*{Question 1.}
\subsection*{a.}
\begin{align*}
  \int_0^4 \cos(\sqrt{x}) dx &= \int_0^2 2u\cos(u) du \\
  &= 2u \sin(u) |_0^2 - 2\int_0^2 \sin(u) du \\
  &= 4 \sin(2) + 2\cos(u) |_0^2 \\
  &= 4 \sin(2) + 2\cos(2) -2 \\ 
  &\approx 0.8049
\end{align*}
\subsection*{b.}
\begin{figure}[h]
  \centering
  \includegraphics[width=0.8\textwidth]{/home/dani/Downloads/2024-12-05_11-03.png}
  \caption{}
  \label{fig:2024-12-05_11-03}
\end{figure}
Thus Simpson gives a more accurate estimation.
\subsection*{c.}
\begin{figure}[h]
  \centering
  \includegraphics[width=0.8\textwidth]{/home/dani/Downloads/2024-12-05_11-07.png}
  \caption{}
  \label{fig:2024-12-05_11-07}
\end{figure}
Minimum $m$ is thus 3.
\clearpage 
\section*{2.}

\begin{figure}[h]
  \centering
  \includegraphics[width=0.5\textwidth]{/home/dani/Downloads/2024-12-05_11-28.png}
  \caption{}
  \label{fig:2024-12-05_11-28}
\end{figure}

\begin{figure}[h]
  \centering
  \includegraphics[width=0.8\textwidth]{/home/dani/Downloads/Jacobi.png}
  \caption{}
  \label{fig:jacobi}
\end{figure}

\begin{figure}[h]
  \centering
  \includegraphics[width=0.8\textwidth]{/home/dani/Downloads/GS.png}
  \caption{}
  \label{fig:gs}
\end{figure}
\subsection*{d.}
Based on the sum of absolute value of the difference, the more accurate method is Gauss-Seidel and the fastest one is Gauss-Seidel.
\clearpage 
\section*{3.}

\begin{figure}[h]
  \centering
  \includegraphics[width=0.9\textwidth]{/home/dani/Downloads/2024-12-05_11-46.png}
  \caption{}
  \label{fig:2024-12-05_11-46}
\end{figure}
\clearpage 
\section*{4.}
\subsection*{a.}
\[
  \int_{-\pi/2}^{\pi/2} 1^2 dx = \pi 
\]
\[
  \int_{-\pi/2}^{\pi/2} x^2 dx = \displaystyle\frac{\pi^3}{12} 
\]
\[
  \int_{-\pi/2}^{\pi/2} (x^2 - \pi^2/12)^2 dx = \displaystyle\frac{\pi^5}{180} 
\]

\[
  \int_{-\pi/2}^{\pi/2} x dx = 0 
\]
\[
  \int_{-\pi/2}^{\pi/2} (x^2 - \pi^2/12) dx = 0
\]
\[
  \int_{-\pi/2}^{\pi/2} x(x^2 - \pi^2/12) dx = 0 
\]
\subsection*{b.}

\begin{figure}[h]
  \centering
  \includegraphics[width=0.8\textwidth]{/home/dani/Downloads/2024-12-05_12-07.png}
  \caption{}
  \label{fig:2024-12-05_12-07}
\end{figure}
\clearpage 
\section*{5.}
\subsection{a.}
We can first rewrite it 
\[
  F(x,y,z) = (x^2 + y-37, x-y^2-5,x+y+z-3)
\]
and the jacobian is 
\[
  \begin{pmatrix}
    2x & 1 & 0\\
    1 & -2y & 0\\
    1 & 1 & 1
  \end{pmatrix}
\]
\subsection*{b.}

\begin{figure}[h]
  \centering
  \includegraphics[width=0.8\textwidth]{/home/dani/Downloads/2024-12-05_12-31.png}
  \caption{}
  \label{fig:2024-12-05_12-31}
\end{figure}
\end{document}
