
\documentclass[11pt]{article}
    \title{\textbf{Math 217 Homework I}}
    \author{Khac Nguyen Nguyen}
    \date{}
    
    \addtolength{\topmargin}{-3cm}
    \addtolength{\textheight}{3cm}
    
\usepackage{amsmath}
\usepackage{mathtools}
\usepackage{amsthm}
\usepackage{amssymb}
\usepackage{pgfplots}
\usepackage{xfrac}  
\usepackage{hyperref}
\usepackage{xcolor}
\definecolor{Mybackground}{RGB}{40,49,51}
\pagecolor{Mybackground}
\color{white}


\newtheorem{definition}{Definition}[section]
\newtheoremstyle{mystyle}%                % Name
  {}%                                     % Space above
  {}%                                     % Space below
  {\itshape}%                                     % Body font
  {}%                                     % Indent amount
  {\bfseries}%                            % Theorem head font
  {}%                                    % Punctuation after theorem head
  { }%                                    % Space after theorem head, ' ', or \newline
  {\thmname{#1}\thmnumber{ #2}\thmnote{ (#3)}}%                                     % Theorem head spec (can be left empty, meaning `normal')

\theoremstyle{mystyle}
\newtheorem{theorem}{Theorem}[section]
\theoremstyle{definition}
\newtheorem*{exmp}{Example}
\begin{document}
\section*{2.}
\subsection*{a.}
Imagine the point on a 2 dimensianl planes Oxy. Notice that for any points labeled $-1$, the first element is less than 2 and larger than 2 otherwise thus the idea is to create a vertical lineat $2$, which is the line $x = 2$, which can be done by setting $w = [1,0]$ and $b = -2$. Indeed, plugging in we will see that $ w \cdot (x, y) + b = 0 \implies x = -b = 2$ is the line equation.
\subsection*{b.}
Notice that every point labeled -1 have their elements with absolute value 1 and labeled 1 have their elements with absolute value 2. Thus we can use any form of k-th norm to first transform it, we will use norm 1. Let $f$ be that transformation
\[
  f: \mathbb{R}^2 \to \mathbb{R}, \indent (x,y) \to |x| + |y| 
\]
Thus we can rewrite the data points, every data points labeled -1 now have value 2 and labeled 1 now have value 4. Thus we can simply compare that to number 3 which means $w = 1$ and $b = -3$.
\section*{3.}

\end{document}
